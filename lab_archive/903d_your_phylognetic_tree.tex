%!TEX TS-program = lualatex
%!TEX encoding = UTF-8 Unicode

\documentclass[12pt]{exam}
\usepackage{graphicx}
	\graphicspath{{/Users/goby/Pictures/teach/163/lab/}
	{img/}} % set of paths to search for images

\usepackage{geometry}
\geometry{letterpaper, left=1.5in, bottom=1in}                   
%\geometry{landscape}                % Activate for for rotated page geometry
%\usepackage[parfill]{parskip}    % Activate to begin paragraphs with an empty line rather than an indent
\usepackage{amssymb, amsmath}
%\usepackage{mathtools}
%	\everymath{\displaystyle}

\usepackage{fontspec}
\setmainfont[Ligatures={TeX}, BoldFont={* Bold}, ItalicFont={* Italic}, BoldItalicFont={* BoldItalic}, Numbers={OldStyle}]{Linux Libertine O}
\setsansfont[Scale=MatchLowercase,Ligatures=TeX, Numbers={OldStyle}]{Linux Biolinum O}
\usepackage{microtype}

%\usepackage{unicode-math}
%\setmathfont[Scale=MatchLowercase]{Asana Math}
%\setmathfont[Scale=MatchLowercase]{XITS Math}

% To define fonts for particular uses within a document. For example, 
% This sets the Libertine font to use tabular number format for tables.
%\newfontfamily{\tablenumbers}[Numbers={Monospaced}]{Linux Libertine O}
%\newfontfamily{\libertinedisplay}{Linux Libertine Display O}

\usepackage{booktabs}
\usepackage{multicol}
\usepackage[normalem]{ulem}

\usepackage{longtable}
%\usepackage{siunitx}

\usepackage{array}
\newcolumntype{L}[1]{>{\raggedright\let\newline\\\arraybackslash\hspace{0pt}}m{#1}}
\newcolumntype{C}[1]{>{\centering\let\newline\\\arraybackslash\hspace{0pt}}p{#1}}
\newcolumntype{R}[1]{>{\raggedleft\let\newline\\\arraybackslash\hspace{0pt}}p{#1}}

\usepackage{enumitem}
\setlist[enumerate]{font=\normalfont\scshape}
\setlist[enumerate,1]{leftmargin=*}

\usepackage{stmaryrd}

\usepackage{hyperref}
%\usepackage{placeins} %PRovides \FloatBarrier to flush all floats before a certain point.
\usepackage{hanging}

\usepackage[sc]{titlesec}

\makeatletter
\def\SetTotalwidth{\advance\linewidth by \@totalleftmargin
\@totalleftmargin=0pt}
\makeatother


\pagestyle{headandfoot}
\firstpageheader{BI 063: Evolution and Ecology}{}{\ifprintanswers\textbf{KEY}\else Name: \enspace \makebox[2.5in]{\hrulefill}\fi}
\runningheader{}{}{\footnotesize{pg. \thepage}}
\footer{}{}{}
\runningheadrule

\begin{document}


\subsection*{Depicting your hypothesis as a phylogenetic tree}

Below is a table of 21 organisms. If you do not know what some of the 
organisms are, try looking them up in a dictionary, your text, or on the 
internet (Google, as usual, is a good place to start). You will use this 
list to to make a hypothesis that shows how these organisms are or 
are not related to each other. Your hypothesis must take the form of 
a phylogenetic tree or trees. Your hypothesis should reflect \textit{your} 
ideas about whether the organisms are related in any way to each other. 

You cannot make an incorrect hypothesis, as long as you follow the simple rules
outlined below. Do not copy a tree from the internet or
from a friend. You can make as many or as few trees as
needed to convey your idea. Below, I give you few tips to help
you make a good hypothesis with correctly drawn trees.


\begin{longtable}[c]{@{}L{1in}L{0.6in}|L{1in}L{0.6in}|L{1in}L{0.6in}@{}}
\toprule
\multicolumn{2}{c|}{Organism} & 
\multicolumn{2}{c|}{Organism} & 
\multicolumn{2}{c}{Organism}\tabularnewline
%
alligator & \includegraphics[width=0.5in]{alligator_small} & 
\emph{E. coli} & \includegraphics[width=0.5in]{ecoli_small} &
marine worm & \includegraphics[width=0.5in]{marine_worm}\tabularnewline
%
bass & \includegraphics[width=0.5in]{bass} &
frog & \includegraphics[width=0.5in]{frog} &
\emph{Paramecium} & \includegraphics[width=0.5in]{paramecium}\tabularnewline
%
bat & \includegraphics[width=0.5in]{bat} & 
fungus & \includegraphics[width=0.4in]{fungus} &
pigeon & \includegraphics[width=0.5in]{pigeon}\tabularnewline
%
bison & \includegraphics[width=0.5in]{bison} &
\emph{Homo sapiens} & \includegraphics[width=0.5in]{human} &
praying mantis & \includegraphics[width=0.4in]{praying_mantis}\tabularnewline
%
cactus & \includegraphics[width=0.35in]{cactus} &
land snail & \includegraphics[width=0.5in]{land_snail} &
snake & \includegraphics[width=0.4in]{snake}\tabularnewline
%
cat & \includegraphics[width=0.4in]{cat} & 
macaque & \includegraphics[width=0.5in]{macaque} & 
wasp & \includegraphics[width=0.5in]{wasp}\tabularnewline
%
chimpanzee & \includegraphics[width=0.5in]{chimpanzee} &
maple tree & \includegraphics[width=0.4in]{maple_tree} &
whale & \includegraphics[width=0.6in]{whale}\tabularnewline
%
\bottomrule
\end{longtable}


\vspace*{\baselineskip}

\noindent\textsc{General tips}

\begin{enumerate}

\item
  Your hypothesis needs to account for all 21
  organisms listed above. This means that you \emph{must} include
  all 21 on your tree.

\item
  An easy way to start is put all the organisms across the top, grouping
  them by any relationships you will be drawing. All 21 of those are
  alive today, so all 21 had better be at the top of the tree, because
  that is the present time. 

%\end{enumerate}

\begin{longtable}[c]{@{}ll@{}}

First draw them: & Then draw the lines:\tabularnewline
%
\includegraphics[width=0.33\textwidth]{draw_step1} &
\includegraphics[width=0.33\textwidth]{draw_step2} \tabularnewline
%
\end{longtable}

%\begin{enumerate}

\item
  The vertical (Y) axis represents time. Present time is at the top. Farther
  down the tree or page represents farther back in time. The oldest
  organism is thus the farthest back, or at the bottom of your tree. If
  the first organism or organisms are among the 21 listed above,
  you can put their names at the bottom too, but you do not have to. In
  fact, I recommend that you do not. Just starting a line is okay, as
  shown in earlier exercises. Obviously, if your first organisms are 
  not among the 21, you will just have to start with lines at the bottom.

\item The horizontal (X) axis does not represent anything in most phylogenetic trees.

\item
  A line (branch) indicates an ancestor to descendant relationship, where
  successive generations would be the (invisible) points that make up
  the line leading from the start of life at the bottom to the present
  day at the top. Of course, if you want an organism to start sometime
  later, just do not start its line all the way down at the bottom. A
  fork in the line indicates a lineage splitting, as you saw in the 
  Phylogenetic Forest exercise.

\item
  You \emph{must} include a time scale labeled along the Y-axis. Do
  not simply write Present and Past. You must have a specific time
  scale, but the actual units (e.g, thousands of years, millions of
  years) is part of your hypothesis.

\item
  Only two organisms can branch from one point. That is, you should get
  a split like a “goal post” (preferred) or {\Large$\Ydown$} with one kind of 
  organism on one branch, a
  different organism on the other. This means that two groups of individuals
  from the same species got separated and one of the groups changed
  enough that they are no longer the same species. It would not happen
  that three groups would all do this at precisely the same moment.
  Hence,

\newpage

\begin{longtable}[c]{@{}lL{0.2in}lL{0.2in}l@{}}
this is possible, & &
this is \emph{not}, & &
but this is!\tabularnewline
%
\includegraphics[height=2in]{03d_possible1} & &
\includegraphics[height=2in]{03d_possible_not} & &
\includegraphics[height=2in]{03d_possible2}\tabularnewline
%
\end{longtable}


\item
  The image below shows nine examples of tree drawing, with explanations of each tree.   Some trees show common mistakes made when drawing the trees (\textsc{b}--\textsc{e}).
  The other trees show a few things that are okay (\textsc{a, f--i}).\\[1\baselineskip] \textit{Do not do \textsc{b, c,  d,} or \textsc{e}!}

\end{enumerate}

\noindent\includegraphics[width=\textwidth]{03d_example_do_dont}

\begin{enumerate}
\def\labelenumi{\alph{enumi}.}


\item
  This shows a \emph{Paramecium} giving rise first to \emph{E. coli} and
  then to \emph{Amoeba}. All three are alive today—they are
  \emph{extant}. This is okay, and a testable
  hypothesis.

\item
  This shows a \emph{Paramecium} giving rise first to \emph{E. coli} and
  \emph{Amoeba}. The \emph{Paramecium} is not alive today, though,
  according to this hypothesis. It is extinct. How do I know? It
  is not represented as a branch leading to the top of the tree. What
  this says is that it evolved into something else, but no
  \emph{Paramecium} line lasted to the present. You would not want to do
  this because \emph{Paramecium} is a living organism
  today.

\item
  This shows \emph{Paramecium} giving rise first to \emph{E. coli} and
  then \emph{Amoeba}. The \emph{E. coli} went extinct a little less
  than 2 million years ago (\textsc{mya}), the other two are extant. How do I
  know this? The \emph{E. coli} line branches off about 3 \textsc{mya} but
  stops less than 2 \textsc{mya}; this is when it became extinct.
  Again, \emph{E. coli} is a modern organism, so you cannot do
  this.

\item
  This is probably impossible. It shows \emph{Amoeba} and \textit{Paramecium}
  merging to become one organism. Other than a few rare cases with
  plants, this just does not happen. \emph{E. coli} is shown joining
  them later. Again, this is highly unlikely.
  
  Now, you may be  thinking that some organisms are formed by having two different
  species mate to produce a new species. In fact, one definition of a
  species in biology is a group of organisms that can mate with each
  other and produce fertile offspring, but \emph{cannot} mate with
  members of another species to produce fertile offspring.
  
  Horses and donkeys are two species; they
  can mate, but the offspring are mules and are sterile. This method
  of producing new species just does not work, except in a few instances
  involving \emph{very} closely related animals, or slightly less closely
  related plants. Usually, with animals, such matings only work with closely
  related species within the same genus, and not always then. 

On the other hand, members of a species \emph{can} become so different
from the rest of the species that they can no longer mate with the
others and produce fertile offspring. We then say that they constitute
a new species. So, diagrams like \textsc{a}, \textsc{b}, and \textsc{c} can happen—we have
actually observed the process of new species forming in the laboratory.
But the process shown in \textsc{d} has only been observed in a few cases with
closely related organisms.


\item
  This is a common mistake when drawing a tree. What this shows is that
  \emph{Paramecium} evolved into \emph{Amoeba} which evolved into
  \emph{E. coli}. This says that only \emph{E. coli} is alive today,
  the other two organisms are extinct, and of course, they actually
  are not.

\item
  This says that \emph{Paramecium} is now as it always has been and that
  it did not evolve into anything or evolve from anything. This is a
  plausible, testable hypothesis.

\item
  This shows the same thing as \textsc{f}. You do not need to put the name at the
  bottom. This is okay.

\item
  This shows that \emph{Paramecium} just appeared on Earth very
  recently. This is also okay.

\item
  This shows that \emph{Paramecium} appeared on Earth a little more than
  2 \textsc{mya}. Again, this is okay.

\end{enumerate}

\noindent\textbf{N\textsc{ote}: Use \emph{only} the 21 organisms listed on the first page
of this assignment. Do not include the other organisms, such as
\emph{Amoeba}, dog or gorilla, used in the examples above, or extinct organisms like dinosaurs.}


\end{document}  