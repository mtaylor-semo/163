%!TEX TS-program = lualatex
%!TEX encoding = UTF-8 Unicode

%\documentclass[12pt, addpoints, hidelinks]{exam}
\documentclass[12pt, hidelinks]{exam}
\usepackage{graphicx}
	\graphicspath{{/Users/goby/Pictures/teach/163/lab/}
	{img/}} % set of paths to search for images

\usepackage{geometry}
\geometry{letterpaper, left=1.5in, bottom=1in}                   
%\geometry{landscape}                % Activate for for rotated page geometry
\usepackage[parfill]{parskip}    % Activate to begin paragraphs with an empty line rather than an indent
\usepackage{amssymb, amsmath}
\usepackage{mathtools}
	\everymath{\displaystyle}

\usepackage{fontspec}
\setmainfont[Ligatures={TeX}, BoldFont={* Bold}, ItalicFont={* Italic}, BoldItalicFont={* BoldItalic}, Numbers={OldStyle}]{Linux Libertine O}
\setsansfont[Scale=MatchLowercase,Ligatures=TeX]{Linux Biolinum O}
\setmonofont[Scale=MatchLowercase]{Inconsolatazi4}
\usepackage{microtype}


% To define fonts for particular uses within a document. For example, 
% This sets the Libertine font to use tabular number format for tables.
 %\newfontfamily{\tablenumbers}[Numbers={Monospaced}]{Linux Libertine O}
% \newfontfamily{\libertinedisplay}{Linux Libertine Display O}

\usepackage{booktabs}
\usepackage{multicol}
\usepackage[normalem]{ulem}

\usepackage{longtable}
%\usepackage{siunitx}
\usepackage{array}
\newcolumntype{L}[1]{>{\raggedright\let\newline\\\arraybackslash\hspace{0pt}}p{#1}}
\newcolumntype{C}[1]{>{\centering\let\newline\\\arraybackslash\hspace{0pt}}p{#1}}
\newcolumntype{R}[1]{>{\raggedleft\let\newline\\\arraybackslash\hspace{0pt}}p{#1}}

\usepackage{enumitem}
\usepackage{hyperref}
%\usepackage{placeins} %PRovides \FloatBarrier to flush all floats before a certain point.
\usepackage{hanging}

\usepackage[sc]{titlesec}

%% Commands for Exam class
\renewcommand{\solutiontitle}{\noindent}
\unframedsolutions
\SolutionEmphasis{\bfseries}

\renewcommand{\questionshook}{%
	\setlength{\leftmargin}{-\leftskip}%
}

%Change \half command from 1/2 to .5
\renewcommand*\half{.5}

\pagestyle{headandfoot}
\firstpageheader{\textsc{bi}\,063 Evolution and Ecology}{}{\ifprintanswers\textbf{KEY}\else Name: \enspace \makebox[2.5in]{\hrulefill}\fi}
\runningheader{}{}{\footnotesize{pg. \thepage}}
\footer{}{}{}
\runningheadrule

\newcommand*\AnswerBox[2]{%
    \parbox[t][#1]{0.92\textwidth}{%
    \begin{solution}#2\end{solution}}
%    \vspace*{\stretch{1}}
}

\newenvironment{AnswerPage}[1]
    {\begin{minipage}[t][#1]{0.92\textwidth}%
    \begin{solution}}
    {\end{solution}\end{minipage}
    \vspace*{\stretch{1}}}

\newlength{\basespace}
\setlength{\basespace}{5\baselineskip}

\pointsinmargin{\pointformat{}}
%\printanswers


\begin{document}

\subsection*{Homology or analogy: comparing skeletons}

\begin{questions}

\question
Consider the uses of teeth in catching and killing prey,
cutting tough plant material, grinding hard seeds or nuts, etc. Look at
the teeth in the three jaws provided in lab. These are also shown below.
What kinds of food do you think were eaten by each skeleton's former
owner? \emph{Justify your answer} based on what you observe about
their teeth. (You can easily figure out what the
animals are. Do not base your explanation on what you know the animal
eats. Base it on what you see in its teeth.)

\begin{parts}

\part Organism 1
\begin{longtable}[l]{@{}ll@{}}
\includegraphics[height=3cm]{05_jaw_cat1} & 
\includegraphics[height=3cm]{05_jaw_cat2}\tabularnewline
\end{longtable}

\AnswerBox{1\baselineskip}{Carnivore. Sharp teeth for capturing and tearing prey.}

\part Organism 2-Hour
\begin{longtable}[l]{@{}ll@{}}
\includegraphics[height=3cm]{05_jaw_horse1} & 
\rotatebox{90}{\includegraphics[width=3cm]{05_jaw_horse2}}\tabularnewline
\end{longtable}

\AnswerBox{1\baselineskip}{Herbivore. Large flat teeth for grinding plant material.}

\part Organism 3
\begin{longtable}[l]{@{}ll@{}}
\includegraphics[height=3cm]{05_jaw_chimp1}  &
\includegraphics[height=3cm]{05_jaw_chimp2}\tabularnewline
\end{longtable}

\AnswerBox{1\baselineskip}{Omnivore. Sharp teeth for tearing and flat teeth for grinding.}

\end{parts}

\newpage

\question[2]
Consider the canine teeth of a cat and a Mexican wolf. The canine teeth
are elongated relative to other teeth. Both species are carnivores. %Elongated canine teeth are
%typical of carnivores, the meat-eating mammals, and this is a structural
%similarity in these animals.

\begin{longtable}[c]{@{}cc@{}}
\toprule
\includegraphics[height=3.5cm]{05_skull_cat} &
\includegraphics[height=3.5cm]{05_skull_mexican_wolf}\tabularnewline
%\midrule
Cat Skull & Mexican Wolf\tabularnewline
\bottomrule
\end{longtable}

\begin{parts}

\part
Is the similarity of elongated canines \emph{necessary for
the function} in these carnivores? Explain.

\AnswerBox{3\baselineskip}{Most students will answer yes, necessary to capture prey and tear meat.}

\part
What do you conclude? Is the similarity a homology or
analogy?

\AnswerBox{3\baselineskip}{Most students will conclude analogy. Accept homology if well justified above.}

\end{parts}

\question[2]
Now consider the canine teeth of fruit bats. They have elongated canine teeth but they are
\emph{not} carnivores. They are frugivores (fruit-eaters) or nectivores 
(lap nectar with the tongue).

\begin{longtable}[c]{@{}c@{}}
\toprule
\includegraphics[height=3.5cm]{05_skull_fruit_bat}\tabularnewline
Fruit Bat\tabularnewline
\bottomrule
\end{longtable}

\begin{parts}
\part
Is the similarity of elongated canines necessary for
the function? Explain. 

\AnswerBox{3\baselineskip}{No. Canines are not necessary, especially for nectivores. Some will argue that canines help frugivores but point out they can eat fruit without enlarged canines.}

%\newpage

\part What do you conclude? Is the similarity a homology or
analogy?

\AnswerBox{3\baselineskip}{Most students will conclude homology.}

\end{parts}


Now have a look at the manus (``hand'') of the bat and the macaque 
(muh-KACK, a type of primate; more specifically, an Old World monkey). 
The manus is the appendage on the end of the forelimb—most of the 
wing of the bat and the a``front foot'' in the macaque. 

\question
Consider the following structural similarity: \emph{Each manus
has five digits or ``fingers.''}
\begin{parts}
	\part What is the function of the manus in the bat?

	\AnswerBox{3\baselineskip}{Flight. \emph{This question has no point value.}}

	\part What is the function of the manus in the macaque?

	\AnswerBox{3\baselineskip}{Grasping, climbing. \emph{This question has no point value.}}

	\part[1] Is the specific similarity described above (that is, having precisely 5
digits, not 4, 6, or some other number) necessary in order to serve the
functions you described above? In other words, are the structures
serving the same functions in the bat and the macaque? Explain.
(\emph{Remember: if a structure serves different functions, the specific
structural similarity is not absolutely required to perform any} one
\emph{of those functions}.

	\AnswerBox{4\baselineskip}{Should determine that five digits is not necessary to serve any one function.}

	\part[1] What do you conclude? Is the structural similarity I described above
explained by homology or analogy? Explain.

	\AnswerBox{4\baselineskip}{Homology because the structures serve different functions.}
\end{parts}


Now compare the manus of the alligator and the
\textit{Necturus} (a type of salamander), to the manus of the macaque.

\question
Consider the following similarity: \emph{Each digit has three
phalanges} (individual bones in each digit).

\begin{parts}
	\part What is the function of the three phalanges in the macaque?

	\AnswerBox{3\baselineskip}{Allow the digits to bend for grasping, etc. \emph{This question has no point value.}}


	\part What is the function of the three phalanges in the alligator and
\emph{Necturus}?

	\AnswerBox{3\baselineskip}{Support, spread the weight. \emph{This question has no point value.}}

	\part[1] Is the specific similarity described above (that is, having precisely 3
phalanges, not 2, 4, or some other number) necessary in order to serve
the specific functions you described above in the macaque, the alligator
and \emph{Necturus}? Explain.

	\AnswerBox{4\baselineskip}{No. The phalanges serve different functions. Could bend
	with more phalanges or even one less.}

	\part[1] What do you conclude? Is the structural similarity I described above
explained by homology or analogy? Explain.

	\AnswerBox{4\baselineskip}{Homology. The phalanges serve different functions.}


\end{parts}

\question
Now, compare the clavicle (``collar bone'') of a macaque to the clavicle of a cat. In both animals, the clavicle is located between the top of the sternum and the shoulder.

\begin{longtable}[c]{@{}rll@{}}
\toprule
Macaque & \includegraphics[width=0.3\textwidth]{05_clavicle_macaque1} &
\includegraphics[width=0.3\textwidth]{05_clavicle_macaque2}\tabularnewline
\midrule
Cat & \includegraphics[width=0.3\textwidth]{05_clavicle_cat2} &
\includegraphics[width=0.3\textwidth]{05_clavicle_cat1}\tabularnewline
\bottomrule
\end{longtable}

In the macaque (upper row), the clavicle is the only bone connection between
the forelimb and the trunk of the body. Note how it makes contact with
the sternum at one end and the shoulder blade at the other. 


In the cat (lower row), you can see looking down (left photo) that
there is no bone connection between the shoulder blade and the trunk of
the body (it didn't have those metal springs in there when it was
alive). The arrow in the right photo of a cat's shoulder shows
the cat's clavicle. It is approximately the same shape as the
macaque's, and is located in the same area of the body. (It also forms
the same way during embryonic development.) The cat's clavicle doesn't touch
any other bone or cartilage in the cat; it is embedded in a muscle on the
chest. The clavicle's position here is slightly off to the side from where it's
found in the living cat because it has been attached with a wire to the
humerus.

\begin{parts}

\part Describe their structural similarity. What is similar about their 
structure, that is, their shape, location, composition, etc.?

	\AnswerBox{4\baselineskip}{\emph{This question has no point value.}}

\part What is the function of the clavicle in each organism? For the 
macaque, consider how monkeys climb trees, hanging from branches
and swinging. How would a bone connection between the arm and chest be
helpful? For the cat, what good is a tiny little bone that doesn't touch
any other bone?

macaque: \ifprintanswers{\textbf{\textit{This question has no point value.}}}\fi\\[2\baselineskip]

cat: \\[2\baselineskip]

\part[1] Is the similarity you described above necessary in order to
serve the functions? Explain.

	\AnswerBox{4\baselineskip}{No. The clavicle serves the function of strengthening the shoulder girdle in the macaque (they will not say shoulder girdle). The clavicle does not serve a function in the cat.}

\part[1] What do you conclude? Is the structural similarity you
described explained by homology or analogy?

	\AnswerBox{4\baselineskip}{Homology. The clavicle serves different functions between the two animals.}

\part[1] If you concluded homology of the clavicle, do you think this
homology applies only to cats and macaques? What if we compared the
clavicle of the cat to the clavicle of the frog, the human, the
alligator, or the bat? Would you conclude that the clavicle is a
homology or analogy? Explain.

	\AnswerBox{4\baselineskip}{Homology. In most vertebrates, the clavicle provides rigidity and support. But, not for the cat. If each vertebrate is compared to the cat, would conclude homology.}

\end{parts}

\question
Consider the forelimb of the human and the pigeon. Each
has a first segment containing a single bone, the humerus, followed by a
second segment containing two bones, the radius and ulna. The structural similarity here is {the presence of two bones in the
second section of the forelimb of pigeon and human.}


\begin{longtable}[c]{@{}L{2.5in}L{2.5in}l@{}}
\toprule
\centering \includegraphics[height=3cm]{05_radius_human1} & \centering \includegraphics[height=3cm]{05_radius_human2}\tabularnewline
\centering \includegraphics[height=3cm]{05_radius_pigeon} & \centering \includegraphics[height=3cm]{05_radius_human3}\tabularnewline

The pigeon's wing is folded. The joint between the humerus and
the radius and ulna (``elbow'') is along the back of the body. The ``hand'' of the pigeon
contains three digits, two of them mostly fused together. & The upper photo shows the radius
and ulna articulating with the wrist in the human arm. The lower photo shows the
human ulna (lower left) and radius (upper left) forming the elbow joint
with the humerus (right).\tabularnewline
\bottomrule
\end{longtable}

\begin{parts}
	\part What is the function of the radius and ulna in each
animal? Hint: feel your own arm at the wrist and move your hand
various ways. Twist your wrist. What sort of motion do two bones allow?
Do birds make similar motions with their wings?

human: \ifprintanswers{\textbf{\textit{This question has no point value}}}\fi\\[2\baselineskip]

pigeon: \\[2\baselineskip]

\part[1] Is the similarity (having two bones in the forearm)
necessary in order to serve the \emph{function(s)}? Explain.

	\AnswerBox{4\baselineskip}{No. The radius and ulna allows for rotation in human but not pigeon.}

\part[1] What do you conclude? Is the structural similarity you
described explained by homology or analogy? Explain.

	\AnswerBox{4\baselineskip}{Homology. The radius and ulna serve different functions.}

\part[1] If you concluded homology of the radius and ulna, do you
think this homology applies only to humans and pigeons? What if we
compared the radius and ulna of the human to the other vertebrate
organisms that have them? What might you conclude? Explain.

	\AnswerBox{2\baselineskip}{The radius and ulna are homologous among the vertebrates. In most cases, they do not allow rotation like humans.}
\end{parts}

\end{questions}

\end{document}  