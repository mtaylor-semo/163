%!TEX TS-program = lualatex
%!TEX encoding = UTF-8 Unicode

\documentclass[t]{beamer}

%%%% HANDOUTS For online Uncomment the following four lines for handout
%\documentclass[t,handout]{beamer}  %Use this for handouts.
%\includeonlylecture{student}
%\usepackage{handoutWithNotes}
%\pgfpagesuselayout{3 on 1 with notes}[letterpaper,border shrink=5mm]


%%% Including only some slides for students.
%%% Uncomment the following line. For the slides,
%%% use the labels shown below the command.

%% For students, use \lecture{student}{student}
%% For mine, use \lecture{instructor}{instructor}


%\usepackage{pgf,pgfpages}
%\pgfpagesuselayout{4 on 1}[letterpaper,border shrink=5mm]

% FONTS
\usepackage{fontspec}
\def\mainfont{Linux Biolinum O}
\setmainfont[Ligatures={Common,TeX}, Contextuals={NoAlternate}, Numbers={Proportional, OldStyle}]{\mainfont}
\setsansfont[Ligatures={Common,TeX}, Scale=MatchLowercase, Numbers={Proportional,OldStyle}, BoldFont={* Bold}, ItalicFont={* Italic},]\mainfont

\newfontface\lining[Numbers={Lining}]\mainfont

\usepackage{graphicx}
	\graphicspath{{/Users/goby/pictures/teach/163/exam/}
	{/Users/goby/pictures/teach/common/}} % set of paths to search for images

%\usepackage{units}
\usepackage{booktabs}
\usepackage{multicol}
%\usepackage{textcomp}

\usepackage{tikz}
%	\tikzstyle{every picture}+=[remember picture,overlay]

\mode<presentation>
{
  \usetheme{Lecture}
  \setbeamercovered{invisible}
  \setbeamertemplate{items}[square]
}

%\usefonttheme[onlymath]{serif}
%\usecolortheme[named=blue7]{structure}

\newcommand{\btVFill}{\vskip0pt plus 1filll}

\begin{document}

{
\usebackgroundtemplate{\includegraphics[width=\paperwidth]{aztec_calendar} }
\begin{frame}[b]

\hfill \tiny Aztec Sun Calendar, Wikimedia Commons, \ccbysa{3}
\end{frame}
}


\begin{frame}{A typical exam consists of}

	\hangpara \textasciitilde10 matching questions (w/ word bank),
	
	\hangpara \textasciitilde10 multiple choice questions,
	
	\hangpara \textasciitilde10 true/false questions, and
	
	\hangpara short answer questions.
	
	\hangpara \highlight{Don't forget to study your lab material!}
\end{frame}
%

\begin{frame}{Do not assume multiple choice questions will be easy.}

\hangpara Which of the following is the correct answer?

\hangpara \quad A. distractor answer

\hangpara \quad B. distractor answer

\hangpara \quad C. correct answer

\hangpara \quad D. distractor answer

\hangpara \quad E. distractor answer


\end{frame}

\begin{frame}

The Greater Prairie Chicken was once widespread throughout midwestern prairies but now remains in only a few very small populations. The decrease occurred rapidly due to habitat loss. Studies showed that the remaining populations have little or no genetic variation. The best explanation for this result is:


\hangpara \textsc{a.} A founder event from the widespread population to the smaller populations caused an increase in homozygosity.

\hangpara \textsc{b.} Genetic drift caused the loss of heterozygosity. 

\hangpara \textsc{c.} Inbreeding caused an increase in heterozygosity. 

\hangpara \textsc{d.} Natural selection favored only one or two alleles that increased survivorship.

\hangpara \textsc{e.} A bottleneck caused the loss of homozygosity. %EndChoice

\end{frame}

\begin{frame}{\highlight{Rise} to the occasion.}
%
	\hangpara \begin{quote}The exams do not need to be so hard and tricky in order to measure how much we have learned. You can tell how much we have learned with an easier test.
	\end{quote}
%
	\hangpara \begin{quote}I really do enjoy your tests. It is not just definitions, you made us apply what we learn to difficult questions.
	\end{quote}
%
\end{frame}

\begin{frame}{You can start the test around 7:30~\textsc{am}.}

You must start by 8:00~\textsc{am} and end by 8:55~\textsc{am}.
\end{frame}

\end{document}
