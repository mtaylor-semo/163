%!TEX TS-program = lualatex
%!TEX encoding = UTF-8 Unicode

%\documentclass[t]{beamer}

%%%% HANDOUTS For online Uncomment the following four lines for handout
\documentclass[t,handout]{beamer}  %Use this for handouts.
\usepackage{handoutWithNotes}
\pgfpagesuselayout{3 on 1 with notes}[letterpaper,border shrink=5mm]

\includeonlylecture{student}

% FONTS
\usepackage{fontspec}
\def\mainfont{Linux Biolinum O}
\setmainfont[Ligatures={Common,TeX}, Contextuals={NoAlternate}, BoldFont={* Bold}, ItalicFont={* Italic}, Numbers={OldStyle}]{\mainfont}
\setsansfont[Ligatures={Common,TeX}, Scale=MatchLowercase, Numbers=OldStyle, BoldFont={* Bold}, ItalicFont={* Italic},]{Linux Biolinum O} 
\usepackage{microtype}

\usepackage{graphicx}
	\graphicspath{{/Users/goby/pictures/teach/163/lecture/}
	{/Users/goby/pictures/teach/common/}} % set of paths to search for images

\usepackage{multicol}
%\usepackage{booktabs}
%\usepackage{textcomp}

\usepackage{tikz}
	\tikzstyle{every picture}+=[remember picture,overlay]
%	\usetikzlibrary{arrows}

\usepackage{amsmath}
%\usepackage{units}
%\usepackage{booktabs}

%\usepackage{tikz}

%\usepackage{ifthen}

\mode<presentation>
{
  \usetheme{Lecture}
  \setbeamercovered{invisible}
%  \setbeamertemplate{items}[square]
}

\usefonttheme[onlymath]{serif}


\begin{document}

% Lecture goals
\begin{frame}{Our goal for this lecture is to}
	
	\hangpara introduce the Hardy-Weinberg equations, and 
	
	\hangpara learn how to calculate allele frequencies and genotype frequencies for a population. 
	
	\vspace*{2\baselineskip}
	
	\hangpara How can we test for the evolution of populations?
	
\end{frame}
%

\begin{frame}[t]{The \highlight{Hardy-Weinberg} equations test for equilibrium of allele frequencies in a population.}

	\vspace*{-\baselineskip}
	
	\begin{multicols}{2}
	
		{\centering \reflectbox{\includegraphics[height=0.7\textheight]{hardy}}\\
		G.H. Hardy\par
		}
		
	\columnbreak
	
		{\centering \includegraphics[height=0.7\textheight]{weinberg}\\
		Wilhelm Weinberg\par
		}
		
	\end{multicols}
	
\end{frame}

\begin{frame}[t]{Two equations show that populations cannot evolve if a population meets five assumptions.}

	\vspace*{-\baselineskip}
	
	\begin{multicols}{2}

	\hangpara \highlight{No} genetic drift,
	
	\hangpara \highlight{No} gene flow,

	\hangpara \highlight{No} mutation,
	
	\hangpara \highlight{No} migration, and
	
	\hangpara \highlight{No} natural selection.
	
	\columnbreak

	\hangpara $p + q = 1$
	
	\hangpara $p^2 + 2pq + q^2 = 1$
	
	\end{multicols}

\end{frame}

% Hardy-Weinberg Equilibrium
\begin{frame}[t]{The Hardy-Weinberg equations act as a null hypothesis.}

	\hangpara When the five assumptions are met, the equations show a population will not evolve.
	
	\hangpara If at least one assumption is violated, then a population will evolve.
	
	\hangpara We can learn how populations evolve by comparing real populations to the null hypothesis.

	\hangpara
	
	\hangpara To learn how to apply the equations, we must first learn how to calculate allele and genotype frequencies.


\end{frame}


% Genotype and phenotype
{
\usebackgroundtemplate{\includegraphics[width=\paperwidth]{chromosomes}}
\begin{frame}[c,plain]{\highlight{Diploid} organisms have two alleles for each gene.}
	\begin{tikzpicture}[remember picture, overlay]

		\node at (3.45,1) (Callele) {$C$};
		\node at (3.45,-1.75) {$C$};

		\node at (6.45,1) (aallele) {$a$};
		\node at (6.45,-1.75) {$a$};

		\node at (9.45,1) (Tallele) {$T$};
		\node at (9.45,-1.75) {$t$};

		\node at (6.4, 2.75) (allele) {Allele};
		\draw (allele.south west) -- (Callele.north east);
		\draw (allele.south) -- (aallele.north);
		\draw (allele.south east) -- (Tallele.north west);

	\end{tikzpicture}
\end{frame}
}
%
\begin{frame}{The total number of alleles in a population for one gene is the population size times 2.}

	\hangpara Assume we have a population of 100 individuals. 
	
	\hangpara If the population size is 100 individuals, then the total number of alleles is $100 \times 2 = 200.$

	\hangpara Assume that one gene in the population has both \textit{T} and \textit{t} alleles.
	
\end{frame}
%
\begin{frame}{\highlight{Allele frequency} is a measure of how common each allele is in the population.}

	\hangpara Assume that 120 of the 200 alleles are \textit{T} and 80 of the 200 are \textit{t.}
	
	\hangpara The allele frequency of \textit{T} is $120/200 = 0.6.$

	\hangpara The allele frequency of \textit{t} is $80/200 = 0.4.$

	\hangpara Frequencies are expressed as decimal fractions between 0 and 1.
	
\end{frame}

%

% Consider a population: Phenotype
\begin{frame}{Calculate allele frequencies for 500 individuals.}
	\vspace{-1\baselineskip}
	\begin{center}
		\includegraphics[width=0.7\textwidth]{genotype_carnations.jpg}
	\end{center}
	
	\vspace{-\baselineskip}
	\hangpara How many total alleles are in the population?
	\pause
		
	\hangpara How many $C^R$ alleles are in the population?
	\pause

	\hangpara How many $C^W$ alleles are in the population?
	\pause

	\hangpara What is the frequency (proportion) of each allele in the population?
	
\end{frame}

% First Hardy-Weinberg Equation
\begin{frame}{Allele frequencies for a gene must sum to 1.}
	
	\hangpara If a gene has only two alleles, then

	\hangpara $p=$ the frequency of allele 1 in a population, and
	\pause
		
	\hangpara $q=$ the frequency of allele 2 in a population, therefore
	\pause

	\hangpara $p+q=1.$ 
	
	\vspace{\baselineskip}
	\hangpara \highlight{The allele frequencies must sum to 1.}
	
\end{frame}
%
\begin{frame}{\highlight{Genotype frequency} is a measure of how common each genotype is in the population.}

	\hangpara With two alleles, only three genotypes are possible: \textit{TT, Tt,} and \textit{tt.}
	
	\hangpara Assume that 36 individuals are \textit{TT,} 48 individuals are \textit{Tt,} and 16 individuals are \textit{tt.}
	
	\hangpara The genotype frequency of \textit{TT} is $36/100 = 0.36.$

	\hangpara The genotype frequency of \textit{Tt} is $48/100 = 0.48.$
	
	\hangpara The genotype frequency of \textit{tt} is $16/100 = 0.16.$

	

\end{frame}

% Consider a population: Genotype
\begin{frame}{Consider a population of 500 diploid individuals.}

	\vspace{-\baselineskip}
	\begin{center}
		\includegraphics[width=0.7\textwidth]{genotype_carnations.jpg}
	\end{center}

	\vspace{-\baselineskip}
	\hangpara Given two alleles, how many genotypes are \textit{possible}?
	\pause
		
	\hangpara What is the frequency of the $C^RC^R$ genotype?
	\pause

	\hangpara How the frequency of the $C^RC^W$ genotype?
	\pause

	\hangpara What is the frequency of the $C^WC^W$ genotype?
	
\end{frame}

% Second Hardy-Weinberg Equation
\begin{frame}{Genotype frequencies for a gene must sum to 1.}

	\hangpara If a gene has only three genotypes, then

	\hangpara $p^2=$ the frequency of homozygote 1 in a population, and
	\pause
		
	\hangpara $q^2=$ the frequency of homozygote 2 in a population, and
	\pause

	\hangpara $2pq=$ the frequency of heterozygotes in a population, therefore
	\pause
	 
	\hangpara $p^2 + 2pq + q^2=1.$

	\vspace{\baselineskip}
	\hangpara \highlight{The genotype frequencies must sum to 1.}
\end{frame}

% Relate the equations
\begin{frame}{How do these equations relate?}
	\vspace{\baselineskip}
	\centering
	\hspace{6em}$p+q=1$\hfill$p^2 + 2pq + q^2=1$\hspace{6em}
	
\end{frame}

\end{document}
