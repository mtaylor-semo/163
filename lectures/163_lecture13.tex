%!TEX TS-program = lualatex
%!TEX encoding = UTF-8 Unicode

\documentclass[t]{beamer}

%%%% HANDOUTS For online Uncomment the following four lines for handout
%\documentclass[t,handout]{beamer}  %Use this for handouts.
%\includeonlylecture{student}
%\usepackage{handoutWithNotes}
%\pgfpagesuselayout{3 on 1 with notes}[letterpaper,border shrink=5mm]


%%% Including only some slides for students.
%%% Uncomment the following line. For the slides,
%%% use the labels shown below the command.

%% For students, use \lecture{student}{student}
%% For mine, use \lecture{instructor}{instructor}

% Fonts
\usepackage{fontspec}
\def\mainfont{Linux Biolinum O}
\setmainfont[Ligatures={Common,TeX}, Contextuals={NoAlternate}, BoldFont={* Bold}, ItalicFont={* Italic}, Numbers={OldStyle}]{\mainfont}
\setsansfont[Ligatures={Common,TeX}, Scale=MatchLowercase, Numbers=OldStyle, BoldFont={* Bold}, ItalicFont={* Italic},]{Linux Biolinum O} 
\usepackage{microtype}

\usepackage{graphicx}
	\graphicspath{{/Users/mtaylor/pictures/teach/163/lecture/}
	{/Users/mtaylor/pictures/teach/common/}} % set of paths to search for images

\usepackage{multicol}
\usepackage{booktabs}
\usepackage{array}
\newcolumntype{L}[1]{>{\raggedright\let\newline\\\arraybackslash\hspace{0pt}}p{#1}}
\newcolumntype{C}[1]{>{\centering\let\newline\\\arraybackslash\hspace{0pt}}p{#1}}
\newcolumntype{R}[1]{>{\raggedleft\let\newline\\\arraybackslash\hspace{0pt}}p{#1}}
%\usepackage{textcomp}
%\usepackage{mhchem}
\usepackage{enumitem}
%\usepackage[export]{adjustbox}


\usepackage{tikz}
	\tikzstyle{every picture}+=[remember picture,overlay]
	\usetikzlibrary{arrows}
\usetikzlibrary{positioning}

\mode<presentation>
{
  \usetheme{Lecture}
  \setbeamercovered{invisible}
  \setbeamertemplate{items}[square]
}


\begin{document}

\lecture{student}{student}

\begin{frame}{Our goal for this lecture is to }

%	\hangpara learn about the \highlight{hierarchical organization} of life.
	
	\hangpara introduce \highlight{ecology,} and
	
	\hangpara demonstrate \highlight{population growth,}
	
	\hangpara using \highlight{exponential} and \highlight{logistic} growth models.

\end{frame}
%
%\begin{frame}{Life is a hierarchy of increasing complexity.}
%\vspace*{-0.5\baselineskip}	
%\hangpara \begin{tabular}{@{}L{34mm}L{34mm}L{34mm}@{}}
%\toprule
%\onslide<1->Cellular 	&
%Organismal 		&
%Biosphere 		\\
%\midrule
% &	&	\\
%\onslide<2->Cell		&
%\onslide<3->Organism	&
%\onslide<9->\alt<handout>{\rule{30mm}{0.4pt}}{Landscape}\\[1em]
%%
%\onslide<2->Organelle		&
%\onslide<3->Organ System	&
%\onslide<8->\alt<handout>{\rule{30mm}{0.4pt}}{Ecosystem}\\[1em]
%%
%\onslide<2->Macromolecules	&
%\onslide<3->Organ			&
%\onslide<7->\alt<handout>{\rule{30mm}{0.4pt}}{Community} \\[1em]
%%
%\onslide<2->Molecule		&
%\onslide<3->Tissue			&
%\onslide<6->\alt<handout>{\rule{30mm}{0.4pt}}{Species}	\\[1em]
%%
%\onslide<2->Atom			&
%						&
%\onslide<5->\alt<handout>{\rule{30mm}{0.4pt}}{Population}\\[1em]
%%\bottomrule
%\end{tabular}
%
%\only<4>{\alt<handout>{}{\hangpara In groups, discuss what levels make up the biosphere.}}
%
%\onslide<10->\hangpara\highlight{Emergent properties} are properties of life present at one level that were not present in lower levels.
%
%\onslide<11->\hangpara What levels do you think are encompassed by evolution?  By ecology? 
%\end{frame}
%
{
\usebackgroundtemplate{\includegraphics[width=\paperwidth]{ecosystem} }
\begin{frame}[b]{\hfill What is ecology?}
	\tiny\textcolor{white}{Jeff Taylor, Wikimedia, \ccbysa{3}.}%Elwha River Photo by
\end{frame}
}
%
\begin{frame}[t]{Ecology has many fields of study.}
	\begin{minipage}[t]{0.6\textwidth}
		\vspace{0pt}
		\includegraphics[width=0.99\textwidth]{ecology_disciplines}
	\end{minipage}
	\hfill
	\begin{minipage}[t]{0.3\textwidth}
		\vspace{0pt}
		\highlight{Population}\\
		\highlight{Community}\\
		\highlight{Ecosystem}\\
		Landscape\\
		Evolutionary\\
		Behavioral\\
		Nutritional\\
		many more\dots
	\end{minipage}	
\end{frame}
%
{
\usebackgroundtemplate{\includegraphics[width=\paperwidth]{population_meerkats} }
\begin{frame}[b]{\textcolor{white}{What is a population?}}

	\tiny\textcolor{white}{Spencer Wright, Flickr, \ccby{2}.}
\end{frame}
}
%
{
\usebackgroundtemplate{\includegraphics[width=\paperwidth]{population_dynamics} }
\begin{frame}[b]{Populations vary in space and time.}

	\hfill \tiny Fig. 53.3 \textcopyright Pearson Education, Inc.
\end{frame}
}
%
{
\usebackgroundtemplate{\includegraphics[width=\paperwidth]{Ecoli} }
\begin{frame}[b]{\textcolor{white}{Meet \textit{Escherichia coli.}}}
	\tiny \hfill \textcolor{white}{NIAID, National Institutes of Health, public domain.}%E. coli
\end{frame}
}
%
\begin{frame}{}
	\hangpara\Large What factors do we need to account for population growth under \highlight{idealized} conditions? 
\end{frame}
%
\begin{frame}{Population growth depends on birth and death rates.}

	\hangpara \textit{b} = per capita birth rate (number born per 1000 individuals). \pause
	
	\hangpara \textit{d} = per capita death rate (number died per 1000 individuals). \pause
	
	\hangpara \textit{r} = \textit{b} – \textit{d} = intrinsic rate of population growth. \pause
	
	\hangpara If \textit{r} \textgreater{} 0, then population is growing larger.
	
	\hangpara If \textit{r} \textless{} 0 then population is growing smaller.
	
	
\end{frame}
%
\begin{frame}{\highlight{Exponential growth} occurs only in idealized conditions.}
	\begin{center}
		\Huge $\frac{\Delta N}{\Delta t} = rN$
	\end{center}
	
	\hangpara $\Delta N = $ change in population size,

	\hangpara $\Delta t = $ change in time,
	
	\hangpara $r = $ intrinsic rate of population growth, and
	
	\hangpara $N = $ size of the population.
	
\end{frame}
%
{
\usebackgroundtemplate{\includegraphics[width=\paperwidth]{exponential_growth}}
\begin{frame}[b]
\end{frame}
}
%
{
\usebackgroundtemplate{\includegraphics[width=\paperwidth]{human_population_growth}}
\begin{frame}[b]
\end{frame}
}
%
\begin{frame}{}

	\hangpara\Large Most populations do not grow under idealized conditions.
	
	\hangpara\Large What else do we need to account for?
	
	\pause
	\hangpara \highlight{Carrying capacity.}
	
\end{frame}
%
\begin{frame}{\highlight{Logistic growth} occurs in natural conditions.}
	\begin{center}
		\Huge $\frac{\Delta N}{\Delta t} = rN\left(\frac{K-N}{K}\right)$
	\end{center}
	
	\hangpara $\Delta N = $change in population size,

	\hangpara $\Delta t = $ change in time,
	
	\hangpara $r = $ intrinsic rate of population growth, 
	
	\hangpara $N = $ size of the population, and
	
	\hangpara \highlight{$K = $ carrying capacity.}
	
\end{frame}
%
\begin{frame}
	\hangpara Assume $r = 1.0$.
	
%	\hangpara If $K = 3000$ and $N = 1000$, \pause then
	\hangpara If $K = 3000$ and $N = 30$, \pause then
	
	\hangpara {$\dfrac{K-N}{K} = \dfrac{(3000-30)}{3000} =  \dfrac{2930}{3000} = 0.99$}
	
	\hangpara This means 99\% of resources are \emph{still available} for population growth. \pause Therefore,
	
	\hangpara $\dfrac{\Delta N}{\Delta t} = rN\left(\dfrac{K-N}{K}\right) = (1.0)(30)\left(\dfrac{2930}{3000}\right) = 29.7$ individuals added. \pause
	
	\hangpara The population nearly \emph{doubled} in size.
	
	\hangpara \textcolor{gray}{Exponential growth: $rN = (1)(30) = 30$ individuals added.}
\end{frame}
%
\begin{frame}
	\hangpara Assume $r = 1.0$.
	
	\hangpara If $K = 3000$ and $N = 2700$, \pause then
	
	\hangpara {$\dfrac{K-N}{K} = \dfrac{(3000-2700)}{3000} =  \dfrac{300}{3000} = 0.10$.}
	
	\hangpara This means that 10\% of resources are available for population growth. \pause Therefore,
	
	\hangpara $\dfrac{\Delta N}{\Delta t} = rN\left(\dfrac{K-N}{K}\right) = (1.0)(2700)\left(\dfrac{300}{3000}\right) = 270$ individuals added. \pause
	
	\hangpara The population increased by only 10\%.
	
	\hangpara \textcolor{gray}{Exponential growth: $rN = (1)(2700) = 2700$ individuals added.}
\end{frame}
%
{
\usebackgroundtemplate{\includegraphics[width=\paperwidth]{logistic_growth}}
\begin{frame}
\end{frame}
}
%
\lecture{instructor}{instructor}

\begin{frame}[t]{What is the new population size?}
	\begin{columns}
		\begin{column}{0.24\textwidth}
			
			\hangpara $N_0 = 2000.$
	
			\hangpara $K = 4000.$
	
			\hangpara $b = 0.1.$
	
			\hangpara $d = 0.05.$ \pause
	
		\end{column}
		\begin{column}{0.74\textwidth}
			
			\hangpara $r = 0.1 - 0.05 = 0.05.$ \pause
		
			\hangpara $rN = 0.05 \times 2000 = 100.$ \pause
		
			\hangpara $\frac{K-N}{K} = \frac{4000-2000}{4000} = 0.5.$ \pause
		
			\hangpara $\frac{\Delta N}{\Delta T} = rN\left(\frac{K-N}{K}\right) = 100 \times 0.5 = 50.$ \pause
		
			\hangpara $N_1 = 2000 + 50 = 2050.$
		\end{column}
	\end{columns}
\end{frame}
%
\begin{frame}[t]{What is the new population size?}
	\begin{columns}
		\begin{column}{0.24\textwidth}
			
			\hangpara $N_0 = 4000.$
			
			\hangpara $ K = 3000.$
			
			\hangpara $b = 0.1.$
			
			\hangpara $d = 0.4.$ \pause
			
		\end{column}
		\begin{column}{0.74\textwidth}
			
			\hangpara $r = 0.1 - 0.4 = -0.3.$ \pause
			
			\hangpara $rN = -0.3 \times 4000 = -1200.$ \pause
			
			\hangpara $\frac{K-N}{K} = \frac{3000-4000}{3000} = -0.33\overline{3}.$ \pause
			
			\hangpara $\frac{\Delta N}{\Delta T} = rN\left(\frac{K-N}{K}\right) = -1200 \times -0.33\overline{3} = 400.$ \pause
			
			\hangpara $N_1 = 4000 + 400 = 4400.$ 

			\hangpara \highlight{WTF?} How did the population grow larger?
		\end{column}
	\end{columns}
\end{frame}
\end{document}
