%!TEX TS-program = lualatex
%!TEX encoding = UTF-8 Unicode

\documentclass[t]{beamer}

%%%% HANDOUTS For online Uncomment the following four lines for handout
%\documentclass[t,handout]{beamer}  %Use this for handouts.
%\usepackage{handoutWithNotes}
%\pgfpagesuselayout{3 on 1 with notes}[letterpaper,border shrink=5mm]
%	\setbeamercolor{background canvas}{bg=black!5}

%\includeonlylecture{student}

%%% Including only some slides for students.
%%% Uncomment the following line. For the slides,
%%% use the labels shown below the command.

%% For students, use \lecture{student}{student}
%% For mine, use \lecture{instructor}{instructor}

% FONTS
\usepackage{fontspec}
\def\mainfont{Linux Biolinum O}
\setmainfont[Ligatures={Common,TeX}, Contextuals={NoAlternate}, BoldFont={* Bold}, ItalicFont={* Italic}, Numbers={OldStyle}]{\mainfont}
\setsansfont[Ligatures={Common,TeX}, Scale=MatchLowercase, Numbers=OldStyle, BoldFont={* Bold}, ItalicFont={* Italic},]{Linux Biolinum O} 
\usepackage{microtype}

\usepackage{graphicx}
	\graphicspath{{/Users/goby/pictures/teach/163/lecture/}
	{/Users/goby/pictures/teach/common/}} % set of paths to search for images

%\usepackage{multicol}
\usepackage{booktabs}
%\usepackage{textcomp}

\usepackage{tikz}
	\tikzstyle{every picture}+=[remember picture,overlay]
%	\usetikzlibrary{arrows}

\mode<presentation>
{
  \usetheme{Lecture}
  \setbeamercovered{invisible}
  \setbeamertemplate{items}[square]
}

%%% Creative Commons Licenses. Establish, then add to Beamer template.
%\newcommand{\ccbysa}[1][4]{\textsc{cc by-sa #1.0}} % Use version 4.0 as default.
\newcommand{\ccby}[1]{%
	\ifx&#1&
	{\textsc{cc by}}%
\else
	{\textsc{cc by #1.0}}
\fi}


\newcommand{\ccbysa}[1]{%
	\ifx&#1&
	{\textsc{cc by-sa}}%
\else
	{\textsc{cc by-sa #1.0}} 
\fi}

\newcommand{\ccbyncsa}[1]{%
	\ifx&#1&
	{\textsc{cc by-nc-sa}}%
\else
	{\textsc{cc by-nc-sa #1.0}}
\fi}

\newcommand{\ccbync}[1]{%
	\ifx&#1&
	{\textsc{cc by-nc}}%
\else
	{\textsc{cc by-nc #1.0}}
\fi}


\begin{document}

\lecture{instructor}{instructor}
\begin{frame}{High score = 98. Mean score = 74.1.}

\includegraphics[width=\textwidth, page=1]{163exam1grades.pdf}

\end{frame}

\lecture{student}{student}
\begin{frame}{Our goal for this lecture is to learn}
	
	\hangpara how \highlight{reproductive isolation} evolves,
	
	\hangpara learn how \highlight{allopatric} distributions contribute to speciation,
	
	\hangpara learn what happens in \highlight{hybrid zones}, 
	
	\hangpara learn what are \highlight{ring species,} and
	
	\hangpara ask whether speciation can occur with gene flow.
	
	\hangpara 

\end{frame}
%


\begin{frame}{\highlight{Speciation} occurs when two new species evolve from single ancestor.}

	\centering
	\begin{tikzpicture}
		[remember picture, overlay,
		myLine/.style={color=black,very thick},
		myArrow/.style={color=orange6, thick, ->}]

		\draw [myLine] (-3,-2) -- (0,-2);
		\draw [myLine] (0,-2) -- (0,-0.7); % vertical lines
		\draw [myLine] (0,-2) -- (0,-3.3);
		\draw [myLine] (0,-0.7) -- (3,-0.7); % descendants
		\draw [myLine] (0,-3.3) -- (3,-3.3); 

		% Speciation
		\node [color=orange6] (speciation) at (-1.5,-3.1) {Speciation};
		\node [circle, draw=orange6, thick, minimum size=3mm] (speccirc) at (0,-2) {};
		\draw [color=orange6, thick] (speciation) edge (speccirc);
		
		% Ancestor 
		\node [color=blue6] at (-1.5,-1.8) {Ancestor};
		
		% Descendants
		\node [color=blue6] at (1.5,-0.5) {New Species};
		\node [color=blue6] at (1.5,-3.1) {New Species};

	\end{tikzpicture}
\end{frame}

\begin{frame}{Isolating mechanisms are \highlight{prezygotic} or \highlight{postzygotic}.}
	\vspace{2\baselineskip}
	\centering
	\begin{tabular}{l l l}
	\toprule
	{\large \highlight{Prezygotic}}	& or &	{\large Postzygotic}\\
	\midrule
	Habitat	& & Sterility \\
	Behavioral & &	Inviability \\
	Temporal	& &	Breakdown \\
	Mechanical 	& & 	\\
	Gamete Incompatibility	& & \\
	\bottomrule
	\end{tabular}

	\hangpara Study pages 502--503 for details of mechanisms we do not cover in lecture.
\end{frame}

{
\usebackgroundtemplate{\includegraphics[width=\paperwidth]{prezygotic_habitat} }
\begin{frame}[b]{\textcolor{white}{Columbine flowers live in different \textcolor{orange5}{habitats.}} }
\tiny \textcolor{white}{Dcrjsr, Wikimedia, \ccbysa{3}. \hfill Steve Berardi, Flickr, \ccbyncsa{2}. \href{https://www.youtube.com/watch?v=Iwfs2TDYg-8}{Video}}
\end{frame}
}

{
\usebackgroundtemplate{\includegraphics[width=\paperwidth]{prezygotic_behavior} }
\begin{frame}[b]{}
\hfill \tiny  \textcolor{white}{Vince Smith, Wikimedia, \ccbysa{2}. \href{https://www.youtube.com/watch?v=z922by9_6Fw}{Video} }
\end{frame}
}

{
\usebackgroundtemplate{\includegraphics[width=\paperwidth]{prezygotic_gamete} }
\begin{frame}[b]{\textcolor{white}{Many marine species have} \textcolor{orange5}{incompatible gametes.}}
\tiny \textcolor{white}{Emma Hickerson, \textsc{noaa}, public domain. \href{https://www.youtube.com/watch?v=wsaZ8-I7akg}{Video}}
\end{frame}
}


{
\usebackgroundtemplate{\includegraphics[width=\paperwidth]{allo_sym_speciation} }
\begin{frame}[b]{\highlight{Allopatric speciation} occurs from geographic isolation.}
\hfill \tiny \textcopyright Pearson Education, Inc.
\end{frame}
}

{
\usebackgroundtemplate{\includegraphics[width=\paperwidth]{allopatric_chipmunks} }
\begin{frame}[b]
\hfill \tiny \textcopyright Pearson Education, Inc.
\end{frame}
}

{
\usebackgroundtemplate{\includegraphics[width=\paperwidth]{allopatric_snapping_shrimp} }
\begin{frame}[b]{Allopatric speciation can be detected with phylogenetic trees.}
\tiny  Julio Duarte, Flickr, \ccbyncsa{2} \hfill Knowlton et al. 1993. Science 260: 1629.
\end{frame}
}

{
\usebackgroundtemplate{\includegraphics[width=\paperwidth]{allopatric_shrimp2} }
\begin{frame}[b]
\tiny  \textcopyright Pearson Education, Inc.
\end{frame}
}

\begin{frame}{Isolating mechanisms are \highlight{prezygotic} or \highlight{postzygotic}.}
	\vspace{2\baselineskip}
	\centering
	\begin{tabular}{l l l}
	\toprule
	{\large Prezygotic}	& or &	{\large \highlight{Postzygotic}}\\
	\midrule
	Habitat	& & Sterility \\
	Behavioral & &	Inviability \\
	Temporal	& &	Breakdown \\
	Mechanical 	& & 	\\
	Gamete Incompatibility	& & \\
	\bottomrule
	\end{tabular}

	\hangpara Study pages 502--503 for details of mechanisms we do not cover in lecture.
\end{frame}

{
\usebackgroundtemplate{\includegraphics[width=\paperwidth]{postzygotic_sterility} }
\begin{frame}[b]{\highlight{Hybrid sterility} means that hybrid offspring cannot reproduce.}
\hfill \tiny\textcolor{white}{\textcopyright Pearson Education, Inc.}
\end{frame}
}


{
\usebackgroundtemplate{\includegraphics[width=\paperwidth]{postzygotic_inviability} }
\begin{frame}[b]{}
\hfill \tiny\textcolor{white}{\textcopyright Sinauer Associates, Inc.}
\end{frame}
}

{
\usebackgroundtemplate{\includegraphics[width=\paperwidth]{hybrid_zone_toads} }
\begin{frame}[b]{\textit{Bombina} hybrids have reduced fitness relative to parent species.}
\tiny \textcopyright Pearson Education, Inc.
\end{frame}
}

\begin{frame}[t]{\highlight{Hybrid zones} can form if reproductive isolation has not evolved completely.}

	\alt<handout>{}{\includegraphics<1>[width=\textwidth]{hybrid_zone1}
	\includegraphics<2>[width=\textwidth]{hybrid_zone2}}
	\includegraphics<3>[width=\textwidth]{hybrid_zone3}

	\vfilll 
	
\onslide<1->\hfill \tiny \textcopyright Pearson Education, Inc.	
\end{frame}

\begin{frame}[t]{Hybrid zones can have three possible outcomes. \phantom{if isolation has not evolved completely.}}

	\includegraphics[width=\textwidth]{hybrid_zone4}

	\vfilll 

\hfill \tiny \textcopyright Pearson Education, Inc.	
\end{frame}

{
\usebackgroundtemplate{\includegraphics[width=\paperwidth]{hybrid_zone_toads} }
\begin{frame}[b]{\textit{Bombina} toads have a stable hybrid zone.}
\tiny \textcopyright Pearson Education, Inc.
\end{frame}
}

\lecture{instructor}{instructor}
{
\usebackgroundtemplate{\includegraphics[width=\paperwidth]{hybrid_reinforcement1} }
\begin{frame}[b]{Two flycatcher species may hybridize with each other.}
\hfill \tiny \textcopyright Pearson Education, Inc.
\end{frame}
}
%
\lecture{student}{student}
{
\usebackgroundtemplate{\includegraphics[width=\paperwidth]{hybrid_reinforcement2} }
\begin{frame}[b]{\highlight{Reinforcement} reduces chance of forming hybrids.}
\hfill \tiny \textcopyright Pearson Education, Inc.
\end{frame}
}

\begin{frame}[t]
	\alt<handout>{}{\frametitle<1>{How many species do you see?}
	\frametitle<2>{Do you still see the same number?}}
	\frametitle<3>{Females of these species use visual cues to identify males.}

	\centering
	\alt<handout>{}{\includegraphics<1>[width=\textwidth]{cichlid_species1}
	\includegraphics<2>[width=\textwidth]{cichlid_species2}}
	\includegraphics<3>[width=\textwidth]{cichlid_species3}
		
	\vfilll
	
	\hfill \tiny \textcopyright Pearson Education, Inc.
\end{frame}


\begin{frame}[t]{\highlight{Fusion} reverses the speciation process.}

	\centering
	\includegraphics[height=0.85\textheight]{hybrid_fusion}

	\vfilll
	
	\hfill \tiny \textcopyright Pearson Education, Inc.
\end{frame}

{
\usebackgroundtemplate{\includegraphics[width=\paperwidth]{ensatina_ring_species} }
\begin{frame}[b]{\highlight{Ring species} can interbreed along a ring but not where the ends of the ring meet.}
	
\tiny Devitt et al. 2011. BMC Evolutionary Biology 11: 245.
\end{frame}
}

\end{document}
