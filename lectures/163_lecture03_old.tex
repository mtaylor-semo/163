%!TEX TS-program = lualatex
%!TEX encoding = UTF-8 Unicode

\documentclass[t]{beamer}

%%%% HANDOUTS For online Uncomment the following four lines for handout
%\documentclass[t,handout]{beamer}  %Use this for handouts.
%\usepackage{handoutWithNotes}
%\pgfpagesuselayout{3 on 1 with notes}[letterpaper,border shrink=5mm]

%\includeonlylecture{student}

%%% Including only some slides for students.
%%% Uncomment the following line. For the slides,
%%% use the labels shown below the command.

%% For students, use \lecture{student}{student}
%% For mine, use \lecture{instructor}{instructor}


%\usepackage{pgf,pgfpages}
%\pgfpagesuselayout{4 on 1}[letterpaper,border shrink=5mm]

% FONTS
\usepackage{fontspec}
\def\mainfont{Linux Biolinum O}
\setmainfont[Ligatures={Common,TeX}, Contextuals={NoAlternate}, BoldFont={* Bold}, ItalicFont={* Italic}, Numbers={OldStyle}]{\mainfont}
\setsansfont[Ligatures={Common,TeX}, Scale=MatchLowercase, Numbers=OldStyle]{Linux Biolinum O} 
\usepackage{microtype}

\usepackage{graphicx}
	\graphicspath{{/Users/goby/pictures/teach/163/lecture/}
	{/Users/goby/pictures/teach/common/}} % set of paths to search for images

%\usepackage{units}
\usepackage{booktabs}
%\usepackage{textcomp}
\usepackage{multicol}

\usepackage{tikz}
	\tikzstyle{every picture}+=[remember picture,overlay]
	\usetikzlibrary{trees}

\mode<presentation>
{
  \usetheme{Lecture}
  \setbeamercovered{invisible}
%  \setbeamertemplate{items}[default]
}


\begin{document}


% Lecture goals
\begin{frame}{Our goal for this lecture is to}
	
	\hangpara practice solving the Hardy-Weinberg equations.
		
\end{frame}
%

%% Consider a population: Phenotype
%\begin{frame}{Calculate allele frequencies for 500 individuals.}
%	\vspace{-1\baselineskip}
%	\begin{center}
%		\includegraphics[width=0.7\textwidth]{genotype_carnations}
%	\end{center}
%	
%	\vspace{-\baselineskip}
%	\hangpara How many total alleles are in the population?
%	\pause
%		
%	\hangpara How many $C^R$ alleles are in the population?
%	\pause
%
%	\hangpara How many $C^W$ alleles are in the population?
%	\pause
%
%	\hangpara What is the frequency (proportion) of each allele in the population?
%	
%\end{frame}
%
%% Consider a population: Genotype
%\begin{frame}{Consider a population of 500 diploid individuals.}
%
%	\vspace{-\baselineskip}
%	\begin{center}
%		\includegraphics[width=0.7\textwidth]{genotype_carnations.jpg}
%	\end{center}
%
%	\vspace{-\baselineskip}
%	\hangpara Given two alleles, how many genotypes are \textit{possible}?
%	\pause
%		
%	\hangpara What is the frequency of the $C^RC^R$ genotype?
%	\pause
%
%	\hangpara How the frequency of the $C^RC^W$ genotype?
%	\pause
%
%	\hangpara What is the frequency of the $C^WC^W$ genotype?
%	
%\end{frame}
%
%
%
%\begin{frame}[t]{Two equations show that populations cannot evolve if a population meets five assumptions.}
%
%	\vspace*{-\baselineskip}
%	
%	\begin{multicols}{2}
%
%	\hangpara \highlight{No} mutation,
%	
%	\hangpara \highlight{No} gene flow,
%
%	\hangpara \highlight{No} genetic drift,
%	
%	\hangpara \highlight{No} non-random mating, and
%	
%	\hangpara \highlight{No} natural selection.
%	
%	\columnbreak
%
%	\hangpara $p + q = 1$
%	
%	\hangpara $p^2 + 2pq + q^2 = 1$
%	
%	\end{multicols}
%
%\end{frame}
%%
%% Hardy-Weinberg Equilibrium
%\begin{frame}[t]{The Hardy-Weinberg equations act as a null hypothesis.}
%
%	\hangpara When the five assumptions are met, the equations show a population will not evolve.
%	
%	\hangpara If at least one assumption is violated, then a population will evolve.
%	
%	\hangpara We can learn how populations evolve by comparing real populations to the null hypothesis.
%
%
%\end{frame}
%%

\end{document}