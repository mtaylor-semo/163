%!TEX TS-program = lualatex
%!TEX encoding = UTF-8 Unicode

\documentclass[t]{beamer}

%%%% HANDOUTS For online Uncomment the following four lines for handout
%\documentclass[t,handout]{beamer}  %Use this for handouts.
%\includeonlylecture{student}
%\usepackage{handoutWithNotes}
%\pgfpagesuselayout{3 on 1 with notes}[letterpaper,border shrink=5mm]


%%% Including only some slides for students.
%%% Uncomment the following line. For the slides,
%%% use the labels shown below the command.

%% For students, use \lecture{student}{student}
%% For mine, use \lecture{instructor}{instructor}

% Fonts
\usepackage{fontspec}
\def\mainfont{Linux Biolinum O}
\setmainfont[Ligatures={Common,TeX}, Contextuals={NoAlternate}, Numbers={Proportional, OldStyle}]{\mainfont}
\setsansfont[Ligatures={Common,TeX}, Scale=MatchLowercase, Numbers={Proportional,OldStyle}, BoldFont={* Bold}, ItalicFont={* Italic},]\mainfont

\newfontface\lining[Numbers={Lining}]\mainfont

%\usepackage[draft]{microtype}


\mode<presentation>
{
  \usetheme{Lecture}
  \setbeamercovered{invisible}
  \setbeamertemplate{items}[default]
}



\usepackage{amsmath,amssymb}
\usepackage{unicode-math}
%\setmathfont[Scale=MatchLowercase]{TeX Gyre Pagella Math}

\usepackage{graphicx}
	\graphicspath{{/Users/goby/pictures/teach/163/lecture/}
	{/Users/goby/pictures/teach/common/}} % set of paths to search for images

\usepackage{xcolor}

\usepackage{multicol}
\usepackage{booktabs}
\usepackage{array}
\newcolumntype{L}[1]{>{\raggedright\let\newline\\\arraybackslash\hspace{0pt}}p{#1}}
\newcolumntype{C}[1]{>{\centering\let\newline\\\arraybackslash\hspace{0pt}}p{#1}}
\newcolumntype{R}[1]{>{\raggedleft\let\newline\\\arraybackslash\hspace{0pt}}p{#1}}
%\usepackage{textcomp}
\usepackage{mhchem}
\mhchemoptions{textfontcommand=\lining}
\usepackage{enumitem}
%\usepackage[export]{adjustbox}

\usepackage{calc} %For widthof, in one slide below.
\makeatletter
\def\@hspace#1{\begingroup\setlength\dimen@{#1}\hskip\dimen@\endgroup}
\makeatother


\usepackage{tikz}
%		%tikzstyle{every picture} conflicts for some reason with the pgfplots stuff above.
	\tikzstyle{every picture}+=[remember picture,overlay]
	\usetikzlibrary{arrows, decorations.pathreplacing}
%\usetikzlibrary{positioning}

% Use the to temporarily set a background grid for positioning.
%\setbeamertemplate{background}[grid][step=1em]

\begin{document}

\lecture{student}{student}

\begin{frame}{Our goals for this lecture are to learn}

	\hangpara how climate determines the distribution of ecosystems,
	
	\hangpara the current human population and \highlight{demographics,}
	
	\hangpara about the \highlight{greenhouse effect} and
	
	\hangpara how it affects \highlight{climate change.} 
	
\end{frame}
%
{
	\usebackgroundtemplate{\includegraphics[width=\paperwidth]{ecosystem_distribution} }
	\begin{frame}[b]{The distribution of terrestrial ecosystems is determined by \highlight{climate}.}
		
	\end{frame}
}
%


{
	\usebackgroundtemplate{\includegraphics[width=\paperwidth]{climate_solar_input} }
	\begin{frame}[b]{Climate is driven by variation of \highlight{solar radiation} from low to high latitudes.}
		
		\hfill \tiny Fig.~52.3\textsc{a} \copyright Pearson Education, Inc.
	\end{frame}
}
%
{
	\begin{frame}[t]{Solar input creates wind and precipitation patterns.}
		
		\includegraphics[width=\textwidth]{climate_circulation_precipitation}
		
		\vfilll
		
		\hfill \tiny Fig.~52.3\textsc{b} \copyright Pearson Education, Inc.
	\end{frame}
}
%


{
	\usebackgroundtemplate{\includegraphics[width=\paperwidth]{climograph} }
	\begin{frame}[b]{Ecosystem distribution is determined by mean annual temperature and precipitation.}
		
	\end{frame}
}
%
\begin{frame}[t]{Mountains create \highlight{rain shadows} that affect regional climate.}
	
	\bigskip
	
	\includegraphics[width=\textwidth]{rain_shadow}
	
	\vfilll
	
	\hfill \tiny Fig.~52.6 \copyright\,Pearson Education, Inc.
\end{frame}
%
{
	\usebackgroundtemplate{\includegraphics[width=\paperwidth]{precipitation_national} }
	\begin{frame}[b]
		
	\end{frame}
}
% May move elsewhere, such as population ecology, or convert to exercise. Did not
% use in Spring 2017.

% BEGIN HUMAN POP.
{
\usebackgroundtemplate{\includegraphics[width=\paperwidth]{human_population_growth} }
\begin{frame}[b]{The global population is \textgreater7,910,000,000 people.}

	\hfill \tiny Fig.~53.22 \copyright\,Pearson Education, Inc.
\end{frame}
}
%
\begin{frame}[t]{The global birth rate for 2020 was 18.1 per 1000 people.}

	\includegraphics[width=\textwidth]{global_birth_rate}
	
	\hangpara More than 9,000 people will be born during class time today.
	
	\vfilll
	
	\hfill \tiny Ali Zifan, Wikimedia Commons, \textsc{cc0}
\end{frame}
%
\begin{frame}[t]{The global death rate for 2020 was 7.6 per 1000 people.}

	\includegraphics[width=\textwidth]{global_death_rate}\par
	
	$r \approx 0.0181 - 0.0076 \approx 0.01$\vspace*{\baselineskip}
	
	People added = $\frac{\Delta N}{\Delta t} =rN \approx 0.01 \times 7,910,000,000 \approx 79,100,000$
		
	\vfilll
	
	\hfill \tiny United Nations, Wikimedia Commons, \ccbysa{3}
\end{frame}

{
\usebackgroundtemplate{\includegraphics[width=\paperwidth]{human_growth_rate} }
\begin{frame}[b]{The human population growth \emph{rate} is slowing.}

	\hfill \tiny Fig.~53.23 \copyright\,Pearson Education, Inc.
\end{frame}
}
%
\begin{frame}[t]{\highlight{Demography} describes the age structure, and birth and death rates of a population.}

	\includegraphics[width=\textwidth]{human_demography}
	
	\vfilll
	
	\hfill \tiny Fig.~53.24 \copyright\,Pearson Education, Inc.
	
\end{frame}
%
{
\usebackgroundtemplate{\includegraphics[width=\paperwidth]{us_demographics} }
%\setbeamertemplate{background}[grid][step=1em]
\begin{frame}[b]{Demographics of the U.S. population.}

	\begin{tikzpicture}
		\draw  [decorate, decoration={brace, amplitude=5pt}] (9em,1em) -- (9em, 3.6em) node [left,midway, xshift=-9pt] {\small Z};

		\draw  [decorate, decoration={brace, amplitude=5pt}] (9em,3.7em) -- (9em, 6.9em) node [left,midway, xshift=-9pt] {\small \highlight{Millenial}};
	
		\draw  [decorate, decoration={brace, amplitude=5pt}] (9em,7em) -- (9em, 10.2em) node [left,midway, xshift=-9pt] {\small X};

		\draw  [decorate, decoration={brace, amplitude=5pt}] (9em,10.3em) -- (9em, 14em) node [left,midway, xshift=-9pt] {\small \highlight{Baby Boomers}};
	
		\draw  [decorate, decoration={brace, amplitude=5pt}] (9em,14.1em) -- (9em, 17.1em) node [left,midway, xshift=-9pt] {\small Silent};
	
		\draw  [decorate, decoration={brace, amplitude=5pt}] (9em,17.2em) -- (9em, 19em) node [left,midway, xshift=-9pt] {\small G.I.};
	
	\end{tikzpicture}
	
	\tiny Delphia234, Wikimedia Commons, \textsc{cc0}

\end{frame}
}

%
% Did not use in Fall 2016 nor spring 2017.
%{
%\usebackgroundtemplate{\includegraphics[width=\paperwidth]{us_birth_rate} }
%\begin{frame}[b]{U.S. birth rate, 1934--present.}
%
%	\hfill \tiny Before My Ken, Wikimedia, public domain
%\end{frame}
%}
%


{
\usebackgroundtemplate{\includegraphics[width=\paperwidth]{human_energy_use} }
\begin{frame}[b]{The average person uses 5.9$\times$ more energy than is sustainable.}

	\hfill \tiny Fig.~53.26 \copyright\,Pearson Education, Inc.
\end{frame}
}

%
\begin{frame}[t]{Is global warming real? Do liars lie?}
	\begin{center}
		\includegraphics[width=\textwidth]{climate_denier_no_warming}
	\end{center}

	Let us begin with \ce{CO2}.
	
	\vfilll
	
	\hfill \tiny \copyright~Marc Morano, Climate Depot
\end{frame}

%


% Begin Climate Change.
%
\begin{frame}[t]{Consuming energy adds \ce{CO2} to the atmosphere.}

	\includegraphics[width=\textwidth]{industrial_revolution}

	\vfilll
	
	\hfill \tiny D.\,W.\,F.~Hardie, Wikimedia, public domain

\end{frame}
%
{
\usebackgroundtemplate{\includegraphics[width=\paperwidth]{greenhouse_effect} }
\begin{frame}[b]

	\hfill \tiny \textcolor{white}{\textsc{us epa}, Wikimedia Commons, public domain}
\end{frame}
}
%
\begin{frame}{Pliocene \ce{CO2} was higher in the past than the present.}
	
	\includegraphics[width=\textwidth]{co2_pliocene}\par
	
	\hangpara But\dots
	
	\vfilll
	
	\hfill \tiny Gavin Foster, \href{http://www.thefosterlab.org/blog/2015/11/11/is-this-the-last-year-below-400-ppm}{thefosterlab.org}
\end{frame}
%
\begin{frame}{Current \ce{CO2} levels are rising \emph{very} rapidly.}
	
	{\centering
		\includegraphics[height=0.82\textheight]{co2_pleistocene_current}\par
	}
	
	\vfilll
	
	\hfill \tiny \textsc{noaa}, public domain
\end{frame}
%
\begin{frame}[t]{This figure uses “cherry-picked” data to support a fake narrative.}
	
	{\centering
		\includegraphics[width=\textwidth]{climate_denier_no_warming}\par
	}
	
	The figure tries to show no global temperature change. But\dots
	
	\vfilll
	
	\hfill \tiny \copyright~Marc Morano, Climate Depot
\end{frame}
%
{
\usebackgroundtemplate{\includegraphics[width=\paperwidth]{co2_temperature_increase} }
\begin{frame}[b]{The accumulation of \ce{CO2} causes global temperature to increase.}

	\hfill \tiny Fig.~52.8 \copyright Pearson Education, Inc.
\end{frame}
}
%

\begin{frame}[t]{Temperature is actually increasing at a \emph{rate} faster than ever recorded.}
	
	\vspace*{-0.5\baselineskip}
	
	{\centering\includegraphics[height=0.75\textheight]{warming_rate}\par%
	}
	
	\vfilll
	
	\tiny Hagelaars, J. 2013. http://ourchangingclimate.wordpress.com/2013/03/19/the-two-epochs-of-marcott/\\
Based on Shakun et al. 2012. Nature 484: 49; Marcott et al. 2013. Science 339: 1198. 

\end{frame}
%
{
	\usebackgroundtemplate{\includegraphics[width=\paperwidth]{ecosystem_distribution} }
	\begin{frame}[b]{Climate change will affect the distribution of ecosystems.}
		
		\hfill \tiny Fig.~52.8 \copyright Pearson Education, Inc.
	\end{frame}
}
%
\begin{frame}{Ecosystems changed during the last ice age.}

	\includegraphics[width=\textwidth]{pleistocene_ecosystem_proportions}

	\vfilll
	
	\hfill \tiny Fig.~9.17, \emph{Biogeography}, \copyright\,Sinauer Associates, Inc.
\end{frame}
%
\begin{frame}[t]{Climate change will cause the distribution of modern species to change.}

	\includegraphics[width=\textwidth]{climate_change_range_shift}\par
	
	\hangpara Many species will not be able to adapt to rapid change.
	
	\vfilll
	
	\hfill \tiny Fig.~52.7 \copyright Pearson Education, Inc.

\end{frame}
%
{
\usebackgroundtemplate{\includegraphics[width=\paperwidth]{climate_ecosystem_change} }
\begin{frame}[b]{Global climate change will change the types and locations of modern ecosystems.}

	\hfill \tiny Fig.~53.26 \copyright\,Pearson Education, Inc.
\end{frame}
}
%

\begin{frame}{If we have time\dots}

\href{http://cci-reanalyzer.org/wx/DailySummary/}{Today's global weather}\bigskip

\href{http://nsidc.org/arcticseaicenews/charctic-interactive-sea-ice-graph/}{Arctic sea ice extent}
\end{frame}
%%
\end{document}
