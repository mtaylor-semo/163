%!TEX TS-program = lualatex
%!TEX encoding = UTF-8 Unicode

\documentclass[t]{beamer}

%%%% HANDOUTS For online Uncomment the following four lines for handout
%\documentclass[t,handout]{beamer}  %Use this for handouts.
%\usepackage{handoutWithNotes}
%\pgfpagesuselayout{3 on 1 with notes}[letterpaper,border shrink=5mm]

%\includeonlylecture{student}

% FONTS
\usepackage{fontspec}
\def\mainfont{Linux Biolinum O}
\setmainfont[Ligatures={Common,TeX}, Contextuals={NoAlternate}, BoldFont={* Bold}, ItalicFont={* Italic}, Numbers={OldStyle}]{\mainfont}
\setsansfont[Ligatures={Common,TeX}, Scale=MatchLowercase, Numbers=OldStyle, BoldFont={* Bold}, ItalicFont={* Italic},]{Linux Biolinum O} 
\usepackage{microtype}

\usepackage{graphicx}
	\graphicspath{{/Users/goby/pictures/teach/163/lecture/}
	{/Users/goby/pictures/teach/common/}} % set of paths to search for images

\usepackage{multicol}
%\usepackage{booktabs}
%\usepackage{textcomp}

\usepackage{tikz}
	\tikzstyle{every picture}+=[remember picture,overlay]
%	\usetikzlibrary{arrows}

\usepackage{amsmath}
%\usepackage{units}
%\usepackage{booktabs}

%\usepackage{tikz}

%\usepackage{ifthen}

\mode<presentation>
{
  \usetheme{Lecture}
  \setbeamercovered{invisible}
%  \setbeamertemplate{items}[square]
}

\usefonttheme[onlymath]{serif}

\begin{document}

% Lecture goals
\lecture{student}{student}
\begin{frame}{Our goal for this lecture is to}
	
	\hangpara practice solving \highlight{Hardy-Weinberg} problems.

	\hangpara Solve each problem yourself. Then compare. But first\dots
	
\end{frame}
%
\lecture{instructor}{instructor}

\begin{frame}{Without looking at your notes, what are the two equations?}

	\pause \vspace{-2\baselineskip}

	{\centering
	\large
	\begin{align*}
		p + q = 1 \\
		p^2 + 2pq + q^2 = 1
	\end{align*}	
	}
	
	\pause 
	\hangpara In the following problems, check your answers against each other using the following:
	
	\vspace{-\baselineskip}
	{\centering
	\large
	\begin{align*}
		p = \sqrt{p^2} \\
		q = \sqrt{q^2}
	\end{align*}	
	}
	
\end{frame}
%
\lecture{student}{student}

\begin{frame}{Example Problem. }
	\hangpara In a group of 100 students, 36 were homozygous for a trait (\emph{TT}), 48 were heterozygous (\emph{Tt}), and 16 were homozygous for the other trait (\emph{tt}).

	\hangpara 1. Calculate the total number of alleles and the two allele frequencies.

	\hangpara 2. Calculate the three genotype frequencies.

\end{frame}
%
{
\setbeamercovered{%
	still covered={\opaqueness<1->{0}},
	again covered={\opaqueness<1->{25}}
	}
\begin{frame}{Example Problem. Solution.}
\begin{multicols}{2}

	\onslide<1-5,9,10>\hangpara \textbf{Allele Frequencies}
	
	\onslide<1-3>\hangpara Total alleles $100 \times 2 = 200$
	
	\onslide<2>\hangpara \emph{T} alleles: $(36 \times 2) + 48 = 120$
	
	\onslide<2,4,9>\hangpara $p = \dfrac{120}{200} = \highlight{\mathbf{0.6}}$

	\onslide<3>\hangpara \emph{t} alleles: $(16 \times 2) + 48 = 80$ 

	\onslide<3,4,10>\hangpara $q = \dfrac{80}{200} = \highlight{\mathbf{0.4}}$
	
	\onslide<4>\hangpara $0.6 + 0.4 = 1$ \checkmark

\columnbreak

	\onslide<5-10>\hangpara \textbf{Genotype Frequencies}
	
	\onslide<5,8,9>\hangpara $p^2=  \dfrac{36}{100} =\highlight{\mathbf{0.36}}$  

	\onslide<6,8>\hangpara $2pq = \dfrac{48}{100} = \highlight{\mathbf{0.48}}$

	\onslide<7-8,10>\hangpara $q^2 = \dfrac{16}{100} = \highlight{\mathbf{0.16}}$

	\onslide<8>\hangpara $0.36 + 0.48 + 0.16 = 1$ \checkmark
	
	\onslide<9-10>\rule{0.4\textwidth}{0.1pt}\vspace{-0.5\baselineskip}
	
	\onslide<9>\hangpara \highlight{$p = \sqrt{0.36} = 0.6$} and
	
	\onslide<10>\highlight{$q = \sqrt{0.16} = 0.4$}

\end{multicols}

\end{frame}
}
%
\begin{frame}{Is this population in Hardy--Weinberg equilibrium?}
	\hangpara You study a group of 274 flowering plants.  136 plants have red flowers, 114 plants have pink flowers and the remaining plants have white flowers. Plants with red flowers are homozygous for the \emph{R} allele, plants with white flowers are homozygous for the \emph{r} allele, and plants with pink flowers are heterozygous.

	\hangpara 1. Calculate the total number of alleles and the two allele frequencies.

	\hangpara 2. Calculate the three genotype frequencies.

	\hangpara Round to 3 digits after each calculation.
\end{frame}
%
\lecture{instructor}{instructor}
{
\setbeamercovered{%
	still covered={\opaqueness<1->{0}},
	again covered={\opaqueness<1->{25}}
	}
\begin{frame}{Is this population in Hardy--Weinberg? Solution.}
\begin{multicols}{2}

	\onslide<1-4>\hangpara \textbf{Allele Frequencies}
	
	\onslide<1-3>\hangpara Total alleles: $274 \times 2 = 548$
	
	\onslide<2>\hangpara \emph{R} alleles: $(136 \times 2) + 114 = 386$
	
	\onslide<2,4,9>\hangpara $p = \dfrac{386}{548} = \highlight{\mathbf{0.704}}$

	\onslide<3>\hangpara \emph{r} alleles: $(24 \times 2) + 114 = 162$ 

	\onslide<3,4,10>\hangpara $q = \dfrac{162}{548} = \highlight{\mathbf{0.296}}$
	
	\onslide<4>\hangpara $0.704 + 0.296 = 1$ \checkmark

\columnbreak

	\onslide<5-8>\hangpara \textbf{Genotype Frequencies}
	
	\onslide<5,8,9>\hangpara $p^2 = \dfrac{136}{274} = \highlight{\mathbf{0.496}}$ 

	\onslide<6,8>\hangpara $2pq = \dfrac{114}{274} = \highlight{\mathbf{0.416}}$

	\onslide<7-8,10>\hangpara $q^2 = \dfrac{24}{274} = \highlight{\mathbf{0.088}}$

	\onslide<8>\hangpara $0.496 + 0.416 + 0.088 = 1$ \checkmark
	
	\onslide<9-10>\rule{0.4\textwidth}{0.1pt}\vspace{-0.5\baselineskip}
	
	\onslide<9>\hangpara \highlight{$p = \sqrt{0.496} = 0.704$} and
	
	\onslide<10>\highlight{$q = \sqrt{0.088} = 0.296$}

\end{multicols}
\end{frame}
}
%
\lecture{student}{student}

\begin{frame}{Is this population in Hardy--Weinberg equilibrium?}
	\vspace{-1\baselineskip}
	\hangpara Most flowers have a female reproductive structure called the pistil. Pistil length is controlled by a single gene with two alleles. In a group of 500 plants, you found that 320 had short pistils (homozygotes), 175 had medium pistils (heterozygotes), and 5 flowers had long pistils (homozygotes). 

	\hangpara 1. Calculate the two allele frequencies.

	\hangpara 2. Calculate the three genotype frequencies.

	\hangpara Round to 3 digits after each calculation.
\end{frame}
%
\lecture{instructor}{instructor}
{
\setbeamercovered{%
	still covered={\opaqueness<1->{0}},
	again covered={\opaqueness<1->{25}}
	}
\begin{frame}{Is this population in Hardy--Weinberg? Solution.}
\vspace{-1\baselineskip}
\begin{multicols}{2}

	\onslide<1-4>\hangpara \textbf{Allele Frequencies}
	
	\onslide<1-3>\hangpara Total alleles: $500 \times 2 = 1000$
	
	\onslide<2>\hangpara \emph{S} alleles: $(320 \times 2) + 175 = 815$
	
	\onslide<2,4,11>\hangpara $p = \dfrac{815}{1000} = \highlight{\mathbf{0.815}}$

	\onslide<3>\hangpara \emph{s} alleles: $(5 \times 2) + 175 = 185$ 

	\onslide<3,4,12>\hangpara $q = \dfrac{185}{1000} = \highlight{\mathbf{0.185}}$
	
	\onslide<4,9-10>\hangpara $0.815 + 0.185 = 1$ \checkmark

\columnbreak

	\onslide<5-8>\hangpara \textbf{Genotype Frequencies}
	
	\onslide<5,8,11>\hangpara $p^2 = \dfrac{320}{500} = \highlight{\mathbf{0.64}}$ 

	\onslide<6,8>\hangpara $2pq = \dfrac{175}{500} = \highlight{\mathbf{0.35}}$

	\onslide<7-8,12>\hangpara $q^2 = \dfrac{5}{500} = \highlight{\mathbf{0.01}}$

	\onslide<8-10>\hangpara $0.64 + 0.35 + 0.01 = 1$ \checkmark
	
	\onslide<9-12>\hangpara \highlight{\textbf{Not in HWE!}}
	\onslide<10>\textbf{\ WTF?}\vspace{-0.5\baselineskip}
	
	\onslide<11>\hangpara \highlight{$p = \sqrt{0.64} = 0.8 \ne 0.815$} and
	
	\onslide<12>\highlight{$q = \sqrt{0.01} = 0.1 \ne 0.185$}

\end{multicols}
\end{frame}
}

\lecture{instructor}{instructor}
\begin{frame}{Why is this last population not in \textsc{hwe}?}
%
	\hangpara We will explore reasons upcoming lectures and lab.
	
%	\hangpara The difference may be due to \highlight{random sampling}, or
%	\pause
%	
%	\hangpara due to \highlight{evolutionary processes.}
%	\pause
%	
%	\hangpara Perform statistical test to find out.
\end{frame}

\end{document}
