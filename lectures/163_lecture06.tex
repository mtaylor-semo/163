%!TEX TS-program = lualatex
%!TEX encoding = UTF-8 Unicode

\documentclass[t]{beamer}

%%%% HANDOUTS For online Uncomment the following four lines for handout
%\documentclass[t,handout]{beamer}  %Use this for handouts.
%\usepackage{handoutWithNotes}
%\pgfpagesuselayout{3 on 1 with notes}[letterpaper,border shrink=5mm]

%\includeonlylecture{student}

%%% Including only some slides for students.
%%% Uncomment the following line. For the slides,
%%% use the labels shown below the command.

%% For students, use \lecture{student}{student}
%% For mine, use \lecture{instructor}{instructor}

% FONTS
\usepackage{fontspec}
\def\mainfont{Linux Biolinum O}
\setmainfont[Ligatures={Common,TeX}, Contextuals={NoAlternate}, BoldFont={* Bold}, ItalicFont={* Italic}, Numbers={OldStyle}]{\mainfont}
\setsansfont[Ligatures={Common,TeX}, Scale=MatchLowercase, Numbers=OldStyle, BoldFont={* Bold}, ItalicFont={* Italic},]{Linux Biolinum O} 
\usepackage{microtype}

\usepackage{graphicx}
	\graphicspath{{/Users/goby/pictures/teach/163/lecture/}
	{/Users/goby/pictures/teach/common/}} % set of paths to search for images

%\usepackage{multicol}
%\usepackage{longtable}
%\usepackage{booktabs}
%\usepackage{textcomp}

%\usepackage{tikz}
%	\tikzstyle{every picture}+=[remember picture,overlay]
%	\usetikzlibrary{arrows}

\mode<presentation>
{
  \usetheme{Lecture}
  \setbeamercovered{invisible}
%  \setbeamertemplate{items}[square]
}



\begin{document}
%
\lecture{student}{student}

\begin{frame}{Our goals for this lecture are to }
	
	\hangpara explain \highlight{homology} and \highlight{analogy,}
	
	\hangpara relate homology and analogy to \highlight{convergent evolution,}

	\hangpara explain why natural selection is selfish, and
	
	% \hangpara why ``perfect species'' cannot evolve by natural selection, and
	
	\hangpara explore \highlight{sexual selection} as a type of natural selection.

\end{frame}
%
{
\usebackgroundtemplate{\includegraphics[width=\paperwidth]{descent_with_modification1} }
\begin{frame}[b]{What is \highlight{descent with modification?}}

	\hfill \tiny \copyright Pearson Education, Inc.

\end{frame}
}
%
{
\usebackgroundtemplate{\includegraphics[width=\paperwidth]{homologous_bird_wings} }
\begin{frame}[b]{A \highlight{homology} is a similar character shared among species due to common ancestry.}

	\vfilll

	\tiny Pixabay, \cc \hfill derdento, Pixabay \cc

\end{frame}
}
%
{
\usebackgroundtemplate{\includegraphics[width=\paperwidth]{homologous_appendages} }
\begin{frame}[t]{Homologies result from modification of existing traits through evolutionary processes.}

	\vfilll

	\hfill \tiny \copyright Pearson Education, Inc.

\end{frame}
}
%
{
\usebackgroundtemplate{\includegraphics[width=\paperwidth]{convergent_evolution_wings}}
\begin{frame}[b]{An \highlight{analogy} is a similar character due to similar function, \emph{not} common ancestry.}


\tiny Top: Conty, Wikimedia, \ccby{3} \hfill Bottom: Martin Hauser, Wikimedia \ccby{3}
\end{frame}
}
%
{
\usebackgroundtemplate{\includegraphics[width=\paperwidth]{convergent_evolution_streamlined}}
\begin{frame}[t]{Analogies are the result of \highlight{convergent evolution.}}

	\vspace{3.85cm}

	\tiny \textcolor{white}{Alexander Vasenin, Wikimedia, \ccbysa{3}
	\hfill Terry Goos, Wikimedia, \ccby{2}}

	\vfilll

	\tiny Nobu Tamura, Wikimedia, \ccby{2} \hfill \textcolor{white}{Ken Funakoshi, Wikimedia, \ccby{2}}

\end{frame}
}
%
{
\usebackgroundtemplate{\includegraphics[width=\paperwidth]{tetrapod_convergence1}}
\begin{frame}[b]{Marine tetrapods evolved streamlining independently.}

	\hfill \tiny Kelley and Pyenson 2015. Science 348: 301.

\end{frame}
}
%
{
\usebackgroundtemplate{\includegraphics[width=\paperwidth]{tetrapod_convergence2}}
\begin{frame}[b]{Marine tetrapods also evolved “paddles” independently.}

	\hfill \rotatebox{90}{\tiny Kelley and Pyenson 2015. Science 348: 301.}

\end{frame}
}
%
{
\usebackgroundtemplate{\includegraphics[width=\paperwidth]{homologous_flippers}}
\begin{frame}[b]

	\hfill \tiny Kelley and Pyenson 2015. Science 348: 301.

\end{frame}
}
%
{
\usebackgroundtemplate{\includegraphics[width=\paperwidth]{penguin_huddle} }
\begin{frame}[b]
\end{frame}
}
%
{
\usebackgroundtemplate{\includegraphics[width=\paperwidth]{penguin_huddle_biggest} }
\begin{frame}[b]
\end{frame}
}
%
{
\usebackgroundtemplate{\includegraphics[width=\paperwidth]{penguin_huddle_outside} }
\begin{frame}[b]

	\tiny\textcolor{white}{\href{http://www.youtube.com/watch?v=Cdiapvktzkw}{Link to Video}
	\hfill
	Based on Waters et al. 2012. Modeling huddling penguins. PLoS ONE 7: e50277.}

\end{frame}
}
%
{
\usebackgroundtemplate{\includegraphics[width=\paperwidth]{fitness_deer_harvest} }
\begin{frame}[b]

	\tiny William V., Flickr Creative Commons.\hfill Scott Bauer, USDA, Public Domain.

\end{frame}
}

{
\usebackgroundtemplate{\includegraphics[width=\paperwidth]{bird_paradise} }
\begin{frame}[b]{\hfill\highlight{Sexual selection}}

	\tiny \textcolor{white}{\copyright Tim Laman, All Rights Reserved. 
	\hfill \href{http://www.youtube.com/watch?v=KIYkpwyKEhY}{Link to Video} }
	
\end{frame}
}
%
{
\usebackgroundtemplate{\includegraphics[width=\paperwidth]{intrasexual_selection_giraffes} }
\begin{frame}[b]{\highlight{Intrasexual} selection is competition between males.}

	\tiny \textcolor{white}{Luca Galuzzi, \href{http://www.galuzi.it}{www.galuzi.it}, \ccbysa{2.5}
	\hfill \href{http://www.youtube.com/watch?v=C7HCIGFdBt8}{Link to Video}}

\end{frame}
}
%
{
\usebackgroundtemplate{\includegraphics[width=\paperwidth]{superb_bird_of_paradise} }
\begin{frame}[b]{\textcolor{orange5}{Intersexual} \textcolor{white!95!black}{selection is females choosing mates.}}

%	\tiny \textcolor{white}{photographer unknown
	\tiny \textcolor{white}{\href{https://youtu.be/IPfW7iolmgc}{Link to Video} 
	\hfill \href{https://youtu.be/1XkPeN3AWIE}{Bonus Video}}
	
\end{frame}
}
%
{
\usebackgroundtemplate{\includegraphics[width=\paperwidth]{phalarope} }
\begin{frame}[b]{\textcolor{white}{Males are not \textit{always} the pretty ones!}}

	\tiny \textcolor{white}{U.S. Fish \& Wildlife Service.
	\hfill \href{https://youtu.be/15ZXpbKrZfI}{Link to Video}}
	
\end{frame}
}
%
{
\usebackgroundtemplate{\includegraphics[width=\paperwidth]{sexual_dimorphism_beetles} }
\begin{frame}[b]{Sexual selection causes phenotype differences between sexes, called \highlight{sexual dimorphism.}}

	\tiny Didier Descouens, Wikimedia \ccbysa{4}
	\hfill \href{http://www.youtube.com/watch?v=_VBz0FaXN1c}{Link to Video}
	
\end{frame}
}
%
\end{document}
