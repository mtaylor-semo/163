%!TEX TS-program = lualatex
%!TEX encoding = UTF-8 Unicode

\documentclass[t]{beamer}

%%%% HANDOUTS For online Uncomment the following four lines for handout
%\documentclass[t,handout]{beamer}  %Use this for handouts.
%\usepackage{handoutWithNotes}
%\pgfpagesuselayout{3 on 1 with notes}[letterpaper,border shrink=5mm]

%\includeonlylecture{student}

%%% Including only some slides for students.
%%% Uncomment the following line. For the slides,
%%% use the labels shown below the command.

%% For students, use \lecture{student}{student}
%% For mine, use \lecture{instructor}{instructor}



% Fonts
\usepackage{fontspec}
\def\mainfont{Linux Biolinum O}
\setmainfont[Ligatures={Common,TeX}, Contextuals={NoAlternate}, BoldFont={* Bold}, ItalicFont={* Italic}, Numbers={Proportional, OldStyle}]{\mainfont}
\setsansfont[Ligatures={Common,TeX}, Scale=MatchLowercase, Numbers={Proportional,OldStyle}, BoldFont={* Bold}, ItalicFont={* Italic},]{Linux Biolinum O} 
\usepackage{microtype}

\usepackage{amsmath,amssymb}
%\usepackage{unicode-math}
%\setmathfont[Scale=MatchLowercase]{TeX Gyre Termes Math}

\usepackage{graphicx}
	\graphicspath{{/Users/goby/pictures/teach/163/lecture/}
	{/Users/goby/pictures/teach/common/}} % set of paths to search for images

\usepackage{color}
%\definecolor{biolum}{RGB}{111,253,254}
\definecolor{biolum}{RGB}{6,174,249}

\usepackage{multicol}
\usepackage{booktabs}
\usepackage{array}
\newcolumntype{L}[1]{>{\raggedright\let\newline\\\arraybackslash\hspace{0pt}}p{#1}}
\newcolumntype{C}[1]{>{\centering\let\newline\\\arraybackslash\hspace{0pt}}p{#1}}
\newcolumntype{R}[1]{>{\raggedleft\let\newline\\\arraybackslash\hspace{0pt}}p{#1}}
%\usepackage{textcomp}
%\usepackage{mhchem}
\usepackage{enumitem}
%\usepackage[export]{adjustbox}


\usepackage{tikz}
	\tikzstyle{every picture}+=[remember picture,overlay]
	\usetikzlibrary{arrows}
\usetikzlibrary{positioning}

\mode<presentation>
{
  \usetheme{Lecture}
  \setbeamercovered{invisible}
  \setbeamertemplate{items}[square]
}


\begin{document}

\lecture{student}{student}

\begin{frame}{Our goal for this lecture is to}

	\hangpara learn how \highlight{source} and \highlight{sink} populations\\are interconnected to form \highlight{metapopulations.}
	
\end{frame}
%
%{
%\usebackgroundtemplate{\includegraphics[width=\paperwidth]{wombat} }
%\begin{frame}[b]
%	\hfill \tiny \textcolor{white}{J.J. Harrison, Wikimedia, \ccbysa{3}}
%\end{frame}
%}
%%
%{
%\usebackgroundtemplate{\includegraphics[width=\paperwidth]{wombat_range} }
%\begin{frame}{}
%
%\end{frame}
%}
%
\begin{frame}{\highlight{Metapopulations} are larger groups of smaller subpopulations.}

	\hangpara They have patches of suitable habitat surrounded by unsuitable habitat.
	
	\hangpara They are subject to local extinction and recolonization.
	
	\hangpara Interactions occur more often within subpopulations than among subpopulations.

\end{frame}
%
{
\usebackgroundtemplate{\includegraphics[width=\paperwidth]{metapopulation_glanville_fritillary} }
\begin{frame}[b]
	\hfill \tiny	\textcolor{white}{Marie-Lan Nguyen, Flickr, \ccby{2}}
\end{frame}
}
%
{
\usebackgroundtemplate{\includegraphics[width=\paperwidth]{metapopulation_fritillary_patches} }
\begin{frame}[b]
	\tiny \textcopyright Pearson Education, Inc.
\end{frame}
}
%
\begin{frame}{A hypothetical metapopulation.}
	\centering
	\includegraphics[height=0.8\textheight]{metapopulation_class}
	
	\vfilll
	
	\hfill \tiny \textcopyright T. G. Barnes, University of Kentucky Cooperative Extension Service.
\end{frame}

%
\begin{frame}{What do \highlight{source} and \highlight{sink} mean?}
	\centering
	\includegraphics[height=0.8\textheight]{metapopulation_class}
	
	\vfilll
	
	\hfill \tiny \textcopyright T. G. Barnes, University of Kentucky Cooperative Extension Service.
\end{frame}
%
\begin{frame}[c]{}
	\begin{multicols}{2}
		\onslide<1->\includegraphics[width=0.45\textwidth]{metapopulation_class}


	\columnbreak
		\onslide<1->\hangpara \highlight{Source populations} have high emigration and low local extinction.
		
		\onslide<2>\hangpara \highlight{Sink populations} have high local extinction but are recolonized by occasional immigration. 
	
	\end{multicols}

\end{frame}
%
\begin{frame}{Why are some habitat patches \highlight{sources} and others \highlight{sinks?}}
	\centering
	\includegraphics[height=0.8\textheight]{metapopulation_class}

	\vfilll
	
	\hfill \tiny \textcopyright T. G. Barnes, University of Kentucky Cooperative Extension Service.
	
\end{frame}
%
\begin{frame}[t]{What are the characteristics of a source?}
	\begin{minipage}{0.5\textwidth}
		\begin{center}
			\vspace{1\baselineskip}
			\includegraphics[width=0.9\textwidth]{metapopulation_class}
		\end{center}
	\end{minipage}\begin{minipage}{0.5\textwidth}
		\flushleft
		
		\hangpara \textcolor{white}{A}\vspace{-1\baselineskip}
		
		\hangpara \makebox[0.8\textwidth]{\hrulefill}
		
		\hangpara \makebox[0.8\textwidth]{\hrulefill}
		
		\hangpara \makebox[0.8\textwidth]{\hrulefill}
		
		\hangpara \makebox[0.8\textwidth]{\hrulefill}
		
	\end{minipage}	
\end{frame}

%
\lecture{instructor}{instructor}

\begin{frame}[t]{Source populations}
	\begin{minipage}{0.5\textwidth}
		\begin{center}
			\vspace{1\baselineskip}
			\includegraphics[width=0.9\textwidth]{metapopulation_class}
		\end{center}
	\end{minipage}\begin{minipage}{0.5\textwidth}
		\flushleft
		
		\hangpara \textcolor{white}{A}\vspace{-1\baselineskip}
		
		\hangpara are in the center of the range,\pause
		
		\hangpara have high \highlight{habitat quality,}\pause
		
		\hangpara have more births than deaths, \pause
		
		\hangpara and so have \highlight{more emigration} than immigration.
		
	\end{minipage}	
\end{frame}

%
\lecture{student}{student}

\begin{frame}[t]{What are the characteristics of a sink?}
	\begin{minipage}{0.5\textwidth}
		\begin{center}
			\vspace{1\baselineskip}
			\includegraphics[width=0.9\textwidth]{metapopulation_class}
		\end{center}
	\end{minipage}\begin{minipage}{0.5\textwidth}
		\flushleft
		
		\hangpara \textcolor{white}{A}\vspace{-1\baselineskip}
		
		\hangpara \makebox[0.8\textwidth]{\hrulefill}
		
		\hangpara \makebox[0.8\textwidth]{\hrulefill}
		
		\hangpara \makebox[0.8\textwidth]{\hrulefill}
		
		\hangpara \makebox[0.8\textwidth]{\hrulefill}
		
	\end{minipage}	
\end{frame}
%
\lecture{instructor}{instructor}

\begin{frame}[t]{Sink populations}
	\begin{minipage}{0.5\textwidth}
		\begin{center}
			\vspace{1\baselineskip}
			\includegraphics[width=0.9\textwidth]{metapopulation_class}
		\end{center}
	\end{minipage}\begin{minipage}{0.5\textwidth}
		\flushleft
		
		\hangpara \textcolor{white}{A}\vspace{-1\baselineskip}
		
		\hangpara are near the margins of the range,\pause
		
		\hangpara have low habitat quality,\pause
		
		\hangpara have more deaths than births,\pause
		
		\hangpara and so have \highlight{more immigration} than emigration.
		
	\end{minipage}	
\end{frame}
%
\lecture{student}{student}

\begin{frame}{Why do sink populations go extinct?}
	\begin{minipage}{0.5\textwidth}
		\begin{center}
			\vspace{1\baselineskip}
			\includegraphics[width=0.9\textwidth]{metapopulation_class}
		\end{center}
	\end{minipage}\begin{minipage}{0.5\textwidth}
		\flushleft
		
		\hangpara \textcolor{white}{A}\vspace{-1\baselineskip}
		
		\hangpara \makebox[0.8\textwidth]{\hrulefill}
		
		\hangpara \makebox[0.8\textwidth]{\hrulefill}
		
		\hangpara \makebox[0.8\textwidth]{\hrulefill}
		
		\hangpara \makebox[0.8\textwidth]{\hrulefill}
		
		\hangpara \makebox[0.8\textwidth]{\hrulefill}
		
	\end{minipage}	
\end{frame}
%
\lecture{instructor}{instructor}

\begin{frame}[t]{Sink populations go extinct because of}
	\begin{minipage}{0.5\textwidth}
		\begin{center}
			\vspace{1\baselineskip}
			\includegraphics[width=0.9\textwidth]{metapopulation_class}
		\end{center}
	\end{minipage}\begin{minipage}{0.5\textwidth}
		\flushleft
		
		\hangpara \textcolor{white}{A}\vspace{-1\baselineskip}
		
		\hangpara habitat quality,\pause
		
		\hangpara $d + e > b + i$,\pause
		
		\hangpara distance of sink from source,\pause
		
		\hangpara size of sink patch,\pause
		
		\hangpara and random events.
		
	\end{minipage}	
\end{frame}
%
\lecture{student}{student}

\begin{frame}{What does sink recolonization depend on?}
	\begin{minipage}{0.5\textwidth}
		\begin{center}
			\vspace{1\baselineskip}
			\includegraphics[width=0.9\textwidth]{metapopulation_class}
		\end{center}
	\end{minipage}\begin{minipage}{0.5\textwidth}
		\flushleft
		
		\hangpara \textcolor{white}{A}\vspace{-1\baselineskip}
		
		\hangpara \makebox[0.8\textwidth]{\hrulefill}
		
		\hangpara \makebox[0.8\textwidth]{\hrulefill}
		
		\hangpara \makebox[0.8\textwidth]{\hrulefill}
		
		\hangpara \makebox[0.8\textwidth]{\hrulefill}
		
		\hangpara \makebox[0.8\textwidth]{\hrulefill}
		
		\hangpara \makebox[0.8\textwidth]{\hrulefill}		
	\end{minipage}		
\end{frame}
%
\lecture{instructor}{instructor}

\begin{frame}[t]{Sink recolonization depends on}
	\begin{minipage}{0.5\textwidth}
		\begin{center}
			\vspace{1\baselineskip}
			\includegraphics[width=0.9\textwidth]{metapopulation_class}
		\end{center}
	\end{minipage}\begin{minipage}{0.5\textwidth}
		\flushleft
		
		\hangpara \textcolor{white}{A}\vspace{-1\baselineskip}
		
		\hangpara habitat quality,\pause
		
		\hangpara rate of source emigration,\pause
		
		\hangpara distance of sink from source,\pause
		
		\hangpara size of sink patch,\pause
		
		\hangpara unsuitable habitat between patches,\pause
		
		\hangpara and random events.
		
	\end{minipage}	
\end{frame}
%
\lecture{student}{student}

\begin{frame}[t]{Organisms can move among patches through \highlight{corridors} of suitable habitat.}

	\centering \includegraphics[width=0.85\textwidth]{metapopulation_corridors}

	\vfilll
	
	\hfill \tiny \textcopyright T. G. Barnes, University of Kentucky Cooperative Extension Service.
\end{frame}
%
\lecture{instructor}{instructor}

{
	\usebackgroundtemplate{\includegraphics[width=\paperwidth]{metapopulation_exercise} }
	\begin{frame}[b]{}
	\end{frame}
}
%
\begin{frame}[c]{}
	\begin{minipage}{0.8\textwidth}
		\begin{center}
			\vspace{1\baselineskip}
			\includegraphics[width=0.9\textwidth]{metapopulation_exercise}
		\end{center}
	\end{minipage}\begin{minipage}{0.19\textwidth}
		\flushleft
		
		\textbf{Source}\\
		
		\vspace*{9\baselineskip}
		
		\vspace*{2ex}
		
		\textbf{Sink} 
		
	\end{minipage}	
\end{frame}
%
\begin{frame}[c]{}
	\begin{minipage}{0.8\textwidth}
		\begin{center}
			\vspace{1\baselineskip}
			\includegraphics[width=0.9\textwidth]{metapopulation_exercise}
		\end{center}
	\end{minipage}\begin{minipage}{0.19\textwidth}
		\flushleft
		
		\textbf{Source}\\
		4H\\
		7H\\
		
		\vspace{1ex}
		8H=5H\\
		6M\\
		2M\\
		10L\\
		
		\vspace{1ex}
		9L\\
		1M\\
		3L\\
		\textbf{Sink} 
		
	\end{minipage}	
\end{frame}
%

%\lecture{instructor}{instructor}
%
%{
%\usebackgroundtemplate{\includegraphics[width=\paperwidth]{deep_sea_adaptations_biolum} }
%\begin{frame}[b]{\textcolor{biolum}{Cool stuff in biology: bioluminescence}}
%
%	\tiny\textcolor{white}{\href{https://www.youtube.com/watch?v=UXl8F-eIoiM}{Link to video}}
%\end{frame}
%}
%%
%{
%\usebackgroundtemplate{\includegraphics[width=\paperwidth]{cool_shit_vomit_biolum} }
%\begin{frame}[b]{\textcolor{biolum}{Don't you wish that you could vomit bioluminescence?}}
%
%\hfill \tiny \textcolor{biolum}{\textsc{noaa} Ocean Explorer, Flickr, \ccbysa{2}}
%\end{frame}
%}
%%
%{
%\usebackgroundtemplate{\includegraphics[width=\paperwidth]{cool_shit_dragonfish} }
%\begin{frame}[b]
%
%\tiny \textcolor{biolum}{\textcopyright\,Edith Widder, Ocean Research \& Conservation Assoc.}
%\end{frame}
%}

%
\end{document}
