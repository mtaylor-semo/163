%!TEX TS-program = lualatex
%!TEX encoding = UTF-8 Unicode

\documentclass[t]{beamer}

%%%% HANDOUTS For online Uncomment the following four lines for handout
%\documentclass[t,handout]{beamer}  %Use this for handouts.
%\usepackage{handoutWithNotes}
%\pgfpagesuselayout{3 on 1 with notes}[letterpaper,border shrink=5mm]

%\includeonlylecture{student}

%%% Including only some slides for students.
%%% Uncomment the following line. For the slides,
%%% use the labels shown below the command.

%% For students, use \lecture{student}{student}
%% For mine, use \lecture{instructor}{instructor}

% FONTS
\usepackage{fontspec}
\def\mainfont{Linux Biolinum O}
\setmainfont[Ligatures={Common,TeX}, Contextuals={NoAlternate}, BoldFont={* Bold}, ItalicFont={* Italic}, Numbers={OldStyle}]{\mainfont}
\setsansfont[Ligatures={Common,TeX}, Scale=MatchLowercase, Numbers=OldStyle, BoldFont={* Bold}, ItalicFont={* Italic},]{Linux Biolinum O} 
\newfontfamily{\liningtab}[Numbers=Lining]{Linux Biolinum O}
\usepackage{microtype}

\usepackage{graphicx}
	\graphicspath{{/Users/goby/pictures/teach/163/lecture/}
	{/Users/goby/pictures/teach/common/}} % set of paths to search for images

\usepackage[pgf]{adjustbox}

\usepackage{multicol}
\usepackage{longtable}
\usepackage{booktabs}
%\usepackage{textcomp}
\usepackage{mhchem}

\usepackage[justification=raggedright, labelsep=period]{caption}
\captionsetup{singlelinecheck=off}
\captionsetup{skip=0.2em}

\usepackage{tikz}
	\tikzstyle{every picture}+=[remember picture,overlay]
	\usetikzlibrary{arrows, arrows.meta}
	\usetikzlibrary{positioning, backgrounds}

\usepackage{forest}
\forestset{
    every leaf node/.style={
        if n children=0{#1}{}
    },
    every tree node/.style={
        if n children=0{}{#1}
    },
    mytree/.style={
        for tree={
            edge path={
            \noexpand\path [draw, very thick, \forestoption{edge}] (!u.parent anchor) |- (.child anchor)\forestoption{edge label};
            },
            every tree node={draw=none,inner sep=0, outer sep=0, minimum size=0},
            every leaf node/.style={align=left},
            grow=east,
            parent anchor=east, 
            child anchor=west,
            anchor=west,
            l sep=5mm,
            s sep=3mm,
            draw=none,
    			if n children=0{tier=word}{}
        }
    }
}

\mode<presentation>
{
  \usetheme{Lecture}
  \setbeamercovered{invisible}
}

\begin{document}

\lecture{student}{student}
\begin{frame}{Our goal for this lecture is to}
	
	\hangpara learn about fossils and the \highlight{fossil record},
	
	\hangpara get a sense of the scale of geological and evolutionary time, 
	
	\hangpara learn how eukaryotic cells evolved through \highlight{endosymbiosis,} and

	\hangpara explore the history of life on Earth.
	

\end{frame}
%


\begin{frame}{What is a \highlight{fossil?} }


	\alt<handout>{}{\onslide<2->{%
	
	\hangpara{\itshape “A preserved remnant or impression of an organism that lived in the past.” } \hfill— Campbell Biology, 10th ed.
	
	
	\hangpara{\itshape “Any remains, impression, or trace of a living thing of a former geologic age, as a skeleton, footprint, etc.”}\hfill — dictionary.com}
	
	}

\end{frame}
%
\begin{frame}[t]{Many types of fossils are known, including}

	\vspace*{-\baselineskip}

	\begin{multicols}{2}

		\includegraphics[width=0.48\textwidth]{fossil_permineralized}\vspace*{\baselineskip}

		\includegraphics[width=0.48\textwidth]{fossil_tissue}

	\columnbreak

		\includegraphics[width=0.48\textwidth]{fossil_trace}\vspace*{\baselineskip}

		Permineralized fossils (upper left)\\
		Trace fossils (upper right)\\
		Soft tissues (lower left)\\
		Chemical fossils (not shown)

	\end{multicols}

	\vfilll

	\tiny top row: Mark Wilson, Wikimedia public domain (top row). Lower left: \textcopyright Mary Schweitzer, North Carolina State University.

\end{frame}
%
{
\usebackgroundtemplate{\includegraphics[width=\paperwidth]{fossil_formation} }
\begin{frame}[b]{Fossils form under specific conditions.}
\hfill \tiny \textcopyright Roberts \& Company
\end{frame}
}
%
{
\usebackgroundtemplate{\includegraphics[width=\paperwidth]{fossil_radiometric_dating} }
\begin{frame}[b]{\highlight{Radiometric dating} is used to help age fossils.}
\hfill \tiny \textcopyright Roberts \& Company
\end{frame}
}
%
\begin{frame}[t]{Combinations of isotopes provide cross check of age.}

	\includegraphics[width=\textwidth]{radiometric_elements}

	\vfilll
	
	\hfill \tiny mitopencourseware, Flickr, \ccbyncsa{2}.

\end{frame}
%
\begin{frame}[t]{The parent element decreases by half for each half-life period. The daughter element increases in proportion.}

    {\liningtab
    \begin{longtable}[c]{@{}lccccccc@{}}
    \multicolumn{8}{l}{Percent remaining after each half-life.}\tabularnewline
    \toprule
    	& \multicolumn{7}{c}{Half-life Number}\tabularnewline
    	\cmidrule{2-8}
    Isotope	& 0 & 1 & 2 & 3 & 4 & 5 & 6 \tabularnewline
    \midrule
    Parent element 	& 100 & 50 & 25 & 12.5 & 6.25   & 3.125 & 1.5625 \tabularnewline
    Daugher element & 0     & 50 & 75 & 87.5 & 93.75 & 96.875 & 98.4375 \tabularnewline 
    \bottomrule
    \end{longtable}
    
    \pause
    
    \begin{longtable}{@{}lccccccc@{}}
    \multicolumn{8}{l}{Number of atoms remaining after each half-life.}\tabularnewline
    \toprule
    	& \multicolumn{7}{c}{Half-life Number}\tabularnewline
    	\cmidrule{2-8}
    Isotope	& 0 & 1 & 2 & 3 & 4 & 5 & 6 \tabularnewline
    \midrule
    Parent Element 	& 256 & 128 & 64 & 32 & 16   & 8 & 4 \tabularnewline
    Daugher element & 0   & 128 & 192 & 224 & 240 & 248 & 252 \tabularnewline 
    \bottomrule
    \end{longtable}
    }%liningtab
    
\end{frame}
%
\begin{frame}{Simple example: What is the age of a sample if\dots}

	\vspace{-\baselineskip}

	\onslide<1->{
		\hangpara Isotope A has a half-life of 50 million years, and the sample contains \\
		25\% of the parent element and 75\% of the daughter element?

	\hangpara A: Age $=$ 2 half lives $\times$ 50 \textsc{my}/half life $=$ \highlight{100 million years old.}
	}%onslide1

	\onslide<2->{
		\hangpara Isotope B has a half-life of 20 million years, and the sample contains \\
		3.125\% of the parent element and 96.875\% of the daughter element?

		\hangpara B: Age $=$ 5 half lives $\times$ 20 \textsc{my}/half life $=$ \highlight{100 million years old.}
	}%onslide2
	
	
	\onslide<1->{
	    {\liningtab
    		\begin{longtable}[c]{@{}lccccccc@{}}
	    \toprule
    		& \multicolumn{7}{c}{Half-life Number}\tabularnewline
    		\cmidrule{2-8}
	    Isotope	& 0 & 1 & 2 & 3 & 4 & 5 & 6 \tabularnewline
    		\midrule
	    Parent element 	& 100 & 50 & 25 & 12.5 & 6.25   & 3.125 & 1.5625 \tabularnewline
    		Daugher element & 0     & 50 & 75 & 87.5 & 93.75 & 96.875 & 98.4375 \tabularnewline 
	    \bottomrule
    		\end{longtable}
    		}%liningtab
	}%onslide1

\end{frame}
%
{
\usebackgroundtemplate{\includegraphics[width=\paperwidth]{fossil_relative_dating} }
\begin{frame}[b]
\tiny \textcopyright Roberts \& Company \hfill Jill Curie, Wikimedia, \ccbysa{3} 
\end{frame}
}
%
\begin{frame}[t]{Organisms leave distinct chemical signatures.}

	\includegraphics[width=\textwidth]{fossil_chemicals}

	\vfilll
	
	\hfill \tiny Schubotz 2009. Ph.D. Disseration, Üniversität Bremen.

\end{frame}
%
{
\usebackgroundtemplate{\includegraphics[width=\paperwidth]{geological_record_lower} }
\begin{frame}[b]
\hfill \tiny \textcopyright Pearson Education, Inc.
\end{frame}
}
%
{
\usebackgroundtemplate{\includegraphics[width=\paperwidth]{geological_record_upper} }
\begin{frame}[b]
\hfill \tiny \textcopyright Pearson Education, Inc.
\end{frame}
}
%
\lecture{instructor}{instructor}
{
\usebackgroundtemplate{\includegraphics[width=\paperwidth]{fossil_clock1} }
\begin{frame}[b]
\hfill \tiny \textcopyright Pearson Education, Inc.
\end{frame}
}
%
{
\usebackgroundtemplate{\includegraphics[width=\paperwidth]{fossil_clock2} }
\begin{frame}[b]
\hfill \tiny \textcopyright Pearson Education, Inc.
\end{frame}
}
%
\lecture{student}{student}
{
\usebackgroundtemplate{\includegraphics[width=\paperwidth]{fossil_clock3} }
\begin{frame}[b]
\hfill \tiny \textcopyright Pearson Education, Inc.
\end{frame}
}
%
\begin{frame}[t]{Life on Earth appeared at least 3.8 billion years ago.}

	\includegraphics[width=\textwidth]{fossil_stromatolite}
	
	\vfilll
	
	\hfill \tiny \textcopyright Pearson Education, Inc.
\end{frame}

{
\usebackgroundtemplate{\includegraphics[width=\paperwidth]{fossil_oxygen_increase} }
\begin{frame}[b]{Cyanobacteria caused atmospheric \ce{O2} to increase about 2.4 billion years ago.}
\tiny \textcopyright Pearson Education, Inc.
\end{frame}
}
%
{
\usebackgroundtemplate{\includegraphics[width=\paperwidth]{fossil_eukaryote} }
\begin{frame}[b]{Eukaryotes appeared \emph{at least} 2.1 billion years ago.}
\tiny \textcolor{white}{Xvazquez, Wikimedia, \ccbysa{2}}
\end{frame}
}
%
\begin{frame}[t]{Eukaryotic cells evolved through \highlight{endosymbiosis.}}
	\includegraphics[width=\textwidth]{fossil_endosymbiosis}
	
	\vfilll
	
	\hfill \tiny \textcopyright Pearson Education, Inc.
\end{frame}
%
{
\usebackgroundtemplate{\includegraphics[width=\paperwidth]{endosymbiosis_appearance} }
\begin{frame}

\end{frame}
}
%
\begin{frame}[t]{Evidence for \highlight{endosymbiosis.} }

	\hangpara Mitochondria and chloroplasts
	
	\hangpara \hspace*{1em} have circular \textsc{dna} like bacteria,
	
	\hangpara \hspace*{1em} divide like bacteria,
	
	\hangpara \hspace*{1em} have plasma membranes like bacteria, and
	
	\hangpara \hspace*{1em} are most closely related to bacteria.
	
\end{frame}


\end{document}
