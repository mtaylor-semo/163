%!TEX TS-program = lualatex
%!TEX encoding = UTF-8 Unicode

\documentclass[t]{beamer}

%%%% HANDOUTS For online Uncomment the following four lines for handout
%\documentclass[t,handout]{beamer}  %Use this for handouts.
%\includeonlylecture{student}
%\usepackage{handoutWithNotes}
%\pgfpagesuselayout{3 on 1 with notes}[letterpaper,border shrink=5mm]


%% For students, use \lecture{student}{student}
%% For mine, use \lecture{instructor}{instructor}

% FONTS
\usepackage{fontspec}
\def\mainfont{Linux Biolinum O}
\setmainfont[Ligatures={Common,TeX}, Contextuals={NoAlternate}, BoldFont={* Bold}, ItalicFont={* Italic}, Numbers={OldStyle}]{\mainfont}
\setsansfont[Ligatures={Common,TeX}, Scale=MatchLowercase, Numbers=OldStyle, BoldFont={* Bold}, ItalicFont={* Italic},]{Linux Biolinum O} 
\usepackage{microtype}

\usepackage{graphicx}
	\graphicspath{{/Users/mtaylor/pictures/teach/163/lecture/}
	{/Users/mtaylor/pictures/teach/common/}} % set of paths to search for images

\usepackage{multicol}
\usepackage{mhchem}

\usepackage{tikz}
	\tikzstyle{every picture}+=[remember picture,overlay]

\mode<presentation>
{
  \usetheme{Lecture}
  \setbeamercovered{invisible}
}


\begin{document}

\lecture{student}{student}

\begin{frame}{Our goal for this lecture is to}
	
	\hangpara learn some important macroevolutionary changes, and
	
	\hangpara learn some of the \highlight{transitional} forms that show the changes.

\end{frame}
%
{
\usebackgroundtemplate{\includegraphics[width=\paperwidth]{fossil_ediacaran_biota} }
\begin{frame}[b]{The \highlight{Ediacaran Period} was 635--542 \textsc{mya}.}
\hfill \tiny \textcopyright Roberts \& Company
\end{frame}
}
%
\lecture{instructor}{instructor}
{
\setbeamercolor{background canvas}{bg=black}
\begin{frame}[t]
	\includegraphics[width=\textwidth]{continents_ediacaran}
	
	\vfilll
	
	\hfill \tiny \textcolor{white}{\href{http://scotese.com}{scotese.com}}
\end{frame}
}
%
\lecture{student}{student}

{
\usebackgroundtemplate{\includegraphics[width=\paperwidth]{ediacaran_fauna_time} }
\begin{frame}[b]
\hfill \tiny \textcopyright Roberts \& Company
\end{frame}
}
%
{
\usebackgroundtemplate{\includegraphics[width=\paperwidth]{ediacaran_diversity} }
\begin{frame}[b]{The fauna had diverse body forms that do not exist today.}
\hfill \tiny \textcopyright Roberts \& Company
\end{frame}
}
%
\begin{frame}{Ancestors of modern phyla first evolved in the Ediacaran or even earlier. }

	\includegraphics[width=\textwidth]{modern_phyla_origins}

	\vfilll
	
	\hfill \tiny \textcopyright Roberts \& Company

\end{frame}
%
\lecture{instructor}{instructor}
{
\setbeamercolor{background canvas}{bg=black}
\begin{frame}[t]
	\includegraphics[width=\textwidth]{continents_cambrian}
	
	\vfilll
	
	\hfill \tiny \textcolor{white}{\href{http://scotese.com}{scotese.com}}
\end{frame}
}
%
\lecture{student}{student}

\begin{frame}{Modern phyla diversified during the \highlight{Cambrian radiation.} }

	\includegraphics[width=\textwidth]{cambrian_radiation}
	
	\hangpara There was no explosion!

	\vfilll
	
	\hfill \tiny \textcopyright Roberts \& Company

\end{frame}
%
{
\usebackgroundtemplate{\includegraphics[width=\paperwidth]{adaptive_radiation_honeycreepers} }
\begin{frame}[b]
\hfill \tiny \textcopyright Roberts \& Company
\end{frame}
}
%
{
\usebackgroundtemplate{\includegraphics[width=\paperwidth]{adaptive_radiation_silverswords} }
\begin{frame}[b]{Radiations can be identified on phylogenetic trees.}
\hfill \tiny \textcopyright Roberts \& Company
\end{frame}
}
%
\lecture{instructor}{instructor}
{
\setbeamercolor{background canvas}{bg=black}
\begin{frame}[t]
	\includegraphics[width=\textwidth]{continents_devonian}
	
	\vfilll
	
	\hfill \tiny \textcolor{white}{\href{http://scotese.com}{scotese.com}}
\end{frame}
}
%
\lecture{student}{student}

{
\usebackgroundtemplate{\includegraphics[width=\paperwidth]{tetrapod_trackways} }
\begin{frame}[b]{Tetrapods colonized land about 390 \textsc{mya}.}
\hfill \tiny \textcopyright Roberts \& Company
\end{frame}
}

{
\usebackgroundtemplate{\includegraphics[width=\paperwidth]{fish_ichthyostega} }
\begin{frame}[t]{What would you predict for an intermediate fossil?}

\begin{multicols}{2}

\phantom{fred}

\columnbreak

\textit{Eusthenopteron} was a lobe-finned fish that lived ca. 408 \textsc{mya}.

\vspace*{\baselineskip}

\textit{Ichthyostega} was an early “amphibian” that lived ca. 350 \textsc{mya}.

\vspace*{\baselineskip}

They share homologies like appendages and skull structure.

\vspace*{\baselineskip}

When would an intermediate organism have lived? In what kind of habitat? What kind of features might it have had?

\end{multicols}

\vfilll

\hfill \tiny \textcopyright Roberts \& Company
\end{frame}
}
%
\begin{frame}[t]{\textit{Tiktaalik} is a \highlight{transitional form} that lived \textasciitilde375 \textsc{mya}.}
	
	\vspace*{-\baselineskip}
	
	\begin{multicols}{2}
	
	\begin{center}
	\includegraphics[width=0.42\textwidth]{tiktaalik_skull} \\ \vspace*{1ex}
	\includegraphics[width=0.42\textwidth]{tiktaalik_paddle}\\
	\includegraphics[width=0.42\textwidth]{tiktaalik_prop}
	\end{center}

	\columnbreak
	
	It has some fish-like characters:\\
	\hspace*{1em} scales\\	
	\hspace*{1em} fins\\
	\hspace*{1em} gills \emph{and} lungs

	\vspace*{\baselineskip}
	
	It has some tetrapod-like characters:\\
	\hspace*{1em} neck	\\
	\hspace*{1em} enlarged ribs\\
	\hspace*{1em} flattened head\\
	\hspace*{1em} functional wrist
	
	\end{multicols}

\end{frame}
%
{
\usebackgroundtemplate{\includegraphics[width=\paperwidth]{transitional_form_fishes} }
\begin{frame}[b]
\hfill \tiny \textcopyright Roberts \& Company
\end{frame}
}
%
{
\usebackgroundtemplate{\includegraphics[width=\paperwidth]{transitional_form} }
\begin{frame}[b]
\hfill \tiny \textcopyright Roberts \& Company
\end{frame}
}
%
{
\usebackgroundtemplate{\includegraphics[width=\paperwidth]{inner_fish} }
\begin{frame}[b]

\onslide<2>{ 
\begin{tikzpicture}

	\draw (-0.15, 2.7) [thick, color=orange5] rectangle (9.25, 9.25);
		
\end{tikzpicture}}

\hfill \tiny Amemiya et al. 2013. Nature 496: 311.
\end{frame}
}
%
\begin{frame}[t]{Reptiles, mammals, and birds share an amniote ancestor.}
	\vspace*{-0.5\baselineskip}
	\begin{center}
		\includegraphics[width=\textwidth]{amniote_phylogeny}
	\end{center}

	\vfilll
	
	\hfill \tiny \textcopyright Pearson Education, Inc.
\end{frame}
%
{
\usebackgroundtemplate{\includegraphics[width=\paperwidth]{reptile_like_mammals} }
\begin{frame}[b]{Reptile-like mammals suggest that mammals evolved from reptilian ancestors.}

\end{frame}
}
%
{
\usebackgroundtemplate{\includegraphics[width=\paperwidth]{reptile_mammal_skulls} }
\begin{frame}[b]

	\hfill \tiny \textcopyright Pearson Education, Inc.
\end{frame}
}
%
{
\usebackgroundtemplate{\includegraphics[width=\paperwidth]{articular_quadrate_reduction} }
\begin{frame}[b]

	\Tiny \textcopyright Kardong 2002. \textit{Vertebrates: Comparative Anatomy, Function, Evolution},  3rd ed.
\end{frame}
}
%
\begin{frame}[t]{The mammalian ear bones evolved from the reptilian jaw bones.}

	\vspace*{-\baselineskip}
	
	\begin{center}
		\includegraphics[height=0.8\textheight]{jawjoints_earbones}
	\end{center}
	\vfilll
	
	\hfill \tiny \href{http://evolution.berkeley.edu/evolibrary/article/homology_06}{evolution.berkeley.edu}
\end{frame}
%
\lecture{instructor}{instructor}
{
\setbeamercolor{background canvas}{bg=black}
\begin{frame}[t]
	\includegraphics[width=\textwidth]{continents_triassic}
	
	\vfilll
	
	\hfill \tiny \textcolor{white}{\href{http://scotese.com}{scotese.com}}
\end{frame}
}
%
\lecture{student}{student}

{
\usebackgroundtemplate{\includegraphics[width=\paperwidth]{archaeopteryx_lithographica}}
\begin{frame}[t]
\end{frame}
}
%
{
\usebackgroundtemplate{\includegraphics[width=\paperwidth]{archaeopteryx_transition}}
\begin{frame}[t]
\end{frame}
}
%
{
\usebackgroundtemplate{\includegraphics[width=\paperwidth]{theropod_phylogeny1}}
\begin{frame}[b]

\hfill \tiny \textcopyright Roberts \& Company.
\end{frame}
}
%
\begin{frame}[t]{The embryonic ostrich manus has five digits.}
	\centering\includegraphics[height=0.82\textheight]{ostrich_manus}
	
	\vfilll
	
	\hfill \tiny Feduccia and Nowicki 2002. Naturwissenschaften 89: 391.
\end{frame}
%
{
\usebackgroundtemplate{\includegraphics[width=\paperwidth]{sinosauropteryx_fossil}}
\begin{frame}[t,plain]{\textcolor{white}{\textit{Sinosauropteryx}, a theropod dinosaur, had filamentous feathers.}}
\end{frame}
}
%
{
\usebackgroundtemplate{\includegraphics[width=\paperwidth]{zhenyuanlong_fossil}}
\begin{frame}[b]{\textit{Zhenuanlong} had flight-like feathers but did not fly.}

\hfill \tiny \textcolor{white}{Lü and Brusatte 2015, Scientific Reports 5: 11775.}
\end{frame}
}
%
{
\usebackgroundtemplate{\includegraphics[width=\paperwidth]{feathered_dinosaurs}}
\begin{frame}[t]
\end{frame}
}
%
{
\usebackgroundtemplate{\includegraphics[width=\paperwidth]{theropod_phylogeny} }
\begin{frame}[b]{Birds evolved from theropod dinosaurs about 150 \textsc{mya}.}
\hfill \tiny \textcopyright Roberts \& Company
\end{frame}
}
%
{
\usebackgroundtemplate{\includegraphics[width=\paperwidth]{t_rex_feathers}}
\begin{frame}[b]

\onslide<2>{ 
\begin{tikzpicture}

	\draw (0.1, 6.55) [thick, color=orange5] rectangle (4.5, 8.15);
		
\end{tikzpicture}}

\tiny Brusatte et al. 2014. Current Biology 24: 2386. \hfill Xu et al. 2012. Nature 484: 92. 
\end{frame}
}
%
{
\begin{frame}[b]{Dinosaurs are not extinct. They fly around us every day.}
	\centering
	\includegraphics[height=0.85\textheight]{dinosaur_genetics}

	\vfilll

	\tiny\hfill Schweitzer et al. 2009. Science 324: 626.
\end{frame}
}
%
%{
%\usebackgroundtemplate{\includegraphics[width=\paperwidth]{cetacean_transition} }
%\begin{frame}[b]
%
%\end{frame}
%}
%%
%{
%\usebackgroundtemplate{\includegraphics[width=\paperwidth]{auditory_bulla} }
%\begin{frame}[b]{Whales and even-toed mammals have different types of a bony ear structure.}
%
%\end{frame}
%}
%%
%\begin{frame}[t]{But, whales and terrestrial cetaceans have the same type of bony ear structure.}
%	\includegraphics[width=\textwidth]{auditory_bulla_indohyus}
%\end{frame}
%%
%\begin{frame}[t]{Terrestrial cetaceans and artiodactyls have a double-pulley ankle joint.}
%	\includegraphics[width=\textwidth]{double_pulley_ankle}
%	
%	\vfilll
%	
%	\hfill \tiny Pearson Education, Inc.
%\end{frame}
%%
%{
%\usebackgroundtemplate{\includegraphics[width=\paperwidth]{cetacean_appendages} }
%\begin{frame}[b]
%
%\end{frame}
%}
%
%{
%\usebackgroundtemplate{\includegraphics[width=\paperwidth]{human_transition} }
%\begin{frame}[b]
%\hfill \tiny \textcopyright Roberts \& Company
%\end{frame}
%}
%%
%{
%\usebackgroundtemplate{\includegraphics[width=\paperwidth]{bipedalism} }
%\begin{frame}[b]{What is the advantage of \highlight{bipedalism?}}
%\hfill \tiny \textcopyright McGraw-Hill
%\end{frame}
%}
%%
%{
%\begin{frame}[b]{Humans first evolved in east central Africa.}
%	\centering
%	\includegraphics[width=\textwidth]{human_dispersal}
%
%	\vfilll
%
%	\tiny\hfill McGraw-Hill.
%\end{frame}
%}
%%
%{
%\usebackgroundtemplate{\includegraphics[width=\paperwidth]{mass_extinctions} }
%\begin{frame}[b]{Five \highlight{mass extinctions} have occurred since the Cambrian.}
%\hfill \tiny \textcopyright Pearson Education, Inc.
%\end{frame}
%}
%%

\end{document}
