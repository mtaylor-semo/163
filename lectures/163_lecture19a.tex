\documentclass[t]{beamer}

%%%% HANDOUTS For online Uncomment the following four lines for handout
%\documentclass[t,handout]{beamer}  %Use this for handouts.
%\usepackage{handoutWithNotes}
%\pgfpagesuselayout{3 on 1 with notes}[letterpaper,border shrink=5mm]
%\pgfpagesuselayout{4 on 1}[border shrink=2mm]
%	\setbeamercolor{background canvas}{bg=black!5}

\includeonlylecture{student}



%\usepackage{pgf,pgfpages}
%\pgfpagesuselayout{4 on 1}[letterpaper,border shrink=5mm]

\usepackage{amsmath,amssymb}
\usepackage{graphicx}
	\graphicspath{{/Users/goby/pictures/teach/153/lecture/}} % set of paths to search for images
%\usepackage{units}
\usepackage{booktabs}
\usepackage{multicol}
%	\setlength{\columnsep=1em}
%\usepackage[utf8]{inputenc}
\usepackage[T1]{fontenc}
\usepackage{helvet}
\usepackage{textcomp}
\usepackage{setspace}
\usepackage{tikz}
	\tikzstyle{every picture}+=[remember picture,overlay]

\mode<presentation>
{
  \usetheme{Lecture}
  \setbeamercovered{invisible}
  \setbeamertemplate{items}[square]
}

\usefonttheme[onlymath]{serif}

\begin{document}

%\begin{frame}{Anscombe's Quartet \tiny (Informational only.)}
%	\begin{center}
%	\small
%	\begin{tabular}{@{}rrcrrcrrcrr@{}} \toprule
%	\multicolumn{2}{c}{I} & \phantom{|} & \multicolumn{2}{c}{II} & \phantom{l} & \multicolumn{2}{c}{III} & \phantom{l} & \multicolumn{2}{c}{IV}\\
%
%\cmidrule{1-2} \cmidrule{4-5} \cmidrule{7-8} \cmidrule{10-11}
%$x$ & $y$ && $x$ & $y$ && $x$ & $y$ && $x$ & $y$\\	
%10.0 & 8.04 & & 10.0 & 9.14  & &  10.0 & 7.46  & &  8.0 & 6.58\\
%8.0 & 6.95 & & 8.0 & 8.14 & & 8.0 & 6.77 & & 8.0 & 5.76\\
%13.0 & 7.58 & & 13.0 & 8.74 & & 13.0 & 12.74 & & 8.0 & 7.71\\
%9.0 & 8.81 & & 9.0 & 8.77 & & 9.0 & 7.11 & & 8.0 & 8.84\\
%11.0 & 8.33 & & 11.0 & 9.26 & & 11.0 & 7.81 & & 8.0 & 8.47\\
%14.0 & 9.96 & & 14.0 & 8.10 & & 14.0 & 8.84 & & 8.0 & 7.04\\
%6.0 & 7.24 & & 6.0 & 6.13 & & 6.0 & 6.08 & & 8.0 & 5.25\\
%4.0 & 4.26 & & 4.0 & 3.10 & & 4.0 & 5.39 & & 19.0 & 12.50\\
%12.0 & 10.84 & & 12.0 & 9.13 & & 12.0 & 8.15 & & 8.0 & 5.56\\
%7.0 & 4.82 & & 7.0 & 7.26 & & 7.0 & 6.42 & & 8.0 & 7.91\\
%5.0 & 5.68 & & 5.0 & 4.74 & & 5.0 & 5.73 & & 8.0 & 6.89\\
%\bottomrule
%\end{tabular}
%\end{center}
%
%\vspace{-1\baselineskip}
%\hangpara For all $x$ columns, $\bar{X} = 9.00$ and $\sigma^2 = 11.00$\\
%For all $y$ columns, $\bar{Y} = 7.50$ and $\sigma^2 = 4.12$
%\end{frame}
%
%{
%\usebackgroundtemplate{\includegraphics[width=\paperwidth]{anscombes_quartet} }
%\begin{frame}{}
%\end{frame}
%}
%
%{
%\usebackgroundtemplate{\includegraphics[width=\paperwidth]{iris_results.pdf} }
%\begin{frame}{}
%\end{frame}
%}


% Lecture goals

\begin{frame}[t]{A sobering view\dots}

	\vspace*{-\baselineskip}
	
	\hangpara $N$ = 164 \hfill $\overline{X}$ = 66.9 \pm 1.1 points. \hfill High = 100 (2 students)\\[2ex]

	\includegraphics[width=\textwidth]{153exam1grades_sp16}
	
\end{frame}

\begin{frame}{Our goal for this lecture is to}

	\hangpara show how real populations grow compared to the logistic growth curve, and
	
	\hangpara learn how \highlight{negative feedback} controls \highlight{density-dependent} population growth, and
	
	\hangpara learn how \highlight{density-independent} factors control population growth.
	
\end{frame}

{
\usebackgroundtemplate{\includegraphics[width=\paperwidth]{real_populations_daphnia} }
\begin{frame}{Populations initially \highlight{exceed} carrying capacity.}
\end{frame}
}

{
\usebackgroundtemplate{\includegraphics[width=\paperwidth]{real_populations_sparrow} }
\begin{frame}{Populations \highlight{fluctuate around} carrying capacity.}
\end{frame}
}


\begin{frame}{}

	\hangpara \highlight{Density-dependent} population regulation occurs when high population density \highlight{decreases birth rate} or \highlight{increases death rate}.
	
	\hangpara What would happen to population size ($N$) over time?
	
\end{frame}

{
\usebackgroundtemplate{\includegraphics[width=\paperwidth]{density_dependent_song_sparrow} }
\begin{frame}[b]{}
\Tiny\textcolor{gray!20!white}{Song sparrow \textit{Melospiza melodia} photo by Keith, Wikimedia Commons.}
\end{frame}
}

{
\usebackgroundtemplate{\includegraphics[width=\paperwidth]{density_dependent_sparrow_nest} }
\begin{frame}[b]{}
\Tiny\textcolor{gray!20!white}{Song sparrow \textit{Melospiza melodia} nest photo by K.P. McFarland, Flickr, Creative Commons.}
\end{frame}
}


\begin{frame}[b]{How does clutch size change? Why?}
	\begin{center}
		\includegraphics[height=0.8\textheight]{density_dependent_clutchsize.pdf}
	\end{center}	

	\tiny Arcese and Smith 1988, J. Animal Ecology 57: 119-136.
\end{frame}

\begin{frame}[b]{How does nest failure change? Why?}
	\begin{center}
		\includegraphics[height=0.8\textheight]{density_dependent_failure.pdf}
	\end{center}	

	\tiny Arcese and Smith 1988, J. Animal Ecology 57: 119-136.
\end{frame}


{
\usebackgroundtemplate{\includegraphics[width=\paperwidth]{density_dependent_territory} }
\begin{frame}[b]{}
\Tiny\textcolor{orange5}{Gannets \textit{Morus serrator}, Muriwai Regional Park, New Zealand photo by Foliash, Wikimedia Commons.}
\end{frame}
}

{
\usebackgroundtemplate{\includegraphics[width=\paperwidth]{density_dependent_lynx_hare} }
\begin{frame}[b]{}
\end{frame}
}


\begin{frame}{}

	\hangpara \highlight{Density-independent} population regulation decreases birth rate or increases death rate \highlight{regardless} of current population size.
		
\end{frame}

{
\usebackgroundtemplate{\includegraphics[width=\paperwidth]{density_independent_dunes} }
\begin{frame}[b]{}
\hfill\Tiny Dune grass in primary dune photo by Sandy Richard, Flickr, Creative Commons.
\end{frame}
}

{
\usebackgroundtemplate{\includegraphics[width=\paperwidth]{isle_royale} }
\begin{frame}{}
\end{frame}
}

{
\usebackgroundtemplate{\includegraphics[width=\paperwidth]{isle_royale_popsize} }
\begin{frame}{}
\end{frame}
}



\end{document}
