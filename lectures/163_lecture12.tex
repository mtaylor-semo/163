%!TEX TS-program = lualatex
%!TEX encoding = UTF-8 Unicode

\documentclass[t]{beamer}

%%%% HANDOUTS For online Uncomment the following four lines for handout
%\documentclass[t,handout]{beamer}  %Use this for handouts.
%\usepackage{handoutWithNotes}
%\pgfpagesuselayout{3 on 1 with notes}[letterpaper,border shrink=5mm]
%	\setbeamercolor{background canvas}{bg=black!5}

%\includeonlylecture{student}

%%% Including only some slides for students.
%%% Uncomment the following line. For the slides,
%%% use the labels shown below the command.

%% For students, use \lecture{student}{student}
%% For mine, use \lecture{instructor}{instructor}

% FONTS
\usepackage{fontspec}
\def\mainfont{Linux Biolinum O}
\setmainfont[Ligatures={Common,TeX}, Contextuals={NoAlternate}, BoldFont={* Bold}, ItalicFont={* Italic}, Numbers={OldStyle}]{\mainfont}
\setsansfont[Ligatures={Common,TeX}, Scale=MatchLowercase, Numbers=OldStyle, BoldFont={* Bold}, ItalicFont={* Italic},]{Linux Biolinum O} 
\usepackage{microtype}

\usepackage{graphicx}
	\graphicspath{{/Users/goby/pictures/teach/163/lecture/}
	{/Users/goby/pictures/teach/common/}} % set of paths to search for images

\usepackage{multicol}
%\usepackage{booktabs}
%\usepackage{textcomp}
\usepackage{mhchem}

%\usepackage{tikz}
%	\tikzstyle{every picture}+=[remember picture,overlay]
%	\usetikzlibrary{arrows}

\mode<presentation>
{
  \usetheme{Lecture}
  \setbeamercovered{invisible}
  \setbeamertemplate{items}[square]
}

%%% Creative Commons Licenses. Establish, then add to Beamer template.
%\newcommand{\ccbysa}[1][4]{\textsc{cc by-sa #1.0}} % Use version 4.0 as default.
\newcommand{\ccby}[1]{%
	\ifx&#1&
	{\textsc{cc by}}%
\else
	{\textsc{cc by #1.0}}
\fi}


\newcommand{\ccbysa}[1]{%
	\ifx&#1&
	{\textsc{cc by-sa}}%
\else
	{\textsc{cc by-sa #1.0}} 
\fi}

\newcommand{\ccbyncsa}[1]{%
	\ifx&#1&
	{\textsc{cc by-nc-sa}}%
\else
	{\textsc{cc by-nc-sa #1.0}}
\fi}

\newcommand{\ccbync}[1]{%
	\ifx&#1&
	{\textsc{cc by-nc}}%
\else
	{\textsc{cc by-nc #1.0}}
\fi}


\begin{document}

\lecture{student}{student}
\begin{frame}{Our goal for this lecture is to}
	
	\hangpara learn about fossils and the \highlight{fossil record},
	
	\hangpara get a sense of the scale of geological and evolutionary time, and

	\hangpara explore the history of life on Earth.
	

\end{frame}
%


\begin{frame}{What is a \highlight{fossil?} }


	\alt<handout>{}{\onslide<2->{%
	
	\hangpara{\itshape “A preserved remnant or impression of an organism that lived in the past.” } \hfill— Campbell Biology, 10th ed.
	
	
	\hangpara{\itshape “Any remains, impression, or trace of a living thing of a former geologic age, as a skeleton, footprint, etc.”}\hfill — dictionary.com}
	
	}

\end{frame}
%
\begin{frame}[t]{Many types of fossils are known, including}

	\vspace*{-\baselineskip}

	\begin{multicols}{2}

		\includegraphics[width=0.48\textwidth]{fossil_permineralized}\vspace*{\baselineskip}

		\includegraphics[width=0.48\textwidth]{fossil_tissue}

	\columnbreak

		\includegraphics[width=0.48\textwidth]{fossil_trace}\vspace*{\baselineskip}

		Permineralized fossils (upper left)\\
		Trace fossils (upper right)\\
		Soft tissues (lower left)\\
		Chemical fossils (not shown)

	\end{multicols}

	\vfilll

	\tiny top row: Mark Wilson, Wikimedia public domain (top row). Lower left: \textcopyright Mary Schweitzer, North Carolina State University.

\end{frame}
%
{
\usebackgroundtemplate{\includegraphics[width=\paperwidth]{fossil_formation} }
\begin{frame}[b]{Fossils form under specific conditions.}
\hfill \tiny \textcopyright Roberts \& Company
\end{frame}
}
%
{
\usebackgroundtemplate{\includegraphics[width=\paperwidth]{fossil_radiometric_dating} }
\begin{frame}[b]{\highlight{Radiometric dating} is used to age fossils.}
\hfill \tiny \textcopyright Roberts \& Company
\end{frame}
}
%
\begin{frame}[t]{Organisms leave distinct chemical signatures.}

	\includegraphics[width=\textwidth]{radiometric_elements}

	\vfilll
	
	\hfill \tiny mitopencourseware, Flickr, \ccbyncsa{2}.

\end{frame}
%
{
\usebackgroundtemplate{\includegraphics[width=\paperwidth]{fossil_relative_dating} }
\begin{frame}[b]
\tiny \textcopyright Roberts \& Company \hfill Jill Curie, Wikimedia, \ccbysa{3} 
\end{frame}
}
%
\begin{frame}[t]{Organisms leave distinct chemical signatures.}

	\includegraphics[width=\textwidth]{fossil_chemicals}

	\vfilll
	
	\hfill \tiny Schubotz 2009. Ph.D. Disseration, Üniversität Bremen.

\end{frame}
%
{
\usebackgroundtemplate{\includegraphics[width=\paperwidth]{geological_record_lower} }
\begin{frame}[b]
\hfill \tiny \textcopyright Pearson Education, Inc.
\end{frame}
}
%
{
\usebackgroundtemplate{\includegraphics[width=\paperwidth]{geological_record_upper} }
\begin{frame}[b]
\hfill \tiny \textcopyright Pearson Education, Inc.
\end{frame}
}
%
\lecture{instructor}{instructor}
{
\usebackgroundtemplate{\includegraphics[width=\paperwidth]{fossil_clock1} }
\begin{frame}[b]
\hfill \tiny \textcopyright Pearson Education, Inc.
\end{frame}
}
%
{
\usebackgroundtemplate{\includegraphics[width=\paperwidth]{fossil_clock2} }
\begin{frame}[b]
\hfill \tiny \textcopyright Pearson Education, Inc.
\end{frame}
}
%
\lecture{student}{student}
{
\usebackgroundtemplate{\includegraphics[width=\paperwidth]{fossil_clock3} }
\begin{frame}[b]
\hfill \tiny \textcopyright Pearson Education, Inc.
\end{frame}
}
%
\begin{frame}[t]{Life on Earth appeared at least 3.8 billion years ago.}

	\includegraphics[width=\textwidth]{fossil_stromatolite}
	
	\vfilll
	
	\hfill \tiny \textcopyright Pearson Education, Inc.
\end{frame}

{
\usebackgroundtemplate{\includegraphics[width=\paperwidth]{fossil_oxygen_increase} }
\begin{frame}[b]{Cyanobacteria caused atmospheric \ce{O2} to increase about 2.4 billion years ago.}
\tiny \textcopyright Pearson Education, Inc.
\end{frame}
}
%
{
\usebackgroundtemplate{\includegraphics[width=\paperwidth]{fossil_eukaryote} }
\begin{frame}[b]{Eukaryotes appeared \emph{at least} 2.1 billion years ago.}
\tiny \textcolor{white}{Xvazquez, Wikimedia, \ccbysa{2}}
\end{frame}
}
%
\begin{frame}[t]{Eukaryotic cells evolved through \highlight{endosymbiosis.}}
	\includegraphics[width=\textwidth]{fossil_endosymbiosis}
	
	\vfilll
	
	\hfill \tiny \textcopyright Pearson Education, Inc.
\end{frame}

\end{document}
