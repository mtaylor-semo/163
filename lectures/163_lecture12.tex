%!TEX TS-program = lualatex
%!TEX encoding = UTF-8 Unicode

\documentclass[t]{beamer}

%%%% HANDOUTS For online Uncomment the following four lines for handout
%\documentclass[t,handout]{beamer}  %Use this for handouts.
%\usepackage{handoutWithNotes}
%\pgfpagesuselayout{3 on 1 with notes}[letterpaper,border shrink=5mm]
%	\setbeamercolor{background canvas}{bg=black!5}

%\includeonlylecture{student}

%%% Including only some slides for students.
%%% Uncomment the following line. For the slides,
%%% use the labels shown below the command.

%% For students, use \lecture{student}{student}
%% For mine, use \lecture{instructor}{instructor}

% FONTS
\usepackage{fontspec}
\def\mainfont{Linux Biolinum O}
\setmainfont[Ligatures={Common,TeX}, Contextuals={NoAlternate}, BoldFont={* Bold}, ItalicFont={* Italic}, Numbers={OldStyle}]{\mainfont}
\setsansfont[Ligatures={Common,TeX}, Scale=MatchLowercase, Numbers=OldStyle, BoldFont={* Bold}, ItalicFont={* Italic},]{Linux Biolinum O} 
\usepackage{microtype}

\usepackage{graphicx}
	\graphicspath{{/Users/goby/pictures/teach/163/lecture/}
	{/Users/goby/pictures/teach/common/}} % set of paths to search for images

\usepackage{multicol}
\usepackage{booktabs}
%\usepackage{textcomp}

\usepackage{tikz}
	\tikzstyle{every picture}+=[remember picture,overlay]
%	\usetikzlibrary{arrows}


\mode<presentation>
{
  \usetheme{Lecture}
  \setbeamercovered{invisible}
  \setbeamertemplate{items}[square]
}

%\usefonttheme[onlymath]{serif}
%\usecolortheme[named=blue7]{structure}


%%% Creative Commons Licenses. Establish, then add to Beamer template.
%\newcommand{\ccbysa}[1][4]{\textsc{cc by-sa #1.0}} % Use version 4.0 as default.
\newcommand{\ccby}[1]{%
	\ifx&#1&
	{\textsc{cc by}}%
\else
	{\textsc{cc by #1.0}}
\fi}


\newcommand{\ccbysa}[1]{%
	\ifx&#1&
	{\textsc{cc by-sa}}%
\else
	{\textsc{cc by-sa #1.0}} 
\fi}

\newcommand{\ccbyncsa}[1]{%
	\ifx&#1&
	{\textsc{cc by-nc-sa}}%
\else
	{\textsc{cc by-nc-sa #1.0}}
\fi}

\newcommand{\ccbync}[1]{%
	\ifx&#1&
	{\textsc{cc by-nc}}%
\else
	{\textsc{cc by-nc #1.0}}
\fi}


\begin{document}

\lecture{student}{student}
\begin{frame}{Our goal for this lecture is to }
	
	\hangpara use software to model evolutionary processes, and
	
	\hangpara define what is a \highlight{species}.
	
\end{frame}
%
\lecture{instructor}{instructor}
{
\usebackgroundtemplate{\includegraphics[width=\paperwidth]{founder_effect_silver_eye} }
\begin{frame}[b,plain]{}
\tiny \textcopyright Pearson Education \hfill Toby Hudson, Wikimedia, \ccbysa{3}
\end{frame}
}
{
\usebackgroundtemplate{\includegraphics[width=\paperwidth]{hms_bounty} }
\begin{frame}[b]{}
\hfill \textcolor{white}{\tiny Yasmina, Wikimedia, \ccby{3}.}
\end{frame}
}

{
\usebackgroundtemplate{\includegraphics[width=\paperwidth]{tahiti} }
\begin{frame}[b]{}
\hfill \textcolor{white}{\tiny Moises Gonzalez, Flickr, ccbync{2}.}
\end{frame}
}

{
\usebackgroundtemplate{\includegraphics[width=\paperwidth]{bounty_mutiny} }
\begin{frame}[b]{}
\hfill \textcolor{white}{\tiny National Maritime Museum, Wikimedia, public domain.}
\end{frame}
}

{
\usebackgroundtemplate{\includegraphics[width=\paperwidth]{mutineer_route} }
\begin{frame}[b]{}
\tiny Benton et al. 2015. Investigative Genetics 6: 12.
\end{frame}
}
%
% Subdividing evolution
\begin{frame}{Evolution can be divided into three related processes.}

	\hangpara \alert<1,5,7>{Microevolution:} evolution within populations.
	\pause

	\hangpara \alert<2,6>{Speciation:} the evolution of new species.
	\pause

	\hangpara \alert<3,8>{Macroevolution:} evolution above the species level.
	\pause

	\centering
	\begin{tikzpicture}
		[remember picture, overlay,
		myLine/.style={color=black,very thick},
		myArrow/.style={color=orange6, thick, ->}]

		\visible<4>{
			\draw [myLine] (-3,-2) -- (0,-2);
			\draw [myLine] (0,-2) -- (0,-0.7); % vertical lines
			\draw [myLine] (0,-2) -- (0,-3.3);
			\draw [myLine] (0,-0.7) -- (3,-0.7); % descendants
			\draw [myLine] (0,-3.3) -- (3,-3.3); 
		}

		% Draw the tree
		\visible<5->{
			\draw [myLine] (-3,-2) -- (0,-2);
		} % ancestor

		\visible<6->{
			\draw [myLine] (0,-2) -- (0,-0.7); % vertical lines
			\draw [myLine] (0,-2) -- (0,-3.3);
		}
		
		\visible<7->{
			\draw [myLine] (0,-0.7) -- (3,-0.7); % descendants
			\draw [myLine] (0,-3.3) -- (3,-3.3); 
		}

		% Ancestor microevolution
		\visible<5>{
			\node [color=orange6] at (-1.5,-1.8) {Microevolution};
		} % ancestor
	
		% Speciation
		\visible<6>{
			\node [color=orange6] (speciation) at (-1.5,-3.1) {Speciation};
			\node [circle, draw=orange6, thick, minimum size=3mm] (speccirc) at (0,-2) {};
			\draw [color=orange6, thick] (speciation) edge (speccirc);
		}
		
		% Descendant microevolution
		\visible<7>{
			\node [color=orange6] at (1.5,-0.5) {Microevolution}; % descendants
			\node [color=orange6] at (1.5,-3.1) {Microevolution};
		}

		% Macroevolution
		\visible<8>{
			\node [color=orange6] (macro) at (1.5,-2) {Macroevolution};
			\draw [myArrow] (macro.north) to (1.5,-0.8);
			\draw [myArrow] (macro.south) to (1.5,-3.2);
		}

	\end{tikzpicture}


%	\vspace{\baselineskip}
%	\hangpara Speciation and macroevolution require microevolution to occur.
\end{frame}

\begin{frame}[t]{Let's play with allele frequencies.}
	\vspace*{-\baselineskip}
	\begin{center}
		\includegraphics[width=0.7\textwidth]{genetic_drift}
	\end{center}

\end{frame}

\lecture{student}{student}

{
\usebackgroundtemplate{\includegraphics[width=\paperwidth]{atelopus_zeteki.jpg} }
\begin{frame}[b]{What is a \highlight{species?} }

\hfill\Tiny\textcolor{gray}{\textit{Atelopus zeteki} photo by Brian Gratwicke, Flickr, Creative Commons.}

\end{frame}
}

\begin{frame}{``Species'' is not easily defined.}
	\vspace{2\baselineskip}
	\centering
	\begin{tabular}{l l}
	\toprule
		\highlight{Morphological}	&	Genealogical\\
		\highlight{Biological}		&	Genotypic\\
		General Lineage		&	Recognition\\
		Phylogenetic (I)			&	Phenetic\\
		Phylogenetic (II)		&	Cladistic\\
		Evolutionary			&	Diagnostic\\
		Ecological				&	Polytypic\\
		Typological			&	Population\\
		Cohesion				&	\\
	\bottomrule
	\end{tabular}
\end{frame}

{
\usebackgroundtemplate{\includegraphics[width=\paperwidth]{morphological_species} }
\begin{frame}[b]{}
\hfill\Tiny\textcolor{gray}{\textit{Atelopus limosus} and \textit{Dendrobates auratus} photos by Brian Gratwicke, Flickr, Creative Commons.}
\end{frame}
}


{
\usebackgroundtemplate{\includegraphics[width=\paperwidth]{dendrobates_morphological} }
\begin{frame}[b]{}
\hfill\Tiny\textcolor{gray}{Dart frog photos from Wikimedia Commons.}
\end{frame}
}

\begin{frame}[t]{\textit{Hypsiboas} is a genus of \highlight{cryptic species.}}
\centering
\begin{multicols}{2}
	\includegraphics[width=0.45\textwidth]{hypsiboas_frogs.pdf}
	\vfill
	\columnbreak
	\includegraphics[width=0.5\textwidth]{hypsiboas_map} \vfill
\end{multicols}
\vspace{5ex}
\hfill\Tiny\textcolor{gray}{Funk et al., 2012. Proc. R. Soc. London B 279: 1806-1814.}
\end{frame}



\begin{frame}{}
\vspace{1\baselineskip}
\begin{multicols}{2}
	Body shape is not reliable.\vspace{\baselineskip}
	
	\includegraphics[width=0.46\textwidth]{hypsiboas_body_pc}
	\columnbreak

	\pause
	Mating calls are reliable.\vspace{\baselineskip}
	\includegraphics[width=0.5\textwidth]{hypsiboas_body_calls}
	
\end{multicols}
\vspace{12ex}
\hfill\Tiny\textcolor{gray}{Funk et al., 2012. Proc. R. Soc. London B 279: 1806-1814.}
\end{frame}

\begin{frame}{The \highlight{cryptic species} are genetically distinct.}
	\centering
	\begin{multicols}{2}
		\includegraphics[width=0.45\textwidth]{hypsiboas_frogs.pdf} \vfill
		\columnbreak
		\includegraphics[height=0.8\textheight]{hypsiboas_phylog} \vfill
	\end{multicols}
\vspace{-1ex}
\hfill\Tiny\textcolor{gray}{Funk et al., 2012. Proc. R. Soc. London B 279: 1806-1814.}
\end{frame}



{
\usebackgroundtemplate{\includegraphics[width=\paperwidth]{biological_species_penguins.jpg} }
\begin{frame}[b]
\Tiny\textcolor{gray!50!white}{King penguins photo by Liam Quinn, Flickr, Creative Commons.}

\end{frame}
}

{
\usebackgroundtemplate{\includegraphics[width=\paperwidth]{interbreed_creepers} }
\begin{frame}[b]{\textcolor{white}{Can these sister species} \textcolor{orange5}{interbreed?}}
\Tiny\textcolor{gray!50!white}{Brown creeper photo by Alan Vernon, Wikimedia Commons. Short-Toed Creeper photo by Thomas van de Vosse, Flickr, Creative Commons.}
\end{frame}
}

{
\usebackgroundtemplate{\includegraphics[width=\paperwidth]{interbreed_zebras} }
\begin{frame}[b]{\textcolor{white}{Can these two species} \textcolor{orange5}{interbreed?}}
\Tiny\textcolor{gray!50!white}{Grévy's zebra photo by Rainbirder, Wikimedia Commons. Hagerman's zebra, Utah Museum of Natural History, photo by Daderot, Wikimedia Commons.}
\end{frame}
}

{
\usebackgroundtemplate{\includegraphics[width=\paperwidth]{asexual_rotifer} }
\begin{frame}[b]{}
\Tiny\textcolor{gray!20!white}{Bdelloid rotifer photo by Damián H. Zanette, Wikimedia Commons.}
\end{frame}
}




\end{document}
