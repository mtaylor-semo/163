%!TEX TS-program = lualatex
%!TEX encoding = UTF-8 Unicode

\documentclass[t]{beamer}

%%%% HANDOUTS For online Uncomment the following four lines for handout
%\documentclass[t,handout]{beamer}  %Use this for handouts.
%\usepackage{handoutWithNotes}
%\pgfpagesuselayout{3 on 1 with notes}[letterpaper,border shrink=5mm]

%\includeonlylecture{student}

%%% Including only some slides for students.
%%% Uncomment the following line. For the slides,
%%% use the labels shown below the command.

%% For students, use \lecture{student}{student}
%% For mine, use \lecture{instructor}{instructor}

% FONTS
\usepackage{fontspec}
\def\mainfont{Linux Biolinum O}
\setmainfont[Ligatures={Common,TeX}, Contextuals={NoAlternate}, BoldFont={* Bold}, ItalicFont={* Italic}, Numbers={OldStyle}]{\mainfont}
\setsansfont[Ligatures={Common,TeX}, Scale=MatchLowercase, Numbers=OldStyle, BoldFont={* Bold}, ItalicFont={* Italic},]{Linux Biolinum O} 
\usepackage{microtype}

\usepackage{graphicx}
	\graphicspath{{/Users/mtaylor/pictures/teach/163/lecture/}
	{/Users/mtaylor/pictures/teach/common/}} % set of paths to search for images

\usepackage{multicol}
\usepackage{booktabs}
%\usepackage{textcomp}
%\usepackage{mhchem}
\usepackage{enumitem}
\usepackage[export]{adjustbox}


\usepackage{tikz}
	\tikzstyle{every picture}+=[remember picture,overlay]
%	\usetikzlibrary{arrows}

\mode<presentation>
{
  \usetheme{Lecture}
  \setbeamercovered{invisible}
}


\begin{document}

\lecture{student}{student}

\begin{frame}{Our goal for this lecture is to continue}
	
	\hangpara learning some important macroevolutionary changes, and
	
	\hangpara learning the \highlight{transitional forms} that show the changes, and
	
	\hangpara to briefly discuss \highlight{mass extinctions.}

\end{frame}
%
\lecture{instructor}{instructor}
{
\setbeamercolor{background canvas}{bg=black}
\begin{frame}[t]
	\includegraphics[width=\textwidth]{continents_miocene}
	
	\vfilll
	
	\hfill \tiny \textcolor{white}{\href{http://scotese.com}{scotese.com}}
\end{frame}
}
%
\lecture{student}{student}


{
\usebackgroundtemplate{\includegraphics[width=\paperwidth]{artiodactyla_phylogeny1} }
\begin{frame}[b]

\tiny Spaulding et al. 2009. PLoS \textsc{one} 4(9): e7962.
\end{frame}
}

{
\usebackgroundtemplate{\includegraphics[width=\paperwidth]{artiodactyla_phylogeny2} }
\begin{frame}[b]

\tiny Spaulding et al. 2009. PLoS \textsc{one} 4(9): e7962.
\end{frame}
}

{
\usebackgroundtemplate{\includegraphics[width=\paperwidth]{cetacean_transition} }
\begin{frame}[b]

\end{frame}
}
%
{
\usebackgroundtemplate{\includegraphics[width=\paperwidth]{auditory_bulla} }
\begin{frame}[b]{Whales and even-toed mammals have different types of a bony ear structure.}

\end{frame}
}
%
\begin{frame}[t]{But, whales and terrestrial cetaceans have the same type of bony ear structure.}
	\includegraphics[width=\textwidth]{auditory_bulla_indohyus}
\end{frame}
%
\begin{frame}[t]{Terrestrial cetaceans and artiodactyls have a double-pulley ankle joint.}
	\includegraphics[width=\textwidth]{double_pulley_ankle}
	
	\vfilll
	
	\hfill \tiny \textcopyright Pearson Education, Inc.
\end{frame}
%
{
\usebackgroundtemplate{\includegraphics[width=\paperwidth]{cetacean_appendages} }
\begin{frame}[b]

\end{frame}
}
%
{
\usebackgroundtemplate{\includegraphics[width=\paperwidth]{cetacean_nostrils} }
\begin{frame}[b]{Nostril position shifted with adaption to aquatic habitat.}

\tiny \textcopyright Roberts \& Company
\end{frame}
}
%
{
\usebackgroundtemplate{\includegraphics[width=\paperwidth]{cetacean_habitat} }
\begin{frame}[b]{Their habitat shifted from fresh to salt water.}

\tiny \textcopyright Roberts \& Company
\end{frame}
}
%
\lecture{instructor}{instructor}
{
\setbeamercolor{background canvas}{bg=black}
\begin{frame}[t]
	\includegraphics[width=\textwidth]{continents_pleistocene}
	
	\vfilll
	
	\hfill \tiny \textcolor{white}{\href{http://scotese.com}{scotese.com}}
\end{frame}
}
%
\lecture{student}{student}

\begin{frame}[t]{How many species do you see?}
	\includegraphics[width=\textwidth]{hominid_skulls}
	
	\vspace*{-\baselineskip}
	\pause
	\alt<handout>{}{%
	\begin{multicols}{3}
	\begin{enumerate}[leftmargin=*,label=\textsc{\Alph*}.]
	{\tiny
	\item Chimpanzee, modern\\
	\item \textit{Australopithecus africanus}, 2.6 \textsc{mya}\\
	\item \textit{A. africanus}, 2.5 \textsc{mya}\\
	\item \textit{Homo habilis}, 1.9 \textsc{mya}\\
	\item \textit{H. habilis}, 1.8 \textsc{mya}\\
	\item \textit{H. rudolfensis}, 1.8 \textsc{mya}\\
	\item \textit{H. erectus}, 1.75 \textsc{mya}\\
	\item \textit{H. ergaster}, 1.75 \textsc{mya}\\
	\item \textit{H. heidelbergensis}, 300–125 \textsc{kya}\\
	\item \textit{H. neanderthalensis}, 70 \textsc{kya}\\
	\item \textit{H. neanderthalensis}, 60 \textsc{kya}\\
	\item \textit{H. neanderthalensis}, 45 \textsc{kya}\\
	\item \textit{H. sapiens}, Cro-Magnon, 30 \textsc{kya}\\
	\item \textit{H. sapiens}, modern}\\
	\end{enumerate}
	\end{multicols}
	}
	
\end{frame}
%
{
\usebackgroundtemplate{\includegraphics[width=\paperwidth]{human_transition} }
\begin{frame}[b]
\hfill \tiny \textcopyright Roberts \& Company
\end{frame}
}
%
\begin{frame}[t]{Hominins evolved in east central Africa nearly 4 \textsc{mya}.}
	\includegraphics[width=\textwidth]{hominid_origins}
\end{frame}
%
{
\usebackgroundtemplate{\includegraphics[width=\paperwidth]{hominid_timeline} }
\begin{frame}[b]{Four major groups of hominins show adaptive shifts.}
\hfill \tiny \textcopyright Foley and Gamble 2009. Phil. Trans. R. Soc. B 364:  3267.
\end{frame}
}
%
\begin{frame}[t]{Bipedalism evolved before large brain size}
	\includegraphics[width=\textwidth]{hominin_brain_size}

\end{frame}
%
{
\usebackgroundtemplate{\includegraphics[width=\paperwidth]{bipedalism} }
\begin{frame}[b]{What is the advantage of \highlight{bipedalism?}}
\hfill \tiny \textcopyright McGraw-Hill
\end{frame}
}
%
{
\usebackgroundtemplate{\includegraphics[width=\paperwidth]{hominid_advances} }
\begin{frame}[b]{Adaptations associated wth bipedalism and brain size increase.}
\hfill \tiny \textcopyright Foley and Gamble 2009. Phil. Trans. R. Soc. B 364:  3267.
\end{frame}
}
%
{
\begin{frame}[b]{\textit{Homo} left Africa in waves.}
	\centering
	\includegraphics[width=\textwidth]{human_dispersal}

	\vfilll

	\tiny\hfill \textcopyright McGraw-Hill.
\end{frame}
}
%
{
\usebackgroundtemplate{\includegraphics[width=\paperwidth]{mass_extinctions} }
\begin{frame}[b]{Five \highlight{mass extinctions} have occurred since the Cambrian.}
\hfill \tiny \textcopyright Pearson Education, Inc.
\end{frame}
}
%
\begin{frame}[t]{A mass extinction occurs when at least 75\% of species go extinct.}
\centering
\begin{tabular}{@{}lrr@{}}
	\toprule
	Event & Genera & Species \\
	\midrule
	End Ordovician	&	57\% & 86\% \\
	Late Devonian & 35\% & 75\% \\
	End Permian	& 59\% & 96\% \\
	End Triassic	& 47\% & 80\% \\
	End Cretaceous (\textsc{k/t}) & 40\% & 76\% \\
	\bottomrule
\end{tabular}

\end{frame}
%
\begin{frame}[t]{The causes of mass extinction all cause \highlight{rapid climate change.}}

	\begin{multicols}{2}
	\hangpara Asteroid impact
	
	\hangpara Volcanism
	
	\hangpara Glaciation
	
	\hangpara Methane gas
	
	\hangpara Changing ocean currents
	
	\hangpara Not all of these apply to every mass extinction.
	
	\columnbreak
	
	\includegraphics[width=0.4\textwidth, frame]{kt_asteroid_impact_site}\\
	{\footnotesize 180 km (112 mi) crater created by\\10 km asteroid impact.}

	\end{multicols}
	
\end{frame}
%
\end{document}
