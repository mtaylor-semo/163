%!TEX TS-program = lualatex
%!TEX encoding = UTF-8 Unicode

\documentclass[t]{beamer}

%%%% HANDOUTS For online Uncomment the following four lines for handout
%\documentclass[t,handout]{beamer}  %Use this for handouts.
%\usepackage{handoutWithNotes}
%\pgfpagesuselayout{3 on 1 with notes}[letterpaper,border shrink=5mm]
%	\setbeamercolor{background canvas}{bg=black!5}

%\includeonlylecture{student}

%%% Including only some slides for students.
%%% Uncomment the following line. For the slides,
%%% use the labels shown below the command.

%% For students, use \lecture{student}{student}
%% For mine, use \lecture{instructor}{instructor}

% FONTS
\usepackage{fontspec}
\def\mainfont{Linux Biolinum O}
\setmainfont[Ligatures={Common,TeX}, Contextuals={NoAlternate}, BoldFont={* Bold}, ItalicFont={* Italic}, Numbers={OldStyle}]{\mainfont}
\setsansfont[Ligatures={Common,TeX}, Scale=MatchLowercase, Numbers=OldStyle, BoldFont={* Bold}, ItalicFont={* Italic},]{Linux Biolinum O} 
\usepackage{microtype}

\usepackage{graphicx}
	\graphicspath{{/Users/goby/pictures/teach/163/lecture/}
	{/Users/goby/pictures/teach/common/}} % set of paths to search for images

\usepackage{multicol}
\usepackage{booktabs}
\usepackage{array}
\newcolumntype{L}[1]{>{\raggedright\let\newline\\\arraybackslash\hspace{0pt}}p{#1}}
\newcolumntype{C}[1]{>{\centering\let\newline\\\arraybackslash\hspace{0pt}}p{#1}}
\newcolumntype{R}[1]{>{\raggedleft\let\newline\\\arraybackslash\hspace{0pt}}p{#1}}
%\usepackage{textcomp}
%\usepackage{mhchem}
\usepackage{enumitem}
%\usepackage[export]{adjustbox}


\usepackage{tikz}
	\tikzstyle{every picture}+=[remember picture,overlay]
	\usetikzlibrary{arrows}
\usetikzlibrary{positioning}

\mode<presentation>
{
  \usetheme{Lecture}
  \setbeamercovered{invisible}
  \setbeamertemplate{items}[square]
}

\begin{document}

\lecture{student}{student}

\begin{frame}{Our goal for this lecture is to}

	\hangpara show how real populations grow compared to the logistic growth curve, and
	
	\hangpara learn how \highlight{negative feedback} controls \highlight{density-dependent} population growth, and
	
	\hangpara learn how \highlight{density-independent} factors control population growth.
	
\end{frame}
%
{
\usebackgroundtemplate{\includegraphics[width=\paperwidth]{real_populations_daphnia} }
\begin{frame}{Populations initially \highlight{exceed} carrying capacity.}
\end{frame}
}
%
{
\usebackgroundtemplate{\includegraphics[width=\paperwidth]{real_populations_sparrow} }
\begin{frame}{Populations \highlight{fluctuate around} carrying capacity.}
\end{frame}
}

%
\begin{frame}{}

	\hangpara \highlight{Density-dependent} population regulation occurs when high population density \highlight{decreases birth rate} or \highlight{increases death rate}.
	
	\hangpara What would happen to population size ($N$) over time?
	
\end{frame}
%
{
\usebackgroundtemplate{\includegraphics[width=\paperwidth]{density_dependent_song_sparrow} }
\begin{frame}[b]{}
\tiny\textcolor{gray!20!white}{Keith, Wikimedia, \ccby{2}.}
\end{frame}
}

{
\usebackgroundtemplate{\includegraphics[width=\paperwidth]{density_dependent_sparrow_nest} }
\begin{frame}[b]{}
\tiny\textcolor{gray!20!white}{K.P. McFarland, Flickr, \ccbync{2}.}
\end{frame}
}


\begin{frame}[b]{How does clutch size change? Why?}
	\begin{center}
		\includegraphics[height=0.8\textheight]{density_dependent_clutchsize.pdf}
	\end{center}	

	\tiny Arcese and Smith 1988, J. Animal Ecology 57: 119-136.
\end{frame}

\begin{frame}[b]{How does nest failure change? Why?}
	\begin{center}
		\includegraphics[height=0.8\textheight]{density_dependent_failure.pdf}
	\end{center}	

	\tiny Arcese and Smith 1988, J. Animal Ecology 57: 119-136.
\end{frame}


{
\usebackgroundtemplate{\includegraphics[width=\paperwidth]{density_dependent_territory} }
\begin{frame}[b]{}
\tiny\textcolor{orange5}{Foliash, Wikimedia, public domain.}
\end{frame}
}

{
\usebackgroundtemplate{\includegraphics[width=\paperwidth]{density_dependent_lynx_hare} }
\begin{frame}[b]{}
\end{frame}
}


\begin{frame}{}

	\hangpara \highlight{Density-independent} population regulation decreases birth rate or increases death rate \highlight{regardless} of current population size.
		
\end{frame}

{
\usebackgroundtemplate{\includegraphics[width=\paperwidth]{density_independent_dunes} }
\begin{frame}[b]{}
\hfill\tiny Sandy Richard, Flickr,\ccbyncsa{2} .
\end{frame}
}

{
\usebackgroundtemplate{\includegraphics[width=\paperwidth]{isle_royale} }
\begin{frame}[b]

	\hfill \tiny \textcolor{white}{Google Earth}
\end{frame}
}

{
\usebackgroundtemplate{\includegraphics[width=\paperwidth]{isle_royale_popsize} }
\begin{frame}{}
\end{frame}
}


\end{document}
