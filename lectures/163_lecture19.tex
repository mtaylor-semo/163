%!TEX TS-program = lualatex
%!TEX encoding = UTF-8 Unicode

\documentclass[t]{beamer}

%%%% HANDOUTS For online Uncomment the following four lines for handout
%\documentclass[t,handout]{beamer}  %Use this for handouts.
%\usepackage{handoutWithNotes}
%\pgfpagesuselayout{3 on 1 with notes}[letterpaper,border shrink=5mm]

%\includeonlylecture{student}

%%% Including only some slides for students.
%%% Uncomment the following line. For the slides,
%%% use the labels shown below the command.

%% For students, use \lecture{student}{student}
%% For mine, use \lecture{instructor}{instructor}

% FONTS
\usepackage{fontspec}
\def\mainfont{Linux Biolinum O}
\setmainfont[Ligatures={Common,TeX}, Contextuals={NoAlternate}, Numbers={Proportional, OldStyle}]{\mainfont}
\setsansfont[Ligatures={Common,TeX}, Scale=MatchLowercase, Numbers={Proportional,OldStyle}, BoldFont={* Bold}, ItalicFont={* Italic},]\mainfont

\newfontface\lining[Numbers={Lining}]\mainfont

\usepackage{microtype}


\mode<presentation>
{
  \usetheme{Lecture}
  \setbeamercovered{invisible}
  \setbeamertemplate{items}[default]
}



\usepackage{amsmath,amssymb}
\usepackage{unicode-math}
%\setmathfont[Scale=MatchLowercase]{TeX Gyre Pagella Math}

\usepackage{graphicx}
	\graphicspath{{/Users/goby/pictures/teach/163/lecture/}
	{/Users/goby/pictures/teach/common/}} % set of paths to search for images

\usepackage{xcolor}

\def\mygray{gray!50}

\usepackage{multicol}
\usepackage{booktabs}
\usepackage{array}
\newcolumntype{L}[1]{>{\raggedright\let\newline\\\arraybackslash\hspace{0pt}}p{#1}}
\newcolumntype{C}[1]{>{\centering\let\newline\\\arraybackslash\hspace{0pt}}p{#1}}
\newcolumntype{R}[1]{>{\raggedleft\let\newline\\\arraybackslash\hspace{0pt}}p{#1}}
%\usepackage{textcomp}
%\usepackage{mhchem}
\usepackage{enumitem}
%\usepackage[export]{adjustbox}

\usepackage{calc} %For widthof, in one slide below.
\makeatletter
\def\@hspace#1{\begingroup\setlength\dimen@{#1}\hskip\dimen@\endgroup}
\makeatother

\usepackage{pgfplots}
	\usepgfplotslibrary{colorbrewer}
	\usepgfplotslibrary{fillbetween}
	\pgfplotsset{
		% Set `compat level to 1.11 or higher so you don't need to
        % prefix every tikz coordinate with `axis cs:'
		compat=1.14,
		baseplot/.style	= {
            width				= 10cm,
            height			= 5cm,
            axis lines*		= left,
            axis on top,
            enlargelimits	= upper,
            ylabel style	= {align=right, rotate=-90},
            xlabel			= {Resource Gradient},
            ylabel			= {Relative\\Abundance},
            ticks				= none,
            no markers,
            samples		= 30,
            smooth,
		},
		3dbaseplot/.style={
    			width          	= 10cm,
			view           	= {45}{65},
			axis on top,
			enlargelimits = false,
%			domain         	= 1:4,
%			y domain       = 1:4,
			no markers,
			samples        = 30,
			xlabel         	= {Temperature},
			ylabel         	= {Seed Size},
			zlabel         	= {Relative\\Abundance},
			xlabel style	= {sloped, at={(rel axis cs:0.5,1,1)}, above, sloped like x axis},
			ylabel style	= {sloped, at={(rel axis cs:0,0.5,1)}, above, sloped like y axis},
			zlabel style	= {rotate=-90, align=right},
			ticks				= none,
%			point meta max	= 1,
			smooth,
	},
	/pgf/declare function = {
		normal(\m,\s)=1/(2*\s*sqrt(pi))*exp(-(x-\m)^2/(2*\s^2));
	},
	/pgf/declare function = {
		bivar(\ma,\sa,\mb,\sb)=
    		1/(2*pi*\sa*\sb) * exp(-((x-\ma)^2/\sa^2 + (y-\mb)^2/\sb^2))/2;
	}
}

% create a variable that stores the factor of the standard
% deviations for that the ellipses should be drawn
\pgfmathsetmacro{\factor}{3}
	
% to simplify the input, which repeats all the time, create a command
% here #1 = `name path', #2 = `\mean', #3 = `\stddev'
% the idea is that the normal values are almost zero after 4 standard
% deviations and so the `samples' can be better used in that ±4 standard
% deviation range around the mean value
% (this has the positive side effect that the lines of two neighboring
%  normal plots don't overlap in the "zero" range and thus makes it
%  much easier for TikZ/PGFPlots to identify the "real" intersections.)
\newcommand*\myaddplot[4]{
    \addplot+ [thick, name path=#1,domain=#2-\factor*#3:#2+\factor*#3, color=#4] {normal(#2,#3)};
}
\newcommand*\mydashedplot[4]{
    \addplot+ [thick, dashed, name path=#1,domain=#2-\factor*#3:#2+\factor*#3, color=#4] {normal(#2,#3)};
}
\newcommand*\myaddplotX[4]{
    \addplot3+ [name path=#1,domain=#2-\factor*#3:#2+\factor*#3, color=#4] (x,4,{normal(#2,#3)});
}
\newcommand*\myaddplotY[4]{
    \addplot3+ [name path=#1,domain=#2-\factor*#3:#2+\factor*#3, color=#4] (1,x,{normal(#2,#3)});
}

%\usepackage{tikz}
%		%tikzstyle{every picture} conflicts for some reason with the pgfplots stuff above.
%%	\tikzstyle{every picture}+=[remember picture,overlay]
%	\usetikzlibrary{arrows}
%\usetikzlibrary{positioning}

\begin{document}

\lecture{student}{student}

\begin{frame}{Our goal for this lecture is to}

	\hangpara learn about \highlight{community ecology},
	
	\hangpara understand the types of \highlight{interspecific interactions} in a community, and
	
	\hangpara show how the interactions influence evolution of the species.
	
\end{frame}
%
{
\usebackgroundtemplate{\includegraphics[width=\paperwidth]{community_ecology} }
\begin{frame}[b]{\textcolor{black}{What is a \textcolor{orange6}{community?} }}

	\tiny \textcolor{white}{Mile End Residents, Flickr \ccby{2}}

\end{frame}
}

\begin{frame}[t]{An emergent property of communities is \highlight{interspecific interactions.}}

\hangpara $- \slash -:$ Both populations are negatively affected.\\
	\hspace{\widthof{$- \slash -:$}} e.g., interspecific competition.

\hangpara $+ \slash -:$ One populations is positively affected, and\\
	 \hspace{\widthof{$- \slash -:$}} one population is negatively affected.\\
	\hspace{\widthof{$- \slash -:$}} e.g., predation.

\hangpara $+ \slash +:$ Both populations are positively affected.\\
	\hspace{\widthof{$+ \slash +:$}} e.g., mutualism.


\hangpara Substitute a {\lining0} for a $+$ or $-$ to indicate a neutral interaction.

\end{frame}
%
\begin{frame}[t]{Species use a specific range of a resource gradient.}
	\centering 
	\begin{tikzpicture}
		\begin{axis}[
			baseplot,
			set layers,
			]
    
			\myaddplot{A}{2.7}{0.4}{orange6};

		\end{axis}
		
		\node [right, text width = 10cm] at (-2,-2) {The resource gradient can be abiotic, like temperature, or biotic, like seed size.};
		
	\end{tikzpicture}
\end{frame}
%
%% BEGIN BUILDING TO SHOW HOW THE PLOTS WORK. ADD ORANGE TEMPERATURE,
% then orange soil ph.  

% Orange temperature on x axis.
\begin{frame}[t]{Species use multiple resources. \textcolor{white}{like temperature, pH, precipitation and much more.}}
    \centering

\begin{tikzpicture}[
	declare function = {orMuX=2.0;},
	declare function = {orMuY=3.2;},
	declare function = {blMuX=2.5;},
	declare function = {blMuY=2.7;},
	declare function = {sX=0.25;},
	declare function = {sY=0.15;},
	]
		\begin{axis}[
			3dbaseplot,
			colormap/OrRd,
			set layers,
			ylabel={},
			samples=30,
			clip mode=individual,
  		]
		
            \myaddplotX{B}{blMuX}{sX}{white}
            \myaddplotY{C}{blMuY}{sY}{white}
            \myaddplotY{D}{orMuY}{sY}{white}
            \myaddplotX{A}{orMuX}{sX}{orange6}
         
		% x axis
		\draw [dashed] (orMuX-\factor*sX,\pgfkeysvalueof{/pgfplots/ymin}) -- (orMuX-\factor*sX,\pgfkeysvalueof{/pgfplots/ymax});
		\draw [dashed] (orMuX+\factor*sX, \pgfkeysvalueof{/pgfplots/ymin}) -- (orMuX+\factor*sX, \pgfkeysvalueof{/pgfplots/ymax});

		\node at (xticklabel cs:0.5) [below, sloped like x axis] {Temperature};

  \end{axis}
\end{tikzpicture}

\end{frame}

% Add Soil pH
\begin{frame}[t]{Species use multiple resources. \textcolor{white}{like temperature, pH, precipitation and much more.}}

    \centering

\begin{tikzpicture}[
	declare function = {orMuX=2.0;},
	declare function = {orMuY=3.2;},
	declare function = {blMuX=2.5;},
	declare function = {blMuY=2.7;},
	declare function = {sX=0.25;},
	declare function = {sY=0.15;},
	]
		\begin{axis}[
			3dbaseplot,
			colormap/OrRd,
			set layers,
			samples=30,
			clip mode=individual,
  		]
		
            \myaddplotX{B}{blMuX}{sX}{white}
            \myaddplotY{C}{blMuY}{sY}{white}

            \myaddplotX{A}{orMuX}{sX}{orange6}
            \myaddplotY{D}{orMuY}{sY}{orange6}
         
		% x axis
		\draw [dashed] (orMuX-\factor*sX,\pgfkeysvalueof{/pgfplots/ymin}) -- (orMuX-\factor*sX,\pgfkeysvalueof{/pgfplots/ymax});
		\draw [dashed] (orMuX+\factor*sX, \pgfkeysvalueof{/pgfplots/ymin}) -- (orMuX+\factor*sX, \pgfkeysvalueof{/pgfplots/ymax});

		% y axis
		\draw [dashed] (\pgfkeysvalueof{/pgfplots/xmin}, orMuY-\factor*sY) -- (\pgfkeysvalueof{/pgfplots/xmax}, orMuY-\factor*sY);
		\draw [dashed] (\pgfkeysvalueof{/pgfplots/xmin}, orMuY+\factor*sY) -- (\pgfkeysvalueof{/pgfplots/xmax}, orMuY+\factor*sY);

		\node at (rel axis cs:1,0.5,0) [below, sloped like y axis] {Seed Size};

  \end{axis}
\end{tikzpicture}

\end{frame}

% Both together with surface plot.
\begin{frame}[t]{The \highlight{ecological niche} represents all of the biotic and abiotic resources required by a species.}

    \centering

\begin{tikzpicture}[
	declare function = {orMuX=2.0;},
	declare function = {orMuY=3.2;},
	declare function = {blMuX=2.5;},
	declare function = {blMuY=2.7;},
	declare function = {sX=0.25;},
	declare function = {sY=0.15;},
	]
		\begin{axis}[
			3dbaseplot,
			colormap/OrRd,
			set layers,
			samples=30,
  		]
		
            \myaddplotX{B}{blMuX}{sX}{white}
            \myaddplotY{C}{blMuY}{sY}{white}

            \myaddplotX{A}{orMuX}{sX}{orange6}
            \myaddplotY{D}{orMuY}{sY}{orange6}
         
		% x axis
		\draw [dashed] (orMuX-\factor*sX,\pgfkeysvalueof{/pgfplots/ymin}) -- (orMuX-\factor*sX,\pgfkeysvalueof{/pgfplots/ymax});
		\draw [dashed] (orMuX+\factor*sX, \pgfkeysvalueof{/pgfplots/ymin}) -- (orMuX+\factor*sX, \pgfkeysvalueof{/pgfplots/ymax});

		% y axis
		\draw [dashed] (\pgfkeysvalueof{/pgfplots/xmin}, orMuY-\factor*sY) -- (\pgfkeysvalueof{/pgfplots/xmax}, orMuY-\factor*sY);
		\draw [dashed] (\pgfkeysvalueof{/pgfplots/xmin}, orMuY+\factor*sY) -- (\pgfkeysvalueof{/pgfplots/xmax}, orMuY+\factor*sY);

	\addplot3 [
    		surf,
		domain = orMuX - \factor*sX:orMuX + \factor*sX, 
		domain y = orMuY-\factor*sY:orMuY + \factor*sY,
		opacity=0.5,
		colormap/Oranges] 
		{bivar(orMuX, sX, orMuY, sY)};
		
  \end{axis}
\end{tikzpicture}

\end{frame}
%


\begin{frame}[t]{\highlight{Interspecific competition} can occur when two or more species require the same part of a resource gradient.}
	\centering
	\begin{tikzpicture}
		\begin{axis}[
			baseplot,
			set layers,
			]
    
			\myaddplot{A}{2.9}{0.4}{orange6};
			\myaddplot{B}{3.1}{0.4}{blue7};
      
			\pgfonlayer{pre main}
				\fill[\mygray, intersection segments={of= B and A}];
			\endpgfonlayer
		\end{axis}

		\node [right, text width = 10cm] at (-2,-2) {Interspecific competition $(- / -)$ is strong because both species share most of the  resource gradient (shaded area).};
	\end{tikzpicture}
\end{frame}
%
{
\usebackgroundtemplate{\includegraphics[width=\paperwidth]{competition_coral_reef} }
\begin{frame}[b]{\highlight{Interspecific competition} occurs when species use the same \textit{limited} resource.}

	\hfill \tiny \textcolor{white}{Toby Hudson, Wikimedia \ccbysa{3}}

\end{frame}
}
%


% Add second species, lots of niche overlap, strong competition.
\begin{frame}[t]{\highlight{Competitive exclusion} means that two species with the same niche cannot coexist in a community.}

    \centering

\begin{tikzpicture}[
	declare function = {orMuX=2.0;},
	declare function = {orMuY=3.2;},
	declare function = {blMuX=2.1;},
	declare function = {blMuY=3.1;},
	declare function = {sX=0.25;},
	declare function = {sY=0.15;},
	]
		\begin{axis}[
			3dbaseplot,
			colormap/OrRd,
			set layers,
			samples=30,
  		]
		
            \myaddplotX{E}{2.5}{sX}{white}
            \myaddplotY{F}{2.7}{sY}{white}
         
            \myaddplotX{A}{orMuX}{sX}{orange6}
            \myaddplotX{B}{blMuX}{sX}{blue7}
            \myaddplotY{D}{orMuY}{sY}{orange6}
            \myaddplotY{C}{blMuY}{sY}{blue7}

	\addplot3 [
    		surf,
		domain=orMuX - \factor*0.25:orMuX + \factor*0.25, 
		domain y = orMuY - \factor*0.15:orMuY + \factor*0.15,
		opacity=0.5,
		colormap/Oranges] 
		{bivar(orMuX, sX, orMuY, sY)};
		
	\addplot3 [
		surf, 
		domain=blMuX - \factor*0.25:blMuX + \factor*0.25, 
		domain y = blMuY - \factor*0.15:blMuY + \factor*0.15,
		opacity=0.3,
		colormap/Blues] 
		{bivar(blMuX, sX, blMuY, sY)};

  \end{axis}
\end{tikzpicture}

\end{frame}
%
% Add second species, minimal niche overlap.
\begin{frame}[t]{Species can coexist when they use at least some resources differently from each other.}

    \centering

\begin{tikzpicture}[
	declare function = {orMuX=2.0;},
	declare function = {orMuY=3.2;},
	declare function = {blMuX=2.1;},
	declare function = {blMuY=2.6;},
	declare function = {sX=0.25;},
	declare function = {sY=0.15;},
	]
		\begin{axis}[
			3dbaseplot,
			colormap/OrRd,
			set layers,
			samples=30,
  		]
%		\pgfmathsetmacro{\factor}{2.5}
		
            \myaddplotX{E}{2.5}{sX}{white}
            \myaddplotY{F}{2.7}{sY}{white}

            \myaddplotX{B}{blMuX}{sX}{blue7}
            \myaddplotY{C}{blMuY}{sY}{blue7}
            
            \myaddplotX{A}{orMuX}{sX}{orange6}
            \myaddplotY{D}{orMuY}{sY}{orange6}
         
	\addplot3 [
    		surf,
		domain=orMuX - \factor*sX:orMuX + \factor*sX, 
		domain y = orMuY-\factor*sY:orMuY+\factor*sY,
		opacity=0.5,
		colormap/Oranges] 
		{bivar(orMuX,sX,orMuY,sY)};
		
	\addplot3 [
		surf, 
		domain=blMuX-\factor*sX:blMuX+\factor*sX, 
		domain y = blMuY-\factor*sY:blMuY+\factor*sY,
		opacity=0.5,
		colormap/Blues] 
		{bivar(blMuX, sX, blMuY, sY)};


  \end{axis}
\end{tikzpicture}

\end{frame}
%
\begin{frame}[t]{More species can coexist if they specialize on different parts of the resource gradient.}
	\centering 
	\begin{tikzpicture}
		\begin{axis}[
			baseplot,
			set layers,
			]
            \myaddplot{A}{1.7}{0.1}{orange6}
            \myaddplot{B}{2.2}{0.1}{blue7}
            \myaddplot{C}{2.7}{0.1}{magenta}
            \myaddplot{D}{3.2}{0.1}{black}
            \myaddplot{E}{3.7}{0.1}{green6}

            \pgfonlayer{pre main}
                \fill[\mygray, intersection segments={of=B and A}];
                \fill[\mygray, intersection segments={of=C and B}];
                \fill[\mygray, intersection segments={of=D and C}];
                \fill[\mygray, intersection segments={of=E and D}];
            \endpgfonlayer
        \end{axis}

	\node [right, text width = 10cm] at (-2,-2)  {Interspecific competition is minimized because species do not share most of the resource};
	\end{tikzpicture}
\end{frame}

%
{
\usebackgroundtemplate{\includegraphics[width=\paperwidth]{resource_partitioning} }
\begin{frame}[b]{\highlight{Resource partitioning} allows more species to coexist in the community.}
	\hfill \tiny Fig~54.2 \copyright\,Pearson Education, Inc.
\end{frame}
}

%%
\begin{frame}[t]{The \highlight{fundamental niche} represents the resources that can be \emph{potentially} used  by a species.}

	\vspace*{\baselineskip}
	
	\centering 
	\begin{tikzpicture}
		\begin{axis}[
			baseplot,
			]
    
			\myaddplot{A}{2.7}{0.3}{orange6};
      
		\end{axis}

	\end{tikzpicture}
\end{frame}
%
\begin{frame}[t]{The \highlight{realized niche} represents the resources that are \emph{actually} used  by a species in the community.}

	\vspace*{\baselineskip}

	\centering 
	\begin{tikzpicture}
		\begin{axis}[
			baseplot,
			set layers,
			]
    
            \myaddplot{B}{2.2}{0.11}{blue7}
            \myaddplot{C}{2.7}{0.11}{orange6}
            \myaddplot{D}{3.2}{0.11}{magenta}

            \pgfonlayer{pre main}
                \fill[\mygray, intersection segments={of=C and B}];
                \fill[\mygray, intersection segments={of=D and C}];
            \endpgfonlayer
      
		\end{axis}

		\begin{axis}[
			baseplot,
			]
			\mydashedplot{Z}{2.7}{0.3}{gray};
		\end{axis}		

		
		\node [right, gray] (fundamental) at (5.5,4) {Fundamental niche};

		\draw [gray, thick, ->] (fundamental.west) -- (4.9,2.4);
		
		\node [left, orange6] (realized) at (2.5,4) {Realized niche};
		
		\draw [orange6, thick, ->] (realized.east) -- (3.4,2.6);
		

	\end{tikzpicture}
\end{frame}
{
\usebackgroundtemplate{\includegraphics[width=\paperwidth]{competition_ecological_niche} }
\begin{frame}[b]{Smaller realized niches allow more species to coexist in the community.}

	\tiny Fig.~54.3~\textcopyright\,Pearson Education, Inc.
\end{frame}
}
%
{
\usebackgroundtemplate{\includegraphics[width=\paperwidth]{competition_character_displacement} }
\begin{frame}[b]{\highlight{Character displacement} in sympatry reduces competition.}

	\hfill \tiny Fig.~54.4~\textcopyright\,Pearson Education, Inc.
\end{frame}
}
%
\lecture{instructor}{instructor}

{
\usebackgroundtemplate{\includegraphics[width=\paperwidth]{darters_of_illinois} }
\begin{frame}[b]

	\hfill \tiny \copyright\,The North American Native Fishes Association
\end{frame}
}
%
\lecture{student}{student}

\begin{frame}[t]{More darter species in a stream increase character displacement.}

	\vspace*{-\baselineskip}

	\begin{multicols}{2}
		{\centering
		\includegraphics[height=0.75\textheight]{character_displacement_darters_graph}
		
		}

	\columnbreak
	
			\includegraphics[height=0.75\textheight]{character_displacement_darters}
			
\end{multicols}

\vfilll

\hfill \tiny Knouft 2003. Evolution 57: 2374.
\end{frame}
%
{
%\usebackgroundtemplate{\includegraphics[width=\paperwidth]{predator_prey_heron} }
\usebackgroundtemplate{\includegraphics[width=\paperwidth]{predator_prey_evolution} }
\begin{frame}[b]{\highlight{Predation} $(+ / -)$ affects evolution of predators and prey.}

	\hfill \tiny Marie and Allistair Knock, Flickr \ccbyncsa{2}.
%	\tiny\textcolor{white}{Chad E., Wikimedia \ccbysa{3}}
\end{frame}
}
%
{
\usebackgroundtemplate{\includegraphics[width=\paperwidth]{aposematic_grasshopper} }
\begin{frame}[b]{}
\tiny \textcolor{white}{MrClean1982, Flickr \ccbync{2}}
\end{frame}
}
%
{
\usebackgroundtemplate{\includegraphics[width=\paperwidth]{mimicry} }
\begin{frame}[b]{}
\end{frame}
}
%
\lecture{instructor}{instructor}

{\setbeamercolor{background canvas}{bg=black}
\begin{frame}{}
\centering
	\includegraphics[height=\textheight]{mimicry_rhagoletis}
\end{frame}
}
%
{\setbeamercolor{background canvas}{bg=black}
\begin{frame}{}
\centering
	\includegraphics[width=\textwidth]{mimicry_Goniurellia_tridens}
\end{frame}
}
%
{
\setbeamercolor{background canvas}{bg=black}
\begin{frame}{}
\centering
	\includegraphics[width=\textwidth]{mimicry_moth_spider_legs}
\end{frame}
}
%
{
\usebackgroundtemplate{\includegraphics[width=\paperwidth]{mimicry_cleaner_wrasse} }
\begin{frame}[b]{}
\tiny\textcolor{white}{Brocken Inaglory, Wikimedia \ccbysa{3}}
\end{frame}
}
%
{
\usebackgroundtemplate{\includegraphics[width=\paperwidth]{mimicry_cleaner_mimic} }
\begin{frame}[b]

	\hfill \tiny\textcolor{white}{Jenny Huang, Wikimedia \ccby{2}}
\end{frame}
}

%
\lecture{student}{student}
{
\usebackgroundtemplate{\includegraphics[width=\paperwidth]{camoflage_frog1} }
\begin{frame}{}

\vspace{4\baselineskip}

\hspace{64mm}\hangpara \Large\textcolor{orange5}{Cryptic coloration}

\hspace{60mm}\hangpara \large\textcolor{white}{Where am I? What am I?}
\end{frame}
}
%
\lecture{instructor}{instructor}
{
\usebackgroundtemplate{\includegraphics[width=\paperwidth]{camoflage_frog2} }
\begin{frame}{}
\end{frame}
}
%
{
\usebackgroundtemplate{\includegraphics[width=\paperwidth]{camoflage_frogmouth} }
\begin{frame}{}
\end{frame}
}
%
{
\usebackgroundtemplate{\includegraphics[width=\paperwidth]{camoflage_nightjar} }
\begin{frame}{}
\end{frame}
}
%
{
\usebackgroundtemplate{\includegraphics[width=\paperwidth]{camoflage_ptarmigan} }
\begin{frame}{}
\end{frame}
}
%
{
\usebackgroundtemplate{\includegraphics[width=\paperwidth]{camoflage_pika} }
\begin{frame}{}
\end{frame}
}
%
{
\setbeamercolor{background canvas}{bg=black}
\begin{frame}{}
\centering
	\includegraphics[width=\textwidth]{camoflage_orchid_mantis1}
\end{frame}
}
%
{
\setbeamercolor{background canvas}{bg=black}
\begin{frame}{}
\centering
	\includegraphics[width=\textwidth]{camoflage_orchid_mantis2}
\end{frame}
}
%
{
\usebackgroundtemplate{\includegraphics[width=\paperwidth]{camoflage_giraffe} }
\begin{frame}{}
\end{frame}
}
%
{
\usebackgroundtemplate{\includegraphics[width=\paperwidth]{camoflage_owl} }
\begin{frame}{}
\end{frame}
}
%
{
\usebackgroundtemplate{\includegraphics[width=\paperwidth]{camoflage_leopard} }
\begin{frame}{}
\end{frame}
}
%
{
\usebackgroundtemplate{\includegraphics[width=\paperwidth]{camoflage_parrot} }
\begin{frame}{}
\end{frame}
}
%
{
\usebackgroundtemplate{\includegraphics[width=\paperwidth]{camoflage_wire_coral_shrimp} }
\begin{frame}{}
\end{frame}
}
%
{
\usebackgroundtemplate{\includegraphics[width=\paperwidth]{camoflage_leaftail_gecko} }
\begin{frame}{}
\end{frame}
}
%
{
\usebackgroundtemplate{\includegraphics[width=\paperwidth]{camoflage_bittern1} }
\begin{frame}{}
\end{frame}
}
%
{
\usebackgroundtemplate{\includegraphics[width=\paperwidth]{camoflage_bittern2} }
\begin{frame}{}
\end{frame}
}
%
{
\usebackgroundtemplate{\includegraphics[width=\paperwidth]{camoflage_bittern3} }
\begin{frame}{}
\end{frame}
}
%
{
\usebackgroundtemplate{\includegraphics[width=\paperwidth]{camoflage_bittern1} }
\begin{frame}{}
\end{frame}
}
%
\lecture{student}{student}

{
\usebackgroundtemplate{\includegraphics[width=\paperwidth]{herbivory} }
\begin{frame}[b]{\textcolor{white}{Plants evolve defenses against \textcolor{orange4}{herbivory} $(+ / -)$.}}

	\hfill\tiny\textcolor{white}{Olgierd Rudak, Wikimedia \ccby{2}}

\end{frame}
}
%
\begin{frame}{\highlight{Symbiosis} is a long-term ecological interaction between two species.}

	\hangpara Mutualism: $ + / + $ 
	
	\hangpara Commensalism: $ + / 0$
	
	\hangpara Parasitism: $ + / - $ 
		
\end{frame}

%
{
\usebackgroundtemplate{\includegraphics[width=\paperwidth]{mutualism} }
\begin{frame}[b]{}
\hfill\tiny\textcolor{white}{Kevin  Bryant, Flickr \ccbyncsa{2}}
\end{frame}
}
%
{
\usebackgroundtemplate{\includegraphics[width=\paperwidth]{commensalism} }
\begin{frame}[t]

\vspace*{70mm}

\hfill \tiny \textcolor{white}{\copyright\,Dave Burdick}

\vfilll

\tiny \textcolor{white}{James St. John, Flickr \ccby{2}} \hfill \copyright\,James Van Tassell
\end{frame}
}
%
{
\usebackgroundtemplate{\includegraphics[width=\paperwidth]{parasitism} }
\begin{frame}[b]{}
\hfill\tiny\textcolor{white}{edgeplot, Flickr \ccbyncsa{2}}
\end{frame}
}
%
{
\usebackgroundtemplate{\includegraphics[width=\paperwidth]{parasitism_carpenteria1} }
\begin{frame}[t]{Carpenteria Marsh, California}
\end{frame}
}
%
{
\usebackgroundtemplate{\includegraphics[width=\paperwidth]{parasitism_carpenteria2} }
\begin{frame}[t]{Parasites can influence an entire community.}
\end{frame}
}
%
{
\usebackgroundtemplate{\includegraphics[width=\paperwidth]{parasitism_anglerfish} }
\begin{frame}[b]{}
\hfill\tiny\textcolor{white}{\textcopyright\,Theodore W. Pietsch, University of Washington.}
\end{frame}
}
%
%%
\end{document}
