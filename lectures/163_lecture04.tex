%!TEX TS-program = lualatex
%!TEX encoding = UTF-8 Unicode

\documentclass[t]{beamer}

%%%% HANDOUTS For online Uncomment the following four lines for handout
%\documentclass[t,handout]{beamer}  %Use this for handouts.
%\usepackage{handoutWithNotes}
%\pgfpagesuselayout{3 on 1 with notes}[letterpaper,border shrink=5mm]

%\includeonlylecture{student}

%%% Including only some slides for students.
%%% Uncomment the following line. For the slides,
%%% use the labels shown below the command.

%% For students, use \lecture{student}{student}
%% For mine, use \lecture{instructor}{instructor}


%\usepackage{pgf,pgfpages}
%\pgfpagesuselayout{4 on 1}[letterpaper,border shrink=5mm]

% FONTS
\usepackage{fontspec}
\def\mainfont{Linux Biolinum O}
\setmainfont[Ligatures={Common,TeX}, Contextuals={NoAlternate}, BoldFont={* Bold}, ItalicFont={* Italic}, Numbers={Proportional}]{\mainfont}
\setsansfont[Scale=MatchLowercase]{Linux Biolinum O} 
\usepackage{microtype}

\usepackage{graphicx}
	\graphicspath{{/Users/goby/pictures/teach/163/lecture/}
	{/Users/goby/pictures/teach/common/}} % set of paths to search for images

%\usepackage{units}
\usepackage{booktabs}
\usepackage{multicol}
%\usepackage{textcomp}

\usepackage{tikz}
%	\tikzstyle{every picture}+=[remember picture,overlay]

\mode<presentation>
{
  \usetheme{Lecture}
  \setbeamercovered{invisible}
  \setbeamertemplate{items}[square]
}

%\usefonttheme[onlymath]{serif}
%\usecolortheme[named=blue7]{structure}

\newcommand{\btVFill}{\vskip0pt plus 1filll}

\newcommand\HiddenWord[1]{%
	\alt<handout>{\rule{\widthof{#1}}{\fboxrule}}{#1}%
}

\newcommand\GrayedOut[1]{%
	\alt<handout>{#1}{\textcolor{gray}{#1}}%
}


\begin{document}

\lecture{student}{student}

\begin{frame}[t]{Our goals for this lecture are to}

	\hangpara explore the details of hypothesis testing, and.
	
	\hangpara discuss types of ideas.
		
\end{frame}

\begin{frame}[t]{Successful experiments depend on a well-formed hypothesis.}

	\hangpara Observation: some aspect of the natural world that interests you.

	\hangpara \highlight{Hypothesis:} A plausible explanation, taking into account what is already known.
	
	\hangpara Prediction: Tells what to expect if you do the experiment.\\ “\textit{If} I do X, \textit{then} I will see Y” is a prediction.
	
	\hangpara Experiment and results: Set up the conditions to test the prediction.
	
	\hangpara Conclusion: The results \textit{support} or \textit{falsify} the hypothesis, or are inconclusive.

\end{frame}

{
\usebackgroundtemplate{\includegraphics[width=\paperwidth]{hypothesis_housefly_tasting}}
\begin{frame}[b,plain]{}
\tiny \hspace*{60mm} \href{https://www.shutterstock.com/video/clip-3173380-stock-footage-house-fly-insect-on-food-table-walking-eating-cleaning-his-hands-leaving-dirt-and-disease.html}{Walking} | \href{https://youtu.be/N23E4jYTExk}{Feeding} \hfill Richard Bartz, Wikimedia Commons.
\end{frame}
}

\lecture{instructor}{instructor}
{
\setbeamercolor{background canvas}{bg=black}
\begin{frame}[b,plain]{}
\end{frame}
}

\lecture{student}{student}

{
\usebackgroundtemplate{\includegraphics[width=\paperwidth]{hypothesis_fly_feet}}
\begin{frame}[b,plain]{}
\hfill \tiny Coffee, Pixabay, Public Domain.
\end{frame}
}

{
\usebackgroundtemplate{\includegraphics[width=\paperwidth]{hypothesis_null_alternative}}
\begin{frame}[b,plain]{}
\tiny Fredrik Andreasson, Flickr CC by 2.0.
\end{frame}
}

\lecture{instructor}{instructor}
{
\usebackgroundtemplate{\includegraphics[width=\paperwidth]{hypothesis_fly_experiment1}}
\begin{frame}[b]

\hfill\tiny\textcolor{white}{USGS Bee Inventory and Monitoring, Flickr Public Domain.}
\end{frame}
}

{
\usebackgroundtemplate{\includegraphics[width=\paperwidth]{hypothesis_fly_experiment2}}
\begin{frame}[b]

\hfill\tiny\textcolor{white}{USGS Bee Inventory and Monitoring, Flickr Public Domain.}
\end{frame}
}

{
\usebackgroundtemplate{\includegraphics[width=\paperwidth]{hypothesis_fly_experiment3}}
\begin{frame}[b]

\hfill\tiny\textcolor{white}{USGS Bee Inventory and Monitoring, Flickr Public Domain.}
\end{frame}
}

\lecture{student}{student}

{
\usebackgroundtemplate{\includegraphics[width=\paperwidth]{hypothesis_fly_experiment4}}
\begin{frame}[b]

\hfill\tiny\textcolor{white}{USGS Bee Inventory and Monitoring, Flickr Public Domain.}
\end{frame}
}


\lecture{instructor}{instructor}

{
\usebackgroundtemplate{\includegraphics[width=\paperwidth]{plot_twist}}
\begin{frame}[b,plain]{}
\end{frame}
}

\lecture{student}{student}

{
\usebackgroundtemplate{\includegraphics[width=\paperwidth]{hypothesis_bristles_taste}}
\begin{frame}[b]

\hfill\tiny Katja Schulz, Wikimedia Commons.
\end{frame}
}

{
\usebackgroundtemplate{\includegraphics[width=\paperwidth]{hypothesis_all_possible}}
\begin{frame}[b]{The data can be consistent with multiple hypotheses.}

\end{frame}
}


\begin{frame}[t]{Because it would take an infinite amount of data to rule out all possible hypotheses,}

\hangpara You can \emph{never} prove a hypothesis.

\hangpara You can only \highlight{support} or \highlight{falsifiy} a hypothesis.

\end{frame}


\lecture{instructor}{instructor}

\begin{frame}[t]{Good hypotheses share important characteristics.}

\hangpara They must be falsifiable. Not false—\emph{falsifiable}.

\end{frame}

\lecture{student}{student}

{
\usebackgroundtemplate{\includegraphics[width=\paperwidth]{hypothesis_flying_teapot}}
\begin{frame}[b]

\end{frame}
}

%
\begin{frame}[t]{Good hypotheses share important characteristics.}

\hangpara They must be falsifiable. Not false—\emph{falsifiable}.


\hangpara They must make predictions about something you do not already know.
\pause

\hangpara It must be conceivable that the predictions are wrong. 
\pause

\hangpara If the predictions disagree with the data, then the hypothesis is falsified.

\end{frame}

\begin{frame}[t,plain]{Work in pairs to define the following terms.}
	\vspace{2ex}

	\onslide<1->\hangpara\highlight{Belief}\onslide<2->\alt<handout>{}{\highlight{:} acceptance of, or confidence in, an alleged fact or body of facts as true or right without positive knowledge or proof.}

	\onslide<1->\hangpara\highlight{Fact}\onslide<3->\alt<handout>{}{\highlight{:} a truth known by actual experience or observation.}

	\onslide<4->\hangpara\highlight{Hypothesis}\onslide<5->\alt<handout>{}{\highlight{:} a tentative explanation for an observation, phenomenon, or scientific problem that can be tested by further investigation.}
	
	\onslide<4->\hangpara\highlight{Theory}\onslide<6->\alt<handout>{}{\highlight{:} A theory in technical use is a more or less verified or established explanation accounting for known facts or phenomena.}

\end{frame}
%
{
\usebackgroundtemplate{\includegraphics[width=\paperwidth]{sjgould_background}}
\begin{frame}[t]{Theories explain facts.}


	\hangpara\highlight{Fact:} “\ldots confirmed to such a degree that it would be perverse to withhold provisional consent.”
	
	\hangpara\highlight{Theory:} “\ldots structures of ideas that explain and interpret facts.”
	
	\hangpara\hspace*{60mm}\parbox[t]{2in}{\raggedright —Stephen Jay Gould, Evolution as Fact and Theory, \textit{Discover}, May 1981.}

\end{frame}
}

%
\begin{frame}[t]{Facts are the observations, theories are the natural processes or mechanisms that explain the facts.}


	\hangpara Gravity is a fact, explained by the General Theory of Relativity.
	\pause

	\hangpara The sun is the center of the solar system (fact), explained by the Heliocentric Theory.
	\pause
		
	\hangpara Diseases are caused by various microbial organisms (fact), explained by the Germ Theory of Disease.
	\pause

	\hangpara Evolution is a fact, explained by the Theory of Natural Selection and other related theories.

\end{frame}


\end{document}
