%!TEX TS-program = lualatex
%!TEX encoding = UTF-8 Unicode

\documentclass[t]{beamer}

%%%% HANDOUTS For online Uncomment the following four lines for handout
%\documentclass[t,handout]{beamer}  %Use this for handouts.
%\usepackage{handoutWithNotes}
%\pgfpagesuselayout{3 on 1 with notes}[letterpaper,border shrink=5mm]

%\includeonlylecture{student}

%%% Including only some slides for students.
%%% Uncomment the following line. For the slides,
%%% use the labels shown below the command.

%% For students, use \lecture{student}{student}
%% For mine, use \lecture{instructor}{instructor}

% FONTS
\usepackage{fontspec}
\def\mainfont{Linux Biolinum O}
\setmainfont[Ligatures={Common,TeX}, Contextuals={NoAlternate}, Numbers={Proportional, OldStyle}]{\mainfont}
\setsansfont[Ligatures={Common,TeX}, Scale=MatchLowercase, Numbers={Proportional,OldStyle}, BoldFont={* Bold}, ItalicFont={* Italic},]\mainfont

\newfontface\lining[Numbers={Lining}]\mainfont

\usepackage{microtype}


\mode<presentation>
{
  \usetheme{Lecture}
  \setbeamercovered{invisible}
  \setbeamertemplate{items}[default]
}



\usepackage{amsmath,amssymb}
\usepackage{unicode-math}
%\setmathfont[Scale=MatchLowercase]{TeX Gyre Pagella Math}

\usepackage{graphicx}
	\graphicspath{{/Users/goby/pictures/teach/163/lecture/}
	{/Users/goby/pictures/teach/common/}} % set of paths to search for images

\usepackage{xcolor}

\usepackage{multicol}
\usepackage{booktabs}
\usepackage{array}
\newcolumntype{L}[1]{>{\raggedright\let\newline\\\arraybackslash\hspace{0pt}}p{#1}}
\newcolumntype{C}[1]{>{\centering\let\newline\\\arraybackslash\hspace{0pt}}p{#1}}
\newcolumntype{R}[1]{>{\raggedleft\let\newline\\\arraybackslash\hspace{0pt}}p{#1}}
%\usepackage{textcomp}
%\usepackage{mhchem}
\usepackage{enumitem}
%\usepackage[export]{adjustbox}

\usepackage{calc} %For widthof, in one slide below.
\makeatletter
\def\@hspace#1{\begingroup\setlength\dimen@{#1}\hskip\dimen@\endgroup}
\makeatother


\usepackage{tikz}
%		%tikzstyle{every picture} conflicts for some reason with the pgfplots stuff above.
	\tikzstyle{every picture}+=[remember picture,overlay]
	\usetikzlibrary{arrows}
%\usetikzlibrary{positioning}

% Use the to temporarily set a background grid for positioning.
%\setbeamertemplate{background}[grid][step=1em]

\begin{document}

%\lecture{instructor}{instructor}
%
%\begin{frame}[t]{$\overline{Y} = 74.4, s = 11.9.$}
%	\includegraphics[width=\textwidth]{163exam3grades}\par
%	
%\end{frame}
%

\lecture{student}{student}

\begin{frame}{Our goals for this lecture (and probably the next) are to}

	\hangpara learn how predator and prey species form \highlight{food webs},

	\hangpara reduce food webs to \highlight{trophic levels}, 
	
	\hangpara understand how \highlight{energy flows} through, 
	
	\hangpara \highlight{nutrients recycle} within ecosystems, and
	
	\hangpara learn what determines the distribution of different ecosystems.
	
	\vfilll
	
	\hfill \tiny Whew!
	
\end{frame}
%
{
\usebackgroundtemplate{\includegraphics[width=\paperwidth]{arctic_food_web} }
\begin{frame}{}
\hspace{65mm}\begin{minipage}{0.45\textwidth}
	\flushleft
	\vspace{4\baselineskip}

	\highlight{Food webs} show relationships among predators and prey.
	\vspace{\baselineskip}
	
	They can also shows energy and nutrients move among organisms in an ecosystem.
\end{minipage}

	\vfilll
	
	\hfill \tiny Fig.~54.15 \copyright\,Pearson Education, Inc.
\end{frame}
}
%
{
\usebackgroundtemplate{\includegraphics[width=\paperwidth]{ecosystem_desert_spring} }
\begin{frame}[b]{\textcolor{black}{What is an \textcolor{white}{ecosystem?} }}

\tiny \textcolor{white}{Fig.~55.2 \copyright\,Pearson Education, Inc.}
\end{frame}
}
%
{
\usebackgroundtemplate{\includegraphics[width=\paperwidth]{sunflowers.jpg} }
\begin{frame}[b]

	\hfill \tiny \textcolor{white}{Trey Ratcliff, Flickr \ccbyncsa{2}}
\end{frame}
}
%
{
\begin{frame}[t]{\highlight{Primary production} is the amount of energy converted to organic matter.}

	\vspace*{-0.5\baselineskip}
	
	\includegraphics[width=\textwidth]{primary_productivity}\\[\baselineskip]
	
	Primary production can be measured as gross (\textsc{gpp}) or net (\textsc{npp}).

	\vfilll
	
	\hfill \tiny Fig.~55.6 \copyright\,Pearson Education, Inc.
\end{frame}
}
%
{
\usebackgroundtemplate{\includegraphics[width=\paperwidth]{energy_flow} }
\begin{frame}[b]

	\hfill \tiny Fig.~55.4 \copyright\,Pearson Education, Inc.

\end{frame}
}
%
{
\usebackgroundtemplate{\includegraphics[width=\paperwidth]{arctic_food_web} }
\begin{frame}{}
\hspace{65mm}\begin{minipage}{0.45\textwidth}
	\flushleft
	\vspace{4\baselineskip}
	Food webs can be reduced to food chains that show \highlight{trophic levels.}
\end{minipage}
\end{frame}
}
{
\usebackgroundtemplate{\includegraphics[width=\paperwidth]{trophic_food_chain} }
\begin{frame}[b]

	\hfill \tiny Fig.~54.14 \copyright\,Pearson Education, Inc.

\end{frame}
}
%
{
\usebackgroundtemplate{\includegraphics[width=\paperwidth]{trophic_efficiency} }
\begin{frame}[b]

	\hfill \tiny Fig.~55.11 \copyright\,Pearson Education, Inc.
\end{frame}
}
%
{
\usebackgroundtemplate{\includegraphics[width=\paperwidth]{energy_partitioning} }
\begin{frame}[b]{Where does the energy go?}

	\hfill \tiny Fig.~55.10 \copyright\,Pearson Education, Inc.
\end{frame}
}
%
{
\usebackgroundtemplate{\includegraphics[width=\paperwidth]{nutrient_cycling} }
\begin{frame}[b]

	\begin{tikzpicture}
	
		\node [right] at (0em, 1.5em) {\highlight{What are nutrients?}};
		
	\end{tikzpicture}

	\hfill \tiny Fig.~55.4 \copyright\,Pearson Education, Inc.

\end{frame}
}
%
{
\usebackgroundtemplate{\includegraphics[width=\paperwidth]{decomposers} }
\begin{frame}[b]{\highlight{Decomposers} return nutrients from dead organisms to the ecosystem.}

	\hfill \tiny \textcolor{white}{TimVickers, Wikimedia, public domain}
\end{frame}
}
%

{
\usebackgroundtemplate{\includegraphics[width=\paperwidth]{decomposers_in_perspective} }
\begin{frame}[t]{}

	\vspace*{3\baselineskip}
	
	\hspace*{70mm}\parbox{45mm}{\raggedright Decomposers span trophic levels.}
	
	\bigskip
	
	\hspace*{70mm}\parbox{45mm}{\raggedright Decomposers are consumed by other organisms.}
		
	\bigskip
	
	\hspace*{70mm}\parbox{46mm}{\raggedright Decomposers convert organic molecules to inorganic molecules that can be used by primary producers.}

	\vfilll
	
	\hfill \tiny \copyright\,McGraw-Hill, Inc.
\end{frame}
}
%
{
\usebackgroundtemplate{\includegraphics[width=\paperwidth]{nitrogen_cycle} }
\begin{frame}[b]

	\hfill \tiny Fig.~55.14d \copyright\,Pearson Education, Inc.
\end{frame}
}
%
{
\usebackgroundtemplate{\includegraphics[width=\paperwidth]{water_cycle} }
\begin{frame}[b]

	\hfill \tiny Fig.~55.14a \copyright\,Pearson Education, Inc.
\end{frame}
}
%
{
\usebackgroundtemplate{\includegraphics[width=\paperwidth]{clear_cutting_nutrient_loss} }
\begin{frame}[b]

	\hfill \tiny Fig.~55.16 \copyright\,Pearson Education, Inc.
\end{frame}
}
{
\usebackgroundtemplate{\includegraphics[width=\paperwidth]{ecosystem_distribution} }
\begin{frame}[b]{The distribution of terrestrial ecosystems is determined by \highlight{climate}.}

\end{frame}
}
%
{
\usebackgroundtemplate{\includegraphics[width=\paperwidth]{climograph} }
\begin{frame}[b]{Climate is determined by mean annual temperature and precipitation.}

\end{frame}
}
%
\begin{frame}[t]{Mountains create \highlight{rain shadows} that affect regional climate.}
	
	\bigskip
	
	\includegraphics[width=\textwidth]{rain_shadow}
	
	\vfilll
	
	\hfill \tiny Fig.~52.6 \copyright\,Pearson Education, Inc.
\end{frame}
%
{
\usebackgroundtemplate{\includegraphics[width=\paperwidth]{precipitation_national} }
\begin{frame}[b]

\end{frame}
}

%%
\end{document}
