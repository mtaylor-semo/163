%!TEX TS-program = lualatex
%!TEX encoding = UTF-8 Unicode

\documentclass[t]{beamer}

%%%% HANDOUTS For online Uncomment the following four lines for handout
%\documentclass[t,handout]{beamer}  %Use this for handouts.
%\usepackage{handoutWithNotes}
%\pgfpagesuselayout{3 on 1 with notes}[letterpaper,border shrink=5mm]

%\includeonlylecture{student}

%%% Including only some slides for students.
%%% Uncomment the following line. For the slides,
%%% use the labels shown below the command.

%% For students, use \lecture{student}{student}
%% For mine, use \lecture{instructor}{instructor}


%\usepackage{pgf,pgfpages}
%\pgfpagesuselayout{4 on 1}[letterpaper,border shrink=5mm]

% FONTS
\usepackage{fontspec}
\def\mainfont{Linux Biolinum O}
\setmainfont[Ligatures={Common,TeX}, Contextuals={NoAlternate}, BoldFont={* Bold}, ItalicFont={* Italic}, Numbers={OldStyle}]{\mainfont}
\setsansfont[Ligatures={Common,TeX}, Scale=MatchLowercase, Numbers=OldStyle]{Linux Biolinum O} 
\usepackage{microtype}

\usepackage{graphicx}
	\graphicspath{{/Users/goby/pictures/teach/163/lecture/}
	{/Users/goby/pictures/teach/common/}} % set of paths to search for images

%\usepackage{units}
\usepackage{longtable}
\usepackage{booktabs}
\usepackage{multicol}


\usepackage{array}
\newcolumntype{L}[1]{>{\raggedright\let\newline\\\arraybackslash\hspace{0pt}}p{#1}}
\newcolumntype{C}[1]{>{\centering\let\newline\\\arraybackslash\hspace{0pt}}p{#1}}
\newcolumntype{R}[1]{>{\raggedleft\let\newline\\\arraybackslash\hspace{0pt}}p{#1}}

\usepackage{tikz}
	\tikzstyle{every picture}+=[remember picture,overlay]
	\usetikzlibrary{trees}

\mode<presentation>
{
  \usetheme{Lecture}
  \setbeamercovered{invisible}
}


\begin{document}

\begin{frame}{Our goals for this lecture are to }
	
	\hangpara review \highlight{homology} and \highlight{analogy,}
	
	\hangpara place homology and analogy in the context of phylogenetic trees and recent labs, and

	\hangpara introduce biogeography.
	
\end{frame}

%

%
%
{
\usebackgroundtemplate{\includegraphics[width=\paperwidth]{homologous_bird_wings} }
\begin{frame}[b]{A \highlight{homology} is a similar character shared among species due to common ancestry.}

	\vfilll

	\tiny Pixabay, \cc \hfill derdento, Pixabay \cc

\end{frame}
}
%
{
\usebackgroundtemplate{\includegraphics[width=\paperwidth]{homologous_appendages} }
\begin{frame}[t]{Homologies result from modification of existing traits through evolutionary processes.}

	\vfilll

	\hfill \tiny \copyright Pearson Education, Inc.

\end{frame}
}
%
\begin{frame}[t]{Homologous characters are used to make phylogenetic trees.}

\begin{longtable}[l]{@{}L{0.8in}C{0.9in}C{0.9in}C{0.7in}C{0.7in}@{}}
\toprule
Taxonomic Group & Blastopore becomes anus & Trochophore larvae & Vertebrae & Four Limbs \tabularnewline
\midrule
Clam &
	0 &
	1 & 
	0 & 
	0 \tabularnewline
Crocodile &
	1 & 
	0 & 
	1 & 
	1 \tabularnewline
Earthworm & 
	0 &
	1 & 
	0 & 
	0 \tabularnewline
Goldfish & 
	1 & 
	0 & 
	1 & 
	0 \tabularnewline
Ostrich & 
	1 & 
	0 & 
	1 & 
	1 \tabularnewline
Salamander & 
	1 & 
	0 & 
	1 & 
	1 \tabularnewline
Sea Urchin &
	1 &
	0 & 
	0 & 
	0 \tabularnewline
Tiger & 
	1 & 
	0 & 
	1 & 
	1 \tabularnewline
\bottomrule
\end{longtable}

\end{frame}
%
{
\usebackgroundtemplate{\includegraphics[width=\paperwidth]{convergent_evolution_wings}}
\begin{frame}[b]{An \highlight{analogy} is a similar character due to similar function, \emph{not} common ancestry.}


\tiny Top: Conty, Wikimedia, \ccby{3} \hfill Bottom: Martin Hauser, Wikimedia \ccby{3}
\end{frame}
}
%
{
\usebackgroundtemplate{\includegraphics[width=\paperwidth]{convergent_evolution_streamlined}}
\begin{frame}[t]{Analogies are the result of \highlight{convergent evolution.}}

	\vspace{3.85cm}

	\tiny \textcolor{white}{Alexander Vasenin, Wikimedia, \ccbysa{3}
	\hfill Terry Goos, Wikimedia, \ccby{2}}

	\vfilll

	\tiny Nobu Tamura, Wikimedia, \ccby{2} \hfill \textcolor{white}{Ken Funakoshi, Wikimedia, \ccby{2}}

\end{frame}
}
%
{
\usebackgroundtemplate{\includegraphics[width=\paperwidth]{elacatinus_phylogeny}}
\begin{frame}[b]

	\tiny  Taylor and Hellberg 2005. \hfill Photo: \copyright~Paul Humann

\end{frame}
}

{
\usebackgroundtemplate{\includegraphics[width=\paperwidth]{aquilegia_phylogeny}}
\begin{frame}[b]{Floral diversity results from adaptation to different pollinators.}

	\hfill \tiny  Whittall and Hodges 2007. Nature Letters 447: 706.

\end{frame}
}
%
{
\usebackgroundtemplate{\includegraphics[width=\paperwidth]{canary_islands_gallotia}}
\begin{frame}[b]

	\hfill \tiny  Thorpe et al. 1993. J. Evol. Biol. 6: 725.

\end{frame}
}
%
{
\usebackgroundtemplate{\includegraphics[width=\paperwidth]{tenerife_galloti}}
\begin{frame}[b]

	\hfill \tiny  Modified from Thorpe and Brown 1989. Biol. J. Linn. Soc. 38: 303.

\end{frame}
}
%
\begin{frame}[t]{\highlight{Biogeography} is the study of the distribution of organisms.}

	\includegraphics[width=\textwidth]{island_biogeography_size}

	\vfilll
	
	\hfill \tiny Wikimedia, \ccby{2}
\end{frame}

\begin{frame}[t]{How can we explain the distribution of flightless birds?}

	{\centering \includegraphics[height=0.8\textheight]{ratites}\par
	}

	\vfilll
	
	\hfill \tiny Wikimedia, \ccby{2}

\end{frame}

\begin{frame}[t]{Plate tectonics explains the distribution of flightless birds.}

	{\centering \includegraphics[height=0.8\textheight]{ratite_biogeography}\par
	}

	\vfilll
	
	\hfill \tiny Mitchell et al. 2014. Science 344: 898.
	
\end{frame}

\begin{frame}[t]{Organisms can expand their range by dispersal.}

	{\centering \includegraphics[height=0.8\textheight]{cattle_egret_range}\par
	}
	
\end{frame}
%
{
\usebackgroundtemplate{\includegraphics[width=\paperwidth]{camel_distribution}}
\begin{frame}[b]{Ancestral camels evolved in North America and dispersed elsewhere.}

	\tiny  John O'Neill, Wikimedia \ccbysa{3}

\end{frame}
}
%

\end{document}
