%!TEX TS-program = lualatex
%!TEX encoding = UTF-8 Unicode

\documentclass[t]{beamer}

%%%% HANDOUTS For online Uncomment the following four lines for handout
%\documentclass[t,handout]{beamer}  %Use this for handouts.
%\includeonlylecture{student}
%\usepackage{handoutWithNotes}
%\pgfpagesuselayout{3 on 1 with notes}[letterpaper,border shrink=5mm]


%%% Including only some slides for students.
%%% Uncomment the following line. For the slides,
%%% use the labels shown below the command.

%% For students, use \lecture{student}{student}
%% For mine, use \lecture{instructor}{instructor}

% FONTS

\usepackage{fontspec}
\def\mainfont{Linux Biolinum O}
\setmainfont[Ligatures={Common,TeX}, Contextuals={NoAlternate}, Numbers={Proportional, OldStyle}]{\mainfont}
\setsansfont[Ligatures={Common,TeX}, Scale=MatchLowercase, Numbers={Proportional,OldStyle}, BoldFont={* Bold}, ItalicFont={* Italic},]\mainfont

\newfontface\lining[Numbers={Lining}]\mainfont


\usepackage{amsmath,amssymb}
\usefonttheme[onlymath]{serif}
\usepackage{unicode-math}
\setmathfont[Scale=MatchLowercase]{TeX Gyre Pagella Math}

\usepackage{microtype}


\mode<presentation>
{
	\usetheme{Lecture}
	\setbeamercovered{invisible}
	\setbeamertemplate{items}[default]
%	\usefonttheme{professionalfonts}
}


\usepackage{graphicx}
	\graphicspath{{/Users/mtaylor/pictures/teach/163/lecture/}
	{/Users/mtaylor/pictures/teach/common/}} % set of paths to search for images

\usepackage{xcolor}

\def\mygray{gray!50}

\usepackage{multicol}
\usepackage{booktabs}
\usepackage{array}
\newcolumntype{L}[1]{>{\raggedright\let\newline\\\arraybackslash\hspace{0pt}}p{#1}}
\newcolumntype{C}[1]{>{\centering\let\newline\\\arraybackslash\hspace{0pt}}p{#1}}
\newcolumntype{R}[1]{>{\raggedleft\let\newline\\\arraybackslash\hspace{0pt}}p{#1}}
%\usepackage{textcomp}
%\usepackage{mhchem}
\usepackage{enumitem}
%\usepackage[export]{adjustbox}

\usepackage{calc} %For widthof, in one slide below.
\makeatletter
\def\@hspace#1{\begingroup\setlength\dimen@{#1}\hskip\dimen@\endgroup}
\makeatother

\usepackage{tikz}
%		%tikzstyle{every picture} conflicts for some reason with the pgfplots stuff above.
	\tikzstyle{every picture}+=[remember picture,overlay]
	\usetikzlibrary{arrows}
%\usetikzlibrary{positioning}

\begin{document}

\lecture{student}{student}

\begin{frame}{Our goal for this lecture is to learn about}

	\hangpara \highlight{species diversity},
	
	\hangpara \highlight {keystone species}, and
	
	\hangpara \highlight {ecological succession.}
	
\end{frame}
%
\begin{frame}{\highlight{Species diversity} measures the number and relative abundance of species in a community.}

	\hangpara \highlight{Species richness:} The number of species in a community.
	
	\hangpara \highlight{Evenness:} The relative abundance of each species in a community.
	
	\hangpara \highlight{Shannon diversity index} is a widely used measure of diversity.
	
	\[H' = -\sum p_i\,\mathrm{ln}\,p_i\]
	
\end{frame}
{
\usebackgroundtemplate{\includegraphics[width=\paperwidth]{community_diversity_example} }
\begin{frame}[t]{Which community has the greatest diversity?}

	\pause

	\vspace*{10\baselineskip}
	
	Community 1: \\$H\prime = -(0.25\,\mathrm{ln}\,0.25 + 0.25\,\mathrm{ln}\,0.25 +0.25\,\mathrm{ln}\,0.25 +0.25\,\mathrm{ln}\,0.25) = 1.39$

	\vspace*{\baselineskip}
	
	Community 2: \\$H\prime = -(0.80\,\mathrm{ln}\,0.80 + 0.05\,\mathrm{ln}\,0.05 +0.05\,\mathrm{ln}\,0.05 +0.10\,\mathrm{ln}\,0.10) = 0.71$
	
	\vfilll

	\hfill \tiny Fig.~54.11~Pearson Education, Inc.

\end{frame}
}
%
{
\usebackgroundtemplate{\includegraphics[width=\paperwidth]{keystone_species} }
\begin{frame}[b]

\hfill \tiny \textcolor{white}{\copyright\,Dave Cowles, Walla Walla University}
\end{frame}
}
%
\begin{frame}[t]{\textit{Pisaster} sea stars increased tide pool diversity.}

	\includegraphics[width=\textwidth]{keystone_results}
	
	\vfilll
	
	\hfill \tiny Fig.~54.18 \copyright\,Pearson Education, Inc.
\end{frame}
%
{
\usebackgroundtemplate{\includegraphics[width=\paperwidth]{keystone_herbivore_effects} }
\begin{frame}[b]{Herbivores maintain diversity through keystone actions.}

	\hfill \tiny Fig.~13.4 \copyright\,Sinauer Associates, Inc.
\end{frame}
}
%
\begin{frame}[t]{The loss of sea urchins caused a lose of diversity on Jamaican coral reefs.}

	{\centering \includegraphics[height=0.75\textheight]{keystone_urchin_grazing} \par
	}
	
	\vfilll
	
	\hfill \tiny Pecht and Pecht 2015. PeerJ Preprints.
\end{frame}

%
{
\usebackgroundtemplate{\includegraphics[width=\paperwidth]{disturbance_diversity} }
\begin{frame}[b]

\hfill \tiny \textcolor{white}{\textsc{noaa}, public domain.}
\end{frame}
}
%
{
\usebackgroundtemplate{\includegraphics[width=\paperwidth]{disturbance_intermediate} }
\begin{frame}[b]{Diversity can be greatest at intermediate levels of disturbance.}

	\hfill \tiny Fig.~54.20 \copyright\,Pearson Education, Inc.
\end{frame}
}
%
\lecture{instructor}{instructor}

{
\usebackgroundtemplate{\includegraphics[width=\paperwidth]{grand_tetons_burn} }
\begin{frame}[b]

\hfill \tiny \textcolor{white}{Google Earth}
\end{frame}
}
%
\lecture{student}{student}

{
\usebackgroundtemplate{\includegraphics[width=\paperwidth]{succession_intro} }
\begin{frame}[t]

	\vspace*{0.2\textheight}

	\hspace*{65mm} \parbox{50mm}{\raggedright Disasters lead to \highlight{ecological succession.}}
	
	\vspace*{\baselineskip}
	
	\hspace*{65mm} \parbox{50mm}{\raggedright Ecological succession describes how community structure changes over time.}

	\vspace*{\baselineskip}
	
	\hspace*{65mm} \parbox{50mm}{\raggedright Community structure refers to the composition of species present in the community.}
	
	\vfilll
	
	\hfill \tiny Greg Willis, Wikimedia \ccbysa{2}

\end{frame}
}
%
\lecture{instructor}{instructor}

{
\usebackgroundtemplate{\includegraphics[width=\paperwidth]{succession_helens_before} }
\begin{frame}[b]

	\tiny \textcolor{white}{17 May 1980, one day before eruption. \hfill \textsc{usgs}, public domain}

\end{frame}
}
%
{
\usebackgroundtemplate{\includegraphics[width=\paperwidth]{succession_helens_after} }
\begin{frame}[b]

	\tiny \textcolor{white}{Some time after the eruption. \hfill \textsc{usgs}, public domain}

\end{frame}
}
%
\lecture{student}{student}

{
\usebackgroundtemplate{\includegraphics[width=\paperwidth]{succession_helens} }
\begin{frame}[b]{\highlight{Primary succession} occurs when rock must turn into soil.}

	\begin{tikzpicture}

		\draw [ultra thick, ->] (18em, 17em) -- (18em, 13em);
		
	\end{tikzpicture}

	\hfill \tiny \textcolor{white}{Dabldoyou, Wikimedia, \ccbysa{4}}

\end{frame}
}
%
{
\usebackgroundtemplate{\includegraphics[width=\paperwidth]{succession_helens} }
\begin{frame}[b]{\highlight{Secondary succession} occurs when soil is already present.}

	\begin{tikzpicture}

		\draw [ultra thick, white, ->] (18em, 11em) -- (18em, 7em);

	\end{tikzpicture}

	\hfill \tiny \textcolor{white}{Dabldoyou, Wikimedia, \ccbysa{4}}

\end{frame}
}
%
{
\usebackgroundtemplate{\includegraphics[width=\paperwidth]{succession_stages} }
\begin{frame}[b]{Secondary succession follows a progression after a disturbance to return to a mature community.}

	\begin{tikzpicture}
			
		\onslide<1->\draw (-0.5em, 18.2em) --  node [above, midway] {Mature} (7em, 18.2em);
		
		\onslide<2->\draw (7.8em, 18.2em) --  node [above, midway] {Disturbance removes community to soil} (31.5em, 18.2em);

		\onslide<3->\draw (-0.5em, 4.5em) --  node [below, midway, text width=7em] {Early\\ successional species} (7em, 4.5em);

		\onslide<4->\draw (7.8em, 4.5em) --  node [below, midway, text width=14em] {Intermediate successional species} (23.5em, 4.5em);

		\onslide<5->\draw (24em, 4.5em) --  node [below, midway, text width=7em] {Late\\ successional species} (31.5em, 4.5em);

	\end{tikzpicture}

	\hfill \tiny Katelyn Murphy, Wikimedia, \ccbysa{4}


\end{frame}
%
%% EARLY SUCCESSION
{
\usebackgroundtemplate{\includegraphics[width=\paperwidth]{succession_forest_stages} }
\begin{frame}[b]{\highlight{Early successional species} (pioneer species) are generalists that facilitate the arrival of later species.}

	\begin{tikzpicture}

		% Stage 1
		\node[draw,circle,minimum size=0.35cm,inner sep=0pt] at (8em,18.5em) {\footnotesize\lining 1};
		\node (legend1) [draw,circle,minimum size=0.35cm,inner sep=0pt] at (0em,19em) {\footnotesize\lining 1};
		\node at (legend1.east) [text width = 2em, right] {\scriptsize Rock};

		\node[draw,circle,minimum size=0.3cm,inner sep=0pt] at (9.3em,2.5em) {\scriptsize\lining 1};

		% Stage 2
		\node[draw,circle,minimum size=0.35cm,inner sep=0pt, orange6] at (12.5em,18.5em) {\footnotesize\lining 2};
		\node (legend2) [draw,circle,minimum size=0.35cm,inner sep=0pt, orange6] at (0em,17em) {\footnotesize\lining 2};
		\node at (legend2.east) [orange6, text width = 2em, right] {\scriptsize Mosses\\[-5pt] Grasses};

		\node[draw,circle,minimum size=0.3cm,inner sep=0pt] at (13em,2.5em) {\scriptsize\lining 2};

		%Stage3
		\node[draw,circle,minimum size=0.35cm,inner sep=0pt, orange6] at (17em,18.5em) {\footnotesize\lining 3};
		\node (legend3) [draw,circle,minimum size=0.35cm,inner sep=0pt, orange6] at (0em,15em) {\footnotesize\lining 3};
		\node at (legend3.east) [orange6, text width = 4em, right] {\scriptsize Grasses\\[-5pt] Perennials};

		\node[draw,circle,minimum size=0.3cm,inner sep=0pt] at (16.5em,2.5em) {\scriptsize\lining 3};

		%Stage4
		\node[draw,circle,minimum size=0.35cm,inner sep=0pt] at (20.2em,18.5em) {\footnotesize\lining 4};
		\node (legend4) [draw,circle,minimum size=0.35cm,inner sep=0pt] at (0em,13em) {\footnotesize\lining 4};
		\node at (legend4.east) [text width = 4em, right] {\scriptsize Woody\\[-5pt] pioneers};

		\node[draw,circle,minimum size=0.3cm,inner sep=0pt] at (20em,2.5em) {\scriptsize\lining 4};

		%Stage5
		\node[draw,circle,minimum size=0.35cm,inner sep=0pt] at (24em,18.5em) {\footnotesize\lining 5};
		\node (legend4) [draw,circle,minimum size=0.35cm,inner sep=0pt] at (0em,11em) {\footnotesize\lining 5};
		\node at (legend4.east) [text width = 5em, right] {\scriptsize Fast\\[-5pt] growing trees};

		\node[draw,circle,minimum size=0.3cm,inner sep=0pt] at (23.7em,2.5em) {\scriptsize\lining 5};

		%Stage6
		\node[draw,circle,minimum size=0.35cm,inner sep=0pt] at (28.2em,18.5em) {\footnotesize\lining 6};
		\node (legend4) [draw,circle,minimum size=0.35cm,inner sep=0pt] at (0em,9em) {\footnotesize\lining 6};
		\node at (legend4.east) [text width = 4em, right] {\scriptsize Mature\\[-5pt] forest};

		\node[draw,circle,minimum size=0.3cm,inner sep=0pt] at (27.3em,2.5em) {\scriptsize\lining 6};


	\end{tikzpicture}

	\hfill \tiny LucasMartinFrey, Wikimedia, \ccby{3}

\end{frame}
}
%
%% INTERMEDIATE SUCCESSION
{
\usebackgroundtemplate{\includegraphics[width=\paperwidth]{succession_forest_stages} }
%\setbeamertemplate{background}[grid][color=black!50!white, step=1em]
\begin{frame}[b]{\highlight{Intermediate successional species} are generalists or specialists that facilitate, inhibit, or tolerate other species.}

	\begin{tikzpicture}

		% Stage 1
		\node[draw,circle,minimum size=0.35cm,inner sep=0pt] at (8em,18.5em) {\footnotesize\lining 1};
		\node (legend1) [draw,circle,minimum size=0.35cm,inner sep=0pt] at (0em,19em) {\footnotesize\lining 1};
		\node at (legend1.east) [text width = 2em, right] {\scriptsize Rock};

		\node[draw,circle,minimum size=0.3cm,inner sep=0pt] at (9.3em,2.5em) {\scriptsize\lining 1};

		% Stage 2
		\node[draw,circle,minimum size=0.35cm,inner sep=0pt] at (12.5em,18.5em) {\footnotesize\lining 2};
		\node (legend2) [draw,circle,minimum size=0.35cm,inner sep=0pt] at (0em,17em) {\footnotesize\lining 2};
		\node at (legend2.east) [text width = 2em, right] {\scriptsize Mosses\\[-5pt] Grasses};

		\node[draw,circle,minimum size=0.3cm,inner sep=0pt] at (13em,2.5em) {\scriptsize\lining 2};

		%Stage3
		\node[draw,circle,minimum size=0.35cm,inner sep=0pt] at (17em,18.5em) {\footnotesize\lining 3};
		\node (legend3) [draw,circle,minimum size=0.35cm,inner sep=0pt] at (0em,15em) {\footnotesize\lining 3};
		\node at (legend3.east) [text width = 4em, right] {\scriptsize Grasses\\[-5pt] Perennials};

		\node[draw,circle,minimum size=0.3cm,inner sep=0pt] at (16.5em,2.5em) {\scriptsize\lining 3};

		%Stage4
		\node[draw,circle,minimum size=0.35cm,inner sep=0pt, orange6] at (20.2em,18.5em) {\footnotesize\lining 4};
		\node (legend4) [draw,circle,minimum size=0.35cm,inner sep=0pt, orange6] at (0em,13em) {\footnotesize\lining 4};
		\node at (legend4.east) [orange6, text width = 4em, right] {\scriptsize Woody\\[-5pt] pioneers};

		\node[draw,circle,minimum size=0.3cm,inner sep=0pt] at (20em,2.5em) {\scriptsize\lining 4};

		%Stage5
		\node[draw,circle,minimum size=0.35cm,inner sep=0pt, orange6] at (24em,18.5em) {\footnotesize\lining 5};
		\node (legend4) [draw,circle,minimum size=0.35cm,inner sep=0pt, orange6] at (0em,11em) {\footnotesize\lining 5};
		\node at (legend4.east) [orange6, text width = 5em, right] {\scriptsize Fast\\[-5pt] growing trees};

		\node[draw,circle,minimum size=0.3cm,inner sep=0pt] at (23.7em,2.5em) {\scriptsize\lining 5};

		%Stage6
		\node[draw,circle,minimum size=0.35cm,inner sep=0pt] at (28.2em,18.5em) {\footnotesize\lining 6};
		\node (legend4) [draw,circle,minimum size=0.35cm,inner sep=0pt] at (0em,9em) {\footnotesize\lining 6};
		\node at (legend4.east) [text width = 4em, right] {\scriptsize Mature\\[-5pt] forest};

		\node[draw,circle,minimum size=0.3cm,inner sep=0pt] at (27.3em,2.5em) {\scriptsize\lining 6};


	\end{tikzpicture}

	\hfill \tiny LucasMartinFrey, Wikimedia, \ccby{3}

\end{frame}
}
%
%% LATE SUCCESSION
{
\usebackgroundtemplate{\includegraphics[width=\paperwidth]{succession_forest_stages} }
\begin{frame}[b]{\highlight{Late successional species} (climax species) are specialists that define the mature community.}

	\begin{tikzpicture}

		% Stage 1
		\node[draw,circle,minimum size=0.35cm,inner sep=0pt] at (8em,18.5em) {\footnotesize\lining 1};
		\node (legend1) [draw,circle,minimum size=0.35cm,inner sep=0pt] at (0em,19em) {\footnotesize\lining 1};
		\node at (legend1.east) [text width = 2em, right] {\scriptsize Rock};

		\node[draw,circle,minimum size=0.3cm,inner sep=0pt] at (9.3em,2.5em) {\scriptsize\lining 1};

		% Stage 2
		\node[draw,circle,minimum size=0.35cm,inner sep=0pt] at (12.5em,18.5em) {\footnotesize\lining 2};
		\node (legend2) [draw,circle,minimum size=0.35cm,inner sep=0pt] at (0em,17em) {\footnotesize\lining 2};
		\node at (legend2.east) [text width = 2em, right] {\scriptsize Mosses\\[-5pt] Grasses};

		\node[draw,circle,minimum size=0.3cm,inner sep=0pt] at (13em,2.5em) {\scriptsize\lining 2};

		%Stage3
		\node[draw,circle,minimum size=0.35cm,inner sep=0pt] at (17em,18.5em) {\footnotesize\lining 3};
		\node (legend3) [draw,circle,minimum size=0.35cm,inner sep=0pt] at (0em,15em) {\footnotesize\lining 3};
		\node at (legend3.east) [text width = 4em, right] {\scriptsize Grasses\\[-5pt] Perennials};

		\node[draw,circle,minimum size=0.3cm,inner sep=0pt] at (16.5em,2.5em) {\scriptsize\lining 3};

		%Stage4
		\node[draw,circle,minimum size=0.35cm,inner sep=0pt] at (20.2em,18.5em) {\footnotesize\lining 4};
		\node (legend4) [draw,circle,minimum size=0.35cm,inner sep=0pt] at (0em,13em) {\footnotesize\lining 4};
		\node at (legend4.east) [text width = 4em, right] {\scriptsize Woody\\[-5pt] pioneers};

		\node[draw,circle,minimum size=0.3cm,inner sep=0pt] at (20em,2.5em) {\scriptsize\lining 4};

		%Stage5
		\node[draw,circle,minimum size=0.35cm,inner sep=0pt] at (24em,18.5em) {\footnotesize\lining 5};
		\node (legend4) [draw,circle,minimum size=0.35cm,inner sep=0pt] at (0em,11em) {\footnotesize\lining 5};
		\node at (legend4.east) [text width = 5em, right] {\scriptsize Fast\\[-5pt] growing trees};

		\node[draw,circle,minimum size=0.3cm,inner sep=0pt] at (23.7em,2.5em) {\scriptsize\lining 5};

		%Stage6
		\node[draw,circle,minimum size=0.35cm,inner sep=0pt, orange6] at (28.2em,18.5em) {\footnotesize\lining 6};
		\node (legend4) [draw,circle,minimum size=0.35cm,inner sep=0pt, orange6] at (0em,9em) {\footnotesize\lining 6};
		\node at (legend4.east) [orange6, text width = 4em, right] {\scriptsize Mature\\[-5pt] forest};

		\node[draw,circle,minimum size=0.3cm,inner sep=0pt] at (27.3em,2.5em) {\scriptsize\lining 6};


	\end{tikzpicture}

	\hfill \tiny LucasMartinFrey, Wikimedia, \ccby{3}

\end{frame}
}
%
{
\usebackgroundtemplate{\includegraphics[width=\paperwidth]{succession_quail_habitat} }
\begin{frame}[b]{Bobwhite quail requires early successional habitat.}

	\hfill \tiny \textcolor{white}{\textsc{usda}, Flickr, \ccby{2}}

\end{frame}
}
%
{
\usebackgroundtemplate{\includegraphics[width=\paperwidth]{succession_kirtland_warbler} }
\begin{frame}[b]{Kirtland's warbler requires intermediate successional habitat.}

	\hfill \tiny \textsc{usfws}, Flickr, \ccby{2}

\end{frame}
}
%
%
\end{document}
