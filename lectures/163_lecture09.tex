%!TEX TS-program = lualatex
%!TEX encoding = UTF-8 Unicode

\documentclass[t]{beamer}

%%%% HANDOUTS For online Uncomment the following four lines for handout
%\documentclass[t,handout]{beamer}  %Use this for handouts.
%\usepackage{handoutWithNotes}
%\pgfpagesuselayout{3 on 1 with notes}[letterpaper,border shrink=5mm]

%\includeonlylecture{student}

%%% Including only some slides for students.
%%% Uncomment the following line. For the slides,
%%% use the labels shown below the command.

%% For students, use \lecture{student}{student}
%% For mine, use \lecture{instructor}{instructor}

% FONTS
\usepackage{fontspec}
\def\mainfont{Linux Biolinum O}
\setmainfont[Ligatures={Common,TeX}, Contextuals={NoAlternate}, BoldFont={* Bold}, ItalicFont={* Italic}, Numbers={OldStyle}]{\mainfont}
\setsansfont[Ligatures={Common,TeX}, Scale=MatchLowercase, Numbers=OldStyle, BoldFont={* Bold}, ItalicFont={* Italic},]{Linux Biolinum O} 
\usepackage{microtype}

\usepackage{graphicx}
	\graphicspath{{/Users/goby/pictures/teach/163/lecture/}
	{/Users/goby/pictures/teach/common/}} % set of paths to search for images

\usepackage{multicol}
\usepackage{booktabs}
%\usepackage{textcomp}

\usepackage{tikz}
	\tikzstyle{every picture}+=[remember picture,overlay]
%	\usetikzlibrary{arrows}


\mode<presentation>
{
  \usetheme{Lecture}
  \setbeamercovered{invisible}
	\setbeamercolor{alerted text}{fg=orange6}
}

\begin{document}

\lecture{student}{student}

\begin{frame}{Our goal for this lecture and the next is to learn}
	
	\hangpara what is a \highlight{species,}

	\hangpara how \highlight{reproductive isolation} evolves,
	
	\hangpara learn how \highlight{allopatric} distributions contribute to speciation,
	
	\hangpara learn what happens in \highlight{hybrid zones}, and
	
%	\hangpara learn what are \highlight{ring species,} and
	
	\hangpara ask whether speciation can occur with gene flow.
	
\end{frame}
%
%
% Subdividing evolution
%\begin{frame}{Evolution can be divided into three related processes.}
%
%	\hangpara \alert<1>{Microevolution:} evolution within populations.
%
%	\hangpara \alert<2>{Speciation:} the evolution of new species.
%
%	\hangpara Macroevolution: evolution above the species level.
%
%	\centering
%	\begin{tikzpicture}
%		[remember picture, overlay,
%		myLine/.style={color=black,very thick},
%		myArrow/.style={color=orange6, thick, ->}]
%
%
%		% Draw the tree
%		\visible<1->{
%			\draw [myLine] (-3,-2) -- (0,-2);
%		} % ancestor
%
%		\visible<2->{
%			\draw [myLine] (0,-2) -- (0,-0.7); % vertical lines
%			\draw [myLine] (0,-2) -- (0,-3.3);
%		}
%		
%		\visible<2->{
%			\draw [myLine] (0,-0.7) -- (3,-0.7); % descendants
%			\draw [myLine] (0,-3.3) -- (3,-3.3); 
%		}
%
%		% Ancestor microevolution
%		\visible<1>{
%			\node [color=orange6] at (-1.5,-1.8) {Microevolution};
%		} % ancestor
%	
%		% Speciation
%		\visible<2>{
%			\node [color=orange6] (speciation) at (-1.5,-3.1) {Speciation};
%			\node [circle, draw=orange6, thick, minimum size=3mm] (speccirc) at (0,-2) {};
%			\draw [color=orange6, thick] (speciation) edge (speccirc);
%		}
%		
%		% Descendant microevolution
%		\visible<2>{
%			\node [color=orange6] at (1.5,-0.5) {Microevolution}; % descendants
%			\node [color=orange6] at (1.5,-3.1) {Microevolution};
%		}
%
%%		% Macroevolution
%%		\visible<8>{
%%			\node [color=orange6] (macro) at (1.5,-2) {Macroevolution};
%%			\draw [myArrow] (macro.north) to (1.5,-0.8);
%%			\draw [myArrow] (macro.south) to (1.5,-3.2);
%%		}
%
%	\end{tikzpicture}
%
%
%%	\vspace{\baselineskip}
%%	\hangpara Speciation and macroevolution require microevolution to occur.
%\end{frame}
%
\lecture{student}{student}

{
\usebackgroundtemplate{\includegraphics[width=\paperwidth]{atelopus_zeteki.jpg} }
\begin{frame}[b]{What is a \highlight{species?} }

%	\hfill \tiny	\textcolor{gray}{\textit{Atelopus zeteki} photo by Brian Gratwicke, Flickr, Creative Commons.}
	\hfill \tiny Brian Gratwicke, Flickr \ccby{2}
\end{frame}
}
%
\begin{frame}{``Species'' is not easily defined.}

	\vspace{2\baselineskip}

	\centering
	\begin{tabular}{l l}
	\toprule
		\highlight{Morphological}	&	Genealogical\\
		\highlight{Biological}		&	Genotypic\\
		General Lineage		&	Recognition\\
		Phylogenetic (I)			&	Phenetic\\
		Phylogenetic (II)		&	Cladistic\\
		Evolutionary			&	Diagnostic\\
		Ecological				&	Polytypic\\
		Typological			&	Population\\
		Cohesion				&	\\
	\bottomrule
	\end{tabular}
\end{frame}
%
{
\usebackgroundtemplate{\includegraphics[width=\paperwidth]{morphological_species} }
\begin{frame}[b]{}
%	\hfill \tiny \textcolor{gray}{\textit{Atelopus limosus} and \textit{Dendrobates auratus} photos by Brian Gratwicke, Flickr, Creative Commons.}
	\hfill \tiny Brian Gratwicke, Flickr \ccby{2}
\end{frame}
}
%
{
\usebackgroundtemplate{\includegraphics[width=\paperwidth]{dendrobates_morphological} }
\begin{frame}[b]{}
	\hfill \tiny \textcolor{white}{Wikimedia Commons.}
\end{frame}
}
%
\begin{frame}[t]{\textit{Hypsiboas} is a genus of \highlight{cryptic species.}}
	\centering
	\begin{multicols}{2}
		\includegraphics[width=0.45\textwidth]{hypsiboas_frogs.pdf} \vfill
	\columnbreak
		\includegraphics[width=0.5\textwidth]{hypsiboas_map} \vfill
	\end{multicols}

	\vfilll

	\hfill \tiny Funk et al., 2012. Proc. R. Soc. London B 279: 1806-1814.
\end{frame}
%
\begin{frame}{}
	\vspace{1\baselineskip}
	\begin{multicols}{2}

		\onslide<1->{Body shape is not reliable.\vspace{\baselineskip}
	
		\includegraphics[width=0.46\textwidth]{hypsiboas_body_pc}}

	\columnbreak

		%\pause
		\onslide<2->{Mating calls are reliable.\vspace{\baselineskip}
		\includegraphics[width=0.5\textwidth]{hypsiboas_body_calls}}
	
	\end{multicols}

	\vfilll

	\onslide<1->{\hfill \tiny Funk et al., 2012. Proc. R. Soc. London B 279: 1806-1814.}
\end{frame}

\begin{frame}{The \highlight{cryptic species} are genetically distinct.}

	\vspace*{-0.5\baselineskip}
	
	\centering
	\begin{multicols}{2}
	
		\includegraphics[width=0.45\textwidth]{hypsiboas_frogs.pdf} \vfill
	
	\columnbreak
	
		\includegraphics[height=0.78\textheight]{hypsiboas_phylog} \vfill
	
	\end{multicols}

	\vfilll

	\hfill \tiny Funk et al., 2012. Proc. R. Soc. London B 279: 1806-1814.

\end{frame}
%
{
\usebackgroundtemplate{\includegraphics[width=\paperwidth]{biological_species_penguins.jpg} }
\begin{frame}[b]

	\hfill \tiny \textcolor{white}{Liam Quinn, Flickr \ccbysa{2}}
	%{King penguins photo by Liam Quinn, Flickr, Creative Commons.}

\end{frame}
}
%
{
\usebackgroundtemplate{\includegraphics[width=\paperwidth]{interbreed_creepers} }
\begin{frame}[b]{\textcolor{white}{Can these sister species} \textcolor{orange5}{interbreed?}}

	\tiny\textcolor{white}{Alan Vernon, Wikimedia \ccbysa{2} \hfill Francesco Veronesi, Wikimedia, \ccbysa{2}}
	%{Brown creeper photo by Alan Vernon, Wikimedia Commons. Short-Toed Creeper photo by Thomas van de Vosse, Flickr, Creative Commons.}
\end{frame}
}
%
{
\usebackgroundtemplate{\includegraphics[width=\paperwidth]{interbreed_zebras} }
\begin{frame}[b]{\textcolor{white}{Can these two species} \textcolor{orange5}{interbreed?}}

	\tiny\textcolor{white}{Rainbirder, Wikimedia, \ccbysa{2} \hfill Daderot, Wikimedia public domain}
	%{Grévy's zebra photo by Rainbirder, Wikimedia Commons. Hagerman's zebra, Utah Museum of Natural History, photo by Daderot, Wikimedia Commons.}
\end{frame}
}
%
{
\usebackgroundtemplate{\includegraphics[width=\paperwidth]{asexual_rotifer} }
\begin{frame}[b]{How do you test for interbreeding in \highlight{asexual species?}}

	\tiny Damián H. Zanette, Wikimedia public domain

\end{frame}
}
%
\begin{frame}{\highlight{Speciation} occurs when two new species evolve from single ancestor.}

	\centering
	\begin{tikzpicture}
		[remember picture, overlay,
		myLine/.style={color=black,very thick},
		myArrow/.style={color=orange6, thick, ->}]

		\draw [myLine] (-3,-2) -- (0,-2);
		\draw [myLine] (0,-2) -- (0,-0.7); % vertical lines
		\draw [myLine] (0,-2) -- (0,-3.3);
		\draw [myLine] (0,-0.7) -- (3,-0.7); % descendants
		\draw [myLine] (0,-3.3) -- (3,-3.3); 

		% Speciation
		\node [color=orange6] (speciation) at (-1.5,-3.1) {Speciation};
		\node [circle, draw=orange6, thick, minimum size=3mm] (speccirc) at (0,-2) {};
		\draw [color=orange6, thick] (speciation) edge (speccirc);
		
		% Ancestor 
		\node [color=blue6] at (-1.5,-1.8) {Ancestor};
		
		% Descendants
		\node [color=blue6] at (1.5,-0.5) {New Species};
		\node [color=blue6] at (1.5,-3.1) {New Species};

	\end{tikzpicture}
\end{frame}
%
\begin{frame}{Isolating mechanisms are \highlight{prezygotic} or \highlight{postzygotic}.}
	\vspace{2\baselineskip}
	\centering
	\begin{tabular}{l l l}
	\toprule
	{\large \highlight{Prezygotic}}	& or &	{\large Postzygotic}\\
	\midrule
	Habitat	& & Sterility \\
	Behavioral & &	Inviability \\
	Temporal	& &	Breakdown \\
	Mechanical 	& & 	\\
	Gamete Incompatibility	& & \\
	\bottomrule
	\end{tabular}

	\hangpara Study pages 502--503 for details of mechanisms we do not cover in lecture.
\end{frame}
%
{
\usebackgroundtemplate{\includegraphics[width=\paperwidth]{prezygotic_habitat} }
\begin{frame}[b]{\textcolor{white}{Columbine flowers live in different \textcolor{orange5}{habitats.}} }
	\tiny \textcolor{white}{Dcrjsr, Wikimedia, \ccbysa{3}. \hfill Steve Berardi, Flickr, \ccbyncsa{2}. \href{https://www.youtube.com/watch?v=Iwfs2TDYg-8}{Video}}
\end{frame}
}
%
{
\usebackgroundtemplate{\includegraphics[width=\paperwidth]{prezygotic_behavior} }
\begin{frame}[b]{}
	\hfill \tiny  \textcolor{white}{Vince Smith, Wikimedia, \ccbysa{2}. \href{https://www.youtube.com/watch?v=z922by9_6Fw}{Video} }
\end{frame}
}
%
{
\usebackgroundtemplate{\includegraphics[width=\paperwidth]{prezygotic_gamete} }
\begin{frame}[b]{\textcolor{white}{Many marine species have} \textcolor{orange5}{incompatible gametes.}}

	\hfill \tiny \textcolor{white}{Emma Hickerson, \textsc{noaa}, public domain. \href{https://www.youtube.com/watch?v=wsaZ8-I7akg}{Video}}

\end{frame}
}
%
{
\usebackgroundtemplate{\includegraphics[width=\paperwidth]{allo_sym_speciation} }
\begin{frame}[b]{\highlight{Allopatric speciation} occurs from geographic isolation.}

	\hfill \tiny \textcopyright Pearson Education, Inc.

\end{frame}
}
%
{
\usebackgroundtemplate{\includegraphics[width=\paperwidth]{allopatric_chipmunks} }
\begin{frame}[b]

	\hfill \tiny \textcopyright Pearson Education, Inc.

\end{frame}
}
%
{
\usebackgroundtemplate{\includegraphics[width=\paperwidth]{allopatric_snapping_shrimp} }
\begin{frame}[b]{Allopatric speciation can be detected with phylogenetic trees.}

	\tiny  Julio Duarte, Flickr, \ccbyncsa{2} \hfill Knowlton et al. 1993. Science 260: 1629.

\end{frame}
}
%
{
\usebackgroundtemplate{\includegraphics[width=\paperwidth]{allopatric_shrimp2} }
\begin{frame}[b]

	\tiny  \copyright Pearson Education, Inc.

\end{frame}
}
%
\begin{frame}{Isolating mechanisms are \highlight{prezygotic} or \highlight{postzygotic}.}
	\vspace{2\baselineskip}
	\centering
	\begin{tabular}{l l l}
		\toprule
		{\large Prezygotic}	& or &	{\large \highlight{Postzygotic}}\\
		\midrule
		Habitat	& & Sterility \\
		Behavioral & &	Inviability \\
		Temporal	& &	Breakdown \\
		Mechanical 	& & 	\\
		Gamete Incompatibility	& & \\
		\bottomrule
	\end{tabular}

	\hangpara Study pages 502--503 for details of mechanisms we do not cover in lecture.
\end{frame}
%
{
\usebackgroundtemplate{\includegraphics[width=\paperwidth]{postzygotic_sterility} }
\begin{frame}[b]{\highlight{Hybrid sterility} means that hybrid offspring cannot reproduce.}

	\hfill \tiny\textcolor{white}{\copyright Pearson Education, Inc.}

\end{frame}
}
%
{
\usebackgroundtemplate{\includegraphics[width=\paperwidth]{postzygotic_inviability} }
\begin{frame}[b]{}

	\hfill \tiny\textcolor{white}{\textcopyright Sinauer Associates, Inc.}

\end{frame}
}
%
{
\usebackgroundtemplate{\includegraphics[width=\paperwidth]{hybrid_zone_toads} }
\begin{frame}[b]{\textit{Bombina} hybrids have reduced fitness relative to parent species.}

	\tiny \textcopyright Pearson Education, Inc.

\end{frame}
}
%
\begin{frame}[t]{\highlight{Hybrid zones} can form if reproductive isolation has not evolved completely.}

	\alt<handout>{}{\includegraphics<1>[width=\textwidth]{hybrid_zone1}
	\includegraphics<2>[width=\textwidth]{hybrid_zone2}}
	\includegraphics<3>[width=\textwidth]{hybrid_zone3}

	\vfilll 
	
	\onslide<1->\hfill \tiny \textcopyright Pearson Education, Inc.	
\end{frame}
%
\begin{frame}[t]{Hybrid zones can have three possible outcomes. \phantom{if isolation has not evolved completely.}}

	\includegraphics[width=\textwidth]{hybrid_zone4}

	\vfilll 

	\hfill \tiny \copyright Pearson Education, Inc.	

\end{frame}
%
{
\usebackgroundtemplate{\includegraphics[width=\paperwidth]{hybrid_zone_toads} }
\begin{frame}[b]{\textit{Bombina} toads have a stable hybrid zone.}

	\tiny \textcopyright Pearson Education, Inc.

\end{frame}
}
%
\lecture{instructor}{instructor}
{
\usebackgroundtemplate{\includegraphics[width=\paperwidth]{hybrid_reinforcement1} }
\begin{frame}[b]{Two flycatcher species may hybridize with each other.}

	\hfill \tiny \textcopyright Pearson Education, Inc.

\end{frame}
}
%
\lecture{student}{student}
{
\usebackgroundtemplate{\includegraphics[width=\paperwidth]{hybrid_reinforcement2} }
\begin{frame}[b]{\highlight{Reinforcement} reduces chance of forming hybrids.}

	\hfill \tiny \textcopyright Pearson Education, Inc.

\end{frame}
}
%
\begin{frame}[t]
	\alt<handout>{}{\frametitle<1>{How many species do you see? One or two?}
	\frametitle<2>{Do you still see the same number?}}
	\frametitle<3>{Females of these species use visual cues to identify males.}

	\centering
	\alt<handout>{}{\includegraphics<1>[width=\textwidth]{cichlid_species1}
	\includegraphics<2>[width=\textwidth]{cichlid_species2}}
	\includegraphics<3>[width=\textwidth]{cichlid_species3}
		
	\vfilll
	
	\hfill \tiny \textcopyright Pearson Education, Inc.
\end{frame}
%
\begin{frame}[t]{\highlight{Fusion} reverses the speciation process.}

	\centering
	\includegraphics[height=0.85\textheight]{hybrid_fusion}

	\vfilll
	
	\hfill \tiny \textcopyright Pearson Education, Inc.
\end{frame}
%
%{
%\usebackgroundtemplate{\includegraphics[width=\paperwidth]{ensatina_ring_species} }
%\begin{frame}[b]{\highlight{Ring species} can interbreed along a ring but not where the ends of the ring meet.}
%	
%	\tiny Devitt et al. 2011. BMC Evolutionary Biology 11: 245.
%\end{frame}
%}
%

\end{document}
