%!TEX TS-program = lualatex
%!TEX encoding = UTF-8 Unicode

%\documentclass[t]{beamer}

%%%% HANDOUTS For online Uncomment the following four lines for handout
\documentclass[t,handout]{beamer}  %Use this for handouts.
\usepackage{handoutWithNotes}
\pgfpagesuselayout{3 on 1 with notes}[letterpaper,border shrink=5mm]
%	\setbeamercolor{background canvas}{bg=black!5}

\includeonlylecture{student}

%%% Including only some slides for students.
%%% Uncomment the following line. For the slides,
%%% use the labels shown below the command.

%% For students, use \lecture{student}{student}
%% For mine, use \lecture{instructor}{instructor}


%\usepackage{pgf,pgfpages}
%\pgfpagesuselayout{4 on 1}[letterpaper,border shrink=5mm]

% FONTS
\usepackage{fontspec}
\def\mainfont{Linux Biolinum O}
\setmainfont[Ligatures={Common,TeX}, Contextuals={NoAlternate}, BoldFont={* Bold}, ItalicFont={* Italic}, Numbers={OldStyle}]{\mainfont}
\setsansfont[Ligatures={Common,TeX}, Scale=MatchLowercase, Numbers=OldStyle]{Linux Biolinum O} 
\usepackage{microtype}

\usepackage{graphicx}
	\graphicspath{{/Users/goby/pictures/teach/163/lecture/}
	{/Users/goby/pictures/teach/common/}} % set of paths to search for images

\usepackage{multicol}
\usepackage{booktabs}
%\usepackage{textcomp}

\usepackage{tikz}
	\tikzstyle{every picture}+=[remember picture,overlay]
	\usetikzlibrary{arrows}

\mode<presentation>
{
  \usetheme{Lecture}
  \setbeamercovered{invisible}
  \setbeamertemplate{items}[square]
}

%\usefonttheme[onlymath]{serif}
%\usecolortheme[named=blue7]{structure}

\newcommand{\btVFill}{\vskip0pt plus 1filll}

%\usepackage{ccicons}


%%% Creative Commons Licenses. Establish, then add to Beamer template.
%\newcommand{\ccbysa}[1][4]{\textsc{cc by-sa #1.0}} % Use version 4.0 as default.
\newcommand{\ccby}[1]{%
	\ifx&#1&
	{\textsc{cc by}}%
\else
	{\textsc{cc by #1.0}}
\fi}


\newcommand{\ccbysa}[1]{%
	\ifx&#1&
	{\textsc{cc by-sa}}%
\else
	{\textsc{cc by-sa #1.0}} 
\fi}

\newcommand{\ccbyncsa}[1]{%
	\ifx&#1&
	{\textsc{cc by-nc-sa}}%
\else
	{\textsc{cc by-nc-sa #1.0}}
\fi}

\newcommand{\ccbync}[1]{%
	\ifx&#1&
	{\textsc{cc by-nc}}%
\else
	{\textsc{cc by-nc #1.0}}
\fi}


\begin{document}

\lecture{student}{student}
\begin{frame}[t]{Our goals for this lecture are to learn}

	
	\hangpara	the difference between \highlight{phenotype} and \highlight{genotype,} and

	\hangpara how phenotype changes in response to different \highlight{modes of selection.}
	
\end{frame}

%
{
\usebackgroundtemplate{\includegraphics[width=\paperwidth]{individuals_vary}}
\begin{frame}[b]
\hfill\textcolor{white}{\tiny \textcopyright Entomart, Wikimedia}
\end{frame}
}
% Why do populations vary?
\begin{frame}[t]{The traits of an organism is called the \highlight{phenotype.}}
	\centering
	\includegraphics[width=0.8\textwidth]{individuals_vary_plain}\par
	
	\btVFill
	
	\hfill\tiny \textcopyright Entomart, Wikimedia

\end{frame}
%
\begin{frame}[t]{What causes the \highlight{phenotype} to vary?}
	\centering
	\includegraphics[width=0.8\textwidth]{individuals_vary_plain}
	
		\btVFill
	
	\hfill\tiny \textcopyright Entomart, Wikimedia

\end{frame}
%
\begin{frame}[t]{The phenotype is determined by the \highlight{genotype.}}

	\begin{multicols}{2}
	
		\includegraphics[width=0.45\textwidth]{punnett_square_genotype_phenotype}
	
	\columnbreak

	The genotype is the genetic makeup of the organism.\vspace*{2\baselineskip}
	
	The genotype is the combination of alleles for one or more gene.
	
	\end{multicols}
	
	\btVFill
	
	\tiny Madeleine Price Ball, Wikimedia, \textsc{cc0}

\end{frame}

% Genotype and phenotype
%\lecture{instructor}{instructor}
{
\usebackgroundtemplate{\includegraphics[width=\paperwidth]{chromosomes}}
\begin{frame}[c,plain]
	\begin{tikzpicture}[remember picture, overlay]

	\alt<handout>{}{\visible <1,6->{
		\node at (0.45,2) [right] {The \highlight{genotype} is the genetic makeup of the organism.};}
	
	\visible <7>{
		\node at (0.45,1.4)[right] {The \highlight{phenotype} is the physical expression of the genotype.};
	}}
		
	
	\alt<handout>{}{\visible <2>{
		\draw (3,-2.5) -- (3,0.5) -- (3.9,0.5) -- (3.9,-2.5) -- cycle ;
		\draw (6,-2.5) -- (6,0.5) -- (6.9,0.5) -- (6.9,-2.5) -- cycle ;
		\draw (9,-2.5) -- (9,0.5) -- (9.9,0.5) -- (9.9,-2.5) -- cycle ;
	
		\node at (6.4, 2) (gene) {Gene or Locus};
	
		\draw (gene.south west) -- (3.9,0.5);
		\draw (gene.south) -- (6.45,0.5);
		\draw (gene.south east) -- (9,0.5);
	}}
	
	\visible <3-4>{
		\node at (3.45,0.25) (Callele) {$C$};
		\node at (3.45,-2.25) {$C$};

		\node at (6.45,0.25) (aallele) {$a$};
		\node at (6.45,-2.25) {$a$};
	}

	\visible <3,5>{
		\node at (9.45,0.25) (Tallele) {$T$};
		\node at (9.45,-2.25) {$t$};
	}
	
	\alt<handout>{}{\visible <3>{
		\node at (6.4, 2) (allele) {Allele};
		\draw (allele.south west) -- (Callele.north east);
		\draw (allele.south) -- (aallele.north);
		\draw (allele.south east) -- (Tallele.north west);
	}}

	\visible <4> {
		\draw (3,-2.5) -- (3,0.5) -- (3.9,0.5) -- (3.9,-2.5) -- cycle ;
		\draw (6,-2.5) -- (6,0.5) -- (6.9,0.5) -- (6.9,-2.5) -- cycle ;

		\node at (3.45, 1.25) (hc) {\highlight{Homozygous,} $C$ allele};
		\node at (6.45, -3.5) (ac) {Homozygous, $a$ allele};

		\draw (hc.south) -- (3.45,0.5);
		\draw (ac.north) -- (6.45,-2.5);
	}
	
	\visible <5> {
		\draw (9,-2.5) -- (9,0.5) -- (9.9,0.5) -- (9.9,-2.5) -- cycle ;

		\node at (9.45, 1.25) (tc) {\highlight{Heterozygous,} $T$ allele};

		\draw (tc.south) -- (9.45,0.5);
	}
	\end{tikzpicture}
\end{frame}
}
%
\lecture{student}{student}

\begin{frame}[t]{If natural selection changes the phenotype, what also had to change?}

\hangpara Three \highlight{modes of selection} describe \emph{how} the phenotype changes in a population.

\hangpara The modes do not say \emph{why} the phenotype changes.

\end{frame}
%
\begin{frame}{\highlight{Directional} selection shifts the population toward \emph{one} extreme phenotype.}
	\centering
	\begin{columns}[T]
		\begin{column}{0.5\textwidth}
			\centering
			\includegraphics[width=0.9\textwidth]{mode_original}
		\end{column}
		\begin{column}{0.5\textwidth}
			\pause\includegraphics[width=0.9\textwidth]{mode_directional}
		\end{column}
	\end{columns}
\end{frame}
%
{
\usebackgroundtemplate{\includegraphics[width=\paperwidth]{scissortail_flycatcher} }
\begin{frame}[b]{Females prefer longer tails in male scissortail flycatchers.}

	\hfill\tiny{TexasEagle, Flickr, \ccbync{2}}

\end{frame}
}
%
\begin{frame}{\highlight{Disruptive} selection shifts the population toward \emph{both} extreme phenotypes.}
	\centering
	\begin{columns}[T]
		\begin{column}{0.5\textwidth}
			\centering
			\includegraphics[width=0.9\textwidth]{mode_original}
		\end{column}
		\begin{column}{0.5\textwidth}
			\includegraphics[width=0.9\textwidth]{mode_disruptive}
		\end{column}
	\end{columns}
\end{frame}
%
{
\usebackgroundtemplate{\includegraphics[width=\paperwidth]{geospiza_fortis} }
\begin{frame}[b,plain]{Beak size is diverging in \textit{Geospiza fortis.}}

	\tiny\textcolor{white}{Mark Putney, Flickr, \ccbysa{2}} \pause\hfill\includegraphics[width=0.45\textwidth]{geospiza_fortis_disruptive}

\end{frame}
}
%
\begin{frame}{\highlight{Stabilizing} selection shifts the population toward the intermediate phenotype.}
	\centering
	\begin{columns}[T]
		\begin{column}{0.5\textwidth}
			\centering
			\includegraphics[width=0.9\textwidth]{mode_original}
		\end{column}
		\begin{column}{0.5\textwidth}
			\includegraphics[width=0.9\textwidth]{mode_stabilizing}
		\end{column}
	\end{columns}
\end{frame}
%
{
\usebackgroundtemplate{\includegraphics[width=\paperwidth]{balancing_selection} }
\begin{frame}[b]{}

\tiny\textcopyright Pearson Education, Inc.
\end{frame}
}

\end{document}
