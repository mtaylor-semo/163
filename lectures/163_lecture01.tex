%!TEX TS-program = lualatex
%!TEX encoding = UTF-8 Unicode

\documentclass[t]{beamer}

%%%% HANDOUTS For online Uncomment the following four lines for handout
%\documentclass[t,handout]{beamer}  %Use this for handouts.
%\usepackage{handoutWithNotes}
%\pgfpagesuselayout{3 on 1 with notes}[letterpaper,border shrink=5mm]


%%% Including only some slides for students.
%%% Uncomment the following line. For the slides,
%%% use the labels shown below the command.
%\includeonlylecture{student}

%% For students, use \lecture{student}{student}
%% For mine, use \lecture{instructor}{instructor}


%\usepackage{pgf,pgfpages}
%\pgfpagesuselayout{4 on 1}[letterpaper,border shrink=5mm]

% FONTS
\usepackage{fontspec}
\def\mainfont{Linux Biolinum O}
\setmainfont[Ligatures={Common,TeX}, Contextuals={NoAlternate}, Numbers={Proportional, OldStyle}]{\mainfont}
\setsansfont[Ligatures={Common,TeX}, Scale=MatchLowercase, Numbers={Proportional,OldStyle}, BoldFont={* Bold}, ItalicFont={* Italic},]\mainfont

\newfontface\lining[Numbers={Lining}]\mainfont

\usepackage{graphicx}
	\graphicspath{{/Users/goby/pictures/teach/163/lecture/}
	{/Users/goby/pictures/teach/common/}} % set of paths to search for images

%\usepackage{units}
\usepackage{booktabs}
%\usepackage{textcomp}

\usepackage{tikz}
%	\tikzstyle{every picture}+=[remember picture,overlay]

\mode<presentation>
{
  \usetheme{Lecture}
  \setbeamercovered{invisible}
  \setbeamertemplate{items}[square]
}

%\usefonttheme[onlymath]{serif}
%\usecolortheme[named=blue7]{structure}

\newcommand{\btVFill}{\vskip0pt plus 1filll}

\begin{document}

{
\usebackgroundtemplate{\includegraphics[width=\paperwidth]{hoverflies} }
\begin{frame}[b,plain]{\textcolor{orange7}{\textsc{bi} 163-01: Evolution and Ecology}}

%\begin{center}\LARGE\textcolor{white}{Sit towards front of the room.}\end{center}

\hfill\textcolor{gray}{\Tiny Mating hoverflies photo by Fir0002, Wikimedia Commons.}
\end{frame}
}

{
\usebackgroundtemplate{\includegraphics[width=\paperwidth]{mike_snake}
}
\begin{frame}[t,plain]
	\large
	\vspace{5ex}
	\hangpara\hspace{17em} Mike Taylor

	\hangpara\hspace{17em} \textsc{rh} 217

	\hangpara\hspace{17em} \textsc{m}\,\textsc{t}\,\textsc{w} 11 am --12 noon.

	\hangpara\hspace{17em} mtaylor@semo.edu

\end{frame}
}


\begin{frame}[t]{I do not give grades. You \highlight{earn} them.}

	\hangpara Four exams @ 60\%,

	\hangpara Lab exercises @ 20\%, and

	\hangpara Assignments @20\%.

\end{frame}


\begin{frame}[t]{\href{http://learning.semo.edu}{learning.semo.edu}}
	\begin{center}
		\includegraphics[width=\textwidth]{moodle_logo}
		
		\medskip
		
		
	\end{center}

	
\end{frame}



{
\usebackgroundtemplate{\includegraphics[width=\paperwidth]{kitty_yawn}
}
\begin{frame}{\textcolor{white}{You \highlight{earn} your grades with}}
%\large
\hangpara\textcolor{white}{\highlight{M}iscellaneous assignments,}
\pause

\hangpara\textcolor{white}{\highlight{E}xams,}
\pause

\hangpara\textcolor{white}{\highlight{O}nline homework, and}
\pause

\hangpara\textcolor{white}{\highlight{W}eekly lab assignments.}

%\vspace{20ex}
%\btVFill
%\vskip0pt plus 1filll
\vskip 0pt plus 1filll

\textcolor{white}{\tiny Good Bokeh, Flickr, \ccby{2}}
\end{frame}
}


%{
%\usebackgroundtemplate{\includegraphics[width=\paperwidth]{162_students} }
%\begin{frame}[b,plain]
%\end{frame}
%}

{
\usebackgroundtemplate{\includegraphics[width=\paperwidth]{C_or_better} }
\begin{frame}[b,plain]
\end{frame}
}


\begin{frame}{\highlight{Rise} to the occasion.}

\hangpara\begin{quote}The exams do not need to be so hard and tricky in order to measure how much we have learned. You can tell how much we have learned with an easier test.
\end{quote}

\hangpara\begin{quote}I really do enjoy your tests. It is not just definitions, you made us apply what we learn to difficult questions.
\end{quote}

%\hangpara You have to \highlight{rise to the occasion.} I will not bring the occasion down to you.

\end{frame}

{
\usebackgroundtemplate{\includegraphics[width=\paperwidth]{hotrod}
}

\begin{frame}[t]{To be \highlight{successful}, you must}
\vspace{1ex}
\hspace{22em}\highlight{C}omplete,
\pause

\vspace{1ex}
\hspace{22em}\highlight{A}ttend,
\pause

\vspace{1ex}
\hspace{22em}\highlight{R}ead,
\pause

\vspace{1ex}
\hspace{22em}\highlight{S}tudy.

%\vspace{25ex}
\vskip 0pt plus 1filll

\tiny 1950 Mercury photo \copyright\,John O'Neill, All Rights Reserved.
\end{frame}
}

{
\usebackgroundtemplate{\includegraphics[width=\paperwidth]{take_good_notes}
}
\begin{frame}[b,plain]{To be successful, you must \highlight{take good notes.}}
\textcolor{white}{\tiny Daniel Foster, Flickr, \ccby{2}}
\end{frame}
}

{
\usebackgroundtemplate{\includegraphics[width=\paperwidth]{find_balance_stones}
}
\begin{frame}[b,plain]
\hfill	\tiny bibigeek, Flickr, \ccby{2}
\end{frame}
}


{
\usebackgroundtemplate{\includegraphics[width=\paperwidth]{think_horizontally}
}
\begin{frame}[plain]
\end{frame}
}

{
\usebackgroundtemplate{\includegraphics[width=\paperwidth]{connect_the_dots}
}
\begin{frame}[t,plain]
\end{frame}
}

\lecture{instructor}{instructor}
{
\usebackgroundtemplate{\includegraphics[width=\paperwidth]{connect_the_dots_solved}}
\begin{frame}[t,plain]
\end{frame}
}

%{
%\usebackgroundtemplate{\includegraphics[width=\paperwidth]{sunflowers}
%}
%\begin{frame}[b,plain]
%	\textcolor{white}{\Tiny Sunflowers by Trey Ratcliff, Flickr, Creative Commons.}
%\end{frame}
%}
%
%{
%\usebackgroundtemplate{\includegraphics[width=\paperwidth]{spider_bee}
%}
%\begin{frame}[b,plain]
%\Tiny Goldenrod crab spider photo by Alvesgaspar, Wikimedia Commons.
%\end{frame}
%}
%
%{
%\usebackgroundtemplate{\includegraphics[width=\paperwidth]{marine_iguana}}
%\begin{frame}[b,plain]
%	\textcolor{white}{\Tiny Marine iguana on San Crist\'{o}bal Island, Galapagos by Les Williams, Flickr, Creative Commons.}
%\end{frame}
%}
%
%{
%\usebackgroundtemplate{\includegraphics[width=\paperwidth]{bird_paradise.jpg}}
%\begin{frame}[b,plain]
%	\Tiny\textcolor{white}{Greater Bird of Paradise \textcopyright Tim Laman, All Rights Reserved. \href{http://www.youtube.com/watch?v=KIYkpwyKEhY}{Link to Video} }
%\end{frame}
%}
%
%\lecture{instructor}{instructor}
%
%{
%\usebackgroundtemplate{\includegraphics[width=\paperwidth]{goliath_beetle}}
%\begin{frame}[b,plain]
%%	\hfill\Tiny \textit{Eupatorus gracilicornis}, Didier Descouens, Wikimedia Commons.
%\end{frame}
%}
%
%
%{
%\usebackgroundtemplate{\includegraphics[width=\paperwidth]{beetle_fondness}}
%\begin{frame}[b,plain]
%	\hfill\Tiny \textit{Eupatorus gracilicornis}, Didier Descouens, Wikimedia Commons.
%\end{frame}
%}
%

\end{document}
