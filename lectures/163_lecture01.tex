%!TEX TS-program = lualatex
%!TEX encoding = UTF-8 Unicode

\documentclass[t]{beamer}

%%%% HANDOUTS For online Uncomment the following four lines for handout
%\documentclass[t,handout]{beamer}  %Use this for handouts.
%\usepackage{handoutWithNotes}
%\pgfpagesuselayout{3 on 1 with notes}[letterpaper,border shrink=5mm]

%\includeonlylecture{student}

%%% Including only some slides for students.
%%% Uncomment the following line. For the slides,
%%% use the labels shown below the command.

%% For students, use \lecture{student}{student}
%% For mine, use \lecture{instructor}{instructor}


%\usepackage{pgf,pgfpages}
%\pgfpagesuselayout{4 on 1}[letterpaper,border shrink=5mm]

% FONTS
\usepackage{fontspec}
\def\mainfont{Linux Biolinum O}
\setmainfont[Ligatures={Common,TeX}, Contextuals={NoAlternate}, BoldFont={* Bold}, ItalicFont={* Italic}, Numbers={OldStyle}]{\mainfont}
\setsansfont[Ligatures={Common,TeX}, Scale=MatchLowercase, Numbers=OldStyle]{Linux Biolinum O} 
\usepackage{microtype}

\usepackage{graphicx}
	\graphicspath{{/Users/goby/pictures/teach/163/lecture/}
	{/Users/goby/pictures/teach/common/}} % set of paths to search for images

%\usepackage{units}
\usepackage{booktabs}
%\usepackage{textcomp}

\usepackage{tikz}
%	\tikzstyle{every picture}+=[remember picture,overlay]

\mode<presentation>
{
  \usetheme{Lecture}
  \setbeamercovered{invisible}
  \setbeamertemplate{items}[square]
}


\begin{document}

\lecture{instructor}{instructor}
{
\usebackgroundtemplate{\includegraphics[width=\paperwidth]{connect_the_dots_solved}}
\begin{frame}[t,plain]
\end{frame}
}

{
\usebackgroundtemplate{\includegraphics[width=\paperwidth]{sunflowers}
}
\begin{frame}[b,plain]
	\hfill \tiny \textcolor{white}{Trey Ratcliff, Flickr, \ccbyncsa{2}}
\end{frame}
}

{
\usebackgroundtemplate{\includegraphics[width=\paperwidth]{spider_bee}
}
\begin{frame}[b,plain]
\tiny Alvesgaspar, Wikimedia, \ccbysa{4}
\end{frame}
}

{
\usebackgroundtemplate{\includegraphics[width=\paperwidth]{marine_iguana}}
\begin{frame}[b,plain]
	\textcolor{white}{\tiny Les Williams, Flickr, \ccbysa{2}}
\end{frame}
}

{
\usebackgroundtemplate{\includegraphics[width=\paperwidth]{bird_paradise.jpg}}
\begin{frame}[b,plain]
	\tiny\textcolor{white}{\href{http://www.youtube.com/watch?v=KIYkpwyKEhY}{Link to Video} \hfill \copyright\,Tim Laman, All Rights Reserved}
\end{frame}
}

\lecture{student}{student}

\begin{frame}{Our goals for this lecture are}
	
	\hangpara define evolution, 
	
	\hangpara learn the historical context of evolutionary thought, and

	\hangpara introduce Charles Darwin.

\end{frame}

%

%
\begin{frame}[t,plain]{Work in pairs to answer the following questions.}
	\vspace{2ex}
	
	\onslide<1->\hangpara What is \highlight{evolution?} \onslide<2->\alt<handout>{}{\\Evolution is the genetic change in a population over time.}\\[2\baselineskip]
	

	\onslide<1->\hangpara Who was \highlight{Charles Darwin?}

\end{frame}

%
{
\usebackgroundtemplate{\includegraphics[width=\paperwidth]{evol_history_pre-darwin}}
\begin{frame}[b]

\hfill\tiny\textcopyright Pearson Education, Inc.
\end{frame}
}
%
{
\usebackgroundtemplate{\includegraphics[width=\paperwidth]{evol_history_post-darwin}}
\begin{frame}[b]

\hfill\tiny\textcopyright Pearson Education, Inc.
\end{frame}
}
%
{
\usebackgroundtemplate{\includegraphics[width=\paperwidth]{thomas_robert_malthus}}
\begin{frame}[b]

\tiny\textcolor{white}{Wikimedia, public domain.}
\end{frame}
}
%
{
\usebackgroundtemplate{\includegraphics[width=\paperwidth]{malthus_graph}}
\begin{frame}[t]{Malthus's idea on human population growth and catastrophe.}

\vskip0pt plus 1filll

\hfill\tiny Kravietz, Wikimedia,\ccbysa{3}
\end{frame}
}
%
{
\usebackgroundtemplate{\includegraphics[width=\paperwidth]{lamarck}}
\begin{frame}[b]

\tiny\textcolor{white}{Wikimedia, public domain.}\hspace*{50mm}Petr Kratochvil, public domain.
\end{frame}
}
%
{
\usebackgroundtemplate{\includegraphics[width=\paperwidth]{georges_cuvier}}
\begin{frame}[b]

\tiny\textcolor{white}{Wikimedia, public domain.}
\end{frame}
}
%
{
\usebackgroundtemplate{\includegraphics[width=\paperwidth]{charles_lyell}}
\begin{frame}[b]

\tiny\textcolor{white}{Wikimedia, public domain.}
\end{frame}
}
%
{
\usebackgroundtemplate{\includegraphics[width=\paperwidth]{darwin_young}}
\begin{frame}[b]

\hfill\tiny\textcolor{white}{Wikimedia, public domain.}
\end{frame}
}
%
\begin{frame}[t]{Darwin's father objected to Charles going on the voyage.}

	Such a voyage would reflect badly on his future prospects as a member of the clergy. \\[0.5\baselineskip]
	
	The entire plan seemed adventurous and wild. \\[0.5\baselineskip]

	Why was a naturalist still being considered so close to the start of the voyage? \\[0.5\baselineskip]
	
	Given the above, other people must have been considered. Why had they refused the offer? \\[0.5\baselineskip]
	
	Going on the voyage would prevent Charles from settling down to a real life. \\[0.5\baselineskip]
	
	The accommodations on the ship would be very poor. \\[0.5\baselineskip]
	
	The voyage would offer Charles another excuse to change his focus in life. \\[0.5\baselineskip]
	
	It would be a complete waste of his time.
\end{frame}
%
{
\usebackgroundtemplate{\includegraphics[width=\paperwidth]{hms_beagle}}
\begin{frame}[b]

\hfill\tiny\textcolor{white}{Wikimedia, public domain.}
\end{frame}
}
%
{
\usebackgroundtemplate{\includegraphics[width=\paperwidth]{darwin_cabin}
}
\begin{frame}[plain]
\end{frame}
}
%
{
\usebackgroundtemplate{\includegraphics[width=\paperwidth]{voyage_beagle}}
\begin{frame}[t]{\textcolor{white}{The \textsc{h.m.s.} Beagle sailed around the world during 1831--1836.}}
\end{frame}
}
%
{
\usebackgroundtemplate{\includegraphics[width=\paperwidth]{giant_sloth}}
\begin{frame}[t]{In South America, Darwin witnessed\ldots}

	\hangpara fossils of a giant sloth and other animals,

	\hangpara marine shells on a mountain top, and

	\hangpara a magnitude 8 earthquake,

	\hangpara the Galapagos Islands.

	\vskip0pt plus 1filll

	\hfill\tiny ДиБгд, Wikimedia, public domain.
\end{frame}
}
%
{
\usebackgroundtemplate{\includegraphics[width=\paperwidth]{darwins_finches}}
\begin{frame}[b]

	\hfill\tiny Wikimedia, public domain.
\end{frame}
}
%
{
\usebackgroundtemplate{\includegraphics[width=\paperwidth]{darwin_descent_quote.jpg}}
\begin{frame}[plain]
\end{frame}
}
%
{
\usebackgroundtemplate{\includegraphics[width=\paperwidth]{darwin_phylogeny}}
\begin{frame}[t]
\end{frame}
}
%
{
\usebackgroundtemplate{\includegraphics[width=\paperwidth]{artificial_selection}}
\begin{frame}[t]{}
\end{frame}
}
%
{
\usebackgroundtemplate{\includegraphics[width=\paperwidth]{alfred_russel_wallace}}
\begin{frame}[b]
\hfill\tiny\textcolor{white}{Wikimedia, public domain.}
\end{frame}
}


{
\usebackgroundtemplate{\includegraphics[width=\paperwidth]{origin_of_species_cover}}
\begin{frame}[t]

	\vspace*{3\baselineskip}

	\hangpara\parbox[t]{2.25in}{\raggedright %
There is grandeur in this view of life, with its several powers, having been originally breathed into a few forms or into one; and that, whilst this planet has gone cycling on according to the fixed law of gravity, from so simple a beginning endless forms most beautiful and most wonderful have been, and are being, evolved.}

	\vskip0pt plus 1filll
	
	\hfill\tiny Wikimedia, public domain.
\end{frame}
}
%
{
\usebackgroundtemplate{\includegraphics[width=\paperwidth]{origin_of_species_cover}}
\begin{frame}[t]

	\vspace*{3\baselineskip}
	
	\hangpara Darwin's book had two ideas:
	
	\hangpara\hspace*{1em}\highlight{descent with modification} 
	
	\hangpara\hspace*{1em}\highlight{natural selection.}

	\vskip0pt plus 1filll
	
	\hfill\tiny Wikimedia, public domain.
\end{frame}
}






%Quote from Origin of Species

%{
%\usebackgroundtemplate{\includegraphics[width=\paperwidth]{darwin_descent_quote}
%}
%\begin{frame}
%\end{frame}
%}

%{
%\usebackgroundtemplate{\includegraphics[width=\paperwidth]{kitty_yawn.jpg}
%}
%\begin{frame}{\textcolor{white}{You \highlight{earn} your grades with}}
%\end{frame}
%}

\end{document}
