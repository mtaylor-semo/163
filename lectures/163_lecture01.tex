%!TEX TS-program = lualatex
%!TEX encoding = UTF-8 Unicode

\documentclass[t]{beamer}

%%%% HANDOUTS For online Uncomment the following four lines for handout
%\documentclass[t,handout]{beamer}  %Use this for handouts.
%\includeonlylecture{student}
%\usepackage{handoutWithNotes}
%\pgfpagesuselayout{3 on 1 with notes}[letterpaper,border shrink=5mm]

%%% Including only some slides for students.
%%% Uncomment the following line. For the slides,
%%% use the labels shown below the command.

%% For students, use \lecture{student}{student}
%% For mine, use \lecture{instructor}{instructor}


%\usepackage{pgf,pgfpages}
%\pgfpagesuselayout{4 on 1}[letterpaper,border shrink=5mm]

% FONTS
\usepackage{fontspec}
\def\mainfont{Linux Biolinum O}
\setmainfont[Ligatures={Common,TeX}, Contextuals={NoAlternate}, BoldFont={* Bold}, ItalicFont={* Italic}, Numbers={OldStyle}]{\mainfont}
\setsansfont[Ligatures={Common,TeX}, Scale=MatchLowercase, Numbers=OldStyle]{Linux Biolinum O} 
\usepackage{microtype}

\usepackage{graphicx}
	\graphicspath{{/Users/mtaylor/pictures/teach/163/lecture/}
	{/Users/mtaylor/pictures/teach/common/}} % set of paths to search for images

%\usepackage{units}
\usepackage{booktabs}
%\usepackage{textcomp}

\usepackage{tikz}
	\tikzstyle{every picture}+=[remember picture,overlay]
	\usetikzlibrary{arrows}

\mode<presentation>
{
  \usetheme{Lecture}
  \setbeamercovered{invisible}
  \setbeamertemplate{items}[square]
}


\begin{document}

%\lecture{instructor}{instructor}
%{
%\usebackgroundtemplate{\includegraphics[width=\paperwidth]{connect_the_dots_solved}}
%\begin{frame}[t,plain]
%\end{frame}
%}
%
%{
%\usebackgroundtemplate{\includegraphics[width=\paperwidth]{sunflowers}
%}
%\begin{frame}[b,plain]
%	\hfill \tiny \textcolor{white}{Trey Ratcliff, Flickr, \ccbyncsa{2}}
%\end{frame}
%}
%
%{
%\usebackgroundtemplate{\includegraphics[width=\paperwidth]{spider_bee}
%}
%\begin{frame}[b,plain]
%\tiny Alvesgaspar, Wikimedia, \ccbysa{4}
%\end{frame}
%}
%
%{
%\usebackgroundtemplate{\includegraphics[width=\paperwidth]{marine_iguana}}
%\begin{frame}[b,plain]
%	\textcolor{white}{\tiny Les Williams, Flickr, \ccbysa{2}}
%\end{frame}
%}
%
%{
%\usebackgroundtemplate{\includegraphics[width=\paperwidth]{bird_paradise.jpg}}
%\begin{frame}[b,plain]
%	\tiny\textcolor{white}{\href{http://www.youtube.com/watch?v=KIYkpwyKEhY}{Link to Video} \hfill \copyright\,Tim Laman, All Rights Reserved}
%\end{frame}
%}

\lecture{student}{student}

\begin{frame}{Our goals for this lecture are to}
	
	\hangpara define \highlight{evolution,}
	
	\hangpara learn the relationship between \highlight{characters} and \highlight{genes}, and between \highlight{traits} and \highlight{alleles}, and
	
	\hangpara learn the relationship between \highlight{genotype} and \highlight{phenotype.}

\end{frame}
%
%\lecture{instructor}{instructor}
{
\usebackgroundtemplate{\includegraphics[width=\paperwidth]{dobzhansky_sense}}
\begin{frame}

\end{frame}
}
%
%\lecture{student}{student}

\begin{frame}[t,plain]{Work in pairs to answer the following question.}
	\vspace{2ex}
	
	\onslide<1->\hangpara What is \highlight{evolution?} 

	\onslide<2->\alt<handout>{}{%
	\vspace{2\baselineskip}
	Evolution is the genetic change in a population over time.}
	
%	\onslide<3->\alt<handout>{}{%
%	\vspace{2\baselineskip}
%	Let us break down this definition into its component parts.}
\end{frame}

%
{
\usebackgroundtemplate{\includegraphics[width=\paperwidth]{population_bees}}
\begin{frame}[b]{What is a \highlight{population?}}

\hfill \tiny \textcolor{white}{PolyDot, Pixabay \cc0.}
\end{frame}
}
%
{
\usebackgroundtemplate{\includegraphics[width=\paperwidth]{population_time}}
\begin{frame}[b]

\hfill \tiny OpenClipArt-Vectors, Pixabay \cc0
\end{frame}
}
%
% Pronounced Pod Mer-KAR-oo and Pod Ko-PIS-tee.
{
\usebackgroundtemplate{\includegraphics[width=\paperwidth]{italian_wall_lizard}}
\begin{frame}[b]{\strut Italian wall lizards were introduced to the island of Pod Mr\v{c}aru from Pod Kipi\v{s}te in 1971.}


\hfill\tiny Alexandre Roux, Flickr, \ccbyncsa{2}
\end{frame}
}
%
{
\usebackgroundtemplate{\includegraphics[width=\paperwidth]{lizards_Croatia_map}}
\begin{frame}
\end{frame}
}
%
{
\usebackgroundtemplate{\includegraphics[width=\paperwidth]{lizards_head_size_bite_force}}
\begin{frame}[b]{How did the island populations differ after 36 years?}

\hfill \tiny Herrel et al. 2008. \textsc{pnas} 105:4792.
\end{frame}
}
%
{
\usebackgroundtemplate{\includegraphics[width=\paperwidth]{lizards_diet}}
\begin{frame}[b]{How did the diet change after 36 years?}
	
\hfill \tiny Herrel et al. 2008. \textsc{pnas} 105:4792.
\end{frame}
}
%
{
	\usebackgroundtemplate{\includegraphics[width=\paperwidth]{lizards_cecal_valve}}
	\begin{frame}[b]{A cecal valve evolved in Mr\v{c}aru lizards that keeps plant matter in the gut for longer digestion.}
	
	\hfill \tiny Herrel et al. 2008. \textsc{pnas} 105:4792.
\end{frame}
}
%
% From Martin and Palumbi. Cytochrome b mutation rate slows
% with increasing body size. They hypothesize that generation
% time and metabolic rate explain the pattern, at least in part.
%{
%\usebackgroundtemplate{\includegraphics[width=\paperwidth]{mutation_rate_body_size}}
%\begin{frame}[b]{How fast can \highlight{genetic change} occur?}
%	
%\tiny Martin and Palumbi 1993. \textsc{pnas} 90:4087.
%\end{frame}
%}
%
%\begin{frame}[b]{\highlight{Random mutations} create genetic variation in populations.}
%
%%\tiny\textcolor{white}{Wikimedia, public domain.}
%\end{frame}
%

{
\usebackgroundtemplate{\includegraphics[width=\paperwidth]{chromosome_numbers_hidden}}
\begin{frame}[b]{Which species has the greatest number of chromosomes? The least?}

%\tiny\textcolor{white}{Wikimedia, public domain.}
\end{frame}
}
%
\lecture{instructor}{instructor}
%
{
\usebackgroundtemplate{\includegraphics[width=\paperwidth]{chromosome_numbers_visible}}
\begin{frame}[b]{The number of chromosomes is not related to organismal complexity.}

%\tiny\textcolor{white}{Wikimedia, public domain.}
\end{frame}
}
%
\lecture{student}{student}
%
{
\usebackgroundtemplate{\includegraphics[width=\paperwidth]{chromosome_numbers_hidden}}
\begin{frame}[b]{Which species has the greatest number of genes? The least?}

%\tiny\textcolor{white}{Wikimedia, public domain.}
\end{frame}
}
%
\lecture{instructor}{instructor}
%
{
\usebackgroundtemplate{\includegraphics[width=\paperwidth]{chromosome_numbers_genes}}
\begin{frame}[b]{Organisms share many or most of the same genes.}

%\tiny\textcolor{white}{Wikimedia, public domain.}
\end{frame}
}
%
\lecture{student}{student}
%
{
\usebackgroundtemplate{\includegraphics[width=\paperwidth]{genes_characters}}
\begin{frame}[b]

\tiny Erikeltic, Wikimedia \ccbysa{3}
\end{frame}
}
%
{
\usebackgroundtemplate{\includegraphics[width=\paperwidth]{alleles_traits}}
\begin{frame}[b]
	
\tiny  dmealiffe, Flickr \ccbysa{2} (top)\\Herwig Kavallar, Wikimedia public domain (bottom)
\end{frame}
}
%

{
\usebackgroundtemplate{\includegraphics[width=\paperwidth]{phenotype_genotype}}
\begin{frame}[b]
	
\tiny  Pharaoh Hound, Flickr \ccbysa{2} (top)\\Home-Skilz, Wikimedia public domain (bottom)
\end{frame}
}
%%
%% Old slides for phenotype and genotype. Stored here in case.
%%
%\begin{frame}[t]{The traits of an organism is called the \highlight{phenotype.}}
%	\centering
%	\includegraphics[width=0.8\textwidth]{individuals_vary_plain}\par
%	
%	\vfilll
%	
%	\hfill\tiny \copyright\, Entomart, Wikimedia
%
%\end{frame}
%%
%\begin{frame}[t]{What causes the \highlight{phenotype} to vary?}
%	\centering
%	\includegraphics[width=0.8\textwidth]{individuals_vary_plain}
%	
%		\vfilll
%	
%	\hfill\tiny \copyright\, Entomart, Wikimedia
%
%\end{frame}
%%
%\begin{frame}[t]{The phenotype is determined by the \highlight{genotype.}}
%
%	\begin{multicols}{2}
%	
%		\includegraphics[width=0.45\textwidth]{punnett_square_genotype_phenotype}
%	
%	\columnbreak
%
%	The genotype is the genetic makeup of the organism.\vspace*{2\baselineskip}
%	
%	The genotype is the combination of alleles for one or more gene.
%	
%	\end{multicols}
%	
%	\vfilll
%	
%	\tiny Madeleine Price Ball, Wikimedia, \textsc{cc0}
%
%\end{frame}
%
%% Genotype and phenotype
%%\lecture{instructor}{instructor}
%{
%\usebackgroundtemplate{\includegraphics[width=\paperwidth]{chromosomes}}
%\begin{frame}[c,plain]
%	\begin{tikzpicture}[remember picture, overlay]
%
%	\alt<handout>{}{\visible <1,6->{
%		\node at (0.45,2) [right] {The \highlight{genotype} is the genetic makeup of the organism.};}
%	
%	\visible <7>{
%		\node at (0.45,1.4)[right] {The \highlight{phenotype} is the physical expression of the genotype.};
%	}}
%		
%	
%	\alt<handout>{}{\visible <2>{
%		\draw (3,-2.5) -- (3,0.5) -- (3.9,0.5) -- (3.9,-2.5) -- cycle ;
%		\draw (6,-2.5) -- (6,0.5) -- (6.9,0.5) -- (6.9,-2.5) -- cycle ;
%		\draw (9,-2.5) -- (9,0.5) -- (9.9,0.5) -- (9.9,-2.5) -- cycle ;
%	
%		\node at (6.4, 2) (gene) {Gene or Locus};
%	
%		\draw (gene.south west) -- (3.9,0.5);
%		\draw (gene.south) -- (6.45,0.5);
%		\draw (gene.south east) -- (9,0.5);
%	}}
%	
%	\visible <3-4>{
%		\node at (3.45,0.25) (Callele) {$C$};
%		\node at (3.45,-2.25) {$C$};
%
%		\node at (6.45,0.25) (aallele) {$a$};
%		\node at (6.45,-2.25) {$a$};
%	}
%
%	\visible <3,5>{
%		\node at (9.45,0.25) (Tallele) {$T$};
%		\node at (9.45,-2.25) {$t$};
%	}
%	
%	\alt<handout>{}{\visible <3>{
%		\node at (6.4, 2) (allele) {Allele};
%		\draw (allele.south west) -- (Callele.north east);
%		\draw (allele.south) -- (aallele.north);
%		\draw (allele.south east) -- (Tallele.north west);
%	}}
%
%	\visible <4> {
%		\draw (3,-2.5) -- (3,0.5) -- (3.9,0.5) -- (3.9,-2.5) -- cycle ;
%		\draw (6,-2.5) -- (6,0.5) -- (6.9,0.5) -- (6.9,-2.5) -- cycle ;
%
%		\node at (3.45, 1.25) (hc) {\highlight{Homozygous,} $C$ allele};
%		\node at (6.45, -3.5) (ac) {Homozygous, $a$ allele};
%
%		\draw (hc.south) -- (3.45,0.5);
%		\draw (ac.north) -- (6.45,-2.5);
%	}
%	
%	\visible <5> {
%		\draw (9,-2.5) -- (9,0.5) -- (9.9,0.5) -- (9.9,-2.5) -- cycle ;
%
%		\node at (9.45, 1.25) (tc) {\highlight{Heterozygous,} $T$ allele};
%
%		\draw (tc.south) -- (9.45,0.5);
%	}
%	\end{tikzpicture}
%\end{frame}
%}


\begin{frame}[t]{Evolution summary}

	\hangpara \highlight{Population:} a group of individuals of the \emph{same species} together in the same area, and \emph{potentially} interacting with each other.
	
	\pause\hangpara \highlight{Time:} measurable differences can take dozens to hundreds or thousands of generations.
	
	\pause\hangpara \highlight{Genetic change:} allele frequencies change in the population over time.
	
	\pause\hangpara Evolution of the genotype causes evolution of the phenotype.

\end{frame}


\end{document}
