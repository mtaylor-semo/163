%!TEX TS-program = lualatex
%!TEX encoding = UTF-8 Unicode

%\documentclass[t]{beamer}

%%%% HANDOUTS For online Uncomment the following four lines for handout
\documentclass[t,handout]{beamer}  %Use this for handouts.
\usepackage{handoutWithNotes}
\pgfpagesuselayout{3 on 1 with notes}[letterpaper,border shrink=5mm]

\includeonlylecture{student}

%%% Including only some slides for students.
%%% Uncomment the following line. For the slides,
%%% use the labels shown below the command.

%% For students, use \lecture{student}{student}
%% For mine, use \lecture{instructor}{instructor}


%\usepackage{pgf,pgfpages}
%\pgfpagesuselayout{4 on 1}[letterpaper,border shrink=5mm]

% FONTS
\usepackage{fontspec}
\def\mainfont{Linux Biolinum O}
\setmainfont[Ligatures={Common,TeX}, Contextuals={NoAlternate}, BoldFont={* Bold}, ItalicFont={* Italic}, Numbers={OldStyle}]{\mainfont}
\setsansfont[Ligatures={Common,TeX}, Scale=MatchLowercase, Numbers=OldStyle]{Linux Biolinum O} 
\usepackage{microtype}

\usepackage{graphicx}
	\graphicspath{{/Users/goby/pictures/teach/163/lecture/}
	{/Users/goby/pictures/teach/common/}} % set of paths to search for images

%\usepackage{units}
\usepackage{booktabs}
\usepackage{multicol}

\usepackage{tikz}
	\tikzstyle{every picture}+=[remember picture,overlay]
	\usetikzlibrary{trees}

\mode<presentation>
{
  \usetheme{Lecture}
  \setbeamercovered{invisible}
  \setbeamertemplate{items}[square]
}

%\usefonttheme[onlymath]{serif}
%\usecolortheme[named=blue7]{structure}

\newcommand{\btVFill}{\vskip0pt plus 1filll}

\begin{document}

%
\lecture{instructor}{instructor}
{
\usebackgroundtemplate{\includegraphics[width=\paperwidth]{vertebrate_phylo1}}
\begin{frame}[b]

	\tiny Fig. 34.2, \textcopyright Pearson Education, Inc.
\end{frame}
}
%
\lecture{student}{student}
%
{
\usebackgroundtemplate{\includegraphics[width=\paperwidth]{vertebrate_phylo2}}
\begin{frame}[t]{\onslide<2->{What is a fish?}}

%	\hangpara Phylogenies help our classification.
	
	\btVFill
	
	\tiny Fig. 34.2, \tiny \textcopyright Pearson Education, Inc.
\end{frame}
}
%
\lecture{instructor}{instructor}
{
\usebackgroundtemplate{\includegraphics[width=\paperwidth]{vertebrate_phylo3}}
\begin{frame}[t]{The classes of vertebrates I learned were}

	\vskip0pt plus 1filll
	
	\tiny Fig. 34.2, \tiny \textcopyright Pearson Education, Inc.
\end{frame}
}
%
\lecture{student}{student}
{
\usebackgroundtemplate{\includegraphics[width=\paperwidth]{vertebrate_phylo4}}
\begin{frame}[t]{The classes of vertebrates I learned were}

	\vspace*{-0.5\baselineskip}
	
	\hangpara “Fishes” are shaded. What is the problem with this classification?
		
\end{frame}
}
{
\usebackgroundtemplate{\includegraphics[width=\paperwidth]{vertebrate_phylo5}}
\begin{frame}[t]{The classes of vertebrates I teach are}

	\vspace*{-0.5\baselineskip}
	
	\hangpara So, what is a fish? \alt<handout>{}{\pause \highlight{You are!}}
	
\end{frame}
}
\end{document}
