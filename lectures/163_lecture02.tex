%!TEX TS-program = lualatex
%!TEX encoding = UTF-8 Unicode

\documentclass[t]{beamer}

%%%% HANDOUTS For online Uncomment the following four lines for handout
%\documentclass[t,handout]{beamer}  %Use this for handouts.
%\usepackage{handoutWithNotes}
%\pgfpagesuselayout{3 on 1 with notes}[letterpaper,border shrink=5mm]
%	\setbeamercolor{background canvas}{bg=black!5}


%%% Including only some slides for students.
%%% Uncomment the following line. For the slides,
%%% use the labels shown below the command.
%\includeonlylecture{student}

%% For students, use \lecture{student}{student}
%% For mine, use \lecture{instructor}{instructor}


%\usepackage{pgf,pgfpages}
%\pgfpagesuselayout{4 on 1}[letterpaper,border shrink=5mm]

% FONTS
\usepackage{fontspec}
\def\mainfont{Linux Biolinum O}
\setmainfont[Ligatures={Common,TeX}, Contextuals={NoAlternate}, BoldFont={* Bold}, ItalicFont={* Italic}, Numbers={Proportional}]{\mainfont}
\setsansfont[Scale=MatchLowercase]{Linux Biolinum O} 
\usepackage{microtype}

\usepackage{graphicx}
	\graphicspath{{/Users/goby/pictures/teach/163/lecture/}
	{/Users/goby/pictures/teach/common/}} % set of paths to search for images

%\usepackage{units}
\usepackage{booktabs}
%\usepackage{textcomp}

\usepackage{tikz}
%	\tikzstyle{every picture}+=[remember picture,overlay]

\mode<presentation>
{
  \usetheme{Lecture}
  \setbeamercovered{invisible}
  \setbeamertemplate{items}[square]
}

%\usefonttheme[onlymath]{serif}
%\usecolortheme[named=blue7]{structure}

\newcommand{\btVFill}{\vskip0pt plus 1filll}

\begin{document}

\lecture{student}{student}

\begin{frame}[t]{Our goals for this lecture are to}

	\hangpara learn the \highlight{Linnaean classification system,}
	
	\hangpara define and show examples of \highlight{binomial nomenclature},
	
	\hangpara define and show examples of \highlight{taxonomic classification}, and
	
	\hangpara learn that science is a process for understanding the natural world.
	
	
\end{frame}

\lecture{instructor}{instructor}

{
\usebackgroundtemplate{\includegraphics[width=\paperwidth]{bird_paradise.jpg}}
\begin{frame}[b,plain]
	\Tiny\textcolor{white}{Greater Bird of Paradise \textcopyright Tim Laman, All Rights Reserved. \href{http://www.youtube.com/watch?v=KIYkpwyKEhY}{Link to Video} }
\end{frame}
}

{
\usebackgroundtemplate{\includegraphics[width=\paperwidth]{goliath_beetle}}
\begin{frame}[b,plain]
%	\hfill\Tiny \textit{Eupatorus gracilicornis}, Didier Descouens, Wikimedia Commons.
\end{frame}
}

{
\usebackgroundtemplate{\includegraphics[width=\paperwidth]{beetle_fondness}}
\begin{frame}[b,plain]
	\hfill\tiny \textit{Eupatorus gracilicornis}, Didier Descouens, Wikimedia Commons.
\end{frame}
}

\lecture{student}{student}
{
\usebackgroundtemplate{\includegraphics[width=\paperwidth]{how_many_species}}
\begin{frame}[b,plain]
	\tiny The Emirr, Wikimedia Commons.
\end{frame}
}


\lecture{instructor}{instructor}
{
\usebackgroundtemplate{\includegraphics[width=\paperwidth]{table_of_diversity}}
\begin{frame}[b,plain]{About 1.2 million species of \textasciitilde11 million species have been described.}

	\vspace*{13\baselineskip}
	
	{\large Should these species be organized by similarity or relatedness? }

	\btVFill

	\hfill\tiny Mora et al. 2011, PLoS ONE.
\end{frame}
}

\lecture{student}{student}
{
\usebackgroundtemplate{\includegraphics[width=\paperwidth]{systema_naturae}}
\begin{frame}[b,plain]
\end{frame}
}

\lecture{student}{student}
{
\usebackgroundtemplate{\includegraphics[width=\paperwidth]{classification_hierarchy}}
\begin{frame}[b,plain]
\end{frame}
}

{
\usebackgroundtemplate{\includegraphics[width=\paperwidth]{binomial_nomenclature}}
\begin{frame}[b,plain]
\end{frame}
}

{
\usebackgroundtemplate{\includegraphics[width=\paperwidth]{taxonomic_classification}}
\begin{frame}[b,plain]
\end{frame}
}

% ONLY GOT TO HERE ON FIRST ATTEMPT.

%{
%\usebackgroundtemplate{\includegraphics[width=\paperwidth]{what_is_science}}
%\begin{frame}[b,plain]
%
%\hspace*{83mm}\tiny\textcolor{white}{J.J, Wikimedia Commons.}
%\end{frame}
%}
%
%\lecture{instructor}{instructor}
%
%{
%\usebackgroundtemplate{\includegraphics[width=\paperwidth]{science_tentative_new1}}
%\begin{frame}[b,plain]{Science is \highlight{tentative.}}
%\end{frame}
%}
%
%\lecture{student}{student}
%
%{
%\usebackgroundtemplate{\includegraphics[width=\paperwidth]{science_tentative_new2}}
%\begin{frame}[b,plain]{Science is \highlight{tentative.}}
%\end{frame}
%}
%
%{
%\usebackgroundtemplate{\includegraphics[width=\paperwidth]{science_objective}}
%\begin{frame}[b,plain]{Science is \highlight{objective.}}
%\end{frame}
%}
%
%{
%\usebackgroundtemplate{\includegraphics[width=\paperwidth]{science_testable}}
%\begin{frame}[b,plain]{Science is \highlight{testable.}}
%\end{frame}
%}
%
%{
%\usebackgroundtemplate{\includegraphics[width=\paperwidth]{reasoning_deductive}}
%\begin{frame}[b,plain]
%
%\hfill\tiny Pixabay.com
%\end{frame}
%}
%
%
%{
%\usebackgroundtemplate{\includegraphics[width=\paperwidth]{reasoning_inductive}}
%\begin{frame}[b,plain]
%
%\tiny\textcolor{white}{decltype, Wikimedia Commons.}
%\end{frame}
%}
%
%\lecture{instructor}{instructor}
%
%{
%\usebackgroundtemplate{\includegraphics[width=\paperwidth]{einstein_guitar}}
%\begin{frame}[b,plain]
%\end{frame}
%}

%{
%\usebackgroundtemplate{\includegraphics[width=\paperwidth]{albert_einstein_wrong}}
%\begin{frame}[b,plain]
%\end{frame}
%}

\end{document}
