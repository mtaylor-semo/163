%!TEX TS-program = lualatex
%!TEX encoding = UTF-8 Unicode

\documentclass[t]{beamer}

%%%% HANDOUTS For online Uncomment the following four lines for handout
%\documentclass[t,handout]{beamer}  %Use this for handouts.
%\includeonlylecture{student}
%\usepackage{handoutWithNotes}
%\pgfpagesuselayout{3 on 1 with notes}[letterpaper,border shrink=5mm]


%% For students, use \lecture{student}{student}
%% For mine, use \lecture{instructor}{instructor}


% FONTS
\usepackage{fontspec}
\def\mainfont{Linux Biolinum O}
\setmainfont[Ligatures={Common,TeX}, Contextuals={NoAlternate}, BoldFont={* Bold}, ItalicFont={* Italic}, Numbers={OldStyle}]{\mainfont}
\setsansfont[Ligatures={Common,TeX}, Scale=MatchLowercase, Numbers=Proportional]{Linux Biolinum O} 
\usepackage{microtype}

\usepackage{graphicx}
	\graphicspath{{/Users/mtaylor/pictures/teach/163/lecture/}
	{/Users/mtaylor/pictures/teach/common/}} % set of paths to search for images

%\usepackage{units}
\usepackage{booktabs}
\usepackage{longtable}
%\usepackage{textcomp}
\usepackage{multicol}
\usepackage{amsmath}

\usepackage{tikz}
	\tikzstyle{every picture}+=[remember picture,overlay]
	\usetikzlibrary{arrows}
	\usetikzlibrary{positioning}
	
	
\usepackage{pgf-pie}

	
\mode<presentation>
{
  \usetheme{Lecture}
  \setbeamercovered{invisible}
  \setbeamertemplate{items}[default]
  \setbeamercolor{alerted text}{fg=orange6}
}


\begin{document}
%
%\begin{frame}
%	\begin{tikzpicture}
%		\pie[radius=2.5, sum=auto, pos={3.2,-4}, text=inside, color={black!20, white}]{0.6/\textit{a} allele, 0.40/\textit{A} allele}
%		\pie[radius=2.5, pos={9.6,-4}, text=inside, color={black!20, black!50, white}]{36/\textit{aa}, 48/\textit{Aa}, 16/\textit{AA}}
%	\end{tikzpicture}
%	
%	\begin{tikzpicture}
%		\node at (3.2,-6.5) (allelefreq){Allele Frequency};
%		\node at (9.6,-6.5) (genofreq){Genotype Frequency};
%	\end{tikzpicture}
%	
%\end{frame}
%
\lecture{student}{student}

\begin{frame}{Our goals for this lecture are to}
	
	\hangpara learn how to calculate \highlight{allele frequencies} and \highlight{genotype frequencies,}
	
	\hangpara learn the mathematical relationship between allele and genotype frequencies, and
	
	\hangpara learn about \highlight{Hardy-Weinberg equilibrium.}
	
\end{frame}
%
\begin{frame}[t]{Evolution summary}

	\hangpara \highlight{Population:} a group of individuals of the \emph{same species} together in the same area, and \emph{potentially} interacting with each other.
	
	\hangpara \highlight{Time:} measurable differences can take dozens to hundreds or thousands of generations.
	
	\hangpara \highlight{Genetic change:} allele frequencies change in the population over time.
	
	\hangpara \highlight{How do we calculate allele frequencies?}

\end{frame}

\begin{frame}[t]{Here is how I will represent alleles.}

	\hangpara  I will use upper case letters to represent genes. I will use superscripts to indicate different alleles for that gene. 
		
	\hangpara $A_1$ and $A_2$, $C_1$ and $C_2$, etc.
	
	\hangpara For now, ignore dominant and recessive alleles. They do not matter for calculating allele frequencies.
%	
%%	\hangpara The population has \emph{only} two unique alleles for each gene.\\ \quad \textcolor{gray}{Real populations often have many unique alleles for each gene.}
%	
%	\hangpara Individuals in the population are diploid; that is, each individual has two alleles for each gene.
%	
%	\hangpara For now, ignore dominant and recessive alleles. I will use upper case letters to represent genes. I will use superscripts to indicate different alleles for that gene. 
%	
%	\hangpara $A_1$ and $A_2$, $C_1$ and $C_2$, etc.
%	
\end{frame}
%
\begin{frame}{Here are some alleles. Answer the following questions.}
	\hfil $A_2$ \quad $A_2$ \quad $A_1$ \quad $A_2$ \quad $A_1$ \quad $A_1$ \quad $A_2$ \quad $A_2$ \quad $A_2$ \quad $A_1$ \hfil
	
	\vspace*{2\baselineskip}
		
	\hangpara How many total alleles are shown? \rule{0.5in}{0.4pt}
	
	\hangpara How many $A_1$ alleles? \rule{0.5in}{0.4pt}
	
	\hangpara How many $A_2$ alleles? \rule{0.5in}{0.4pt}
	
\end{frame}
%
\begin{frame}[t]{\highlight{Allele frequency} is a measure of how common an allele is in a population.}
	
	%	\vspace{\baselineskip}
	
	\hangpara Allele frequency (proportion) is calculated as the count of one allele divided by the total count for all alleles in a population.
	
	\hangpara Divide the number of $A_1$ alleles by the total number of alleles. \rule{0.5in}{0.4pt}
	
	\hangpara Divide the number of $A_2$ alleles by the total number of alleles. \rule{0.5in}{0.4pt}
	
	\hangpara This is the frequency of each allele in this population. Frequency is \emph{always} expressed as a decimal fraction.
	 
\end{frame}
%
\begin{frame}{The frequencies should always add up to 1.}
	\begin{tikzpicture}
	
	\pie[radius=2.5, sum=auto, pos={6,-3}, text=inside, color={black!20, white}]{0.60/\textit{$A_2$} allele, 0.40/\textit{$A_1$} allele}
%	\pie[radius=2.5, pos={9.6,-4}, text=inside, color={black!20, black!50, white}]{36/\textit{aa}, 48/\textit{Aa}, 16/\textit{AA}}
	\end{tikzpicture}
	
	\begin{tikzpicture}
	\node at (6,-5.5) (allelefreq){Allele Frequency};
%	\node at (9.6,-6.5) (genofreq){Genotype Frequency};
	\end{tikzpicture}
	
\end{frame}
%
\lecture{instructor}{instructor}

\begin{frame}[t]{Calculate the frequency of these five alleles.}
	
	\begin{longtable}[c]{@{}crc@{}}
		\toprule
		Allele	& Num.  & Frequency \tabularnewline
		\midrule
		& & \tabularnewline[-1ex]
		$Z_1$ & 6 & \rule{0.5in}{0.4pt} \tabularnewline[1.5ex]
		$Z_2$ & 12 & \rule{0.5in}{0.4pt} \tabularnewline[1.5ex]
		$Z_3$ & 18 & \rule{0.5in}{0.4pt} \tabularnewline[1.5ex]
		$Z_4$ & 1 & \rule{0.5in}{0.4pt} \tabularnewline[1.5ex]
		$Z_5$ & 3 & \rule{0.5in}{0.4pt} \tabularnewline[1.5ex]
		\bottomrule
	\end{longtable}
\end{frame}
%

\begin{frame}[t]{How did you do?}
	
	\begin{longtable}[c]{@{}crc@{}}
		\toprule
		Allele	& Num.  & Frequency \tabularnewline
		\midrule
		& & \tabularnewline[-1ex]
		$Z_1$ & 6 & {0.15} \tabularnewline[1.5ex]
		$Z_2$ & 12 & {0.30} \tabularnewline[1.5ex]
		$Z_3$ & 18 & {0.45} \tabularnewline[1.5ex]
		$Z_4$ & 1 & {0.025} \tabularnewline[1.5ex]
		$Z_5$ & 3 & {0.075} \tabularnewline[1.5ex]
		\bottomrule
	\end{longtable}
\end{frame}
%
\begin{frame}{The frequencies add up to 1.}
	\begin{tikzpicture}
	
	\pie[rotate=36, radius=2.5, sum=auto, pos={6,-3}, color={white}]{0.15/\textit{$Z_1$}, 0.30/\textit{$Z_2$}, 0.45/\textit{$Z_3$}, 0.025/\textit{$Z_4$}, 0.075/\textit{$Z_5$}}
	%	\pie[radius=2.5, pos={9.6,-4}, text=inside, color={black!20, black!50, white}]{36/\textit{aa}, 48/\textit{Aa}, 16/\textit{AA}}
	\end{tikzpicture}
	
%	\begin{tikzpicture}
%	\node at (6,-5.5) (allelefreq){Allele Frequency};
	%	\node at (9.6,-6.5) (genofreq){Genotype Frequency};
%	\end{tikzpicture}
	
\end{frame}

\lecture{student}{student}
%

%%%%%%%%
%\begin{frame}[t]{\highlight{Allele frequency} is a measure of how common an allele is in a population.}
%
%%	\vspace{\baselineskip}
%	
%	\hangpara Allele frequency (proportion) is calculated as the count of one allele divided by the total count for all alleles in a population.
%	
%	\[ p_i = \dfrac{x_i}{n}\]
%
%	\hangpara where $i$ is one allele, $x_i$ is the number of copies of the $i$th allele, and $n$ is the total number of alleles.
%	
%%	\hangpara $\sum\limits_{i}^{n}$ tells you to sum together $x_i$ for all $n$ alleles.
%	
%\end{frame}
%%
%\begin{frame}[t]{Review: 10 alleles were sampled from a population with two unique alleles.}
%
%	\hfil $p_i = \dfrac{x_i}{n}$
%	\hfil \emph{$A_2$ $A_2$ $A_1$ $A_2$ $A_1$ $A_1$ $A_2$ $A_2$ $A_2$ $A_1$}
%	\pause \hfil $n = 10$ \hfil
%
%	\pause \hangpara \emph{$A_1$} is allele 1 so $x_1 = 4$. 
%	\pause \qquad \emph{$A_2$} is allele 2 so $x_2 = 6$. 
%	
%	\pause \hangpara $\sum\limits_{i}^{n} = x_1 + x_2 = 4 + 6 = 10.$
%	
%	\pause \hangpara The frequency of \emph{$A_1$} is $p_1 = \dfrac{4}{10} = 0.4.$
%
%	\hangpara The frequency of \emph{$A_2$} is $p_2 = \dfrac{6}{10} = 0.6.$
%\end{frame}
%
\begin{frame}[t]{Learn these rules about allele frequencies:}

	\hangpara Frequencies are expressed as decimal fractions between 0 and 1.
	
	\hangpara The frequencies of all alleles in a population \emph{must} sum to 1.
	
	\hangpara Frequencies can be converted to percentages. A frequency of 0.30 = 30\% of the total alleles.
	
	\hangpara Allele frequencies represent the probability of drawing any one allele at random from the population. An allele with a frequency of 0.05 has a 5\% chance of being sampled at random from the population.

\end{frame}
%
\begin{frame}[t]{We will assume the following:}

	\hangpara The population has \emph{only} two unique alleles for each gene.\\ \quad \textcolor{gray}{Real populations often have many unique alleles for each gene.}
	 

	\hangpara Individuals in the population are diploid; that is, each individual has two alleles for each gene.
	
\end{frame}
%
\begin{frame}[t]{With only two alleles in a diploid population, then only three \highlight{genotypes} are possible.}

\Large
\hfil \alert<2>{$Y_1Y_1$} \hfil \alert<3>{$Y_1Y_2$} \hfil \alert<2>{$Y_2Y_2$} \hfill

\normalsize
	\hangpara \onslide<2->An individual with two copies of the same allele has a \alert<2>{homozygous} genotype.
	
	\hangpara \onslide<3->An individual with one copy each of different alleles has a \alert<3>{heterozygous} genotype.
	
	\hangpara \onslide<4->\textsc{\highlight{Important:}} Genotype refers to the combination of alleles for \emph{one or more} genes. We will use homozygote and heterozygote to refer to the genotype of just one gene.

\end{frame}
%
\begin{frame}[t]{\highlight{Genotype frequency} is a measure of how common a genotype is in a population.}

	\hangpara Calculate genotype frequencies for these three genotypes, just as you did for allele frequencies.
	
	\hangpara \quad $H_1H_1$: 27 individuals. \onslide<2>{0.15}
	
	\hangpara \quad	$H_1H_2$: 81 individuals. \onslide<2>{0.45}
	
	\hangpara \quad $H_2H_2$: 72 individuals. \onslide<2>{0.40}

\end{frame}
%
\begin{frame}[t]{Learn these rules about genotype frequencies:}

	\hangpara Frequencies are expressed as decimal fractions between 0 and 1.
	
	\hangpara The frequencies of all genotypes in a population \emph{must} sum to 1.
	
	\hangpara Frequencies can be converted to percentages. A frequency of 0.39 = 39\% of the total alleles.
	
	\hangpara Genotype frequencies represent the probability of drawing any one genotype at random from the population. A genotype with a frequency of 0.77 has a 77\% chance of being sampled at random from the population.

\end{frame}
%
\begin{frame}[t]{Allele and genotype frequencies have a mathematical relationship.}
	
	\hangpara Let $p$ be the frequency of one allele.
	
	\hangpara Let $q$ be the frequency of the other allele.
	
	\hangpara Allele frequencies must sum to 1. Therefore,
	
	\highlight{\[ p + q = 1.\]}
	
	\pause
	
	\vspace{\baselineskip}
	
	\hangpara \highlight{Trivia:} Why use $p$ and $q$ instead of gene names?  Because real genes have names like \emph{tinman,} \emph{sonic hedgehog,} and \emph{ms(3)k81}.
\end{frame}
%
\begin{frame}[t]{Assume a very large population of randomly mating males and females.}

	\hangpara Eggs will contain each allele in proportion to the allele frequencies. If $p = 0.6$ and $q = 0.4$, then 60\% of the eggs in the population will contain the $p$ allele and 40\% will contain the $q$ allele.
	
	\hangpara Sperm will also contain each allele in proportion to the allele frequencies. 
	
	\hangpara Therefore, in a large randomly mating population, 
	
	\[ \stackrel{\text{eggs}}{(p + q)}
	\stackrel{\text{sperm}}{(p + q)}\, 
	= \highlight{p^2 + 2pq + q^2 = 1} \]
	
	\hangpara where $p^2$ and $q^2$ represent the frequencies of the two homozygous genotypes and $2pq$ represents the frequency of the heterozygous genotype in the population.
\end{frame}
%
{
\setbeamertemplate{element name}{arguments}
\begin{frame}[c]%[t]{The mathematical relationship can be represented graphically with a Punnett square.}

\centering
\begin{tikzpicture}[scale=0.55]
\draw[thick] (-5,4) rectangle (5,-6) node (square){};
%    \foreach \row in {0,1, ..., 9} {
%        \foreach \column in {0, ..., 3} {
%    \fill ({2*\column + mod(\row,2)}, -\row) rectangle +(1,-1);
%        }
%    }
\node [above] at (0,5){Eggs};
\node [above] at (-2,4){$p = 0.6$};
\node [above] at (3,4){$q = 0.4$};
\draw [gray] (1,4) -- (1,-6);
\draw [gray] (-5,-2) -- (5,-2); 
\node [left] at (-5,1){$p=0.6$};
\node [left] at (-5,-4){$q=0.4$};
\node [left] at (-7,-1) {Sperm};

\pause
\node at (-2,1){$p^2 = 0.36$};
\node at (3,1){$pq = 0.24$};
\node at (-2,-4){$pq = 0.24$};
\node at (3,-4){$q^2 = 0.16$};

\pause
\node at (0,-7){$p^2 + pq + pq + q^2 = \highlight{p^2 + 2pq + q^2 = 1}$};
\end{tikzpicture}
\end{frame}
}
% 
\begin{frame}[t]{The \highlight{Hardy-Weinberg} equations test for equilibrium of allele and genotype frequencies in a population.}

	\vspace*{-\baselineskip}
	
	\begin{multicols}{2}
	
	\vspace*{0\baselineskip}
	
	\hangpara $p + q = 1$
	
	\hangpara $p^2 + 2pq + q^2 = 1$
	
	\vspace{\baselineskip}
	
	\hangpara If allele frequencies do not change, then genotype frequencies will not change.
	
	\vspace{\baselineskip}
	
	\hangpara \highlight{If allele frequencies do not change, then a population cannot evolve.}
	
	\columnbreak
	
		{\centering \includegraphics[height=0.34\textheight]{hardy}\\[-1ex]
		{\small G.\,H.\,Hardy}\\[1ex]

		\includegraphics[height=0.34\textheight]{weinberg}\\[-1ex]
		{\small Wilhelm Weinberg}\par
		}
		
	\end{multicols}
	
\end{frame}

\end{document}
