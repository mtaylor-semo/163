%!TEX TS-program = lualatex
%!TEX encoding = UTF-8 Unicode

\documentclass[letterpaper]{tufte-handout}

%\geometry{showframe} % display margins for debugging page layout

\usepackage{fontspec}
\def\mainfont{Linux Libertine O}
\setmainfont[Ligatures={Common,TeX}, Contextuals={NoAlternate}, BoldFont={* Bold}, ItalicFont={* Italic}, Numbers={OldStyle,Lining}]{\mainfont}
\setsansfont[Scale=MatchLowercase]{Linux Biolinum O} 
\usepackage{microtype}

\usepackage{graphicx} % allow embedded images
  \setkeys{Gin}{width=\linewidth,totalheight=\textheight,keepaspectratio}
  \graphicspath{{img/}} % set of paths to search for images
\usepackage{amsmath}  % extended mathematics
\usepackage{booktabs} % book-quality tables
\usepackage{units}    % non-stacked fractions and better unit spacing
\usepackage{multicol} % multiple column layout facilities
%\usepackage{fancyvrb} % extended verbatim environments
%  \fvset{fontsize=\normalsize}% default font size for fancy-verbatim environments

\usepackage{enumitem}

\makeatletter
% Paragraph indentation and separation for normal text
\renewcommand{\@tufte@reset@par}{%
  \setlength{\RaggedRightParindent}{1.0pc}%
  \setlength{\JustifyingParindent}{1.0pc}%
  \setlength{\parindent}{1pc}%
  \setlength{\parskip}{0pt}%
}
\@tufte@reset@par

% Paragraph indentation and separation for marginal text
\renewcommand{\@tufte@margin@par}{%
  \setlength{\RaggedRightParindent}{0pt}%
  \setlength{\JustifyingParindent}{0.5pc}%
  \setlength{\parindent}{0.5pc}%
  \setlength{\parskip}{0pt}%
}
\makeatother

% Set up the spacing using fontspec features
   \renewcommand\allcapsspacing[1]{{\addfontfeatures{LetterSpace=15}#1}}
   \renewcommand\smallcapsspacing[1]{{\addfontfeatures{LetterSpace=10}#1}}

\title{\scshape{bi} 163 Study Guide 02}

\date{} % without \date command, current date is supplied

\begin{document}

\maketitle	% this prints the handout title, author, and date

%\printclassoptions
\section*{The History of Evolutionary Thought}

We\marginnote{\textbf{Read:} pgs. 21, 462--470.} discussed types of ideas, defined evolution, and provided historical context for evolutionary thought that led to the publication of \textit{On the Origin of Species} by Charles Darwin.

\section*{Vocabulary}

\vspace{-1\baselineskip}
\begin{multicols}{2}
Thomas Robert Malthus\\
Jean Baptiste de Lamarck \\
Georges Cuvier \\
Charles Lyell \\
Charles Darwin \\
\textsc{h.m.s.} Beagle \\
Alfred Russel Wallace \\
artificial selection 
\end{multicols}

\section*{Concepts}

You should \emph{write} clear and concise answers to each question in the Concepts section.  Remember to ``think horizontally'' and to ``connect the dots.'' 

\begin{enumerate}

	\item You do not need to know most of the background that I presented for Charles Darwin (e.g., his childhood, \&c.). We will cover the two ideas from his book in subsequent lectures.
	
	\item You should know these few facts about these scientists for their historical context:

	\begin{enumerate}[label=\alph*.]

	\item Thomas Robert Malthus wrote the \textit{Principle of Populations} that described how the human population would suffer if the rate of population growth exceeded the rate of food production. Darwin incorporated this idea into natural selection, where organisms would “struggle” for limited resources. 
	
	\item Jean Baptiste de Lamarck was one of the first scientists to propose that organisms change over time. He proposed that the mechanism of change was “inheritance of acquired characteristics,” which stated that if a structure (like a giraffe's neck) or physiology of an individual changed through use, then the change would be inherited by that individual's offspring. That hypothesis turned out to be incorrect.

	\item George Cuvier is sometimes called the “father of paleontology.” Paleontology is the study of fossils and the fossil record. Cuvier was the first to propose that many species that lived in the past went extinct. Cuvier also popularized the idea that the Earth underwent great catastrophes that caused rapid change to the environment and caused the extinctions.

	\item Charles Lyell was a geologist. He supported ``uniformitarianism,'' the idea that the Earth changed very slowly over long periods of time. This is the opposite of Cuvier's catastrophism. Most importantly, Lyell was the first to argue that the Earth was much older than 6000 years old, as was widely believed at the time.

	\item Alfred Russel Wallace was a naturalist that spent lots of time around Australia and the Malay Peninsula (southeast Asia). While working around the Malaysian Peninsula, Wallace developed his own idea of natural selection, independent of Charles Darwin.  Darwin and Wallace published a paper outlining the process of natural selection. Darwin then went on to publish \textit{On the Origin of Species.}

	\end{enumerate}
	
	\item It is always good to know the foundational history of scientific thought. Embrace it!

\end{enumerate}

\section*{Example exam questions}

These are examples only and not exhaustive. One or more of these questions may or may not appear on the exam.

\begin{fullwidth}

\bigskip

\noindent \rule{1in}{0.4pt} Argued that Earth changed uniformly for much more than 6000 years.

\bigskip

\noindent \rule{1in}{0.4pt} Title of the book that influenced Darwin's thinking about organisms competing for\\
\noindent\hspace*{1in} limited resources.

\bigskip

\noindent True\hspace{1em}False\hspace{1em} Alfred Russel Wallace also developed a concept of natural selection. 

\bigskip

\noindent True\hspace{1em}False\hspace{1em} Darwin used his principles of natural selection to explain how artificial selection works \\
\noindent \phantom{True\hspace{1em}False\hspace{1em}} in show animals, farm animals, and crops.

\bigskip

\noindent His independent development of the process of natural selection caused Darwin to publish \textit{On the Origin of Species.}

\smallskip

\noindent \textsc{a}. Charles Lyell\\
\noindent \textsc{b}. Alfred Russel Wallace\\
\noindent \textsc{c}. Georges Cuvier\\
\noindent \textsc{d}. Robert Thomas Malthus\\
\noindent \textsc{e}. Jean Baptiste de Lamarck\\

\end{fullwidth}


\end{document}