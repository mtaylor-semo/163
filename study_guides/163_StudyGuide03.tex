%!TEX TS-program = lualatex
%!TEX encoding = UTF-8 Unicode

\documentclass[letterpaper]{tufte-handout}

%\geometry{showframe} % display margins for debugging page layout

\usepackage{fontspec}
\def\mainfont{Linux Libertine O}
\setmainfont[Ligatures={Common,TeX}, Contextuals={NoAlternate}, BoldFont={* Bold}, ItalicFont={* Italic}, Numbers={OldStyle}]{\mainfont}
\setsansfont[Scale=MatchLowercase, Numbers={OldStyle}]{Linux Biolinum O} 
\usepackage{microtype}

\usepackage{graphicx} % allow embedded images
  \setkeys{Gin}{width=\linewidth}
  \graphicspath{	{/Users/goby/teach/163/lectures/}}%}%

%\usepackage{amsmath}  % extended mathematics
%\usepackage{booktabs} % book-quality tables
%\usepackage{units}    % non-stacked fractions and better unit spacing
%\usepackage{multicol} % multiple column layout facilities
%\usepackage{fancyvrb} % extended verbatim environments
%  \fvset{fontsize=\normalsize}% default font size for fancy-verbatim environments

\usepackage{enumitem}

\makeatletter
% Paragraph indentation and separation for normal text
\renewcommand{\@tufte@reset@par}{%
  \setlength{\RaggedRightParindent}{1.0pc}%
  \setlength{\JustifyingParindent}{1.0pc}%
  \setlength{\parindent}{1pc}%
  \setlength{\parskip}{0pt}%
}
\@tufte@reset@par

% Paragraph indentation and separation for marginal text
\renewcommand{\@tufte@margin@par}{%
  \setlength{\RaggedRightParindent}{0pt}%
  \setlength{\JustifyingParindent}{0.5pc}%
  \setlength{\parindent}{0.5pc}%
  \setlength{\parskip}{0pt}%
}
\makeatother

% Set up the spacing using fontspec features
   \renewcommand\allcapsspacing[1]{{\addfontfeatures{LetterSpace=15}#1}}
   \renewcommand\smallcapsspacing[1]{{\addfontfeatures{LetterSpace=10}#1}}

\newcommand\lecturefile{163_lecture03_fullsize}

\title{{\scshape bi} 163 Study Guide 03}

\date{} % without \date command, current date is supplied

\begin{document}

\maketitle	% this prints the handout title, author, and date

%\printclassoptions
\section*{Descent with modification, natural selection, and relative fitness}

We\marginnote{\textbf{Read:} pgs. 12--15, 467--470, 491--492.} discussed Darwin's concepts of descent with modification and natural selection. We also discussed how differences in the relative fitness among individuals in a population contributes to the process of natural selection.

\section*{Vocabulary}

\vspace{-1\baselineskip}
\begin{multicols}{2}
descent with modification\\
common (shared) ancestor \\
descendants \\
natural selection \\
population \\
adaptation \\
reproductive output \\
relative fitness
\end{multicols}

\section*{Concepts}

You should \emph{write} clear and concise answers to each question in the Concepts section.  Remember to ``think horizontally'' and to ``connect the dots.'' 

\begin{enumerate}

	\item Explain descent with modification as developed by Darwin. 
	\begin{marginfigure}
		\includegraphics[page=2]{\lecturefile}
	\end{marginfigure}

	\item Identify common ancestors and descendants on a phylogenetic tree.

	\item Why can descent with modification\marginnote{We will spend more time with phylogenetic trees in a couple of weeks.} be represented with phylogenetic trees? 

	\item Describe how natural selection works in a population. Explain each the key points that we covered in lecture (given in the text as two observations and two inferences) that explain how natural selection works.

	\item\label{nsOne} Explain what is meant by ``individuals in a population vary.''

	\item Explain what is meant by ``populations overproduce offspring.'' Does that mean every individual in a population overproduces offspring? Why or why not?

	\item Explain what is meant by ``not all individuals in a population will survive.'' Does this mean that only individuals that live a full, natural life will reproduce? Does it mean that individuals either die or survive?

	\item Explain what is meant by ``traits determine survival.'' Does it mean that only the strongest or fastest individuals survive? Does it mean that only the strongest or fastest individuals reproduce? Why or why not?

	\item\label{nsFour} Explain what is meant by ``more survival, more offspring.'' Does living a long life mean that individual is guaranteed to produce more offspring compared to an individual that lives a shorter time? Why or why not? 

	\item Explain what is meant by heritable traits.\marginnote{A trait is not heritable if it is not determined by genetics.}

	\item Explain how the statements in items \ref{nsOne}--\ref{nsFour} lead to traits spreading through a population. Does “spreading of traits” refer to the ability of individuals to move to one area or another?\marginnote{Remember what is a population.} Why or why not?

	\item Explain what are adaptations. Relate the evolution of adaptations to the process of natural selection.
	\begin{marginfigure}
		\includegraphics[page=11]{\lecturefile}
	\end{marginfigure}
	
	\item Explain how the \emph{process} of natural selection cause the \emph{pattern} of descent with modification.

	\item Explain relative fitness. Why is fitness relative? Explain how relative fitness influences the process of natural selection and the evolution of adaptations.  
	
	You must remember that fitness does not refer to physical fitness as associated with exercise. It refers instead to the number of offspring produced by an individual over its life time. Remember too that fitness is comparative. That is, fitness is compared among individuals in a population. Some individuals have relatively greater fitness than other individuals. An individual that reproduces once and has 20 offspring has lower relative fitness than an individual that reproduces twice and has 11 offspring each time (22 offspring total).

	\item Why is ``survival of the fittest'' not a good\marginnote{The phrase was coined by Herbert Spencer, not Darwin. Darwin did not include the phrase until the 5th edition of \textit{On the Origin of Species.}} explanation for how natural selection works?


\end{enumerate}

\section*{Example exam questions}

These are examples only and not exhaustive. One or more of these questions may or may not appear on the exam.

\bigskip

\noindent \rule{1in}{0.4pt} Trait that increases fitness in the current environment.

\bigskip

\noindent True\hspace{1em}False\hspace{1em} Adaptations are the result of descent with modification.\marginnote{Think carefully about the relationship between descent with modification and natural selection. One is a pattern and one is a process. Processes cause patterns.} 

\bigskip

\noindent Given a population that contains genetic variation, what is the correct sequence of the following events, under the influence of natural selection?

\smallskip

\noindent \textsc{a}. Well-adapted individuals leave more offspring than do poorly adapted individuals.\\\marginnote{Most multiple choice questions will have \emph{at least} five choices to reduce your chances of getting correct answers by guessing. Regular study reduces the need for guessing. Still, if you don't know, then guess!}
\noindent \textsc{b}. A change occurs in the environment.\\
\noindent \textsc{c}. Genetic frequencies within the population change.\\
\noindent \textsc{d}. Poorly adapted individuals have decreased survivorship.\\

\bigskip

\noindent (5 points) What concept published by Robert Thomas Malthus did Darwin use to develop the concept of natural selection? Explain how Malthus‘s idea helps to explain how natural selection works.

\smallskip

Notice how this question connects lecture 02 to this lecture.

\end{document}