%!TEX TS-program = lualatex
%!TEX encoding = UTF-8 Unicode

\documentclass[letterpaper]{tufte-handout}

%\geometry{showframe} % display margins for debugging page layout

\usepackage{fontspec}
\def\mainfont{Linux Libertine O}
\setmainfont[Ligatures={Common,TeX}, Contextuals={NoAlternate}, BoldFont={* Bold}, ItalicFont={* Italic}, Numbers={OldStyle,Proportional}]{\mainfont}
\setsansfont[Scale=MatchLowercase]{Linux Biolinum O} 
\usepackage{microtype}

\usepackage{graphicx} % allow embedded images
  \setkeys{Gin}{width=\linewidth,totalheight=\textheight,keepaspectratio}
  \graphicspath{	{/Users/goby/teach/163/lectures/}} % set of paths to search for images
\usepackage{amsmath}  % extended mathematics
\usepackage{booktabs} % book-quality tables
\usepackage{units}    % non-stacked fractions and better unit spacing
\usepackage{multicol} % multiple column layout facilities
%\usepackage{fancyvrb} % extended verbatim environments
%  \fvset{fontsize=\normalsize}% default font size for fancy-verbatim environments

\usepackage{enumitem}

\makeatletter
% Paragraph indentation and separation for normal text
\renewcommand{\@tufte@reset@par}{%
  \setlength{\RaggedRightParindent}{1.0pc}%
  \setlength{\JustifyingParindent}{1.0pc}%
  \setlength{\parindent}{1pc}%
  \setlength{\parskip}{0pt}%
}
\@tufte@reset@par

% Paragraph indentation and separation for marginal text
\renewcommand{\@tufte@margin@par}{%
  \setlength{\RaggedRightParindent}{0pt}%
  \setlength{\JustifyingParindent}{0.5pc}%
  \setlength{\parindent}{0.5pc}%
  \setlength{\parskip}{0pt}%
}
\makeatother

% Set up the spacing using fontspec features
   \renewcommand\allcapsspacing[1]{{\addfontfeatures{LetterSpace=15}#1}}
   \renewcommand\smallcapsspacing[1]{{\addfontfeatures{LetterSpace=10}#1}}

\newcommand\lecturefile{163_lecture01_fullsize}
  
\title{{\scshape bi} 163 Study Guide 01}

\date{} % without \date command, current date is supplied

\begin{document}

\maketitle	% this prints the handout title, author, and date

%\printclassoptions
\section*{Evolution, genotypes, and phenotypes.}

We\marginnote{\textbf{Read:} pgs. 480--481, 270 (Mendel's model), 272.} defined and explained evolution, and explained the relationships between genes and characters, alleles and traits, and genotypes and phenotypes. You will learn more about genes and alleles in~{\scshape bi} 173. For this course, you need to understand them only in the context of evolution.

\section*{Vocabulary}

\vspace{-1\baselineskip}
\begin{multicols}{2}
evolution\\
population \\
gene \\
allele \\
character \\
trait \\
genotype \\
phenotype \\
genome
\end{multicols}

\section*{Concepts}

You should \emph{write} clear and concise answers to each question in the Concepts section.  Remember to ``think horizontally'' and to ``connect the dots.'' 

\begin{enumerate}

	\item Define evolution. 
	
	\item Explain evolution.\marginnote{In other words, knowing a simple definition is not usually sufficient. You must be able to explain concepts.} That is, explain evolution by explaining what is mean by populations, time, and genetic change. 
	
	\item Explain the difference between the following pairs of terms.
	
	\begin{enumerate}
		\item characters and traits.
		
		\item genes and alleles.
		
		\item characters and genes.
		
		\item traits and alleles.
		
	\begin{marginfigure}
		\includegraphics[page=14]{\lecturefile}
	\end{marginfigure}
	
		\item genotype and phenotype.
		
		\item alleles and phenotype.
		
		\item genes and genotype.
		
	\end{enumerate}

	\item Why is evolution in a population usually measured in terms of generation time instead of years?

	\item Describe how the phenotype changed for Italian wall lizards that were introduced to Pod Mr\v{c}aru\marginnote{Pronounced Mer-KAR-oo.} in 1971. About how long did it take for the phenotypic differences to evolve? 
	
	\item What was different about Pod Mr\v{c}aru compared to their original island (Pod Kipi\v{s}te.)\marginnote{Pronounced Ko-PIS-tee} that caused the phenotype to change? Was it a different type of habitat? Interactions with different types of species? Differences in diet? Differences in climate? Something else? Explain.
	

	\item Is it safe to say that \emph{only} the phenotype evolved for the Italian Wall Lizards?  Explain why or why not. If not, explain what else must have evolved in the lizards. That is, what determines the phenotype? 
	
\end{enumerate}

\section*{Example exam questions}

These are examples only and not exhaustive. One or more of these questions may or may not appear on the exam. Do not forget that the study guide questions above, or variations of the questions, might be used as short answer questions on the exam. 

\begin{fullwidth}

\bigskip

\noindent \rule{1in}{0.4pt} The combination of alleles in an organism for one or more genes.

\bigskip

\noindent \rule{1in}{0.4pt} All of the genes in an organism.

\bigskip

\noindent True\hspace{1em}False\hspace{1em} Complex organisms always have more chromosomes than simple organisms. 

\bigskip

\noindent True\hspace{1em}False\hspace{1em} Evolution always takes at least tens of thousands of years.

\bigskip

\noindent The combination of \rule{1in}{0.4pt} in the genome determine the phenotype of an organism.

\smallskip

\noindent \textsc{a}. genes\\
\noindent \textsc{b}. alleles\\
\noindent \textsc{c}. characters\\
\noindent \textsc{d}. traits\\
\noindent \textsc{e}. genotypes\\

\end{fullwidth}


\end{document}