%!TEX TS-program = lualatex
%!TEX encoding = UTF-8 Unicode

\documentclass[letterpaper]{tufte-handout}

%\geometry{showframe} % display margins for debugging page layout

\usepackage{fontspec}
\def\mainfont{Linux Libertine O}
\setmainfont[Ligatures={Common,TeX}, Contextuals={NoAlternate}, BoldFont={* Bold}, ItalicFont={* Italic}, Numbers={OldStyle}]{\mainfont}
\setsansfont[Scale=MatchLowercase, Numbers={OldStyle}]{Linux Biolinum O} 
\setmonofont{Linux Libertine O}
\usepackage{microtype}

\usepackage{graphicx} % allow embedded images
  \setkeys{Gin}{width=\linewidth,totalheight=\textheight,keepaspectratio}
  \graphicspath{{img/}} % set of paths to search for images
\usepackage{amsmath}  % extended mathematics
\usepackage{booktabs} % book-quality tables
\usepackage{units}    % non-stacked fractions and better unit spacing
\usepackage{multicol} % multiple column layout facilities
%\usepackage{fancyvrb} % extended verbatim environments
%  \fvset{fontsize=\normalsize}% default font size for fancy-verbatim environments

\usepackage{enumitem}
\usepackage{mhchem}

\makeatletter
% Paragraph indentation and separation for normal text
\renewcommand{\@tufte@reset@par}{%
  \setlength{\RaggedRightParindent}{1.0pc}%
  \setlength{\JustifyingParindent}{1.0pc}%
  \setlength{\parindent}{1pc}%
  \setlength{\parskip}{0pt}%
}
\@tufte@reset@par

% Paragraph indentation and separation for marginal text
\renewcommand{\@tufte@margin@par}{%
  \setlength{\RaggedRightParindent}{0pt}%
  \setlength{\JustifyingParindent}{0.5pc}%
  \setlength{\parindent}{0.5pc}%
  \setlength{\parskip}{0pt}%
}
\makeatother

% Set up the spacing using fontspec features
   \renewcommand\allcapsspacing[1]{{\addfontfeatures{LetterSpace=15}#1}}
   \renewcommand\smallcapsspacing[1]{{\addfontfeatures{LetterSpace=10}#1}}

\title{{\scshape bi} 163 Study Guide 16}

\date{} % without \date command, current date is supplied

\begin{document}

\maketitle	% this prints the handout title, author, and date

%\printclassoptions
\section*{Transitional forms 2 and mass extinctions}

We\marginnote{\textbf{Read:} 475--476, 534--536, 742--748.} covered evidence for transitional forms that represent major macroevolutionary events. We also covered mass extinctions.

\section*{Vocabulary}

\vspace{-1\baselineskip}
\begin{multicols}{2}
Artiodactyla (even-toed mammals) \\
terrestrial cetaceans \\
auditory bulla \\
involucrum \\
double-pulley ankle \\
\textit{Homo sapiens} \\
australopithecines \\
\textit{Australopithecus} \\
\textit{Homo} \\
bipedalism \\
mass extinction \\
\end{multicols}

\section*{Concepts}

You should \emph{write} clear and concise answers to each question in the Concepts section.  Remember to ``think horizontally'' and to ``connect the dots.'' \vspace*{\baselineskip}

\begin{enumerate}

	\item What morphological evidence\marginnote{\url{http://evolution.berkeley.edu/evolibrary/article/evograms_03}} suggests that whales are descended from artiodactyls? Explain the auditory bulla and double-pulley ankle joints. Explain \emph{why} these are important traits that link artiodactyls and whales.
	
	\item What morphological features \marginnote{\url{https://ocean.si.edu/ocean-videos/evolution-whales-animation}} did terrestrial cetaceans like \textit{Ambulocetus} and \textit{Rodhocetus} have to suggest they are transitional fossils between artiodactyls and whales.
	
	\item The position of the nostrils of fossil terrestrial cetaceans (the terrestrial fossils like \textit{Ambulocetus} and \textit{Rodhocetus}) also show evidence of a transition from land to water. Explain this. How does this match with evidence from dolphin embryos.
	
	\item What type of habitat did terrestrial cetaceans live in?
	
	\item When did \textit{Homo sapiens} evolve?\marginnote{\url{http://www.evolution.berkeley.edu/evosite/evo101/IIE2cHumanevop2.shtml}}
	
	\item When and where did the earliest hominins evolve?
	
	\item Which evolved first in hominins\marginnote{The Hominini, or hominins, is a clade that includes chimpanzees, australopithecines, and all species of \textit{Homo}. I use hominins here to discuss \emph{only} australopithecines and humans.}, bipedalism or large brains? In which group (genus) did bipedalism evolve?
	
	\item What are the adaptive advantages of bipedalism?
	
	\item What are some of the adaptations associated with bipedalism and increased brain size?
	
	\item Did \textit{Homo} leave Africa to colonize Europe and Asia only once or more than once?  Explain.
	
	\item Which species of \textit{Homo} was most likely the first to leave Africa? What was the last?
	
	\item What is a mass extinction? How many have occurred since the beginning of the Cambrian Period? Can you name them all?
	
	\item What are some of the causes of mass extinction?\marginnote{\url{http://www.bbc.co.uk/nature/extinction_events}} What do these all have in common?
	
	\item What percentage of species must go extinct to be considered a mass extinction?
	
	\item Which was the largest mass extinction? About what percent of species went extinct?
	
	\item What happened to cause dinosaurs go extinct? This may be a trick question. 
	
\end{enumerate}

\end{document}