%!TEX TS-program = lualatex
%!TEX encoding = UTF-8 Unicode

\documentclass[letterpaper]{tufte-handout}

%\geometry{showframe} % display margins for debugging page layout

\usepackage{fontspec}
\def\mainfont{Linux Libertine O}
\setmainfont[Ligatures={Common,TeX}, Contextuals={NoAlternate}, BoldFont={* Bold}, ItalicFont={* Italic}, Numbers={OldStyle}]{\mainfont}
\setsansfont[Scale=MatchLowercase, Numbers={OldStyle}]{Linux Biolinum O} 
\usepackage{microtype}

\usepackage{graphicx} % allow embedded images
  \setkeys{Gin}{width=\linewidth,totalheight=\textheight,keepaspectratio}
  \graphicspath{{img/}} % set of paths to search for images
\usepackage{amsmath}  % extended mathematics
\usepackage{booktabs} % book-quality tables
\usepackage{units}    % non-stacked fractions and better unit spacing
\usepackage{multicol} % multiple column layout facilities
%\usepackage{fancyvrb} % extended verbatim environments
%  \fvset{fontsize=\normalsize}% default font size for fancy-verbatim environments

\usepackage{enumitem}

\makeatletter
% Paragraph indentation and separation for normal text
\renewcommand{\@tufte@reset@par}{%
  \setlength{\RaggedRightParindent}{1.0pc}%
  \setlength{\JustifyingParindent}{1.0pc}%
  \setlength{\parindent}{1pc}%
  \setlength{\parskip}{0pt}%
}
\@tufte@reset@par

% Paragraph indentation and separation for marginal text
\renewcommand{\@tufte@margin@par}{%
  \setlength{\RaggedRightParindent}{0pt}%
  \setlength{\JustifyingParindent}{0.5pc}%
  \setlength{\parindent}{0.5pc}%
  \setlength{\parskip}{0pt}%
}
\makeatother

% Set up the spacing using fontspec features
   \renewcommand\allcapsspacing[1]{{\addfontfeatures{LetterSpace=15}#1}}
   \renewcommand\smallcapsspacing[1]{{\addfontfeatures{LetterSpace=10}#1}}

\title{{\scshape bi} 163 Study Guide 15}

\date{} % without \date command, current date is supplied

\begin{document}

\maketitle	% this prints the handout title, author, and date

%\printclassoptions
\section*{Molecular clocks and genome evolution}

We\marginnote{\textbf{Read:} 380--381 and Fig. 18-19 for background on homeotic and other developmental genes, 453--458 (especially 457--458), 538--541, 559--562.} discussed molecular clocks, the evolution of genomes by gene duplication, and the role of homeotic genes in macroevolution.

\section*{Vocabulary}

\vspace{-1\baselineskip}
\begin{multicols}{2}
molecular clock \\
macroevolution  \\
genome  \\
gene duplication  \\
pseudogene \\
homeotic genes \\
regulatory genes \\
\textit{Hox} genes \\
\textit{Ubx} \\
\textit{Antp} \\
\textit{Pax6} \\
paedomorphosis

\end{multicols}

\section*{Concepts}

You should \emph{write} clear and concise answers to each question in the Concepts section.  Remember to ``think horizontally'' and to ``connect the dots.'' 

\begin{enumerate}

	\item What is a molecular clock? How can it be calibrated?

	\item Can the same molecular clock be used for any gene? Explain. 

	\item Briefly explain the two possible fates for a duplicated gene.

	\item What is a pseudogene? How do they evolve?\marginnote{Pseudo- means false or fake.}

	\item What are homeotic genes?\marginnote{Homeotic refers to a broad class of genes that determines the body plan, like the position of legs, wings, and abdomen. \textit{Hox} genes are one group of homeotic genes but not all homeotic genes are \textit{Hox} genes. This is a common source of confusion for students learning about homeotic genes.} 	
	
	\item How were homeotic genes first discovered?
	
	\item How do homeotic genes function? How can mutation of the homeotic binding site alter the function of homeotic genes? 
	
	\item What is the function of \textit{Ubx}?\marginnote{Both \textit{Ubx} and \textit{Antp} are \textit{Hox} genes. \textit{Pax6} is not a \textit{Hox} gene but it is a developmental regulatory gene.} What is the function of \textit{Antp}? What is the function of \textit{Pax6}?
	
	\item Explain paedomorphosis.\marginnote{Paedo- means child.  -morphosis means shape.}


\end{enumerate}

\end{document}