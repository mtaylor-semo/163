%!TEX TS-program = lualatex
%!TEX encoding = UTF-8 Unicode

\documentclass[letterpaper]{tufte-handout}

%\geometry{showframe} % display margins for debugging page layout

\usepackage{fontspec}
\def\mainfont{Linux Libertine O}
\setmainfont[Ligatures={Common,TeX}, Contextuals={NoAlternate}, BoldFont={* Bold}, ItalicFont={* Italic}, Numbers={OldStyle}]{\mainfont}
\setsansfont[Scale=MatchLowercase, Numbers={OldStyle}]{Linux Biolinum O} 
\usepackage{microtype}

\usepackage{graphicx} % allow embedded images
  \setkeys{Gin}{width=\linewidth}
  \graphicspath{	{/Users/goby/teach/163/lectures/}}%}%

\usepackage{amsmath}  % extended mathematics
\usepackage{booktabs} % book-quality tables
\usepackage{units}    % non-stacked fractions and better unit spacing
\usepackage{multicol} % multiple column layout facilities
%\usepackage{fancyvrb} % extended verbatim environments
%  \fvset{fontsize=\normalsize}% default font size for fancy-verbatim environments

\usepackage{enumitem}

\makeatletter
% Paragraph indentation and separation for normal text
\renewcommand{\@tufte@reset@par}{%
  \setlength{\RaggedRightParindent}{1.0pc}%
  \setlength{\JustifyingParindent}{1.0pc}%
  \setlength{\parindent}{1pc}%
  \setlength{\parskip}{0pt}%
}
\@tufte@reset@par

% Paragraph indentation and separation for marginal text
\renewcommand{\@tufte@margin@par}{%
  \setlength{\RaggedRightParindent}{0pt}%
  \setlength{\JustifyingParindent}{0.5pc}%
  \setlength{\parindent}{0.5pc}%
  \setlength{\parskip}{0pt}%
}
\makeatother

% Set up the spacing using fontspec features
   \renewcommand\allcapsspacing[1]{{\addfontfeatures{LetterSpace=15}#1}}
   \renewcommand\smallcapsspacing[1]{{\addfontfeatures{LetterSpace=10}#1}}

\newcommand\lecturefile{163_lecture06_fullsize}

\title{{\scshape bi} 163 Study Guide 06}

\date{} % without \date command, current date is supplied

\begin{document}

\maketitle	% this prints the handout title, author, and date

%\printclassoptions
\section*{Genetic drift and other microevolutionary processes}

We\marginnote{\textbf{Read:} pages 355--356, 486 (``Conditions for Hardy-Weinberg Equilibrium,'' including the small box on the lower left), 488--491, 480--481, 500--501} covered ways other than natural selection that populations evolve, including gene flow, mutation, non-random mating, and genetic drift. We introduced the concept of microevolution.

\section*{Vocabulary}

\vspace{-1\baselineskip}
\begin{multicols}{2}
allele frequency \\
gene flow \\
mutation \\
neutral mutations \\
heterozygosity \\
homozygosity \\
non-random mating \\
inbreeding \\
genetic drift \\
bottleneck effect \\
founder effect \\
microevolution \\
speciation \\
macroevolution

\end{multicols}

\section*{Concepts}

You should \emph{write} clear and concise answers to each question in the Concepts section.  Remember to ``think horizontally'' and to ``connect the dots.'' 

\begin{enumerate}

	\item Be sure you have a basic understanding of allele frequency.\marginnote{Microevolution is change of allele frequencies in a population over time.}  Frequency refers to how common something is. An allele that is common in a population has a high frequency. An allele that is rare in a population has a low frequency.
	
	\item What are the five evolutionary processes that cause allele frequencies in a population to change?  Explain how each causes allele frequencies to change.
	
	\begin{marginfigure}[0.5in]
		\includegraphics[page=3]{\lecturefile}
	\end{marginfigure}

	\textsc{Note:} When you read page 486, you'll read a small bit about Hardy-Weinberg equilibrium. This refers to a pair of math equations that show that populations cannot evolve as long as five assumptions are met. The five assumptions are listed in the small box on that page. The five evolutionary processes you just identified are \emph{violations} of those assumptions. You will learn more about these in upcoming lectures; I know you can hardly wait!).
	

	\item Compare and contrast bottleneck and founder effects.\marginnote{It is important to remember that genetic drift is any change in allele frequencies due to random events. Bottleneck and founder effects both involve a random selection of individuals and therefore a random selection of alleles.}  How are they similar?  How do they differ?  Explain why they are types of genetic drift.
		
	\item If heterozygosity is the proportion of heterozygotes in a population, then what is homozygosity?	
	
	\item How does non-random mating decrease heterozygosity?\marginnote{Heterozygosity and homozygosity must add up to 1. If an individual is not heterozygous, it must be heterozygous.} If heterozygosity is decreasing, what must be increasing?

	\item Which two processes (other than natural selection) decrease heterozygosity? Is this a good thing or a bad thing for the population? Why?
	
	\item Which two processes (other than natural selection) tend to increase heterozygosity? Is this a good thing or a bad thing for the population? Why?

	\item Is genetic drift present in all populations?\marginnote{Remember the assumption of infinite population size.} Why? Is genetic drift more pronounced in large populations or small populations?  Why?

	\item Explain the relation between the definition of ``evolution'' and “microevolution?” How are they the same or different? 
\end{enumerate}

\section*{Example exam questions}

These are examples only and not exhaustive. One or more of these questions may or may not appear on the exam.

\bigskip

\noindent \rule{1in}{0.4pt} Inbreeding is an example of this cause of microevolution.

\bigskip

\noindent True\hspace{1em}False\hspace{1em} Gene flow is enough to stop the effects of genetic drift in  \\
\noindent \phantom{True\hspace{1em}False\hspace{1em}}  a large population.

\bigskip

\noindent Prairie chickens are endangered grassland birds. In Illinois, grasslands used to cover most of the state but farming and growth of cities changed the habitat over a fairly short period of time. Now, less than 1\% of the grasslands remain. A study showed that most genes had far fewer alleles than historic populations. The loss of genetic variation is most likely due to

\smallskip

\noindent \textsc{a}. the small population size causing random fluctuation of allele frequencies.\\
\noindent \textsc{b}. a genetic bottleneck caused by the loss of habitat. \\
\noindent \textsc{c}. a founder event caused by individuals moving to the remaining grasslands. \\
\noindent \textsc{d}. directional selection that favored survivorship in small habitats. \\
\noindent \textsc{e}. balancing selection, which maintains genetic variation in small populations.


\end{document}