%!TEX TS-program = lualatex
%!TEX encoding = UTF-8 Unicode

\documentclass[letterpaper]{tufte-handout}

%\geometry{showframe} % display margins for debugging page layout

\usepackage{fontspec}
\def\mainfont{Linux Libertine O}
\setmainfont[Ligatures={Common,TeX}, Contextuals={NoAlternate}, BoldFont={* Bold}, ItalicFont={* Italic}, Numbers={OldStyle}]{\mainfont}
\setsansfont[Scale=MatchLowercase, Numbers={OldStyle}]{Linux Biolinum O} 
\usepackage{microtype}

\usepackage{graphicx} % allow embedded images
  \setkeys{Gin}{width=\linewidth}
  \graphicspath{	{/Users/goby/teach/163/lectures/}}%}%

\usepackage{amsmath}  % extended mathematics
\usepackage{booktabs} % book-quality tables
\usepackage{units}    % non-stacked fractions and better unit spacing
\usepackage{multicol} % multiple column layout facilities
%\usepackage{fancyvrb} % extended verbatim environments
%  \fvset{fontsize=\normalsize}% default font size for fancy-verbatim environments

\usepackage{enumitem}

\makeatletter
% Paragraph indentation and separation for normal text
\renewcommand{\@tufte@reset@par}{%
  \setlength{\RaggedRightParindent}{1.0pc}%
  \setlength{\JustifyingParindent}{1.0pc}%
  \setlength{\parindent}{1pc}%
  \setlength{\parskip}{0pt}%
}
\@tufte@reset@par

% Paragraph indentation and separation for marginal text
\renewcommand{\@tufte@margin@par}{%
  \setlength{\RaggedRightParindent}{0pt}%
  \setlength{\JustifyingParindent}{0.5pc}%
  \setlength{\parindent}{0.5pc}%
  \setlength{\parskip}{0pt}%
}
\makeatother

% Set up the spacing using fontspec features
   \renewcommand\allcapsspacing[1]{{\addfontfeatures{LetterSpace=15}#1}}
   \renewcommand\smallcapsspacing[1]{{\addfontfeatures{LetterSpace=10}#1}}

\newcommand\lecturefile{163_lecture06_fullsize}

\title{{\scshape bi} 163 Study Guide 06}

\date{} % without \date command, current date is supplied

\begin{document}

\maketitle	% this prints the handout title, author, and date

%\printclassoptions
\section*{Homology, analogy, convergent evolution, and sexual selection}

We\marginnote{\textbf{Read:} pgs 473--475, 493--494.} covered homologies, analogies, convergent evolution, the selfish nature of natural selection, and forms of sexual selection.

\section*{Vocabulary}

\vspace{-1\baselineskip}
\begin{multicols}{2}
homology \\
analogy\\
convergent evolution \\
sexual selection\\
intrasexual selection\\
intersexual selection\\
sexual dimorphism

\end{multicols}

\section*{Concepts}

You should \emph{write} clear and concise answers to each question in the Concepts section.  Remember to ``think horizontally'' and to ``connect the dots.'' 

\begin{enumerate}

	\item Explain convergent evolution.\marginnote{Analogies are struturally similar traits that serve similar functions in unrelated species. Analogies evolve through convergent evolution by natural selection.} Why does it occur? How does this relate to analogous structures? How does this differ from homologous structures?

	\item	Why is hunting younger adult deer (4 points or less) illegal in Arkansas but hunting larger deer (6 or more points) legal?  
	
	\begin{marginfigure}[0.5in]
		\includegraphics[page=13]{\lecturefile}
	\end{marginfigure}
	
	The key to this question is to think about relative fitness and the passing on of alleles.  Younger adult deer are fully capable of reproducing. Imagine a deer that has alleles that would allow it to grow large and produce large sets of antlers (favored by hunters, as well as important for territorial competition). If that deer is harvested when it is still young, before it had time to reproduce (little to zero relative fitness), then those alleles would be removed from the population. Over time, the population would be unable to produce large bucks with large antlers.	
	
	On the other hand, if the deer is protected from hunting, then is gets a chance to reproduce (higher relative fitness). This allows copies of his alleles to remain in the population so the herd as a whole keeps the ability to produce large bucks with large sets of antlers.
	
	This is an important management technique used for many commercially important species, although it is not always smaller individuals that have higher fitness. In Louisiana, for example, the redfish (\textit{Sciaenops oscellatus}) is a popular (and tasty) game fish. A person can take up to five fishes with a minimum length of 16 inches but only 1 fish can be larger than 27 inches. For this species, the largest individuals have very high relative fitness. By leaving most of the largest individuals, the population is able to sustain itself and continue to produce large individuals.
	

	\item	Explain why natural selection is ``selfish.''\marginnote{“We are survival machines—robot vehicles blindly programmed to preserve the selfish molecules known as genes. This is a truth which still fills me with astonishment.” — Richard Dawkins, \textit{The Selfish Gene} }

	\item Compare and contrast intersexual selection and intrasexual selection. Provide an example of each. Do not repeat the examples from class. Find different examples.   

	\item Although sexual selection is a form of natural selection, it differs from natural selection in a fundamental way; that is, how it acts on the reproductive success of the individual. Most forms of natural selection affect survival, which \emph{indirectly} affects reproductive success. Individuals that live longer will have more reproductive opportunities and so, on average, tend to have higher relative fitness. But, living longer does not \emph{guarantee} reproductive success. Individuals that live a long time can have low relative fitness.   
	
	Sexual selection can \emph{directly} affect reproductive success.\marginnote{Secondary sexual traits are the traits related directly to sexual selection, such as antlers on a moose or the bright coloration of male birds.} A male that wins a territorial competition will reproduce. A male that is chosen by a female will reproduce. Thus, sexual selection can affect the evolution of secondary sexual traits.
	
	\item How does sexual selection lead to the evolution of sexual dimorphism?



\end{enumerate}

\section*{Example exam questions}

These are examples only and not exhaustive. One or more of these questions may or may not appear on the exam.

\bigskip

\noindent \rule{1in}{0.4pt} Unrelated organisms evolve similar traits by adapting to\\
\noindent\hspace*{1in} similar habitats. 

\bigskip

\noindent True\hspace{1em}False\hspace{1em} Both homologies and analogies can evolve by \\
\noindent \phantom{True\hspace{1em}False\hspace{1em}}  natural selection.

\bigskip

\noindent Some beetles and flies have antler-like structures on their heads, much like male deer do. The existence of antlers in beetle, fly, and deer species with strong male-male competition is an example of 


\smallskip

\textsc{a}. homology.\\
\textsc{b}. common ancestry. \\
\textsc{c}. convergent evolution. \\
\textsc{d}. descent with modification. \\
\textsc{e}. intersexual selection.



%\rotatebox[origin=c]{180}{}

\vskip0pt plus 1fill

\hfill Answers \marginnote{\hfill\reflectbox{Convergent evolution. True. E.}}

\end{document}