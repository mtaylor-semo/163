%!TEX TS-program = lualatex
%!TEX encoding = UTF-8 Unicode

\documentclass[letterpaper]{tufte-handout}

%\geometry{showframe} % display margins for debugging page layout

\usepackage{fontspec}
\def\mainfont{Linux Libertine O}
\setmainfont[Ligatures={Common,TeX}, Contextuals={NoAlternate}, BoldFont={* Bold}, ItalicFont={* Italic}, Numbers={OldStyle}]{\mainfont}
\setsansfont[Scale=MatchLowercase, Numbers={OldStyle}]{Linux Biolinum O} 
\usepackage{microtype}

\usepackage{graphicx} % allow embedded images
  \setkeys{Gin}{width=\linewidth}
  \graphicspath{	{/Users/goby/teach/163/lectures/}}%}%

\usepackage{amsmath}  % extended mathematics
\usepackage{booktabs} % book-quality tables
\usepackage{units}    % non-stacked fractions and better unit spacing
\usepackage{multicol} % multiple column layout facilities
%\usepackage{fancyvrb} % extended verbatim environments
%  \fvset{fontsize=\normalsize}% default font size for fancy-verbatim environments

\usepackage{enumitem}

\makeatletter
% Paragraph indentation and separation for normal text
\renewcommand{\@tufte@reset@par}{%
  \setlength{\RaggedRightParindent}{1.0pc}%
  \setlength{\JustifyingParindent}{1.0pc}%
  \setlength{\parindent}{1pc}%
  \setlength{\parskip}{0pt}%
}
\@tufte@reset@par

% Paragraph indentation and separation for marginal text
\renewcommand{\@tufte@margin@par}{%
  \setlength{\RaggedRightParindent}{0pt}%
  \setlength{\JustifyingParindent}{0.5pc}%
  \setlength{\parindent}{0.5pc}%
  \setlength{\parskip}{0pt}%
}
\makeatother

% Set up the spacing using fontspec features
   \renewcommand\allcapsspacing[1]{{\addfontfeatures{LetterSpace=15}#1}}
   \renewcommand\smallcapsspacing[1]{{\addfontfeatures{LetterSpace=10}#1}}

\title{{\scshape bi} 163 Study Guide 08}

\date{} % without \date command, current date is supplied

\begin{document}

\maketitle	% this prints the handout title, author, and date

%\printclassoptions
\section*{Natural selection and relative fitness}

We\marginnote{\textbf{Read:} pgs. 12--14, 468--470, 491--492.} discussed the details of natural selection and how differences in the relative fitness among individuals contributes to the process of natural selection.

\section*{Vocabulary}

\vspace{-1\baselineskip}
\begin{multicols}{2}
natural selection \\
adaptation \\
reproductive output \\
relative fitness \\


\end{multicols}

\section*{Concepts}

You should \emph{write} clear and concise answers to each question in the Concepts section.  Remember to ``think horizontally'' and to ``connect the dots.'' 

\begin{enumerate}

	\item Why is ``survival of the fittest'' not a good explanation for how natural selection works? (Bonus: Who coined the phrase?)

	\item Describe how natural selection works in a population. Explain each the key points (given in the text as two observations and two inferences) that explain how natural selection works.

	\item\label{nsOne} Explain what is meant by ``individuals in a population vary.''

	\item Explain what is meant by ``populations overproduce offspring.'' Does that mean every individual in a population overproduces offspring? Why or why not?

	\item Explain what is meant by ``not all individuals in a population will survive.'' Does this mean that only individuals that live a full, natural life will reproduce? Does it mean that individuals either die or survive?

	\item Explain what is meant by ``traits determine survival.'' Does it mean that only the strongest or fastest individuals survive? Does it mean that only the strongest or fastest individuals reproduce? Why or why not?

	\item\label{nsFour} Explain what is meant by ``more survival, more offspring.'' Does living a long life mean that individual is guaranteed to produce more offspring compared to an individual that lives a shorter time? Why or why not? 

	\item Explain what is meant by heritable traits.

	\item Explain how the statements in items \ref{nsOne}--\ref{nsFour} lead to traits spreading through a population. Does it relate to the ability of individuals to move to one area or another? Why or why not?

	\item Explain what are adaptations. Relate the evolution of adaptations to the process of natural selection.
	
	\item Explain how the \emph{process} of natural selection cause the \emph{pattern} of descent with modification.

	\item Explain relative fitness.  Why is fitness relative? Explain how relative fitness influences the process of natural selection and the evolution of adaptations.  


\end{enumerate}

\end{document}