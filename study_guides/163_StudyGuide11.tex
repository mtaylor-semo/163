%!TEX TS-program = lualatex
%!TEX encoding = UTF-8 Unicode

\documentclass[letterpaper]{tufte-handout}

%\geometry{showframe} % display margins for debugging page layout

\usepackage{fontspec}
\def\mainfont{Linux Libertine O}
\setmainfont[Ligatures={Common,TeX}, Contextuals={NoAlternate}, BoldFont={* Bold}, ItalicFont={* Italic}, Numbers={OldStyle}]{\mainfont}
\setsansfont[Scale=MatchLowercase, Numbers={OldStyle}]{Linux Biolinum O} 
\usepackage{microtype}

\usepackage{graphicx} % allow embedded images
  \setkeys{Gin}{width=\linewidth,totalheight=\textheight,keepaspectratio}
  \graphicspath{{img/}} % set of paths to search for images
\usepackage{amsmath}  % extended mathematics
\usepackage{booktabs} % book-quality tables
\usepackage{units}    % non-stacked fractions and better unit spacing
\usepackage{multicol} % multiple column layout facilities
%\usepackage{fancyvrb} % extended verbatim environments
%  \fvset{fontsize=\normalsize}% default font size for fancy-verbatim environments

\usepackage{enumitem}

\makeatletter
% Paragraph indentation and separation for normal text
\renewcommand{\@tufte@reset@par}{%
  \setlength{\RaggedRightParindent}{1.0pc}%
  \setlength{\JustifyingParindent}{1.0pc}%
  \setlength{\parindent}{1pc}%
  \setlength{\parskip}{0pt}%
}
\@tufte@reset@par

% Paragraph indentation and separation for marginal text
\renewcommand{\@tufte@margin@par}{%
  \setlength{\RaggedRightParindent}{0pt}%
  \setlength{\JustifyingParindent}{0.5pc}%
  \setlength{\parindent}{0.5pc}%
  \setlength{\parskip}{0pt}%
}
\makeatother

% Set up the spacing using fontspec features
   \renewcommand\allcapsspacing[1]{{\addfontfeatures{LetterSpace=15}#1}}
   \renewcommand\smallcapsspacing[1]{{\addfontfeatures{LetterSpace=10}#1}}

\title{{\scshape bi} 163 Study Guide 11}

\date{} % without \date command, current date is supplied

\begin{document}

\maketitle	% this prints the handout title, author, and date

%\printclassoptions
\section*{Genetic drift and other evolutionary microevolutionary processes}

We\marginnote{\textbf{Read:} pages 355--356, 486 (``Conditions for Hardy-Weinberg Equilibrium,'' including the small box on the lower left), 488--491, 480--481, 500--501} covered ways other than natural selection that populations evolve, including gene flow, mutation, non-random mating, and genetic drift. We introduced the concept of microevolution.

\section*{Vocabulary}

\vspace{-1\baselineskip}
\begin{multicols}{2}
allele frequency \\
gene flow \\
mutation \\
neutral mutations \\
heterozygosity \\
homozygosity \\
non-random mating \\
inbreeding \\
genetic drift \\
bottleneck effect \\
founder effect \\
microevolution \\
speciation \\
macroevolution

\end{multicols}

\section*{Concepts}

You should \emph{write} clear and concise answers to each question in the Concepts section.  Remember to ``think horizontally'' and to ``connect the dots.'' 

\begin{enumerate}

	\item Be sure you have a basic understanding of allele frequency.  Frequency refers to how common something is. An allele that is common in a population has a high frequency. An allele that is rare in a population has a low frequency.
	
	\item What are the five evolutionary processes that cause allele frequencies in a population to change?  Explain how each causes allele frequencies to change.
	
	\textsc{Note:} When you read page 486, you'll read a small bit about Hardy-Weinberg equilibrium. This refers to a pair of math equations that demonstrate populations will not evolve as long as five assumptions are met. The five assumptions are listed in the small box on that page. The five evolutionary processes you jsut identified are \emph{violations} of those assumptions. You will learn more about these in \textsc{bi} 283 (Genetics; I know you can hardly wait!).

	\item Compare and contrast bottleneck and founder effects.  How are they similar?  How do they differ?  Explain why they are types of genetic drift.
		
	\item If heterozygosity is the proportion of heterozygotes in a population, then what is homozygosity?	
	
	\item How does non-random mating decrease heterozygosity? If heterozygosity is decreasing, what must be increasing?

	\item Which two processes (other than natural selection) decrease heterozygosity? Is this a good thing or a bad thing for the population? Why?
	
	\item Which two processes (other than natural selection) tend to increase heterozygosity? Is this a good thing or a bad thing for the population? Why?

	\item Is genetic drift present in all populations? Why? Is genetic drift more pronounced in large populations or small populations?  Why?

	\item Explain the relation between the definition of ``evolution'' and microevolution? How are they the same or different? 
\end{enumerate}

\end{document}