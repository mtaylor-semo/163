%!TEX TS-program = lualatex
%!TEX encoding = UTF-8 Unicode

\documentclass[letterpaper]{tufte-handout}

%\geometry{showframe} % display margins for debugging page layout

\usepackage{fontspec}
\def\mainfont{Linux Libertine O}
\setmainfont[Ligatures={Common,TeX}, Contextuals={NoAlternate}, BoldFont={* Bold}, ItalicFont={* Italic}, Numbers={OldStyle}]{\mainfont}
\setsansfont[Scale=MatchLowercase]{Linux Biolinum O} 
\usepackage{microtype}

\usepackage{graphicx} % allow embedded images
  \setkeys{Gin}{width=\linewidth,totalheight=\textheight,keepaspectratio}
  \graphicspath{{img/}} % set of paths to search for images
\usepackage{amsmath}  % extended mathematics
\usepackage{booktabs} % book-quality tables
\usepackage{units}    % non-stacked fractions and better unit spacing
\usepackage{multicol} % multiple column layout facilities
%\usepackage{fancyvrb} % extended verbatim environments
%  \fvset{fontsize=\normalsize}% default font size for fancy-verbatim environments

\makeatletter
% Paragraph indentation and separation for normal text
\renewcommand{\@tufte@reset@par}{%
  \setlength{\RaggedRightParindent}{1.0pc}%
  \setlength{\JustifyingParindent}{1.0pc}%
  \setlength{\parindent}{1pc}%
  \setlength{\parskip}{0pt}%
}
\@tufte@reset@par

% Paragraph indentation and separation for marginal text
\renewcommand{\@tufte@margin@par}{%
  \setlength{\RaggedRightParindent}{0pt}%
  \setlength{\JustifyingParindent}{0.5pc}%
  \setlength{\parindent}{0.5pc}%
  \setlength{\parskip}{0pt}%
}
\makeatother

% Set up the spacing using fontspec features
   \renewcommand\allcapsspacing[1]{{\addfontfeatures{LetterSpace=15}#1}}
   \renewcommand\smallcapsspacing[1]{{\addfontfeatures{LetterSpace=10}#1}}

\title{{\scshape bi} 063 lab study guide 02}

\date{} % without \date command, current date is supplied

\begin{document}

\maketitle	% this prints the handout title, author, and date

%\printclassoptions
\section*{Science, reasoning, and the scientific method}

We\marginnote{\textbf{Read:} 16--18, 464 (Scala Natur\ae{} and Classification of Species), 548.} discussed classification, taxonomy, and binomial nomenclature.

\section*{Vocabulary}

\vspace{-1\baselineskip}
\begin{multicols}{2}
Carolus Linnaeus\\
\textit{Systema Natur\ae}\\
Binomial nomenclature\\
Taxonomic classification\\
\end{multicols}

\section{Concepts}

You should \emph{write} clear and concise answers to each question in the Concepts section.  Remember to ``think horizontally'' and to ``connect the dots.'' 

\begin{enumerate}
	\item Who was Carolus Linnaeus? What was his contribution to the sciences of taxonomy and classification?

	\item Approximately how many species have been cataloged (described)? About how many species might exist on Earth today?
	
	\item Name all eight levels\marginnote{A handy memory trick is to ask yourself, "Do kings play chess on fine-grained sand?" The first letter of each word matches one level of classification in the proper order, i.e., domain, kingdom, phylum, etc.} in proper hierarchical order, of the Linnaean classification system. Why do scientists use this classification system?
	
	\item What is binomial nomenclature/ Why must all species have a scientific name? Why not just use common names like robin or blackbird?

\end{enumerate}

\end{document}