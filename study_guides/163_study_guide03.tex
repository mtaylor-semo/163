%!TEX TS-program = lualatex
%!TEX encoding = UTF-8 Unicode

\documentclass[letterpaper]{tufte-handout}

%\geometry{showframe} % display margins for debugging page layout

\usepackage{fontspec}
\def\mainfont{Linux Libertine O}
\setmainfont[Ligatures={Common,TeX}, Contextuals={NoAlternate}, BoldFont={* Bold}, ItalicFont={* Italic}, Numbers={OldStyle}]{\mainfont}
\setsansfont[Scale=MatchLowercase, Numbers={OldStyle}]{Linux Biolinum O} 
\usepackage{microtype}

\usepackage{graphicx} % allow embedded images
  \setkeys{Gin}{width=\linewidth}
  \graphicspath{	{/Users/goby/teach/163/lectures/}}%}%

\usepackage{amsmath}  % extended mathematics
%\usepackage{booktabs} % book-quality tables
%\usepackage{units}    % non-stacked fractions and better unit spacing
%\usepackage{multicol} % multiple column layout facilities
%\usepackage{fancyvrb} % extended verbatim environments
%  \fvset{fontsize=\normalsize}% default font size for fancy-verbatim environments

\usepackage{enumitem}

\makeatletter
% Paragraph indentation and separation for normal text
\renewcommand{\@tufte@reset@par}{%
  \setlength{\RaggedRightParindent}{1.0pc}%
  \setlength{\JustifyingParindent}{1.0pc}%
  \setlength{\parindent}{1pc}%
  \setlength{\parskip}{0pt}%
}
\@tufte@reset@par

% Paragraph indentation and separation for marginal text
\renewcommand{\@tufte@margin@par}{%
  \setlength{\RaggedRightParindent}{0pt}%
  \setlength{\JustifyingParindent}{0.5pc}%
  \setlength{\parindent}{0.5pc}%
  \setlength{\parskip}{0pt}%
}
\makeatother

% Set up the spacing using fontspec features
   \renewcommand\allcapsspacing[1]{{\addfontfeatures{LetterSpace=15}#1}}
   \renewcommand\smallcapsspacing[1]{{\addfontfeatures{LetterSpace=10}#1}}

\title{{\scshape bi} 163 Study Guide 03}

\date{} % without \date command, current date is supplied

\begin{document}

\maketitle	% this prints the handout title, author, and date

%\printclassoptions
\section*{Counting alleles in a population}

This study guide%
\marginnote{A good way to practice this is to 
work with a friend. You make up three numbers for each of the
three possible genotypes, and then solve for the allele
frequencies. Your study buddy does the same thing. Then, exchange
problems. You solve your fiend's problem and your friend solves
your problem. Compare answers to be sure you get the same results.} 
%
will show you how to calculate allele frequencies 
based on the number of diploid individuals of each genotype. 
Assume you have 100 individuals in a population, divided among 
three genotypes: \vspace{\baselineskip}

25 $A_1A_1$, \\
50 $A_1A_2$, and \\
25 $A_2A_2$.

\vspace{\baselineskip}

First, count the total number of alleles. The population size
is 100 and each diploid individual has two alleles. Therefore,
\vspace{\baselineskip}

$100 \times 2 = 200$ total alleles.
\vspace{\baselineskip}

\noindent Next, count $A_1$ alleles. Individuals that are homozygous 
$A_1A_1$ each contribute two $A_1$ alleles. Therefore, 
\vspace{\baselineskip}

$25 \times 2 = 50\ A_1$ alleles. 
\vspace{\baselineskip}

\noindent Heterozygous individuals each have one $A_1$ allele, so
\vspace{\baselineskip}

$50 \times 1 = 50\ A_1$ alleles.
\vspace{\baselineskip}

\noindent Add together the number of $A_1$ alleles from the
homozygotes and heterozygotes,
\vspace{\baselineskip}

$50 + 50 = 100$ total $A_1$ alleles.
\vspace{\baselineskip}

Repeat this process for the $A_2$ allele, beginning with 
$A_2A_2$ homozygotes, then heterozygotes.
\vspace{\baselineskip}

$25 \times 2 = 50$,\\
$50 \times 1 = 50$,\\
$50 + 50 = 100$ total $A_2$ alleles.
\vspace{\baselineskip}

Finally, calculate the two allele frequencies. Divide the number of each allele by the total number of alleles. \vspace{\baselineskip}

$A_1$ frequency: $\dfrac{100}{200} = 0.5.$\\[1.5ex]
$A_2$ frequency: $\dfrac{100}{200} = 0.5.$

\subsection{Practice problems}

Calculate allele frequencies. 
Correct answers are included in the right margin
to check your work. If you have problems, see your instructor. \vspace{\baselineskip}

\noindent Problem 1.\vspace{\baselineskip}

16 $T_1T_1$, \\
48 $T_1T_2$, and \\
36 $T_2T_2$.

\vspace{\baselineskip}

Frequency of $T_1$ allele: \rule{1in}{0.4pt} \marginnote{\hfill \reflectbox{$T_1$: 0.4}}

\vspace{\baselineskip}

Frequency of $T_2$ allele: \rule{1in}{0.4pt} \marginnote{\hfill $T_2$: 0.6}

\vspace{\baselineskip}

%%

\noindent Problem 2.\vspace{\baselineskip}

243 $K_1K_1$, \\
54 $K_1K_2$, and \\
3 $K_2K_2$.

\vspace{\baselineskip}

Frequency of $K_1$ allele: \rule{1in}{0.4pt} \marginnote{\hfill $K_1$: 0.9}

\vspace{\baselineskip}

Frequency of $K_2$ allele: \rule{1in}{0.4pt} \marginnote{\hfill $K_2$: 0.1}

\vspace{\baselineskip}

\noindent Problem 3. Some alleles are rare.\vspace{\baselineskip}

1 $R_1R_1$, \\
1998 $R_1R_2$, and \\
998001 $R_2R_2$.

\vspace{\baselineskip}

Frequency of $R_1$ allele: \rule{1in}{0.4pt} \marginnote{\hfill $R_1$: 0.001}

\vspace{\baselineskip}

Frequency of $R_2$ allele: \rule{1in}{0.4pt} \marginnote{\hfill $R_2$: 0.999}

\vspace{\baselineskip}

\noindent Problem 4.\vspace{\baselineskip}

294 $D_1D_1$, \\
252 $D_1D_2$, and \\
54 $D_2D_2$.

\vspace{\baselineskip}

Frequency of $D_1$ allele: \rule{1in}{0.4pt} \marginnote{\hfill $D_1$: 0.7}

\vspace{\baselineskip}

Frequency of $D_2$ allele: \rule{1in}{0.4pt} \marginnote{\hfill $D_2$: 0.3}


\end{document}