%!TEX TS-program = lualatex
%!TEX encoding = UTF-8 Unicode

\documentclass[letterpaper]{tufte-handout}

%\geometry{showframe} % display margins for debugging page layout

\usepackage{fontspec}
\def\mainfont{Linux Libertine O}
\setmainfont[Ligatures={Common,TeX}, Contextuals={NoAlternate}, BoldFont={* Bold}, ItalicFont={* Italic}, Numbers={OldStyle}]{\mainfont}
\setsansfont[Scale=MatchLowercase, Numbers={OldStyle}]{Linux Biolinum O} 
\usepackage{microtype}

\usepackage{graphicx} % allow embedded images
  \setkeys{Gin}{width=\linewidth,totalheight=\textheight,keepaspectratio}
  \graphicspath{{img/}} % set of paths to search for images
\usepackage{amsmath}  % extended mathematics
\usepackage{booktabs} % book-quality tables
\usepackage{units}    % non-stacked fractions and better unit spacing
\usepackage{multicol} % multiple column layout facilities
%\usepackage{fancyvrb} % extended verbatim environments
%  \fvset{fontsize=\normalsize}% default font size for fancy-verbatim environments

\usepackage{enumitem}

\makeatletter
% Paragraph indentation and separation for normal text
\renewcommand{\@tufte@reset@par}{%
  \setlength{\RaggedRightParindent}{1.0pc}%
  \setlength{\JustifyingParindent}{1.0pc}%
  \setlength{\parindent}{1pc}%
  \setlength{\parskip}{0pt}%
}
\@tufte@reset@par

% Paragraph indentation and separation for marginal text
\renewcommand{\@tufte@margin@par}{%
  \setlength{\RaggedRightParindent}{0pt}%
  \setlength{\JustifyingParindent}{0.5pc}%
  \setlength{\parindent}{0.5pc}%
  \setlength{\parskip}{0pt}%
}
\makeatother

% Set up the spacing using fontspec features
   \renewcommand\allcapsspacing[1]{{\addfontfeatures{LetterSpace=15}#1}}
   \renewcommand\smallcapsspacing[1]{{\addfontfeatures{LetterSpace=10}#1}}

\title{{\scshape bi} 163 Study Guide 13}

\date{} % without \date command, current date is supplied

\begin{document}

\maketitle	% this prints the handout title, author, and date

%\printclassoptions
\section*{Speciation}

We\marginnote{\textbf{Read:} 500--513.} learned how reproductive isolation causes new species to evolve. We explored the role of geography, how hybrid zones form, and possible speciation outcomes from hybrid zones.

\section*{Vocabulary}

\vspace{-1\baselineskip}
\begin{multicols}{2}
speciation \\
reproductive isolation \\
prezygotic barriers \\
postzygotic barriers \\
habitat isolation \\
behavioral isolation \\
temporal isolation \\
mechanical isolation \\
gamete incompatibility \\
hybrid sterility \\
hybrid inviability \\
hybrid breakdown \\
allopatric speciation \\
sympatric speciation \\
hybrid zone \\
stability \\
reinforcement \\
fusion \\
ring species 
\end{multicols}

\section*{Concepts}

You should \emph{write} clear and concise answers to each question in the Concepts section.  Remember to ``think horizontally'' and to ``connect the dots.'' 

\begin{enumerate}

	\item What is speciation?  

	\item Explain \marginnote{More than one type of prezygotic and postzygotic barrier can contribute to speciation.}
 the difference between prezygotic and postzygotic isolation barriers. Provide examples for each of the prezygotic and postzygotic barriers.
 
	\item Briefly describe how each of the prezygotic and postzygotic\marginnote{Reproductive isolation occurs when some aspect about the biology of the organism causes the inability to reproduce, such as being adapted to different habitats, or producing sterile hybrid offspring.} isolation barriers can lead to reproductive isolation.

	\item Explain how allopatric speciation can contribute to the evolution of biological reproductive isolation.
	
	\item Explain the difference between allopatric and sympatric speciation. \marginnote{Remember that allopatric and sympatric speciation refer to geography, not biological reproductive isolation.  Organisms can have allopatric distribution but not be different species.}
	
	\item Describe how phylogenetic trees can be used to look for evidence of allopatric speciation.
	
	\item Explain how hybrid zones can form. Compare and contrast each of the three possible outcomes from hybrid zones (stability, reinforcement, fusion). Explain which of the three outcomes continues or maintains the speciation process and which can lead to the loss of species.
	
	\item Explain what is a ring species and how they form.


\end{enumerate}

\end{document}