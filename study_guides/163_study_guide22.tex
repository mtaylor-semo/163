%!TEX TS-program = lualatex
%!TEX encoding = UTF-8 Unicode

\documentclass[letterpaper]{tufte-handout}

%\geometry{showframe} % display margins for debugging page layout

\usepackage{fontspec}
\def\mainfont{Linux Libertine O}
\setmainfont[Ligatures={Common,TeX}, Contextuals={NoAlternate}, BoldFont={* Bold}, ItalicFont={* Italic}, Numbers={OldStyle}]{\mainfont}
\setsansfont[Scale=MatchLowercase, Numbers={OldStyle}]{Linux Biolinum O} 
\usepackage{microtype}

\usepackage{graphicx} % allow embedded images
  \setkeys{Gin}{width=\linewidth,totalheight=\textheight,keepaspectratio}
  \graphicspath{{img/}} % set of paths to search for images
\usepackage{amsmath}  % extended mathematics
\usepackage{booktabs} % book-quality tables
\usepackage{units}    % non-stacked fractions and better unit spacing
\usepackage{multicol} % multiple column layout facilities
%\usepackage{fancyvrb} % extended verbatim environments
%  \fvset{fontsize=\normalsize}% default font size for fancy-verbatim environments

\usepackage{enumitem}

\makeatletter
% Paragraph indentation and separation for normal text
\renewcommand{\@tufte@reset@par}{%
  \setlength{\RaggedRightParindent}{1.0pc}%
  \setlength{\JustifyingParindent}{1.0pc}%
  \setlength{\parindent}{1pc}%
  \setlength{\parskip}{0pt}%
}
\@tufte@reset@par

% Paragraph indentation and separation for marginal text
\renewcommand{\@tufte@margin@par}{%
  \setlength{\RaggedRightParindent}{0pt}%
  \setlength{\JustifyingParindent}{0.5pc}%
  \setlength{\parindent}{0.5pc}%
  \setlength{\parskip}{0pt}%
}
\makeatother

% Set up the spacing using fontspec features
   \renewcommand\allcapsspacing[1]{{\addfontfeatures{LetterSpace=15}#1}}
   \renewcommand\smallcapsspacing[1]{{\addfontfeatures{LetterSpace=10}#1}}

\title{{\scshape bi} 163 study guide 22}
\author{Human population and climate change }
\date{} % without \date command, current date is supplied

\begin{document}

\maketitle	% this prints the handout title, author, and date

%\printclassoptions
We\marginnote{\textbf{Read:} 1201--1205, 1160--1161, 1163--1164, 1272--1274.} finished climate, and thencovered the human population, emphasizing U.S. demographics. We also covered anthropogenic climate change.

\section*{Vocabulary}

\vspace{-1\baselineskip}
\begin{multicols}{2}
climate \\
solar radiation \\
demography \\
sustainability \\
greenhouse effect
\end{multicols}

\section*{Concepts}

You should \emph{write} clear and concise answers to each question in the Concepts section.  Remember to ``think horizontally'' and to ``connect the dots.'' 

\begin{enumerate}

	\item Briefly describe how the input of solar radiation into Earth's atmosphere creates air circulation and precipitation patterns. You should be able to explain this for the equator and 30° latitude (N or S).

	\item Why do deserts occur at 30° of latitude? Why do tropical rain forests occur at the equator?

	\item The human population appears to be growing exponentially but the growth rate is slowing. How do you explain this apparent discrepancy? 

	\item What does it mean to say that the average person uses 5.9$\times$ more energy than is sustainable? Why does this matter?

	\item Explain the greenhouse effect. Use the greenhouse effect to explain the relationship between accumulation of greenhouse gasses and temperature.
	
	\item Why is global climate change a more accurate term than global warming?
	
	\item Evidence suggests that global temperatures have been warmer at some points in the evolutionary past than current temperatures. If so, then why are we concerned about climate change?
	
	\item Explain the effects that climate change may have on the distribution of species and ecosystems.

\end{enumerate}

\end{document}