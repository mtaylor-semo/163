%!TEX TS-program = lualatex
%!TEX encoding = UTF-8 Unicode

\documentclass[12pt, hidelinks]{exam}
\usepackage{graphicx}
	\graphicspath{{/Users/goby/Pictures/teach/163/lab/}
	{img/}} % set of paths to search for images

\usepackage{geometry}
\geometry{letterpaper, left=1.5in, bottom=1in}                   
%\geometry{landscape}                % Activate for for rotated page geometry
\usepackage[parfill]{parskip}    % Activate to begin paragraphs with an empty line rather than an indent
\usepackage{amssymb, amsmath}
\usepackage{mathtools}
	\everymath{\displaystyle}

\usepackage{fontspec}
\setmainfont[Ligatures={TeX}, BoldFont={* Bold}, ItalicFont={* Italic}, BoldItalicFont={* BoldItalic}, Numbers={OldStyle}]{Linux Libertine O}
\setsansfont[Scale=MatchLowercase,Ligatures=TeX]{Linux Biolinum O}
\setmonofont[Scale=MatchLowercase]{Inconsolatazi4}
\usepackage{microtype}


% To define fonts for particular uses within a document. For example, 
% This sets the Libertine font to use tabular number format for tables.
 %\newfontfamily{\tablenumbers}[Numbers={Monospaced}]{Linux Libertine O}
% \newfontfamily{\libertinedisplay}{Linux Libertine Display O}

\usepackage{booktabs}
\usepackage{multicol}
\usepackage[normalem]{ulem}

\usepackage{longtable}
%\usepackage{siunitx}
\usepackage{array}
\newcolumntype{L}[1]{>{\raggedright\let\newline\\\arraybackslash\hspace{0pt}}p{#1}}
\newcolumntype{C}[1]{>{\centering\let\newline\\\arraybackslash\hspace{0pt}}p{#1}}
\newcolumntype{R}[1]{>{\raggedleft\let\newline\\\arraybackslash\hspace{0pt}}p{#1}}

\usepackage{enumitem}
\usepackage{hyperref}
%\usepackage{placeins} %PRovides \FloatBarrier to flush all floats before a certain point.
\usepackage{hanging}

\usepackage[sc]{titlesec}

%% Commands for Exam class
\renewcommand{\solutiontitle}{\noindent}
\unframedsolutions
\SolutionEmphasis{\bfseries}

\renewcommand{\questionshook}{%
	\setlength{\leftmargin}{-\leftskip}%
}

%Change \half command from 1/2 to .5
\renewcommand*\half{.5}

\pagestyle{headandfoot}
\firstpageheader{\textsc{bi}\,063 Evolution and Ecology}{}{\ifprintanswers\textbf{KEY}\else Name: \enspace \makebox[2.5in]{\hrulefill}\fi}
\runningheader{}{}{\footnotesize{pg. \thepage}}
\footer{}{}{}
\runningheadrule

\newcommand*\AnswerBox[2]{%
    \parbox[t][#1]{0.92\textwidth}{%
    \begin{solution}#2\end{solution}}
%    \vspace*{\stretch{1}}
}

\newenvironment{AnswerPage}[1]
    {\begin{minipage}[t][#1]{0.92\textwidth}%
    \begin{solution}}
    {\end{solution}\end{minipage}
    \vspace*{\stretch{1}}}

\newlength{\basespace}
\setlength{\basespace}{5\baselineskip}

\newcommand{\hidepoints}{%
	\pointsinmargin\pointformat{}
}

\newcommand{\showpoints}{%
	\nopointsinmargin\pointformat{(\thepoints)}
}

%\printanswers

\hidepoints

\begin{document}

\subsection*{Vertebrate embryos}%(\numpoints\ points)

Here are some photos of vertebrate embryos. They have been scaled to
the same size, rotated to the same angle, converted to 
black and white, placed on dark backgrounds so that you can compare 
them easily. The photos were taken with different
photographic techniques, so that introduces some differences into their
overall appearance. They are not quite all at the same developmental
stage, but they are close.

\begin{longtable}[c]{@{}lll@{}}
\toprule
\includegraphics[height=5cm]{06_vert_embryo_A_chicken} 	&
\includegraphics[height=5cm]{06_vert_embryo_B_zebrafish}	&
\includegraphics[height=5cm]{06_vert_embryo_C_human}\\
A \ifprintanswers\textbf{\large Chicken}\fi & 
B \ifprintanswers\textbf{\large Zebrafish}\fi & 
C \ifprintanswers\textbf{\large Human}\fi  \\[4ex]
\midrule
\includegraphics[height=5cm]{06_vert_embryo_D_cat}	&
\includegraphics[height=5cm]{06_vert_embryo_E_mouse}	&
\includegraphics[height=5cm]{06_vert_embryo_F_pig}\\
D \ifprintanswers\textbf{\large Cat}\fi &
E \ifprintanswers\textbf{\large Mouse}\fi 	&	
F \ifprintanswers\textbf{\large Pig}\fi \\[4ex]
\bottomrule
\end{longtable}

\begin{questions}

\question
Can you identify the human? How about the
mouse, pig, chicken, cat, and zebrafish? Place the names next to the appropriate letter. 

\ifprintanswers\textbf{Yes I can. See above.}\fi

\newpage

Embryos have many structures that have little or no
function during the embryo stage; they become useful after birth. As a result, 
the structures tend not to be modified much by natural selection. For example,
\emph{all of these embryos, including the human, have tails;} that is,
portions of the backbone that extend beyond the hind legs. You can see the tails 
in the photos on the previous page.

Now consider the following detailed drawings of nine vertebrate embryos.
The illustrated regions correspond to the head, throat and upper body.
In the throat region are \emph{pharyngeal arches}, indicated by the
bracket or arrows. These arches are found in all vertebrate embryos. 
In fishes, pharyngeal arches develop into the gills. In other animals, 
they develop into other structures in the throat region. Like the tail, the pharyngeal
arches do not have a function in the embryo.

\begin{center}
	\includegraphics[width=\textwidth]{06_vert_embryo_pharygeal_arches}
\end{center}

\question[3]
Are the similarities of a tail and pharyngeal arches in these embryos due to homology or analogy? Explain.

\AnswerBox{5\baselineskip}{Homology. The tail and pharyngeal arches do not serve any function. Therefore,
the tail and pharyngeal arches are homologous among vertebrates.}

\question[2]
How do these data about embryos affect your hypothesis? Does it support or
falsify any part of it? Explain why.

\AnswerBox{3\baselineskip}{Answer depends on student hypothesis.}


\subsubsection*{Dolphin embryos}

Consider dolphins, which are a type of whale. They are mammals so they have hair (not much),
they maintain a constant internal temperature, and they feed their young with
milk. They breathe air, with a single nostril located near the top of
the head. Their front limbs are flippers, and they have no hind limbs
(although they do have tiny partial pelvic bones, which serve to anchor
the genitals).

\question[3]
Your hypothesis includes a whale.  Whales and dolphins do not have 
rear limbs. Based on your hypothesis, would you expect to 
see hind limb buds at any stage in the embryo of a dolphin? Explain.

\AnswerBox{5\baselineskip}{Open ended but most write that they would not expect to see rear limb buds because dolphins do not have rear limbs as adults.}

\question[2]
Look at the dolphin embryo images on screen. Do they agree
with your prediction above? Explain how they affect your hypothesis.

\AnswerBox{3\baselineskip}{Answer depends on student hypothesis.}

\end{questions}

\end{document}  