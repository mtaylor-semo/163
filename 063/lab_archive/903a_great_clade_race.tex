%!TEX TS-program = lualatex
%!TEX encoding = UTF-8 Unicode

\documentclass[12pt, hidelinks]{exam}
\usepackage{graphicx}
	\graphicspath{{/Users/goby/Pictures/teach/163/lab/}
	{img/}} % set of paths to search for images

\usepackage{geometry}
\geometry{letterpaper, left=1.5in, bottom=1in}                   
%\geometry{landscape}                % Activate for for rotated page geometry
%\usepackage[parfill]{parskip}    % Activate to begin paragraphs with an empty line rather than an indent
\usepackage{amssymb, amsmath}
\usepackage{mathtools}
	\everymath{\displaystyle}

\usepackage{fontspec}
\setmainfont[Ligatures={TeX}, BoldFont={* Bold}, ItalicFont={* Italic}, BoldItalicFont={* BoldItalic}, Numbers={OldStyle}]{Linux Libertine O}
\setsansfont[Scale=MatchLowercase,Ligatures=TeX]{Linux Biolinum O}
%\setmonofont[Scale=MatchLowercase]{Inconsolatazi4}
\usepackage{microtype}

\usepackage{unicode-math}
\setmathfont[Scale=MatchLowercase]{Asana Math}
%\setmathfont[Scale=MatchLowercase]{XITS Math}

% To define fonts for particular uses within a document. For example, 
% This sets the Libertine font to use tabular number format for tables.
\newfontfamily{\tablenumbers}[Numbers={Monospaced}]{Linux Libertine O}
\newfontfamily{\libertinedisplay}{Linux Libertine Display O}

\usepackage{booktabs}
\usepackage{multicol}
\usepackage[normalem]{ulem}

%\usepackage{tabularx}
\usepackage{longtable}
%\usepackage{siunitx}
\usepackage{array}
\newcolumntype{L}[1]{>{\raggedright\let\newline\\\arraybackslash\hspace{0pt}}p{#1}}
\newcolumntype{C}[1]{>{\centering\let\newline\\\arraybackslash\hspace{0pt}}p{#1}}
\newcolumntype{R}[1]{>{\raggedleft\let\newline\\\arraybackslash\hspace{0pt}}p{#1}}

\usepackage{enumitem}
\usepackage{hyperref}
%\usepackage{placeins} %PRovides \FloatBarrier to flush all floats before a certain point.
\usepackage{hanging}

\usepackage[sc]{titlesec}


\renewcommand{\solutiontitle}{\noindent}
\unframedsolutions
\SolutionEmphasis{\bfseries}

%Change \half command from 1/2 to .5
\renewcommand*\half{.5}


\makeatletter
\def\SetTotalwidth{\advance\linewidth by \@totalleftmargin
\@totalleftmargin=0pt}
\makeatother


\pagestyle{headandfoot}
\firstpageheader{BI 063: Evolution and Ecology}{}{\ifprintanswers\textbf{KEY}\else Name: \enspace \makebox[2.5in]{\hrulefill}\fi}
\runningheader{}{}{\footnotesize{pg. \thepage}}
\footer{}{}{}
\runningheadrule

\newcommand*\AnswerBox[2]{%
    \parbox[t][#1]{0.92\textwidth}{%
    \begin{solution}#2\end{solution}}
%    \vspace*{\stretch{1}}
}

\newenvironment{AnswerPage}[1]
    {\begin{minipage}[t][#1]{0.92\textwidth}%
    \begin{solution}}
    {\end{solution}\end{minipage}
    \vspace*{\stretch{1}}}

\newlength{\basespace}
\setlength{\basespace}{5\baselineskip}

%\printanswers

\begin{document}

\subsection*{The Great Clade Race\footnote{From an exercise developed by Goldstein, D.W. 2003. The American Biology Teacher 65:679--682.}}

Phylogenetic trees are hypotheses about the relationships of organisms. These relationships can be inferred using a technique called cladistics. This technique, like many in biology, uses a set of vocabulary and procedures that are important to learn but can be confusing for beginners. To avoid confusion, you will be introduced to cladistic techniques through a simple game called The Great Clade Race. The game is a puzzle that eases you “cladistic thinking” and concepts without the overwhelming jargon. Later, you will gain greater understanding of the concepts with the proper vocabulary and procedures. Even with greater understanding, perhaps you'll always view phylogenetic hypotheses as a puzzle that needs to be solved.

To play this game, your group will be given a set of nine small cards with various shapes on them and lettered \textsc{a}–\textsc{i} in the upper right. You will use cards \textsc{a--g} to complete two tasks. Set aside cards \textsc{h} and \textsc{i}. You will not need them today. \vspace*{2\baselineskip}

\noindent\textsc{Task 1: classification}

\textit{Organize the cards into distinct groups} using any criteria you want (e.g., types of shapes, number of shapes, or anything else you can think of). You can make as many or as few groups as you like but be sure you can tell what criteria you used for each group. You may be asked to explain or defend your grouping to your classmates. \emph{There is no right answer.} Go for it!

When all tables have finished classifying their cards, I will have a brief discussion with the class about the criteria chosen by the students at each table.\vspace*{1\baselineskip}

You grouped your cards using criteria of your choosing. Imagine doing something similar for all organisms on Earth, using their many different traits (shapes). Your classification would probably differ from another person's classification in many ways. The problem is that your classification based on arbitrarily chosen characters would not be any better or worse than another classification based on other characters.

For example, Carolus Linnaeus created a class in the animal kingdom called Vermes. This class contained any invertebrate that did not have a jointed legs. Vermes included organisms as different as worms, clams, corals, sea urchins, and different kinds of intestinal parasites. Vermes is no longer accepted as a valid class. Why?  As scientists learned more about the organisms, they realized that Vermes contained animals that had far more differences than similarities. Rather than using characters chosen at random, scientists developed a classification scheme that reflects the evolutionary relationships of organisms. The next task will help you understand how to do just that.\vspace*{2\baselineskip}

\newpage

\noindent\textsc{Task 2: shared shapes and the Great Clade Race}

Imagine a footrace through the woods. Eight runners start at the same spot on the edge of the woods. The race ends on the other side of the woods. The racing path forks repeatedly into two paths. Each series of forks leads to a separate finish line on the other side of the woods. Each runner carries one of the cards you used above. The shapes on the cards determine the path the runner will take, following a set of rules. The runners must follow these rules to complete the race. 

\begin{itemize}

	\item At each fork is a sign with one or two shapes on it. The side of the sign a shape is on tells the runner which fork to take. For example, a runner than sees Sign A (below) will take the right fork if her card has a stop sign shape and the left fork if her card does not. A runner that sees Sign B will take the left fork if his card has a square and the right fork if his card has a star. 
	
%	\begin{center}
	
		\includegraphics[height=1.5in]{03a_clade_race_signs}
	
%	\end{center}
	
	\item The runner must take the route specified by matching the shapes to the cards. The runner with square cannot take the fork with the star. 

	\item A runner can take only one of the two forks, forcing each runner to end up at a different finish line on the other side of the woods.
	
	
%	\item The runner \emph{must} take that path if the shape matches the check-in station.
	
%	\item Some paths do not have a check-in station so any runner can pass.
	
	\item Each runner must finish at a separate finish line.

\end{itemize}

\emph{Draw a map of the race course, using the cards and the rules.}  You can draw the map on the back of this handout. You must draw the forking path, the shapes indicating the proper path, and where each runner finishes. Draw the shapes next to the proper path. You do not have to draw miniature signs. \vspace*{1\baselineskip}

Once you have completed your map, study it carefully. Notice that many runners shared the same path early in the race because the shared the same shapes. As the race proceeded, fewer and fewer runners shared the same path. Eventually, each runner ended at a different finish line because each runner had a different combination of shapes overall. This created a hierarchy of shapes that defined the race course. All runners shared the circle shape but only some runners shared the square shape and fewer shared the triangle shape.  All runners that had triangles on their cards also had squares but not all runners with squares had triangles.  None of the runners with the diamond shape had the triangle shape, and so on.  This created a nested hierarchy of similarity based on shared shapes. Study your map to be certain you understand this. 

This concept can be extended to classification of organisms. Organisms have sets of shared characters as well as some unique characters. Some organisms share more characters than others, as shown in the next diagram. Notice that all organisms except the lancelet share a vertebral column. The lamprey has a vertebral column but does not have jaws. Near the top of the tree (the right side), organisms share fewer and fewer characters. Only turtle and leopard share an amnion and only the leopard has hair.  Of course, the ancestor of turtles and leopards and the other organisms did not run through the woods carrying cards with shapes. But, these organisms have different sets of shared characters as a result of following different evolutionary paths.

\begin{center}
	\includegraphics[width=\textwidth]{03a_vertebrate_tree}
\end{center}

The sets of shared characters are used two ways. The first is to make a classification scheme, as shown by the vertical lines above. Organisms in the same taxonomic group share the same sets of characters. Organisms at higher taxonomic levels (e.g., class or order) all share certain characteristics. Organisms at lower taxonomic levels (e.g., families or species) share characters that are not shared with all organisms at higher taxonomic levels. For example, all amniotes have amnionic eggs and all amniotes are tetrapods (organism with four appendages) but not all tetrapods have amniotic eggs. Frogs have four legs but do not have amniotic eggs.

You may wonder how this method is any better than the method you used in Task~1. The advantage is that classification based on shared characters can be used to reflect the evolutionary relationships of the organisms. This is a powerful scientific tool. 

First, though, we have to understand how organisms end up with shared characters.  One way is that the organisms inherit the characters from a common ancestor, much like siblings share characteristics inherited from their biological parents. Another way organisms have similar characters is because organisms may live in the same type of environment. For example, dolphins and sharks both have streamlined bodies for swimming through water. Later, you will learn a technique for deciding whether a character is shared because of common ancestry or because of some other factor.

%\begin{questions}

%\end{questions}

\end{document}  