%!TEX TS-program = lualatex
%!TEX encoding = UTF-8 Unicode

\documentclass[12pt, hidelinks]{exam}
\usepackage{graphicx}
	\graphicspath{{/Users/goby/Pictures/teach/163/lab/}
	{img/}} % set of paths to search for images

\usepackage{geometry}
\geometry{letterpaper, left=1.5in, bottom=1in}                   
%\geometry{landscape}                % Activate for for rotated page geometry
\usepackage[parfill]{parskip}    % Activate to begin paragraphs with an empty line rather than an indent
\usepackage{amssymb, amsmath}
\usepackage{mathtools}
	\everymath{\displaystyle}

\usepackage{fontspec}
\setmainfont[Ligatures={TeX}, BoldFont={* Bold}, ItalicFont={* Italic}, BoldItalicFont={* BoldItalic}, Numbers={OldStyle}]{Linux Libertine O}
\setsansfont[Scale=MatchLowercase,Ligatures=TeX, Numbers=OldStyle]{Linux Biolinum O}
%\setmonofont[Scale=MatchLowercase]{Inconsolatazi4}
\usepackage{microtype}


% To define fonts for particular uses within a document. For example, 
% This sets the Libertine font to use tabular number format for tables.
 %\newfontfamily{\tablenumbers}[Numbers={Monospaced}]{Linux Libertine O}
% \newfontfamily{\libertinedisplay}{Linux Libertine Display O}

\usepackage{booktabs}
\usepackage{multicol}
\usepackage[normalem]{ulem}

\usepackage{longtable}
%\usepackage{siunitx}
\usepackage{array}
\newcolumntype{L}[1]{>{\raggedright\let\newline\\\arraybackslash\hspace{0pt}}m{#1}}
\newcolumntype{C}[1]{>{\centering\let\newline\\\arraybackslash\hspace{0pt}}m{#1}}
\newcolumntype{R}[1]{>{\raggedleft\let\newline\\\arraybackslash\hspace{0pt}}m{#1}}

\usepackage{enumitem}
\setlist{leftmargin=*}
\setlist[1]{labelindent=\parindent}
\setlist[enumerate]{label=\textsc{\alph*}.}
\setlist[itemize]{label=\color{gray}\textbullet}

\usepackage{hyperref}
%\usepackage{placeins} %PRovides \FloatBarrier to flush all floats before a certain point.
\usepackage{hanging}

\usepackage[sc]{titlesec}

%% Commands for Exam class
\renewcommand{\solutiontitle}{\noindent}
\unframedsolutions
\SolutionEmphasis{\bfseries}

\renewcommand{\questionshook}{%
	\setlength{\leftmargin}{-\leftskip}%
}

\newcommand{\hidepoints}{%
	\pointsinmargin\pointformat{}
}

\newcommand{\showpoints}{%
	\nopointsinmargin\pointformat{(\thepoints)}
}

%Change \half command from 1/2 to .5
\renewcommand*\half{.5}

\pagestyle{headandfoot}
\firstpageheader{\textsc{bi}\,063 Evolution and Ecology}{}{\ifprintanswers\textbf{KEY}\else Name: \enspace \makebox[2.5in]{\hrulefill}\fi}
\runningheader{}{}{\footnotesize{pg. \thepage}}
\footer{}{}{}
\runningheadrule

\newcommand*\AnswerBox[2]{%
    \parbox[t][#1]{0.92\textwidth}{%
    \begin{solution}#2\end{solution}}
%    \vspace*{\stretch{1}}
}

\newenvironment{AnswerPage}[1]
    {\begin{minipage}[t][#1]{0.92\textwidth}%
    \begin{solution}}
    {\end{solution}\end{minipage}
    \vspace*{\stretch{1}}}

\newlength{\basespace}
\setlength{\basespace}{5\baselineskip}



%\printanswers

\hidepoints

\begin{document}


\subsection*{Other transitional fossil evidence}

Reptiles and mammals are not the only organisms that show evidence of transitional forms in the fossil record. While the list of transitional fossils is large, you will look at a few important transitions.

\subsubsection*{Birds and reptiles }

If birds and reptiles have a common ancestor, you might expect to find
some intermediates. Birds do not fossilize too well, as their bones are
very light and fragile, but there are some very famous fossils of an
organism called \textit{Archaeopteryx lithographica.} These date to about
150 million years ago (abbreviated \textsc{mya} from here on).

\begin{tabular}[t]{@{}L{0.3\textwidth}R{0.65\textwidth}@{}}


\textit{Compsognathus,} a small
theropod dinosaur

\vspace*{6\baselineskip}

\textit{Archaeopteryx}

\vspace*{7\baselineskip}

\textit{Gallus} (modern chicken)

%\vspace*{1\baselineskip}
\vfill 
 &

\includegraphics[width=0.63\textwidth]{07_dino_bird_comparison}\newline{\tiny Illustration from Strickberger, \emph{Evolution}}\\

\end{tabular}


\newpage

\begin{tabular}{@{}L{0.45\textwidth}R{0.5\textwidth}@{}}


Skull of \textit{Archaeopteryx:} note the teeth and lower jaw bones. &

\includegraphics[width=0.49\textwidth]{07_archaeopteryx_skull} \\



Forelimbs of some relatives of modern birds. 
(A) \textit{Ornitholestes}, a theropod dinosaur, 
(B) \textit{Archaeopteryx}, (C) \textit{Sinornis,} an archaic bird from 
the lower Cretaceous, and (D) the wing of a modern chicken (modified 
from Carroll 1988, 1997). &

\includegraphics[width=0.46\textwidth]{07_dino_bird_wings}\\

\end{tabular}

\vspace*{\baselineskip}

Photo of one of the best fossils of \textit{Archaeopteryx.} Note feather imprints.

\begin{center}\includegraphics[width=0.75\textwidth]{07_archaeopteryx}\end{center}

\begin{questions}

\question[3]
Is \emph{Archaeopteryx} intermediate between reptiles and
birds? Explain. 

\newpage

\subsubsection*{Whales and even-toed land mammals}

One major difference between whales and the land
mammals is that whales have no hind limbs. Actually, they have tiny
pelvic bones and some portion of a femur, but nothing that protrudes
from the body. Presumably, if they evolved from land mammals, there
should be some evidence of gradual reduction of hind limbs in their
ancestors.

\begin{tabular}{@{}L{0.35\textwidth}R{0.6\textwidth}@{}}

Modern even-toed mammals like bison have a unique type of ankle joint called a ``double pulley'' that is not found in other modern land mammals. Fossils like \textit{Indohyus, Pakicetus, Ambulocetus,} and \textit{Rodhocetus} also have a double-pulley ankle joint. 

\vspace*{\baselineskip} 

Whales and dolphins have a uniquely shaped ear bone that is also found in the fossils shown at right but not in other modern land mammals. &

\includegraphics[width=0.57\textwidth]{07_whale_like_fossils}\\[\baselineskip]


\textit{Basilosaurus isis}, a fossil whale with small but complete hind
limbs. &
\includegraphics[width=0.57\textwidth]{07_basilosaurus} \\

\end{tabular}



\question[3]
Do the whale fossils shown here support the hypothesis that
whales evolved from land mammals? Explain.

\newpage


\subsubsection*{Fishes and amphibians}

Fish are the first vertebrates to appear in the fossil record; the first
land vertebrate is the fossil amphibian \emph{Ichthyostega}. If
amphibians came from fish ancestors, you should see some intermediates
indicating the development of feet. \textit{Eusthenopteron} (a lobe-finned
fish) lived about 380 \textsc{mya}. \textit{Acanthostega} and \textit{Ichthyostega} are regarded to be among 
 the earliest amphibians. They lived about 360 \textsc{mya}. All of the limbs shown below have a humerus, radius, and ulna. Notice the reduction in the number of skeletal structures that eventually form the digits.


\begin{longtable}[c]{@{}L{0.49\textwidth}L{0.49\textwidth}@{}}

\includegraphics[width=0.47\textwidth]{07_eusthenopteron_skeleton} &
\includegraphics[width=0.47\textwidth]{07_ichthyostega_skeleton} \tabularnewline
\textit{Eusthenopteron} (a lobe-finned fish) & \emph{Ichthyostega} (early amphibian)\tabularnewline

\multicolumn{2}{c}{\includegraphics[width=\textwidth]{07_fish_amphib_limbs}}\tabularnewline
\multicolumn{2}{l}{Front appendages from several fossils that lived between 380--360 \textsc{mya}.}\tabularnewline

\end{longtable}


\question[3]
Do the fossils shown support the hypothesis that amphibians
evolved from fish? Explain.



\newpage

\subsubsection*{Humans, hominids, and great apes}

Most of you have a phylogenetic tree that shows humans and chimpanzee (an ape)
having a common ancestor. Apes appear in the fossil record
during the Pliocene, about 5 \textsc{mya}. Humans first appear
during the Pleistocene, less than 2 \textsc{mya}. 

\question[2]
Based on this, if a hypothesis says that apes and
humans have a common ancestor, what sort of fossils or sequence of
fossils of ape-like and/or human-like organisms would it predict? (Do not
worry about specific anatomic traits here. You will get to that later.)
Explain.

\AnswerBox{3\baselineskip}{%
Answer here.
}

\question[2]
If a hypothesis says that apes and humans do \emph{not} have a
common ancestor, what sort of fossils of ape-like and/or human-like
organisms would it predict? Explain.

\AnswerBox{3\baselineskip}{%
Answer here.
}

Apes are not monkeys. Monkeys are organisms in two major groups, the Old
World monkeys (found in Africa and Asia) and the New World monkeys
(found in South America). The Macaque is an example of an Old World monkeys. Old World monkeys have relatively short tails;
New World monkeys have long tails which can be used to grasp objects,
swing from branches, etc. Apes are tailless, larger than monkeys, and
have larger brains than monkeys. The apes include chimpanzees (\emph{Pan
troglodytes}), bonobos or pygmy chimpanzees (\emph{Pan paniscus}),
Gorillas (\emph{Gorilla gorilla}), and orangutans (\emph{Pongo
pygmaeus}). Chimps, gorillas, and pygmy chimps are African. Orangutans are
Asian.

Think about ways that apes and humans differ. First, humans are fully
bipedal; that is, we walk on two legs. Apes can walk bipedally for short
distances, but not very efficiently. How are these differences reflected
in skeletal anatomy?

One difference is the position of the opening in the skull where the
spine joins to it, and where the spinal cord nerves connect with the
brain. This opening is called the foramen magnum, which is Latin for
``big hole'.'

\begin{longtable}[c]{@{}L{0.31\textwidth}L{0.31\textwidth}L{0.31\textwidth}@{}}
\includegraphics[width=0.3\textwidth]{07_coyote_magnum} &
\includegraphics[width=0.3\textwidth]{07_chimp_magnum} &
\includegraphics[width=0.3\textwidth]{07_human_magnum} \\
Coyote: foramen magnum is in the back, indicating spine is
horizontal & 
Chimpanzee: foramen magnum is angled between bottom
and back, indicating spine exits at an angle. & 
Human: foramen
magnum is angled directly under or to the front, indicating spine is
vertical.\tabularnewline

\end{longtable}

In quadrupeds like a dog or coyote, the foramen magnum is right at the
back of the skull—thus when the dog walks on all fours, the head is
pointed forward. Chimpanzees have a more upright posture. Like most
apes, they walk on the knuckles of their hands, and their forelimbs are
longer than the hind limbs (chimp skeleton showing knuckle-walking
position). The foramen magnum must be more toward the bottom of the
skull so that the spine can exit at an angle while the organism looks
forward. Humans walk fully upright, with the spine more or less
vertical. The foramen magnum in a human skull is all the way underneath,
even angled slightly toward the front.

\question[2]\label{foramen_prediction}
If intermediate fossils exist between apes and humans, what
would you expect the position of the foramen magnum to be? Explain.

\AnswerBox{4\baselineskip}{%
Answer here.
}


Another skeletal difference in bipeds is the shape of the pelvis (hip bones). If
you look at the chimp's skeleton again, you will see that the hip bones
are long and thin, and the flat blade (ilium) lies flat against the
back. Looking downward along the axis of the spine, you can see this
below.

\begin{longtable}[c]{@{}L{0.35\textwidth}L{0.6\textwidth}@{}}

Chimp pelvis. The blade of the ilium is flat against the back, that is,
lies parallel to a line drawn across the back. &

\includegraphics[width=0.55\textwidth]{07_chimp_pelvis}\tabularnewline


Human pelvis. The blade of the ilium wraps around the side of the body,
almost perpendicular to a line drawn across the back.

\vspace*{\baselineskip}{\tiny Illustrations from Lovejoy, C.O. Evolution of human walking.
\emph{Scientific American} Nov. 1998, 118--125.} &
\includegraphics[width=0.55\textwidth]{07_human_pelvis} \tabularnewline

\end{longtable}

\question[2]\label{pelvic_prediction}
If intermediate fossils exist between apes and humans, what
would you expect their pelvic bones to be like? Explain.

\AnswerBox{4\baselineskip}{%
Answer here.
}

The teeth of humans and apes are also different. Note below how large
the canines are in the chimp compared to the rest of the teeth. Also,
the arrangement of teeth in the jaw is {\sffamily U}-shaped in apes, 
and more {\sffamily C}-shaped in humans.

\begin{longtable}[c]{@{}cc@{}}
\includegraphics[width=0.3\textwidth]{07_chimp_upper_jaw} &
\includegraphics[width=0.3\textwidth]{07_human_upper_jaw} \tabularnewline
Chimp upper jaw &
Human upper jaw \tabularnewline

\end{longtable}

\question[2]\label{jaw_prediction}
If intermediate fossils exist between apes and humans, what would you expect their teeth and jaws to be like?  Explain. 

\newpage

There are in fact some thousands of fossils of organisms that appear to
be intermediate between apes and humans; they have been classified with
us in the family Hominidae, and are known as hominids. These organisms
were originally divided into two genera, \emph{Australopithecus} and
\emph{Homo}. Since then, some of these have been reclassified into more
genera, but use the older names for now. All the
\emph{Australopithecus} fossils (there are several species) have been
found in Africa. Fossil species of Homo include \emph{Homo habilis},
\emph{Homo erectus}, and \emph{Homo sapiens}. All of these are found in
Africa, and the earliest example of each is African.

Most of the discussion here will center on the genus
\emph{Australopithecus.} Fossils of this genus have been found as fossils
 dating from about 4 \textsc{mya} to about 1 \textsc{mya}. Earlier, on 
page~\pageref{foramen_prediction}, you made a prediction about the 
location of the foramen magnum on a hypothesized transitional form.

\begin{longtable}[c]{@{}L{0.31\textwidth}L{0.31\textwidth}L{0.31\textwidth}@{}}

\includegraphics[width=0.3\textwidth]{07_chimp_magnum2} &
\includegraphics[width=0.3\textwidth]{07_australo_magnum} &
\includegraphics[width=0.3\textwidth]{07_human_magnum2} \\
Chimpanzee: foramen magnum is angled between bottom
and back, indicating spine exits at an angle. & 
\textit{Australopithecus boisei}: The foramen magnum is directly under or toward the front. & 
Human: foramen
magnum is angled directly under or to the front, indicating spine is
vertical.\tabularnewline

\end{longtable}


Earlier, on 
page~\pageref{pelvic_prediction}, you made a prediction about the 
 the pelvis from a transitional form. Here is a diagram of a pelvis \textit{Australopithecus.}, followed by thediagrams from humans and chimps on the next page. 


\begin{longtable}[c]{@{}L{0.35\textwidth}L{0.35\textwidth}@{}}

\multicolumn{2}{c}{%
\includegraphics[width=0.33\textwidth]{07_australo_pelvis}
}\tabularnewline

\multicolumn{2}{c}{Pelvis of \textit{Australopithecus afarensis.}}\tabularnewline

\includegraphics[width=0.33\textwidth]{07_chimp_pelvis} &
\includegraphics[width=0.33\textwidth]{07_human_pelvis} \tabularnewline

Chimp pelvis. The blade of the ilium is flat against the back, that is,
lies parallel to a line drawn across the back. &

Human pelvis. The blade of the ilium wraps around the side of the body,
almost perpendicular to a line drawn across the back.\tabularnewline

\end{longtable}

Earlier, on 
page~\pageref{jaw_prediction}, you made a prediction about the 
 the teeth and jaw from a transitional form. Teeth and skulls fossilize particularly well.  Many fossils of Australopithecus teeth and jaws are known.  The illustration below compares the teeth of two \textit{Australopithecus} species to those of chimp and human.

\begin{longtable}[c]{@{}L{0.33\textwidth}L{0.33\textwidth}@{}}

\includegraphics[width=0.3\textwidth]{07_afarensis_upper_jaw} &
\includegraphics[width=0.3\textwidth]{07_boisei_upper_jaw} \tabularnewline
\textit{A. afarensis} upper jaw (one canine pulled partly out from jaw) &
\textit{A. boisei} upper jaw \tabularnewline

\includegraphics[width=0.3\textwidth]{07_chimp_upper_jaw} &
\includegraphics[width=0.3\textwidth]{07_human_upper_jaw} \tabularnewline
Chimp upper jaw &
Human upper jaw \tabularnewline

\end{longtable}

\question[2]
 Do the teeth and jaw of members of the genus \textit{Australopithecus} support the hypothesis that Australopithecus was intermediate between apes and humans?  Explain. 
 
 \end{questions}
 
\end{document}  