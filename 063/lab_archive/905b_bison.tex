%!TEX TS-program = lualatex
%!TEX encoding = UTF-8 Unicode

\documentclass[12pt, addpoints, hidelinks]{exam}
\usepackage{graphicx}
	\graphicspath{{/Users/goby/Pictures/teach/163/lab/}
	{img/}} % set of paths to search for images

\usepackage{geometry}
\geometry{letterpaper, left=1.5in, bottom=1in}                   
%\geometry{landscape}                % Activate for for rotated page geometry
\usepackage[parfill]{parskip}    % Activate to begin paragraphs with an empty line rather than an indent
\usepackage{amssymb, amsmath}
\usepackage{mathtools}
	\everymath{\displaystyle}

\usepackage{fontspec}
\setmainfont[Ligatures={TeX}, BoldFont={* Bold}, ItalicFont={* Italic}, BoldItalicFont={* BoldItalic}, Numbers={OldStyle}]{Linux Libertine O}
\setsansfont[Scale=MatchLowercase,Ligatures=TeX]{Linux Biolinum O}
\setmonofont[Scale=MatchLowercase]{Inconsolatazi4}
\usepackage{microtype}


% To define fonts for particular uses within a document. For example, 
% This sets the Libertine font to use tabular number format for tables.
 %\newfontfamily{\tablenumbers}[Numbers={Monospaced}]{Linux Libertine O}
% \newfontfamily{\libertinedisplay}{Linux Libertine Display O}

\usepackage{booktabs}
\usepackage{multicol}
\usepackage[normalem]{ulem}

\usepackage{longtable}
%\usepackage{siunitx}
\usepackage{array}
\newcolumntype{L}[1]{>{\raggedright\let\newline\\\arraybackslash\hspace{0pt}}p{#1}}
\newcolumntype{C}[1]{>{\centering\let\newline\\\arraybackslash\hspace{0pt}}p{#1}}
\newcolumntype{R}[1]{>{\raggedleft\let\newline\\\arraybackslash\hspace{0pt}}p{#1}}

\usepackage{enumitem}
\usepackage{hyperref}
%\usepackage{placeins} %PRovides \FloatBarrier to flush all floats before a certain point.
\usepackage{hanging}

\usepackage[sc]{titlesec}

%% Commands for Exam class
\renewcommand{\solutiontitle}{\noindent}
\unframedsolutions
\SolutionEmphasis{\bfseries}

\renewcommand{\questionshook}{%
	\setlength{\leftmargin}{-\leftskip}%
}

%Change \half command from 1/2 to .5
\renewcommand*\half{.5}

\pagestyle{headandfoot}
\firstpageheader{\textsc{bi}\,063 Evolution and Ecology}{}{\ifprintanswers\textbf{KEY}\else Name: \enspace \makebox[2.5in]{\hrulefill}\fi}
\runningheader{}{}{\footnotesize{pg. \thepage}}
\footer{}{}{}
\runningheadrule

\newcommand*\AnswerBox[2]{%
    \parbox[t][#1]{0.92\textwidth}{%
    \begin{solution}#2\end{solution}}
%    \vspace*{\stretch{1}}
}

\newenvironment{AnswerPage}[1]
    {\begin{minipage}[t][#1]{0.92\textwidth}%
    \begin{solution}}
    {\end{solution}\end{minipage}
    \vspace*{\stretch{1}}}

\newlength{\basespace}
\setlength{\basespace}{5\baselineskip}

%\printanswers


\begin{document}

\subsection*{Radius and ulna revisited: the bison (\numpoints\ points)}

In the previous assignment, you examined the radius and ulna in
humans, pigeons, and (hopefully) several other organisms. You have
already given some thought to the function of these bones,
which is to allow the wrist to rotate.

Now, consider the bison. Much like cattle, bison have hooves, and they
support a great deal of weight on them. Their front hooves do not rotate
with respect to the rest of the limb (in fact, you can imagine that it
might be pretty dangerous for the ``wrist'' of a bison to turn,
considering the weight it supports).

\begin{questions}

\question[3]
Below are three different hypotheses that include the bison. Based 
on what you know of bison, and of the radius and ulna,
what would each hypothesis below \emph{predict} about the forelimb of
the bison? Would each hypothesis predict that the bison would more
likely have two bones in the second part of the forelimb, or only one?
Give your answer and explain why that answer is the one predicted by
each hypothesis.

\begin{parts}

\part What does hypothesis A predict?

\includegraphics[width=0.9\textwidth]{05_hypA}

\begin{solution}
  Hypothesis A predicts that bison should not have homologies with any other animal in the phylogeny.
\end{solution}

\newpage

\part What does hypothesis B predict?

\includegraphics[width=0.8\textwidth]{05_hypB}

\AnswerBox{5\baselineskip}{Hypothesis B predicts that the bison should not have homologies with the whale, pigeon or bat but should have a homology with the animals on the same tree.}

\part What does hypothesis C predict?

\includegraphics[width=0.8\textwidth]{05_hypC}

\begin{solution}
  Hypothesis C predicts that bison should have homologies with all other animals in the phylogeny.
\end{solution}

\end{parts}

\newpage

Now go down to the student lounge near MG 127 (turn right out the door, 
the left, and halfway down the hall on the left) and have a look at the bison
skeleton in the case. Note that there are two bones in the second section 
of the forelimb of the bison. Look carefully at the "wrist" joint; the two bones 
(radius and ulna) are fused together there (the black vertical line is a wire holding the
skeleton together). In fact, the bones are at least partially fused all
along their length. They start out as two separate bones in the embryo,
but fuse during development. By the time the bison is born, the two
bones can't move separately from each other at all.

\question[2]
Why do you think the bison has two bones in the lower part of
its forelimb? Before you say that it is for additional strength,
consider how much weight the single bone above them (humerus) supports).
Considering this as a similarity with humans, would this be a homology
or an analogy? Explain.

\AnswerBox{6\baselineskip}{It has two bones in the lower forelimb because it shares a common ancestor with other animals that also have a radius and ulna. The radius and ulna serve different functions in humans and bison so the radius and ulna are homologies.}

\question[3]
What can you \emph{conclude} about each of the three class
hypotheses based on this evidence? Tell in each case whether the
hypothesis is supported, falsified, weakened, or if there is no
conclusive evidence provided, and explain your answer. If the evidence
requires some modification in the hypothesis, suggest how it should be
modified.

\begin{parts}

	\part Hypothesis A: 
	
	\AnswerBox{3\baselineskip}{Hypothesis A is falsified. Bison has a homology of the radius and ulna with all other animals on the phylogeny (or at least the human).}


	\part Hypothesis B:

	\AnswerBox{3\baselineskip}{Hypothesis B is falsified (or weakened). Bison has a homology with cat, human, macaque, etc. But, should also have one with pigeon, bat, and whale.}
	
	\part Hypothesis C:

	\AnswerBox{2\baselineskip}{Hypothesis C is supported. Bison has a homology with all of the organisms shown.}
\end{parts}

\end{questions}

\end{document}  