%!TEX TS-program = lualatex
%!TEX encoding = UTF-8 Unicode

\documentclass[12pt, addpoints, hidelinks]{exam}
\usepackage{xcolor}
\usepackage{graphicx}
	\graphicspath{{/Users/goby/Pictures/teach/163/lab/}
	{img/}} % set of paths to search for images

\usepackage{geometry}
\geometry{letterpaper, left=1.5in, bottom=1in}                   
%\geometry{landscape}                % Activate for for rotated page geometry
\usepackage[parfill]{parskip}    % Activate to begin paragraphs with an empty line rather than an indent
\usepackage{amssymb, amsmath}
\usepackage{mathtools}
	\everymath{\displaystyle}

\usepackage{fontspec}
\setmainfont[Ligatures={TeX}, BoldFont={* Bold}, ItalicFont={* Italic}, BoldItalicFont={* BoldItalic}, Numbers={OldStyle}]{Linux Libertine O}
\setsansfont[Scale=MatchLowercase,Ligatures=TeX]{Linux Biolinum O}
%\setmonofont[Scale=MatchLowercase]{Inconsolatazi4}
\usepackage{microtype}


% To define fonts for particular uses within a document. For example, 
% This sets the Libertine font to use tabular number format for tables.
 %\newfontfamily{\tablenumbers}[Numbers={Monospaced}]{Linux Libertine O}
% \newfontfamily{\libertinedisplay}{Linux Libertine Display O}

\usepackage{booktabs}
\usepackage{multicol}
\usepackage[normalem]{ulem}

\usepackage{longtable}
%\usepackage{siunitx}
\usepackage{array}
\newcolumntype{L}[1]{>{\raggedright\let\newline\\\arraybackslash\hspace{0pt}}p{#1}}
\newcolumntype{C}[1]{>{\centering\let\newline\\\arraybackslash\hspace{0pt}}p{#1}}
\newcolumntype{R}[1]{>{\raggedleft\let\newline\\\arraybackslash\hspace{0pt}}p{#1}}

\usepackage{enumitem}
%\setenumerate{label=\Alph.}
\setlist{leftmargin=*}
\setlist[1]{labelindent=\parindent}
\setlist[enumerate]{label=\textsc{\alph*}.}
\setlist[itemize]{label=\color{gray}\textbullet}

\usepackage{hyperref}
%\usepackage{placeins} %PRovides \FloatBarrier to flush all floats before a certain point.
\usepackage{hanging}

\usepackage[sc]{titlesec}

%% Commands for Exam class
\renewcommand{\solutiontitle}{\noindent}
\unframedsolutions
\SolutionEmphasis{\bfseries}

% Shifts margins left. Question numbers appear in margin.
% This allows "fullwidth" to left align with text that is outside 
% if the questions environment.
\renewcommand{\questionshook}{%
	\setlength{\leftmargin}{-\leftskip}%
}

%\renewcommand{\partshook}{%
%	\setlength{\leftmargin}{-\leftskip}%
%}

%Change \half command from 1/2 to .5
\renewcommand*\half{.5}

\pagestyle{headandfoot}
\firstpageheader{\textsc{bi}\,063 Evolution and Ecology}{}{\ifprintanswers\textbf{KEY}\else Name: \enspace \makebox[2.5in]{\hrulefill}\fi}
\runningheader{}{}{\footnotesize{pg. \thepage}}
\footer{}{}{}
\runningheadrule

\newcommand*\AnswerBox[2]{%
    \parbox[t][#1]{0.92\textwidth}{%
    \begin{solution}#2\end{solution}}
%    \vspace*{\stretch{1}}
}

\newenvironment{AnswerPage}[1]
    {\begin{minipage}[t][#1]{0.92\textwidth}%
    \begin{solution}}
    {\end{solution}\end{minipage}
    \vspace*{\stretch{1}}}

\newlength{\basespace}
\setlength{\basespace}{5\baselineskip}


%\usepackage{mdframed}
%\mdfsetup{%
%	innerleftmargin=0pt,%
%	innerrightmargin=0pt,
%	innertopmargin=0pt,
%	innerbottommargin=0pt,
%	hidealllines=true
%}%end mdfsetup

%
%\makeatletter
%\def\SetTotalwidth{\advance\linewidth by \@totalleftmargin
%\@totalleftmargin=0pt}
%\makeatother


%\printanswers

\pointsinmargin\pointformat{}

\begin{document}


\subsection*{Using homology and analogy to test hypotheses (\numpoints\ points)}

At the end of the Great Clade Race exercise, you learned that organisms could
be classified based on sets of shared characters or structures, as shown in this 
diagram. Organisms in the same taxonomic group share a set of similar structures,
such as a vertebral column (vertebrates) or four walking legs (tetrapods).

\noindent\includegraphics[width=0.92\textwidth]{03a_vertebrate_tree}

When you see similar structures shared among two or more species, there are two
testable explanations for it. They are

\begin{enumerate}
	\item \emph{homology:} the structures are similar because the species
inherited the similarity from a common ancestor, and

	\item \emph{analogy:} the similarity is necessary for a particular function.\footnote{The similarity
could be due solely to chance, but this would be \emph{very}
unlikely.} 
\end{enumerate}

How can you distinguish between homology and analogy? If you know the 
common ancestor, you can see if the structure is present in the species and the
ancestor. If the structure is present in the ancestor, then the similarity must be due
to homology. If the structure is absent, then the similarity in the descendants must be due to analogy.

Usually, though, you will be looking at similarities in living organisms to test
hypotheses about their ancestry, without knowing anything about the ancestor.  
If you have a hypothesis that \emph{predicts} that two species have a
common ancestor, then the argument would be: ``These two organisms have a 
similarity which is due to homology; therefore they must have a common ancestor.''
So, suppose you are testing the hypothesis you looked at in an earlier exercise.

\includegraphics[width=\textwidth]{03c_intro_trees}

This tree shows fish and whales as having a common ancestor.
You are testing the hypothesis, so you look at a fish and a whale,
and note a structural similarity: both have structures on their sides that are flattened
and broad (fins and flippers). You then say to a classmate, ``I think
that similarity is due to homology, so it supports this hypothesis''

\begin{questions}


%% CONSIDER removing this question and just explaining by example.
\question[1]
If you are testing the predictions made by your hypothesis,
can you confidently state that you \emph{know} the similarity is due to
homology \emph{because} your hypothesis predicts they got it from a
common ancestor? Explain why or why not.

\AnswerBox{3\baselineskip}{No. That is circular reasoning. }

If homology is to be used as evidence to test a hypothesis about
ancestry, you cannot use the hypothesis as an argument for how you know
something is a homology---that would be a circular argument. You need another 
way to determine if structures are homologous that is not based on a 
hypothesis about ancestry, an independent test for homology. You 
have to consider what you know about the
organisms themselves. 

Remember that an analogy is

\begin{itemize}
\item
  a similarity in structures that serve similar functions, and
\item
  a similarity in structure that is necessary for the function
  to be possible.
\end{itemize}

What you must do is look at the organisms themselves and use them to
test your hypothesis. Under the hypothesis-testing ``program,'' you have made
your hypothesis---the tree---so now you should
make predictions based on the your hypothesis.

\question[1]
If the tree above is a good representation of the
relationships among the organisms shown, should you find any homologies
between fish and whales? Explain.

\AnswerBox{4\baselineskip}{Yes. They hypothesis predicts that fish and whales
have a common ancestor. Therefore, the hypothesis predicts they should have homologies.}

\question[1]
Does the hypothesis shown in the tree predict what specific
homologies they will have? Why or why not? Explain.

\AnswerBox{4\baselineskip}{No. The hypothesis predicts only that the should
have homologies, but not what the homologies might be.}

\question[1]
Does the hypothesis shown by the tree predict that there
will be any homologies between bison and whales? Why or why not? What
if you later found evidence of a homology? Could this have implications
for your hypothesis? Explain.

\AnswerBox{4\baselineskip}{No. The hypothesis predicts that bison and whales do
not have a common ancestor so they should not share homologies. If a homology is found
then the hypothesis has been falsified.}

\question[1]
What about analogies? Does the tree above predict that there
will be analogies between grass and cactus? How about between humans and
marine worms? Explain.

\AnswerBox{4\baselineskip}{No. Analogies could exist between any organisms. 
The hypothesis predicts only the presence of homologies.}

If you think about it, analogies could exist between almost any two
organisms. Analogy does not depend on them being related but instead that they
share an organ, limb, or other structure that had a similar function in both. That could be
the case if they are not related, like bison and whales in this
hypothesis, or if they are related, like cactus and fungus in this
hypothesis. The hypothesis itself does not make any predictions about
analogies---it is just about relationships, and does not say anything
about the functions of organs in the various organisms.

When you are testing hypotheses about ancestry, \emph{analogies are 
always inconclusive} because an analogy can be found both in 
organisms that are related and in ones that are not
related. Analogies just cannot tell you whether two organisms are related.

%

\subsubsection*{Using homology and analogy to test hypotheses}

Using the scientific method, tell what you would conclude about
the hypothesis shown in the tree above if each of the following results
occurred, and explain your reasoning for each answer.

\question[1]
Assume you found an analogy between whales and fish. Is the hypothesis supported, falsified, or is the evidence inconclusive? Explain.

\AnswerBox{4\baselineskip}{Inconclusive. Analogies cannot be used to evaluate hypotheses.}


\question[1]
Assume you found a homology between whales and fish. Is the hypothesis supported, falsified, or is the evidence inconclusive? Explain.

\AnswerBox{4\baselineskip}{The hypothesis is supported. The hypothesis predicts that whales and fish have a common ancestor. The homology suggests they have a common ancestor.}


\question[1]
Assume you found an analogy between bison and whales. Is the hypothesis supported, falsified, or is the evidence inconclusive? Explain.

\AnswerBox{4\baselineskip}{Inconclusive. Analogies cannot be used to evaluate hypotheses.}

\question[1]
Assume you found a homology between bison and whales. Is the hypothesis supported, falsified, or is the evidence inconclusive? Explain.

\AnswerBox{4\baselineskip}{The hypothesis is falsified. The hypothesis predicts that bison and whales do not have a common ancestor but the homology suggests they do.}

%% CONSIDER removing this question. This is often confusing.
\question[1]
What must be true of a trait in the most recent common ancestor if the
similarity is due to homology?

\AnswerBox{2\baselineskip}{The trait in the ancestor must also be a homology.}


\end{questions}

\end{document}  