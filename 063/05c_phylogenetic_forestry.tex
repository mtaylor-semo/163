%!TEX TS-program = lualatex
%!TEX encoding = UTF-8 Unicode

\documentclass[12pt, hidelinks]{exam}
\usepackage{graphicx}
	\graphicspath{{/Users/goby/Pictures/teach/163/lab/}
	{img/}} % set of paths to search for images

\usepackage{geometry}
\geometry{letterpaper, left=1.5in, bottom=1in}                   
%\geometry{landscape}                % Activate for for rotated page geometry
\usepackage[parfill]{parskip}    % Activate to begin paragraphs with an empty line rather than an indent
\usepackage{amssymb, amsmath}
\usepackage{mathtools}
	\everymath{\displaystyle}

\usepackage{fontspec}
\setmainfont[Ligatures={TeX}, BoldFont={* Bold}, ItalicFont={* Italic}, BoldItalicFont={* BoldItalic}, Numbers={OldStyle}]{Linux Libertine O}
\setsansfont[Scale=MatchLowercase,Ligatures=TeX]{Linux Biolinum O}
%\setmonofont[Scale=MatchLowercase]{Inconsolatazi4}
\usepackage{microtype}

%\usepackage{unicode-math}
%\setmathfont[Scale=MatchLowercase]{Asana Math}
%\setmathfont[Scale=MatchLowercase]{XITS Math}

% To define fonts for particular uses within a document. For example, 
% This sets the Libertine font to use tabular number format for tables.
%\newfontfamily{\tablenumbers}[Numbers={Monospaced}]{Linux Libertine O}
%\newfontfamily{\libertinedisplay}{Linux Libertine Display O}

\usepackage{booktabs}
\usepackage{multicol}
\usepackage[normalem]{ulem}

%\usepackage{tabularx}
\usepackage{longtable}
%\usepackage{siunitx}
\usepackage{array}
\newcolumntype{L}[1]{>{\raggedright\let\newline\\\arraybackslash\hspace{0pt}}p{#1}}
\newcolumntype{C}[1]{>{\centering\let\newline\\\arraybackslash\hspace{0pt}}p{#1}}
\newcolumntype{R}[1]{>{\raggedleft\let\newline\\\arraybackslash\hspace{0pt}}p{#1}}

\usepackage{enumitem}
\usepackage{enumitem}
\setlist{leftmargin=*}
\setlist[1]{labelindent=\parindent}
\setlist[enumerate]{label=\textsc{\alph*}.}

%\usepackage{hyperref}
%\usepackage{placeins} %PRovides \FloatBarrier to flush all floats before a certain point.
%\usepackage{hanging}

\usepackage[sc]{titlesec}

\renewcommand{\solutiontitle}{\noindent}
\unframedsolutions
\SolutionEmphasis{\bfseries}

\renewcommand{\questionshook}{%
	\setlength{\leftmargin}{-\leftskip}%
}

%Change \half command from 1/2 to .5
\renewcommand*\half{.5}


%% Allows fullwidth command to break across pages.
%% See 
\makeatletter
\def\SetTotalwidth{\advance\linewidth by \@totalleftmargin
\@totalleftmargin=0pt}
\makeatother


\pagestyle{headandfoot}
\firstpageheader{\textsc{bi} 063: Evolution and Ecology}{}{\ifprintanswers\textbf{KEY}\else Name: \enspace \makebox[2.5in]{\hrulefill}\fi}
\runningheader{}{}{\footnotesize{pg. \thepage}}
\footer{}{}{}
\runningheadrule

\newcommand*\AnswerBox[2]{%
    \parbox[t][#1]{0.92\textwidth}{%
    \begin{solution}#2\end{solution}}
    \vspace{\stretch{1}}
}

\newenvironment{AnswerPage}[1]
    {\begin{minipage}[t][#1]{0.92\textwidth}%
    \begin{solution}}
    {\end{solution}\end{minipage}
    \vspace{\stretch{1}}}

\newlength{\basespace}
\setlength{\basespace}{5\baselineskip}

%\printanswers

\begin{document}

\subsection*{Phylogenetic forestry: learn to interpret phylogenetic trees\footnote{Many of the concepts in this exercise are drawn from Baum, D.A. et al., 2005. The Tree-Thinking Challenge. Science 310: 979-980, and Meisel, R.P., 2010. Teaching tree-thinking to undergraduate biology students. Evo. Edu. Outreach. doi :10.1007/s12052-010-0254-9.}}

You will see and draw phylogenetic trees throughout this course.
Although phylogenetic trees are a convenient way to draw hypotheses about
relationships of different organisms, phylogenetic trees are often
misinterpreted by beginning biology students (and sometimes by advanced
biologists!). This worksheet will help you learn to properly interpret
phylogenetic trees.

Let us begin with a simple hypothesis about two organisms, say a lizard
and a crocodile. %(I omitted the time scale for convenience. Your
%hypotheses \emph{must always} have a vertical time scale.)

%\vspace*{\baselineskip}
\begin{center}%croc/lizard
	\noindent\includegraphics[width=0.9\textwidth]{05c_phylo_forestry1}
\end{center}

\begin{questions}

\question
Do you think these two phylogenies diagram the same
hypothesis? Explain.

\AnswerBox{3\baselineskip}{They are the same. Both crocodile and lizard
share the same ancestor.}

\question
Do the hypotheses predict which organism evolved from the
other? In other words, does the hypothesis on the left predict that a
crocodile evolved from a lizard? Does the hypothesis on the right
predict that a lizard evolved from a crocodile? Explain.

\AnswerBox{3\baselineskip}{They do not. This is common for people to 
interpret but the trees predict they have the same ancestor but the ancestor
is not specified by the trees.}

\end{questions}

In fact, these two trees are identical. Both trees hypothesize that the
crocodile and the lizard share a common ancestor. Both hypotheses
predict that we should find evidence of a
common ancestor between these two organisms. \emph{Importantly, neither
tree predicts that one organism evolved from the other organism}.
Students commonly but incorrectly interpret the longest line as the organism that has
been around the longest.

%\vspace*{\baselineskip}

\begin{center}% Hypothesized ancestor

	\noindent\includegraphics[width=0.9\textwidth]{05c_phylo_forestry2}
\end{center}

Here are the same two phylogenies, but drawn a different way. These
phylogenies still predict that the lizard and the crocodile have a
common ancestor but the ancestor was neither the lizard nor the
crocodile. Instead, they both evolved from a common ancestral organism
that lived in the past but is now extinct. The common ancestor of the
two organisms is represented by the single line or \textbf{branch}, indicated
 by the braces.

Let us explore common ancestry by adding an ostrich to our hypothesis.

%\vspace*{\baselineskip}

\begin{center}
	\noindent\includegraphics[width=\textwidth]{05c_add_ostrich}
\end{center}

\begin{questions}
\setcounter{question}{2}

\question
Do the two hypotheses make different predictions about
whether the crocodile or the lizard is more closely related to the
ostrich? In other words, does the hypothesis on the left predict that
the crocodile is more closely related to the ostrich than the lizard is
to the ostrich? Does the hypothesis on the right predict that the lizard
is more closely related to the ostrich than the crocodile is to the
ostrich? Explain.

\AnswerBox{5\baselineskip}{They do not. Horizontal order does not
matter. The hypothesis predicts that crocodile and lizard share a more
recent ancestor than either does with an ostrich. All three ultimate have
a common ancestor.}

\end{questions}

The order of the crocodile and the lizard across the horizontal axis
does not matter. What does matter is that, in both hypotheses, the
crocodile and the lizard share a more recent common ancestor with each
other than either does with the ostrich. Both hypotheses are identical.

\begin{center}
	\noindent\includegraphics[width=\textwidth]{05c_add_ostrich_labeled}
\end{center}

Remember that the vertical axis represents time. Therefore, according to
these hypotheses, the crocodile and the lizard shared a common ancestor
with each other more recently than either did with the ostrich. All
three organisms shared a common ancestor with each other farther in the
past. You can use this information to determine who is most closely
related. The crocodile and lizard shared the most recent common ancestor
so they are more closely related to each other than to the ostrich. The
ostrich is equally related to both the crocodile and the lizard because
all three trace back to the same common ancestor.

If a branch indicates a common ancestor, then the point at which the
ancestral branch splits into two new branches indicates the time at
which two descendant organisms diverged from their common ancestor. The
solid circles above indicate the time at which the crocodile and lizard
diverged from their common ancestor.

\begin{questions}
\setcounter{question}{3}

%\newpage

\question
What do the white diamonds in the hypothesis above
represent?

\AnswerBox{2\baselineskip}{Diamonds represent the time at which the ostrich
diverged from the common ancestor shared with the ancestor of the crocodile/lizard.}

\end{questions}

Phylogenetic trees can be thought of as nested hierarchies. Organisms
that share more recent common ancestors are nested within larger groups
containing more distantly related organisms and their ancestors. %Compare
%the two diagrams below to the two hypotheses above.

\begin{center}
	\noindent\includegraphics[width=\textwidth]{05c_nested_hierarchy1}
\end{center}

The different shades indicate distinct groups in the hierarchy. The
crocodile and lizard group together in the black rectangle, indicating
that they share a more recent common ancestor than either does with the
ostrich. This shows the closer relationship between the crocodile
and the lizard. The relationship between the crocodile and lizard is
nested within the larger grey rectangle that also contains the ostrich.
This shows that all three organisms share a common but older
ancestor, which again highlights how the ostrich is equally related to
the crocodile and the lizard. Notice that the figure on the right
illustrates the same relationship among all three organisms as the
figure on the left. \emph{The horizontal order on a phylogeny
does not change the relationship among the species.} Compare the diagram 
above to the shaded phylogeny below. The crocodile/lizard cluster is nested within
the larger ostrich/crocodile/lizard cluster.

\begin{center}
	\noindent\includegraphics[width=\textwidth]{05c_nested_phylo}
\end{center}


Here is another diagram, with the wolf added.


\begin{center}
	\noindent\includegraphics{05c_nested_hierarchy2}
\end{center}

\newpage

\begin{questions}
\setcounter{question}{4}

\question\label{wolf_added}
Draw a phylogenetic tree that is consistent with the nested
hierarchy drawn above. Ask your instructor to check your work.

\AnswerBox{10\baselineskip}{Draw it yourself!}

\end{questions}

You should have drawn a phylogeny that shows the crocodile and lizard
sharing the most recent common ancestor with each other. Then, the
crocodile and lizard together share an older common ancestor with the
ostrich. Finally, the ostrich, lizard and crocodile equally share an
even older common ancestor with the wolf.

Ask your instructor to double-check your phylogeny to be certain. Then, use the
phylogeny you just drew for question \ref{wolf_added} to answer the next two
questions.

\begin{questions}
\setcounter{question}{5}

\question
According to the phylogenetic tree, is the wolf more closely
related to the crocodile or to the lizard? Explain.

\AnswerBox{3\baselineskip}{The wolf is equally related to the crocodile and
lizard because they share the same ancestor.}

\question
According to the phylogenetic tree, is the ostrich more
closely related to the crocodile or to the wolf? Explain.

\AnswerBox{3\baselineskip}{The ostrich is more closely related to the 
crocodile because it shares a more recent ancestor with the crocodile (and lizard)
than it does with the wolf.}

\end{questions}

To figure out which organisms are more closely related, you have to see
when they last shared an ancestor. (Remember that the vertical axis
represents time.) Organisms that share a more recent common ancestor are
more closely related to each other than are organisms that share a
common ancestor farther in the past. Therefore, according to the
hypothesis you drew for question \ref{wolf_added}, the crocodile and lizard are most
closely related. The ostrich is more closely related to the crocodile
and lizard than the ostrich is to the wolf because the ostrich shares a
more recent common ancestor with the crocodile and lizard than it does
with the wolf.

\begin{questions}
\setcounter{question}{7}

\question
Convince yourself by drawing on your phylogeny from question~\ref{wolf_added} an
arrow that points to the branch that represents the most recent common
ancestor of the ostrich, the crocodile and the lizard. Label this arrow
\#1. Then, draw an arrow that points to the branch that represents the
most recent common ancestor of all four organisms. Label this arrow \#2.

\AnswerBox{1\baselineskip}{Should have arrows in the appropriate spot.}

\question
Feeling confidant now? Turn the following diagram into
\emph{two} identical phylogenetic trees, each with a different
horizontal order of the names.

\end{questions}

\begin{center}
	\noindent\includegraphics{05c_nested_hierarchy3}
\end{center}

Tree 1: \hspace*{0.4\textwidth}Tree 2:

\AnswerBox{3\baselineskip}{Draw the two trees yourself.}

\newpage

When you are done, ask your instructor to verify the accuracy of your phylogenetic
trees. Then, use one of those two trees to answer the next two
questions.

\begin{questions}
\setcounter{question}{9}

\question
According to the hypothesis, is the penguin or the ostrich
more closely related to the wolf? Explain.

\AnswerBox{3\baselineskip}{Both are equally related to the wolf because 
share the same ancestor with the wolf.}

\question
According to the hypothesis, is the ostrich or the
crocodile more closely related to the wolf? Explain.

\AnswerBox{3\baselineskip}{Both are equally related to the wolf because
both share the same ancestor with the wolf.}


One final note: Rectangular or “goal post” trees, drawn with horizontal and vertical
lines as shown on pages 2 and 3, take just a little more time to draw but
cause much less confusion than the slanted trees shown on the first page.
Both types of trees provide the same information about hypothesized
relationships among organisms but clarity is preferable to confusion.

\question[Checkout]
Turn the last phylogenetic tree (Tree 3) from the previous handout into a nested hierarchy.

\AnswerBox{10\baselineskip}{Nested hierarchy goes here.}

\end{questions}

\end{document}  