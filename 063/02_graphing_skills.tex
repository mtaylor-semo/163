%!TEX TS-program = lualatex
%!TEX encoding = UTF-8 Unicode

\documentclass[12pt, hidelinks]{exam}

%\printanswers

\usepackage{graphicx}
\graphicspath{{/Users/goby/Pictures/teach/163/lab/}
	{img/}} % set of paths to search for images

\usepackage{geometry}
\geometry{letterpaper, left=1.5in, bottom=1in}                   
%\geometry{landscape}                % Activate for for rotated page geometry
\usepackage[parfill]{parskip}    % Activate to begin paragraphs with an empty line rather than an indent
\usepackage{amssymb, amsmath}
\usepackage{mathtools}
\everymath{\displaystyle}

\usepackage{fontspec}
\setmainfont[Ligatures={TeX}, BoldFont={* Bold}, ItalicFont={* Italic}, BoldItalicFont={* BoldItalic}, Numbers={Proportional}]{Linux Libertine O}
\setsansfont[Scale=MatchLowercase,Ligatures=TeX, Numbers={Proportional}]{Linux Biolinum O}
\setmonofont[Scale=MatchLowercase]{Linux Libertine Mono O}
\newfontfamily{\liningnum}[Numbers=Lining]{Linux Libertine O}
\usepackage{microtype}


% To define fonts for particular uses within a document. For example, 
% This sets the Libertine font to use tabular number format for tables.
%\newfontfamily{\tablenumbers}[Numbers={Monospaced}]{Linux Libertine O}
% \newfontfamily{\libertinedisplay}{Linux Libertine Display O}

\usepackage{booktabs}
\usepackage{multicol}
%\usepackage[normalem]{ulem}

\usepackage{longtable}
%\usepackage{siunitx}
\usepackage{array}
\newcolumntype{L}[1]{>{\raggedright\let\newline\\\arraybackslash\hspace{0pt}}p{#1}}
\newcolumntype{C}[1]{>{\centering\let\newline\\\arraybackslash\hspace{0pt}}p{#1}}
\newcolumntype{R}[1]{>{\raggedleft\let\newline\\\arraybackslash\hspace{0pt}}p{#1}}

\usepackage{enumitem}
\setlist{leftmargin=*}
\setlist[1]{labelindent=\parindent}
\setlist[enumerate]{label=\textsc{\alph*}.}
\setlist[itemize]{label=\color{gray}\textbullet}

\usepackage{hyperref}
%\usepackage{placeins} %PRovides \FloatBarrier to flush all floats before a certain point.
\usepackage{hanging}

\usepackage[sc]{titlesec}

%% Commands for Exam class
\renewcommand{\solutiontitle}{\noindent}
\unframedsolutions
\SolutionEmphasis{\bfseries}

\renewcommand{\questionshook}{%
	\setlength{\leftmargin}{-\leftskip}%
}

\pagestyle{headandfoot}
\firstpageheader{\textsc{bi}\,063 Evolution and Ecology}{}{\ifprintanswers\textbf{KEY}\else Name: \enspace \makebox[2.5in]{\hrulefill}\fi}
\runningheader{}{}{\footnotesize{pg. \thepage}}
\footer{}{}{}
\runningheadrule

\newcommand*\AnswerBox[2]{%
	\parbox[t][#1]{0.92\textwidth}{%
		\begin{solution}#2\end{solution}
		%    \vspace{\stretch{0.5}}
		\vskip\stretch{1}}
}

\newenvironment{AnswerPage}[1]
{\begin{minipage}[t][#1]{0.92\textwidth}%
		\begin{solution}}
		{\end{solution}\end{minipage}
	\vspace{\stretch{1}}}

\newlength{\basespace}
\setlength{\basespace}{5\baselineskip}

\newcommand*\AnswerBlank{\rule{0.75in}{0.4pt}\kern0.67pt.}
\newcommand*\xcell[1]{cell~\liningnum{#1}}
\newcommand*\axis[1]{{\scshape #1}-axis}

%
%\makeatletter
%\def\SetTotalwidth{\advance\linewidth by \@totalleftmargin
%\@totalleftmargin=0pt}
%\makeatother


\begin{document}

%% 2010 Version
\subsection*{Creating figures with Microsoft Excel\texttrademark}

Graphical representations of your data in scientific publications are called figures. For this exercise, you will learn to create scientific figures using Microsoft Excel. Excel has many options for making quality figures. Excel also has options that you should never use in scientific figures, most of which, unfortunately, are the default settings.  

This exercise will teach you how to use Excel to create a one-%and two-%
variable column chart (also called a bar chart), a scatterplot, and a line graph suitable for publication. Once learned, you will use this valuable skill to present results in this course, in your future courses, and for the rest of your scientific career. 

\subsubsection*{Column chart: one explanatory variable}

Column charts, also called bar charts, are often used when you want to compare values among categorical variables. For example, you may use a column chart to compare the average \textsc{gpa} among freshmen, sophomores, juniors and seniors on campus. 

First, you will create a column chart to show the number of earthquakes of magnitude 3.0 or higher\footnote{Magnitude represents the strength of the earthquake on the Richter scale. A magnitude 4.0 earthquake releases nearly 32$\times$ more energy than a magnitude 3.0 earthquake.} recorded in Oklahoma from 1980 through last year. \bigskip

\begin{questions}
	
	\question
	Which variable is the explanatory variable? \ifprintanswers \textbf{Year} \else \rule{1.5in}{0.4pt} \fi
	
	%\AnswerBox{1\baselineskip}{Year}
	
	\medskip
	
	\question
	Which variable is the response variable? \ifprintanswers \textbf{Number of earthquakes} \else \rule{1.5in}{0.4pt} \fi
	
%	\AnswerBox{1\baselineskip}{Number of earthquakes}

	\bigskip	
	
	Download the file called “oklahoma\_earthquakes.xlsx” from the course data website:\\ \url{http://mtaylor4.semo.edu/~goby/bi163/}.\bigskip
	
	The first column of data is the year. The second column is the number of earthquakes of magnitude 3.0 or higher that occurred in Oklahoma each year.

\begin{enumerate}
	\item Select cells {\liningnum B1 to B38}.
	
	\item Click on “Insert” on the ribbon menu. Click on the “Column” icon near the center of the ribbon, and then choose the 2-D “Clustered Column,” which is the first icon on the left. If you leave your cursor over the icon for a moment, an explanation of the chart type will appear.
	
	\item Resize the graph to make it larger by dragging from the corners.
	
	\item Right-click on the graph and choose “Select Data\dots” from the menu.
	
	\item Click on “Earthquakes” on the left-hand side of the dialog box, and then click the “Edit” button on the \emph{right-hand} side, for “Horizontal (Category) Axis Labels.” Select cells {\liningnum A2 to A38}. Click the “OK” button, then click the next “OK” button.
	
	You should now have years on the \textsc{x}-axis and the number of earthquakes for each year on the \textsc{y}-axis.
	
\end{enumerate}

The graph is adequate but it can be improved with a few simple changes. First, change and enlarge the font. The font in your graphs should be the same font you use in the body of your report. For example, if you are using Times New Roman in your Word document, you should use Times New Roman in your graph. The font size should be at least the same as you use in your report but is often a little larger. For this exercise, use 12 point font, regular style (not bold).

\begin{enumerate}[resume]
	\item Go to the home ribbon. Select “Times New Roman” and 12 for the font size.
	
	\emph{If that does not work} then do these steps:
	
	\begin{itemize}
		\item Place your cursor over one of the years on the \axis{x}, then right-click. Select “Font\dots”.

		\item In the “Latin text font:” area, you can either delete the current font and type “Times New Roman” (exactly) or click on the small button to the right of the “Font:” area and scroll down to select it from the menu. 

		\item Tab over to or click in the “Size:” area and change the value to 12. Press the Enter key or click the “OK” button.

	\end{itemize}
	
	\item Repeat the last step for the \axis{y}.
	
\end{enumerate}

Scientific figures usually do not have a symbol legend in the figure because the symbols are explained in the figure caption. So, delete the symbol legend.  

\begin{enumerate}[resume]
	\item Choose the “Chart Design” ribbon. Click on “Add Chart Element” on the very left side of the ribbon.  

	\item Scroll down to “Legend” and select “None” from the pop-up to remove the legend.

	\item Once again, click on “Add Chart Element,” scroll down to Chart Title, then click on “None.”

	\item Or, click on the symbol legend and press the delete key. 
\end{enumerate}

Add labels to the \axis{x} and \axis{y}, starting with the \axis{x}.

\begin{enumerate}[resume]
	\item Choose the “Chart Design” ribbon. Click on “Add Chart Element” on the very left side of the ribbon.
	
	\item Select “Axis Titles,” then choose “Primary Horizontal.”

	\item Right-click on the axis label that appears to the left side of the \axis{y}. Set the font to Times New Roman and the font size to 12 points. If necessary, click on the “Font style:” menu and select “Regular” to remove the bold face.

	\item Right-click on the label. Select the text inside the box and type “Year.” Click anywhere outside the label to accept your change.

	\item Repeat the steps for the \axis{y} but choose “Primary Vertical” for the placement of the axis title. Use “Number of earthquakes” for the title. 

	
\end{enumerate}

Change the color of the columns to black. Color figures should be avoided until you know what you are doing. Not all people can perceive the same colors equally, which can cause problems interpreting the figures. Plus, publishing color figures is very expensive (often more than \$600 \emph{per} figure) so black, white, and shades of gray are used unless color is absolutely necessary. 

\begin{enumerate}[resume]
	\item Right-click on any one of columns. Choose “Format Data Series\dots” from the menu.
	
	\item Click on the “Paint Bucket” icon. Click on “Fill” from the choices below. Select the “Solid fill” radio button. Select the small black square from the “Fill” choices.
	
	\item Click on “Border.” Select the “Solid Line” radio button. Select the black square for the color.
	
	\item Click the “X” to close the pane.
	
\end{enumerate}

If your figure contains grid lines and a title, they must be removed.


\begin{enumerate}[resume]
	\item Click on the “Earthquakes” title at the top of the graph. Press the delete key.

	\item Right-click on one of the grid lines. Select “Format Gridlines\dots” from the popup menu. Select the “No line” radio button and close the pane.

	\item Or, click on one of the grid lines and press the delete key.
	
	The \axis{y} is missing a vertical axis line so you must add one with tick marks.
	
	\item Right-click on the \axis{y}. Choose “Format axis\dots. Click the paint can icon. Under the “Line” options, click “Solid Line.” 
	
	\item Click on the chart icon. Under the “Tick Mark” options, choose “Outside” for the Major type.
	
\end{enumerate}

You now have scientific-quality figure that would be suitable for publication.

\question
Fracking has become more common in Oklahoma. Fracking produces waste water that is injected into the ground for disposal. Fracking increased dramatically in Oklahoma in 2010 and later. Based on your figure, how does the number of earthquakes from 1980–2009 compare to the number of earthquakes since 2010?

\AnswerBox{1\baselineskip}{The number of earthquakes since 2010 is far, far greater than the 30 years before hand.}

%\newpage

\question \label{ques:most_quakes}
Which three years have the greatest number of earthquakes? How many earthquakes occurred in each of these years?

\AnswerBox{3\baselineskip}{See if they think to look up the numbers in the Excel sheet.}

The recent increase in the number of earthquakes suggests wastewater disposal from the fracking industry \emph{might} be the cause but this would have to be tested. Such tests are complex but you can do a visual inspection to see if there \emph{appears} to be an association.

Use your web browser to visit this official website published by the Office Of The Secretary Of Energy \& Environment for Oklahoma, and the follow the instructions below.

\url{http://earthquakes.ok.gov/what-we-know/earthquake-map/}

\begin{enumerate}
	\item Uncheck the “Earthquakes - Past 7 days” to hide recent earthquakes.
	
	\item Click the “Arbuckle Waste Water Disposal Wells” check box. The blue circles show the locations of all active waste water disposal wells in Oklahoma.
	
	\item Click \emph{all three} checkboxes for “Earthquakes - 2000 through 2009,” “Earthquakes - 1990 through 1999,” and
	“Earthquakes - 1980 through 1989.”
	
	This shows the earthquakes recorded for 1980–2009. The size of the circles indicates the relative magnitude or strength of the earthquake. You can click on individual circles to see the date and strength of the earthquake.
\end{enumerate}

\question
Does there appear to be an association between the earthquakes and the disposal wells during 1980–2009, or do the earthquakes appear to be located randomly with respect to the wells? Explain.

\AnswerBox{3\baselineskip}{random pattern or weak association.}

%\newpage

\begin{enumerate}
	\item Uncheck the three boxes for 1980–2009. Leave the disposal wells box checked.
	
	\item Check the boxes for the three years with the greatest number of earthquakes. Refer back to question~\ref{ques:most_quakes} or your Excel figure.
	
\end{enumerate}

\question
Does there appear to be an association between the earthquakes and the disposal wells during the three years with greatest earthquake activity, or do the earthquakes appear to be located randomly with respect to the wells? Explain.

\AnswerBox{2\baselineskip}{Definite association between quakes and wells.}

%\newpage

Not all disposal wells cause earthquakes but these data show a clear association between the location of earthquakes and the waste water disposal wells. Other studies using more detailed data strongly support the hypothesis that deep, rapid injection of waste water is the cause of the sudden increase in Oklahoma earthquakes.

%% Effective Fall 2017
%% Not doing. Replaced with students making bar charts of their bacterial resistance data.
%\subsubsection*{Column chart: two explanatory variables}
%
%Create a column chart to show the mean (average) number of insect species collected from two species of trees. Trees were sampled from a treatment plot, where the number of trees per acre was thinned, and from a control plot, where no thinning occurred. Eight trees were surveyed for each tree species.\bigskip
%
%\question
%Which two variables are the explanatory variable? 
%
%\AnswerBox{2\baselineskip}{Tree species and whether the plot was thinned.}
%
%\question
%Which variable is the response variable?
%
%\AnswerBox{2\baselineskip}{Insect richness.}
%
%Download the file called “insect\_richness.xlsx” from the course data website:\\ \url{http://mtaylor4.semo.edu/~goby/bi163/}.\bigskip
%
%The file has three columns. The first is the treatment, indicating whether the plot was thinned or not (the control). The second column represents individual trees, either Douglas Fir (\textsc{df}) or Lodgepole Pine (\textsc{lp}). The third column is insect richness, which is the number of species of insects that was found on each tree. 
%
%The data are not in a format suitable for making a column chart. These data are the raw numbers of insects. You need to report the average richness per tree species so you must summarize the data. You also need to calculate the standard deviation.
%
%\begin{enumerate}
%	\item In \xcell{F3}, type “Douglas Fir.” In \xcell{G3}, type ‘Lodgepole Pine.” In \xcell{E4}, type “Control.” In \xcell{E5}, type “Thinned.” 
%
%	\item \label{tree_mean} Calculate the mean insect species richness for Douglas Fir in the Control Group. Click in \xcell{F4}. Type \texttt{=average(}. Next, select the cells that you want to average. In this case, select cells {\liningnum C2} through {\liningnum C9}, as you did above. After you have selected the last cell, type “)”, and then press the Enter key. 
%
%	What if you are not sure of the function name? Click on “Formulas” in the ribbon menu. Click on the “Insert Function” icon at the very left of the ribbon. In the dialog box that appears, type “Average” into the area where it says “Search for a function:”. Highlight \texttt{AVERAGE} in the choices and click the “OK” button. Next, a dialog box appears for you to enter the arguments for the \texttt{AVERAGE} function. For the Douglas Fir control sample, you must select cells {\liningnum C2:C9} by clicking and holding on \xcell{C2} and dragging down to \xcell{C9}. The range of cells is added to the dialog box. Press the Enter key or click the “OK” button. If you did this correctly, the value should be 65.875.
%
%	\item Repeat Step~\ref{tree_mean} for the other three cells: Control Lodgepole Pine, Thinned Douglas Fir, and Thinned Lodgepole Pine. 
%
%	\item \label{tree_dev} In \xcell{I3}, type ``Douglas Fir s.d.'' In \xcell{J3}, type ``Lodgepole Pine s.d.''
%
%	\item Calculate the standard deviations. In \xcell{I4}, type \texttt{=stdev.s(c2:c9)} (or, select the cells as you did earlier).
%
%	\item Repeat Step~\ref{tree_dev} for the other three cells: Control Lodgepole Pine (cells {\liningnum C10:C17}), Thinned Douglas Fir (cells {\liningnum C18:C25}), and Thinned Lodgepole Pine (cells {\liningnum C26:C33}). 
%
%\end{enumerate}
%
%The data are now arranged and ready for you to make the column chart.
%
%\begin{enumerate}[resume]
%	\item Select cells{\liningnum E3 to G5}.
%	
%	\item Insert a 2-D clustered column chart as you did above and resize it as necessary.
%	
%	\item Change the font and size to Times New Roman, 12 point.
%	
%	\item Delete the legend inside the figure.
%	
%	\item Add a \axis{y} label, Times New Roman, 12 point, regular style, that reads “Insect Richness.” \emph{Be sure to remove the bold face if Excel added it.}
%	
%	\item Remove any horizontal or vertical grid lines, if they appear on your figure.
%	
%	\item Change the colors of the columns to black and white (or grey). Use black for the border color of both columns.
%	
%\end{enumerate}
%
%For this figure, you will add error bars to show the standard deviations, to show the variability of the data around the mean for each species and treatment.
%
%
%\begin{enumerate}[resume]
%	
%	\item Click the left (control) bar for Douglas Fir. It should select both Control bars (for Douglas Fir and Lodgepole Pine). 
%	
%	\item Choose the “Design” tab. Click the “Add Chart Element” on the left side of the ribbon. Choose “Error Bars” from the menu. Select the “More Error Bars Options\dots” from the bottom of that menu.
%	
%	\item  Click on “Vertical Error Bars” on the top left side of the dialog box.
%	
%	\item On the right side of the screen, click on the “Custom” radio button, and then click on “Specify Value”.  If necessary, move the dialog box that opens to one side so that you can see cells {\liningnum I4:J5}.
%	
%	\item \label{fir_error} Delete the “\texttt{=\{1\}}” from the “Positive Error Value” field. With your cursor, select cells {\liningnum I4:J4}. Repeat this for the “Negative Error Value.” Select the same two cells ({\liningnum I4:J4}). Click the “OK” button.
%	
%	\item Repeat Step~\ref{fir_error} for Lodgepole Pine. Select cells {\liningnum I5:J5}.
%	
%\end{enumerate}
%
%
%\question
%Which species of tree has the insect higher richness. Use numbers from the data to support your answer.
%
%\AnswerBox{4\baselineskip}{Douglas fir has higher richness of about 67 (unthinned) or 55 (thinned species)}. 
%
%\question
%Does thinning of trees appear to affect species richness living in each tree species? Tell how you know.
%
%\AnswerBox{3\baselineskip}{Yes. The unthinned controls for both species have greater average richness compared to thinned plots.}
%
%\question
%How do you think the length of the error bars would change if you increased sample size for each tree species and treatment?
%
%\AnswerBox{3\baselineskip}{The amount of variability will \emph{probably} decrease so the error bars would get shorter.}
%
\subsection*{Scatterplot}

As you will learn later this semester, the distribution of ecosystems and their associated dominant plant groups, are determined primarily by mean annual temperature (\textsc{mat}) and mean annual precipitation (rain and snow; \textsc{map}).

\question
Identify the \emph{two} explanatory variables.\smallskip 

	\quad \ifprintanswers \textbf{Mean annual temperature} \else \rule{1.5in}{0.4pt} \fi
	\qquad \ifprintanswers \textbf{Mean annual precipitation} \else \rule{1.5in}{0.4pt} \fi

%\AnswerBox{2\baselineskip}{Mean annual temperature and mean annual precipitation.}

	\medskip
	
\question
Identify the response variable. \ifprintanswers \textbf{The ecosystems/dominant plant groups} \else \rule{1.5in}{0.4pt} \fi

%\AnswerBox{2\baselineskip}{The ecosystems/dominant plant groups.}

Download the file called “ecosystems.xlsx” from the course data website:\\ \url{http://mtaylor4.semo.edu/~goby/bi163/}.\bigskip

You will see the distribution of three plant groups by making a scatterplot of \textsc{mat} on the \axis{x} and \textsc{map} on the \axis{y}. %You will plot temperature on the \axis{x} and precipitation on the \axis{y}. 
The plant groups are Western Redcedar (Cw), Mixed Grassland (Gr) and Subalpine Larch (La). 

\begin{enumerate}
	\item Use your cursor to select cells {\liningnum C2:D21}, which selects \textsc{mat} and \textsc{map} for Western Redcedar.

	\item Click on the “Insert” in the ribbon menu, then on the ‘Scatter” icon, then on the upper left icon (“Scatter with only Markers”).

	Enlarge the graph by dragging from the corners. Fill most of the white space on the screen to the right of the data. Move the graph if it is covering any of the data. 
\end{enumerate}

You now have one plant group plotted but it is called ‘series 1”. We need to add data for the other two plant groups and give them informative names. 

\begin{enumerate}[resume]
	\item Right-click anywhere on the chart, and then click “Select Data\dots.”

	\item Click on “Series 1,” and then click the “Edit” button. Type “Western Redcedar” (do not include the quotes) in the “Series name:” field. Press Enter or click the “OK” button.

	\item Click the “Add” button. Type “Mixed Grasslands” in the “Series name:” field.

	\item Click in the “Series X values:” field.

	\item Select cells {\liningnum C22:C51} to select \textsc{mat} for the mixed grasslands. Notice that the cells you selected appear in the “Series \textsc{x} values” field.

	\item Repeat for \textsc{map}. Click in the “Series \textsc{y} values” field, delete the “=\{1\}”, and then select {\liningnum D22:D51} to select the \textsc{map} values for the \axis{y}. 

	\item Repeat this process for Subalpine Larch. Click the “Add” button, type “Subalpine Larch” for the name, use {\liningnum C52:C81} for “Series \textsc{x} values” and D52:D81 for “Series \textsc{y} values.” 

	\item Click the “OK” button. You may have to scroll back up. All three plant groups should now be displayed in your graph. 
\end{enumerate}

At this point, you have an adequate graph but you can improve it. Notice The \axis{y} crosses the \axis{x} at 0 instead of $-$4, which places the \axis{y} in an awkward position. Change the \axis{y} to cross the \axis{x} at $-$4.

\begin{enumerate}[resume]
	\item Place your cursor over one of the numbers on the \axis{x}, then right-click. Select “Format Axis\dots”.

	\item From the “Format Axis” area, click the “Axis value:” radio button, near the bottom below “Vertical axis crosses:” and then replace the 0.0 in box with $-$4.0. Do not forget the negative sign. Close the pane.

	\item As you did with the previous figures, set the font to Times New Roman, 12 point, regular style for the \textsc{x}- and \textsc{y}-axes. I trust you remember how to do that but, if not, review the instructions above.
	
	\item Delete the symbol legend. Delete any horizontal or vertical grid lines.

	\item Add labels to the \textsc{x}- and \textsc{y}-axes. Format the text of each axis to Times New Roman, 12 point. Set Font style to Regular to remove the bold face.

	\item Edit the text of each label to read “Mean Annual Temperature (degrees C)” for the \axis{x} and “Mean Annual Precipitation (mm) for the \axis{y}.

	\item You can improve the label “(degrees C)” by using the degree symbol (°). Right-click on the label and select “Edit Text.” Double click on “degrees” to select it.

\begin{center}
	\includegraphics[width=0.4\textwidth]{02_insert_symbol}
\end{center}

	\item Click on “Insert” on the ribbon menu and then click on “Symbol” on the right side of the ribbon. Click the degree symbol near the bottom center of the various symbols. Click in the “Insert” button and then the “Close” button. 
	
	\item If necessary, resize your figure so that the axis label does not overlap the axis values.
\end{enumerate}
 
You now have an excellent scatterplot but you can improve it by enlarging the symbols and removing the color, remembering that color is costly and is useless in photocopies.

\begin{enumerate}[resume]
	\item Place your cursor over one of the blue circles (Western Redcedar) and right-click \emph{once}. Select “Format Data Series\dots.” When you do this step, be sure all (or most) of the blue circles are selected. If only one blue circle is selected, click somewhere in the white area of the graph, then try again.  

	\item Click on the “Paint Can” icon. Select the “Marker” tab. Click on “Marker Options.” Click the “Built-in” radio button.  Increase the size to 10.
	
	\item Change the Type to diamond shape. Change the color to Solid Fill black. Change the border to Solid Line black.

	\item Repeat this process for the green circles that represent Subalpine Larch. Change the marker size to 9. Change the Built-In Type to triangle shape. Chang the color to Solid Fill medium gray.  Change the border to Solid Line black.

	\item Repeat for the red circles that represent the Mixed Grasslands. Change the marker size to 9 (size 10 squares might look too big relative to the diamonds and triangles but you are welcome to try size 10). Change the Built-In Type to square shape. Change the color to Solid Fill white.  Change the border to Solid Line black.
\end{enumerate}

\question
Which ecosystem shows the \emph{greatest} variability for mean annual temperature?

\AnswerBox{2\baselineskip}{Mixed Grasslands (squares).}

%\newpage

\question
Which ecosystem shows the \emph{least} variability for mean annual precipitation?

\AnswerBox{2\baselineskip}{Mixed Grasslands (squares).}

\question
Which ecosystem, on average, needs the lowest annual temperatures to be present. What do you estimate is about the average amount of annual precipitation needed by this ecosystem to be present? 

\AnswerBox{2\baselineskip}{Subalpine larch (triangles). About 1100 mm precipitation.}

\question
Which ecosystem, on average, needs the highest annual precipitation to be present. What do you estimate is about the average annual temperature needed by this ecosystem to be present? 

%\AnswerBox{2\baselineskip}{Western Redcedar (diamonds). About 8–9°C temperature.}
\ifprintanswers \textbf{Western Redcedar (diamonds). About 8–9°C temperature.} \fi

\subsection*{Line Graph}

Line graphs are useful for showing trends in data over time or categories, such as increasing tree diameter with age of the tree, or the accumulation of species sampled over time. Here you will plot the mean monthly river stage (a measure of water depth) from the Mississippi River measured at the U.S. Geological Survey gauge near Thebes, IL to see how river depth changes throughout the year. The gauge records the river stage every hour of every day. The data have been reduced to mean monthly discharge. %\bigskip

\smallskip

\question
Identify the explanatory variable. \ifprintanswers \textbf{Year} \else \rule{1.5in}{0.4pt} \fi

%\AnswerBox{2\baselineskip}{Year}

\medskip

\question
Which variable is the response variable? \ifprintanswers \textbf{Mean monthly river stage} \else \rule{1.5in}{0.4pt} \fi

%\AnswerBox{2\baselineskip}{Mean monthly river stage}
\bigskip

Download the file called “mississippi\_river\_stages.xlsx” from the course data website:\\ \url{http://mtaylor4.semo.edu/~goby/bi163/}

By now, you should be good at graphing in Excel so the final graph should be quick.

\begin{enumerate}
	\item Use the cursor to highlight cells {\liningnum B2:C13}, which contains January to December data for the year 2011.

	\item Click “Insert” on the ribbon menu, click the “Line” icon near the left center of the ribbon, and then click the upper left icon to insert the line graph.

	\item Right-click on the graph, choose “Select Data\dots” and add the 2012 series, using the same procedure you followed with the scatterplot. Edit the Series 1 name to change it to “2011”. Name the new series “2012”. \emph{Note}: Use the Gauge Height column for the \textsc{y} values. The Month names for the \textsc{x}-values do not change so do not select them when you add new data series. Select only the Gauge Height values.

	\item Add a primary vertical axis title. Edit the label to read “Gauge Height (feet)”. Add a primary horizontal axis title that reads “Month”. Change both to 12 point, Times New Roman font, regular style.

	\item Edit both axes to use 12 point, Times New Roman Regular font, regular style.

	\item Delete the symbol legend. Delete any horizontal and vertical grid lines.

	\item Right-click on the upper line for 2011. Format the data series so that the Line Color is solid black.

	\item Right-click on the lower line for 2012. Format the color to black and the dash type to dashed (use either the 3rd or 4th choice for a nice dashed line.)
\end{enumerate}

\question
One of the years you plotted was a drought year across the U.S., including southeast Missouri. The other year was a flood year, where the flooding was so bad in southeast Missouri that they had to blow up a levee at Bird's Point to release some of the water from the river. Based on your graph, which line represents the drought year and which year represents the flood year. Explain how you decided.

\AnswerBox{3\baselineskip}{The flood year has a much higher river stage in the spring. The drought year has a much lower river stage overall.}

%\newpage

\question
In both years, the river stage is higher in April and May and lower in September and October. Explain why.

\AnswerBox{3\baselineskip}{Snow melt and spring rains bring the river levels up in spring. No melt and summer dryness leads to low river stages in summer.}


%\subsubsection*{Chart types to avoid}
%
%Pie Charts: Pie charts are not often used with scientific data 
%although they do have some specific uses. People are not good 
%at estimating the area of a circle or comparing the relative 
%sizes of the pie slices. This increases the difficulty of 
%interpreting the figure. In general, use a clustered column 
%chart instead of a pie chart. 
%
%3-D charts: They look fancy but they are actually harder to 
%interpret for most types of data. 3-D charts are rarely 
%appropriate and must be avoided. 

\subsubsection*{Figure Captions}

Every scientific figure must have a figure number and a caption 
\emph{below} the figure. Figure numbers are assigned to figures 
in the order they appear in your report. For example, if your 
report has two figures, the first figure would be labeled Figure~1.
The second figure would be labeled Figure~2.
Even if your report has just one figure, it would still be labeled
Figure~1. 

The figure caption, also called a figure legend, must describe 
briefly what the figure is about, and tell what any symbols, 
colors, or lines represent. The caption might also 
provide sample sizes, tell what any error bars represent, or 
if differences were statistically significant. Thus, the 
figure itself determines what should be in the caption. 
Here is an example of the Mississippi River gauge height graph you 
made earlier.

\vspace*{0.5\baselineskip}

\hfil%
\begin{minipage}{0.75\textwidth}
	\includegraphics[width=\textwidth]{02_mississippi_gauge_height}\\
	Figure 1. Mean monthly gauge height for the Mississippi River near 
	Thebes, Illinois for 2011 (solid line) and 2012 (dashed line). 
\end{minipage}%
\hfil

\vspace{0.5\baselineskip}

The caption tells that the data are the mean (average) gauge heights for 
each month, which river the data are from, the location of the
gauge, which years that data are from, and which line represents each 
year. The reader can look at only that figure and caption and know what
the figure is about. A rule of thumb is that \emph{figures stand alone,} which means 
the caption must provide enough information so that a person who 
has the figure and caption only can still interpret the figure. 


%On the next page, you will write captions for the other two figures you made today.

%\newpage

\question
Write a caption for the Oklahoma earthquake figure. This will be 
a relatively simple caption. Be sure to include the correct figure 
number. Hint: What is the figure number for the above figure?

\hfil%
\begin{minipage}{0.7\textwidth}
	\includegraphics[width=\textwidth]{02_ok_earthquakes}\\
	
	\ifprintanswers 
		\textbf{Figure 2. Number of Oklahoma earthquakes per year for 1980–2016.}
		\vspace*{\baselineskip}
	\else
		\vspace{0.1\textheight}
	\fi
%\newpage
	
\end{minipage}%
\hfil

%\newpage

\question
Write a caption for the ecosystem distribution figure. This
caption will be a bit more complex than the last one your wrote.
Be sure to include  the correct figure number.

\hfil%
\begin{minipage}{0.7\textwidth}
	\includegraphics[width=\textwidth]{02_ecosystem_distribution}\\
	
	\ifprintanswers 
		\textbf{{\small Figure 3. Distribution of three ecosystems based on mean annual precipitation and mean annual temperature. The dominant plant group for each ecosystem is Subalpine Larch (gray triangles), Mixed Grassland (white squares), and Western Redcedar (black diamonds).}}
		%\vspace*{\baselineskip}
	\else
		\vspace{0.01\textheight}
	\fi
	
%	\vspace{3\baselineskip}
%	\vspace{0.15\textheight}
	
\end{minipage}%
\hfil



\subsubsection*{Graph the results of your bacterial resistance data}

Your final task is to make a graph of the bacterial resistance data and
write a caption for it. 
You will graph the mean values of the diameters of the zone of inhibition 
and the halo for your entire lab section. \emph{Graph the mean values for 
the entire lab section, not the values obtained by your group.} 

Your instructor will collect the data from each group. Calculate the mean 
values for each treatment (control and the different concentrations of 
triclosan), and then fill in the table on the next page with the results. 
The control is 0.0\% concentration.

%\newpage

\begin{longtable}[l]{lcc}
	\toprule
	Concentration	&	Inhibition Diameter	&	Halo Diameter \tabularnewline
	\midrule
	& & \tabularnewline[0.5em]
	0.0\%	& \rule{0.75in}{0.4pt} & \rule{0.75in}{0.4pt} \tabularnewline[1.5em]
	0.1\%	& \rule{0.75in}{0.4pt} & \rule{0.75in}{0.4pt} \tabularnewline[1.5em]
	0.3\%	& \rule{0.75in}{0.4pt} & \rule{0.75in}{0.4pt} \tabularnewline[1.5em]
	0.5\%	& \rule{0.75in}{0.4pt} & \rule{0.75in}{0.4pt} \tabularnewline[1.5em]
	0.7\%	& \rule{0.75in}{0.4pt} & \rule{0.75in}{0.4pt} \tabularnewline
	\bottomrule
\end{longtable}

\question \label{ques:graph_type}
Which type of graph is most appropriate for these data? Why?

\AnswerBox{2\baselineskip}{Bar/Column chart}

Enter the above data into Excel. Use the row and column headings shown in the table above. Then make the type of graph you identified for question~\ref{ques:graph_type}. Be sure your graph meets the following requirements:

\begin{enumerate}
	\item Remove all grid lines,
	
	\item properly label the \textsc{x}- and \textsc{y}-axes, including units of measurement and concentration,
	
	\item use 0.0 for the control
	
	\item use black, white, or shades of gray instead of color,
	
	\item use the proper font and font size, and
	
	\item remove the default figure legend.
	
\end{enumerate}

On the next page, write a caption for your figure.

%\newpage

\question
Write the caption for your figure. When finished, show the 
figure and caption to your lab instructor. \emph{Save a copy 
of your spreadsheet or graph so that you can include it in 
your final report for the bacterial resistance experiment.}


\AnswerBox{\basespace}{Be sure legend includes a figure number, a brief description of what the figure represents, and a statement about what each color bar represents. }

\end{questions}



\end{document}  