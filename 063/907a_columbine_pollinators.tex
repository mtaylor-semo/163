%!TEX TS-program = lualatex
%!TEX encoding = UTF-8 Unicode

\documentclass[12pt, hidelinks]{exam}
\usepackage{graphicx}
	\graphicspath{{/Users/goby/Pictures/teach/163/lab/}
	{img/}} % set of paths to search for images

\usepackage{geometry}
\geometry{left=1.5in, bottom=1in}                   
%\geometry{landscape}                % Activate for for rotated page geometry
\usepackage[parfill]{parskip}    % Activate to begin paragraphs with an empty line rather than an indent
%\usepackage{amssymb, amsmath}
%\usepackage{mathtools}
%	\everymath{\displaystyle}

\usepackage{pdflscape}

\usepackage{fontspec}
\setmainfont[Ligatures={TeX}, BoldFont={* Bold}, ItalicFont={* Italic}, BoldItalicFont={* BoldItalic}, Numbers={OldStyle}]{Linux Libertine O}
\setsansfont[Scale=MatchLowercase,Ligatures=TeX]{Linux Biolinum O}
\setmonofont[Scale=MatchLowercase]{Linux Libertine Mono O}
\usepackage{microtype}

%\usepackage{unicode-math}
%\setmathfont[Scale=MatchLowercase]{Asana Math}
%\setmathfont[Scale=MatchLowercase]{XITS Math}

% To define fonts for particular uses within a document. For example, 
% This sets the Libertine font to use tabular number format for tables.
\newfontfamily{\dnatable}[Numbers={Monospaced}]{Linux Libertine Mono O}
\newfontfamily{\regfont}[ItalicFont={* Italic}]{Linux Libertine O}

\usepackage{booktabs}
\usepackage{multicol}

\usepackage[justification=raggedright, labelsep=period]{caption}
\captionsetup{singlelinecheck=off}
\captionsetup{skip=0.2em}

%\usepackage{tabularx}
\usepackage{longtable}
%\usepackage{siunitx}
\usepackage{array}
\newcolumntype{L}[1]{>{\raggedright\let\newline\\\arraybackslash\hspace{0pt}}p{#1}}
\newcolumntype{C}[1]{>{\centering\let\newline\\\arraybackslash\hspace{0pt}}p{#1}}
\newcolumntype{R}[1]{>{\raggedleft\let\newline\\\arraybackslash\hspace{0pt}}p{#1}}

\usepackage{enumitem}
\usepackage{enumitem}
\setlist{leftmargin=*}
\setlist[1]{labelindent=\parindent}
\setlist[enumerate]{label=\textsc{\alph*}.}

\usepackage{hyperref}
%\usepackage{placeins} %PRovides \FloatBarrier to flush all floats before a certain point.
%\usepackage{hanging}

\usepackage[sc]{titlesec}

\renewcommand{\solutiontitle}{\noindent}
\unframedsolutions
\SolutionEmphasis{\bfseries}

\renewcommand{\questionshook}{%
	\setlength{\leftmargin}{-\leftskip}%
}

%Change \half command from 1/2 to .5
\renewcommand*\half{.5}


%% Allows fullwidth command to break across pages.
%% See 
%\makeatletter
%\def\SetTotalwidth{\advance\linewidth by \@totalleftmargin
%\@totalleftmargin=0pt}
%\makeatother


\pagestyle{headandfoot}
\firstpageheader{\textsc{bi} 063: Evolution and Ecology}{}{\ifprintanswers\textbf{KEY}\else Name: \enspace \makebox[2.5in]{\hrulefill}\fi}
\runningheader{}{}{\footnotesize{pg. \thepage}}
\footer{}{}{}
\runningheadrule

\newcommand*\AnswerBox[2]{%
    \parbox[t][#1]{0.92\textwidth}{%
    \begin{solution}#2\end{solution}}
    \vspace{\stretch{1}}
}

\newenvironment{AnswerPage}[1]
    {\begin{minipage}[t][#1]{0.92\textwidth}%
    \begin{solution}}
    {\end{solution}\end{minipage}
    \vspace{\stretch{1}}}

\newlength{\basespace}
\setlength{\basespace}{5\baselineskip}

%\printanswers

\begin{document}

\subsection*{Pollinators and speciation in columbine plants}

Columbines (genus \textit{Aquilegia}), a diverse group of plants
with about 70 known species. Columbine flowers have an unusual
morphology. The inner petals have long spurs that extend backwards
between the outer sepals. The spurs are filled with nectar that attracts
pollinators, such as bumblebees, hummingbirds, and hawkmoths. In
columbines, spur length and flower color vary among species. Spurs vary from
1–15 cm and may be straight or curved. Sepal, petal and spur colors are
typically blue, red, white, or yellow.

\begin{center}
	\includegraphics{06_columbine_parts}

	{\footnotesize
	www.fs.fed.us/wildflowers/beauty/columbines/flower.shtml}
\end{center}

For this exercise, you will work in pairs or in a group of three students. Together,
 you will use a phylogeny of 30 \textit{Aquilegia} species and
varieties to explore how changes of flower color and spur length
correspond to changes in columbine pollinators.

The phylogeny is on the next page. Match the species on the phylogeny
 to the flower color (blue, red, yellow or white), the spur length (short, medium or long), and the
pollinator (bumblebees, hummingbirds, or hawkmoths). Then,
search for evolutionary patterns. For example, do the columbine species
with red flowers share the same common ancestor or did red flower color
evolve multiple times? Do all columbine species with short spurs share a
common ancestor? What about columbine species that are pollinated by
hawkmoths?

Once you have established the patterns on the tree, answer the questions 
on the next page.
%compare your results
%with the results of the other members in your group. Do you find a
%relationship between flower color, spur length and pollinator?


\begin{questions}

\question
What is the association between flower color and pollinator? Explain.

\AnswerBox{2\baselineskip}{Blue: Bumble bees. Red: Hummingbirds. Yellow/white: Hawkmoths.}

\question
What is the association between spur length and pollinator? Explain.

\AnswerBox{2\baselineskip}{short: Bumble bees. Medium: Hummingbirds. Long: Hawkmoths.}


\question
Which type of organism (bumble bees, hummingbirds, hawkmoths) do you think was the original type of pollinator for the common ancestor of columbines? Explain why. \textsc{Hint:} Which columbines are basal? What pollinators do they use?

\AnswerBox{2\baselineskip}{Bumble bees, as the basal groups are pollinated by bumble bees.}

\question
How many pollinator shifts (from one type of pollinator to another) can you identify on the tree? 

\AnswerBox{2\baselineskip}{Lots. At least one shift to hummingbirds and one to hawkmoths but more likely.}

\question
Propose how pollinators might have contributed to reproductive isolation and speciation in columbine plants.

\AnswerBox{3\baselineskip}{Pollinators return to same species, prevent gene flow. Many answers are possible.}

\end{questions}

\newpage

\subsubsection*{Columbine spur length and flower colors}

Spurs are short (7--12 mm), medium (14--24 mm) or long (26--130 mm).  
Flower color is typically blue, red, white or yellow. In some species,
the color is very pale or faded. The color refers to the color of the
spurs and the sepals. In some species, the petals have a different color
than the sepals and spurs. For example, most of the species with red
flowers have yellow petals. The information in the table has been simplified
without affecting the results.

Match spur length and the flower colors to the species codes on the phylogeny.

\begin{tabular}[c]{@{}lll@{}}
\toprule
Species & Spur Length & Flower Color\tabularnewline
\midrule
BA 				& Medium	& Red (often pale)\tabularnewline
BR 				& Short 	& Blue\tabularnewline
CA 				& Medium	& Red\tabularnewline
CH.CHI 		& Long 		& Yellow\tabularnewline
CH.NM 		& Long 		& Yellow\tabularnewline
CHAP 			& Long 		& Yellow\tabularnewline
COAL 			& Long 		& Pale blue to white\tabularnewline
COOC.CO	& Long 		& Pale blue to white\tabularnewline
COOC.UT 	& Long		& White\tabularnewline
COOC.WY 	& Long		& Pale blue to white\tabularnewline
DE 				& Medium	& Red\tabularnewline
EL 				& Medium	& Red\tabularnewline
EX 				& Medium	& Red\tabularnewline
FL 				& Medium	& Yellow\tabularnewline
FO.E 			& Medium	& Red\tabularnewline
FO.W 			& Medium	& Red\tabularnewline
HI 				& Long		& Yellow\tabularnewline
JO				& Short 	& Blue\tabularnewline
LA 				& Short 	& Pale purple to white\tabularnewline
LO.AZ 			& Long		& Yellow\tabularnewline
LO.TX 			& Long		& Yellow\tabularnewline
MI					& Long		& Pink (occasionally pale)\tabularnewline
PI 				& Long 		& White\tabularnewline
PU				& Long 		& White\tabularnewline
SA				& Short		& Blue\tabularnewline
SC 				& Long 		& Blue\tabularnewline
SH 				& Medium	& Red\tabularnewline
SK				& Medium	& Red\tabularnewline
Sp. nov.\footnotemark	& Long & White\tabularnewline
TR 				& Medium	& Red\tabularnewline
\bottomrule
\end{tabular}


\footnotetext{Sp. nov. means \emph{species novum}, which designates a new
species without a formal scientific name.}

\newpage

\subsubsection*{Columbine Species and Pollinators}

Most columbines are pollinated by only bumblebees, only hummingbirds or
only hawkmoths. A few species have two pollinators, which is indicated
in the table. Both pollinators contribute about equally to pollination
success in columbines.

Match spur length and the flower colors to the species codes on the phylogeny.

\begin{tabular}[c]{@{}ll@{}}
\toprule
Species & Pollinator\tabularnewline
\midrule
BA 				& Hummingbirds / Hawkmoths\tabularnewline
BR 				& Bumblebees\tabularnewline
CA 				& Hummingbirds\tabularnewline
CH.CHI 		& Hawkmoths\tabularnewline
CH.NM 		& Hawkmoths\tabularnewline
CHAP 			& Hawkmoths\tabularnewline
COAL 			& Hawkmoths\tabularnewline
COOC.CO	& Bumblebees\tabularnewline
COOC.UT 	& Hawkmoths\tabularnewline
COOC.WY 	& Bumblebees\tabularnewline
DE 				& Hummingbirds\tabularnewline
EL 				& Hummingbirds\tabularnewline
EX 				& Hummingbirds\tabularnewline
FL 				& Hummingbirds\tabularnewline
FO.E 			& Hummingbirds\tabularnewline
FO.W 			& Hummingbirds\tabularnewline
HI 				& Hawkmoths\tabularnewline
JO 				& Bumblebees\tabularnewline
LA 				& Bumblebees\tabularnewline
LO.AZ 			& Hawkmoths\tabularnewline
LO.TX 			& Hawkmoths\tabularnewline
MI 				& Hummingbirds / Hawkmoths\tabularnewline
PI 				& Hawkmoths\tabularnewline
PU 				& Hawkmoths\tabularnewline
SA 				& Bumblebees\tabularnewline
SC 				& Hawkmoths\tabularnewline
SH 				& Hummingbirds\tabularnewline
SK 				& Hummingbirds\tabularnewline
Sp. nov.		& Hawkmoths\tabularnewline
TR 				& Hummingbirds\tabularnewline
\bottomrule
\end{tabular}

\newpage

	\includegraphics[width=0.85\textwidth]{06_columbine_phylogeny}

\end{document}  