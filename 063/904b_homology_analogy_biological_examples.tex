%!TEX TS-program = lualatex
%!TEX encoding = UTF-8 Unicode

\documentclass[12pt, hidelinks]{exam}
\usepackage{xcolor}
\usepackage{graphicx}
	\graphicspath{{/Users/goby/Pictures/teach/163/lab/}
	{img/}} % set of paths to search for images

\usepackage{geometry}
\geometry{letterpaper, left=1.5in, bottom=1in}                   
%\geometry{landscape}                % Activate for for rotated page geometry
\usepackage[parfill]{parskip}    % Activate to begin paragraphs with an empty line rather than an indent
\usepackage{amssymb, amsmath}
\usepackage{mathtools}
	\everymath{\displaystyle}

\usepackage{fontspec}
\setmainfont[Ligatures={TeX}, BoldFont={* Bold}, ItalicFont={* Italic}, BoldItalicFont={* BoldItalic}, Numbers={OldStyle}]{Linux Libertine O}
\setsansfont[Scale=MatchLowercase,Ligatures=TeX]{Linux Biolinum O}
%\setmonofont[Scale=MatchLowercase]{Inconsolatazi4}
\usepackage{microtype}


% To define fonts for particular uses within a document. For example, 
% This sets the Libertine font to use tabular number format for tables.
 %\newfontfamily{\tablenumbers}[Numbers={Monospaced}]{Linux Libertine O}
% \newfontfamily{\libertinedisplay}{Linux Libertine Display O}

\usepackage{booktabs}
\usepackage{multicol}
\usepackage[normalem]{ulem}

\usepackage{longtable}
%\usepackage{siunitx}
\usepackage{array}
\newcolumntype{L}[1]{>{\raggedright\let\newline\\\arraybackslash\hspace{0pt}}p{#1}}
\newcolumntype{C}[1]{>{\centering\let\newline\\\arraybackslash\hspace{0pt}}p{#1}}
\newcolumntype{R}[1]{>{\raggedleft\let\newline\\\arraybackslash\hspace{0pt}}p{#1}}

\usepackage{enumitem}
%\setenumerate{label=\Alph.}
\setlist{leftmargin=*}
\setlist[1]{labelindent=\parindent}
\setlist[enumerate]{label=\textsc{\alph*}.}
\setlist[itemize]{label=\color{gray}\textbullet}

\usepackage{hyperref}
%\usepackage{placeins} %PRovides \FloatBarrier to flush all floats before a certain point.
\usepackage{hanging}

\usepackage[sc]{titlesec}

%% Commands for Exam class
\renewcommand{\solutiontitle}{\noindent}
\unframedsolutions
\SolutionEmphasis{\bfseries}

% Shifts margins left. Question numbers appear in margin.
% This allows "fullwidth" to left align with text that is outside 
% if the questions environment.
\renewcommand{\questionshook}{%
	\setlength{\leftmargin}{-\leftskip}%
}

%\renewcommand{\partshook}{%
%	\setlength{\leftmargin}{-\leftskip}%
%}

%Change \half command from 1/2 to .5
\renewcommand*\half{.5}

\pagestyle{headandfoot}
\firstpageheader{\textsc{bi}\,063 Evolution and Ecology}{}{\ifprintanswers\textbf{KEY}\else Name: \enspace \makebox[2.5in]{\hrulefill}\fi}
\runningheader{}{}{\footnotesize{pg. \thepage}}
\footer{}{}{}
\runningheadrule

\newcommand*\AnswerBox[2]{%
    \parbox[t][#1]{0.92\textwidth}{%
    \begin{solution}#2\end{solution}}
%    \vspace*{\stretch{1}}
}

\newenvironment{AnswerPage}[1]
    {\begin{minipage}[t][#1]{0.92\textwidth}%
    \begin{solution}}
    {\end{solution}\end{minipage}
    \vspace*{\stretch{1}}}

\newlength{\basespace}
\setlength{\basespace}{5\baselineskip}


%\usepackage{mdframed}
%\mdfsetup{%
%	innerleftmargin=0pt,%
%	innerrightmargin=0pt,
%	innertopmargin=0pt,
%	innerbottommargin=0pt,
%	hidealllines=true
%}%end mdfsetup

%
%\makeatletter
%\def\SetTotalwidth{\advance\linewidth by \@totalleftmargin
%\@totalleftmargin=0pt}
%\makeatother


%\printanswers

\pointsinmargin\pointformat{}

\begin{document}


\subsection*{Homology or analogy? Biological examples}

The remainder of this course will be a series of ways to test your
hypothesis about how (or if) organisms are related. In order to do this,
you will need to determine if particular similarities are due to homology
or analogy. Here is a flowchart showing a method for making these
decisions:

\begin{center}
	\includegraphics[width=0.94\textwidth]{04_homology_analogy_flowchart}
\end{center}

\begin{enumerate}

\item
	Describe the structural similarity you wish to test.

    You must describe a specific organ or structure that is similar in the two organisms.
    You canot say a whole organism is analogous or homologous to
    another; you have to work with one structure at a time. You must describe
    \emph{how} the organs or structures are similar. Do not just name the similar structures. 

\item
  Describe the function of the structures in each organism.

    Examine the structure in each organism separately and describe the functions
    operationally. That is, do not say ``they use them to
    move around,'' because moving around could be accomplished with feet,
    wings, jets, or wheels, each of which works quite differently from
    the rest. Instead, say exactly what they do. For example, the wing
    of the goose is used push down on air and move the
    body to fly.

\item
  Do the structures serve the same or different functions? 
  Is the structural similarity necessary in order to have this function?

    For example, all wings push down on air to move the body.

\item Decide whether the structural similarity is an analogy or a homology.

	  Analogy.---An analogy does not provide any evidence
  that organisms are related but it does not falsify the hypothesis
  either. \emph{For judging the hypothesis in the tree, this
  similarity gives no conclusive evidence.}
	
	  Homology.---Homologies can only be concluded if a similarity
  is not necessary for the function. A good way to judge this is whether the 
  structures serve different function. If the structure serve different functions
  in different organisms, then the similarity might not be necessary to serve any 
  one of the functions. 

\end{enumerate}

For this lab, you will examine different structures on several organisms and decide whether a similarity is due to homology or analogy. To make this decision, use the flow chart at
the top of this page and follow the steps. You cannot base your decision on a hypothesis about the ancestry of the organisms. (Remember that, for the
purposes of this course, if we cannot disprove analogy, we will accept it
as a tentative explanation.)

%Each question below is worth 1 point.

\subsection*{Part 1. Wings}

\begin{questions}

\question[1]\label{ques:wings}
Examine the wings of a bird and a cicada. Describe their structural
similarity. Do not just write ``both are wings.''  Describe what is similar about the
structures that we call wings in a bird and a cicada.

\AnswerBox{3\baselineskip}{}

\question[1]
What is the function of the wing in the bird and the cicada? Again, be specific.

\AnswerBox{3\baselineskip}{}

\question[1]
Is the similarity you described above necessary to serve
this function? Explain. If you described multiple similarities,
(flat, thin, lightweight, etc.) consider each similarity separately
and determine if it is necessary for the function.

\AnswerBox{3\baselineskip}{}

\question[1]
What do you conclude? Is the structural similarity you described in
question~\ref{ques:wings} explained by homology or analogy?

%\AnswerBox{3\baselineskip}{}
\newpage

\question[1]
Does this evidence support or falsify the hypothesis that birds and
cicadas have a common ancestor? Or, is the evidence inconclusive?
Explain.

\AnswerBox{3\baselineskip}{}

\subsection*{Part 2. Leaves}

\question[1]\label{ques:leaf}
Examine the leaves of a maple tree and the leaf of
an oak tree.\footnote{Your instructor may provide leaves from other species of trees.} If the leafs are mounted on paper, \emph{please
handle them carefully.} Describe their structural similarity.

\AnswerBox{5\baselineskip}{}

\question[1]
What is the function of the leaves in the oak and the maple?

\AnswerBox{5\baselineskip}{}

\question[1]
Is the similarity you described above necessary in order to serve
this function? Explain. If you described multiple similarities,
e.g., flat, thin, lightweight, etc., consider each similarity separately and
determine if it is necessary for the function.

\AnswerBox{5\baselineskip}{}

\question[1]
What do you conclude? Is the structural similarity you described in
question~\ref{ques:leaf} explained by homology or analogy? Explain.

%\AnswerBox{3\baselineskip}{}
\newpage

\question[1]
Does this evidence support or falsify the hypothesis that oaks and
maple have a common ancestor? Or, is the evidence inconclusive? Explain.

\AnswerBox{3\baselineskip}{}

\question[1]
Most cactus plants do not have leaves, yet they have photosynthesis.
Where do you think photosynthesis takes place in cactus? Does this
knowledge change your thoughts about leaves being necessary for
photosynthesis? You do not need to modify your earlier answers.

\AnswerBox{3\baselineskip}{}

\subsection*{Part 3. Limbs}

Examine the \emph{front} legs of just the cicada and ant and 
answer questions \ref{ques:ant_start}--\ref{ques:ant_end}.  Then, 
look at the front legs of the praying mantis and answer question~\ref{ques:mantis_start}.

\question[1]\label{ques:ant_start}
Describe the structural similarities for the front limbs of an ant and a cicada. 

\AnswerBox{3\baselineskip}{}

\question[1]
Describe the function of the front legs of the ant and cicada.

\AnswerBox{3\baselineskip}{}

\question[1]
Is the similarity you described above necessary in order to serve
this function? Explain.

\AnswerBox{3\baselineskip}{}

\question[1]\label{ques:ant_end}
What do you conclude? Is the structural similarity explained by
homology or analogy, or is the evidence inconclusive? Explain.

%\AnswerBox{3\baselineskip}{}

\newpage

\question[4]\label{ques:mantis_start}
Now look at the front limb of a praying mantis.

\begin{parts}
\part What is its function?

\AnswerBox{3\baselineskip}{}

\part How is it similar to the front leg of a cicada or ant?

\AnswerBox{3\baselineskip}{}

\part Is the similarity to ant and cicada legs that you described above
necessary in order to serve this function in the praying mantis? That is, does the front
limb of the praying mantis serve the same function as the front legs in ants and
cicadas?

\AnswerBox{3\baselineskip}{}

\part\label{ques:mantis_end} What do you conclude? Is the structural similarity of the praying mantis explained by homology or analogy? Explain.

\AnswerBox{3\baselineskip}{}

\end{parts}

\question[1]
Based on your conclusions above for questions~\ref{ques:ant_end} and \ref{ques:mantis_start}\ref{ques:mantis_end}, can you determine whether ant and
cicada share a common ancestor? Explain.

\end{questions}

\end{document}  