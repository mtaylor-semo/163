%!TEX TS-program = lualatex
%!TEX encoding = UTF-8 Unicode

\documentclass[t]{beamer}

%%%% HANDOUTS For online Uncomment the following four lines for handout
%\documentclass[t,handout]{beamer}  %Use this for handouts.
%\usepackage{handoutWithNotes}
%\pgfpagesuselayout{3 on 1 with notes}[letterpaper,border shrink=5mm]


%%% Including only some slides for students.
%%% Uncomment the following line. For the slides,
%%% use the labels shown below the command.
%\includeonlylecture{student}

%% For students, use \lecture{student}{student}
%% For mine, use \lecture{instructor}{instructor}


%\usepackage{pgf,pgfpages}
%\pgfpagesuselayout{4 on 1}[letterpaper,border shrink=5mm]

% FONTS
\usepackage{fontspec}
\def\mainfont{Linux Biolinum O}
\setmainfont[Ligatures={Common,TeX}, Contextuals={NoAlternate}, Numbers={Proportional, OldStyle}]{\mainfont}
\setsansfont[Ligatures={Common,TeX}, Scale=MatchLowercase, Numbers={Proportional,OldStyle}, BoldFont={* Bold}, ItalicFont={* Italic},]\mainfont

\newfontface\lining[Numbers={Lining}]\mainfont

\usepackage{graphicx}
	\graphicspath{{/Users/goby/pictures/teach/163/lab/}
	{/Users/goby/pictures/teach/common/}} % set of paths to search for images

%\usepackage{units}
\usepackage{booktabs}
\usepackage{multicol}
\usepackage{calc}  % Used for the HiddenWord macro below.
%\usepackage{textcomp}

\usepackage{tikz}
%	\tikzstyle{every picture}+=[remember picture,overlay]

\mode<presentation>
{
  \usetheme{Lecture}
  \setbeamercovered{invisible}
  \setbeamertemplate{items}[square]
}

%\usefonttheme[onlymath]{serif}
%\usecolortheme[named=blue7]{structure}

% HiddenWord macro requires the calc package.
\newcommand\HiddenWord[1]{% 
	\alt<handout>{\rule{\widthof{#1}}{\fboxrule}}{#1}%
}

% Chance gray to black if you do not want to gray out text.
\newcommand\GrayedOut[1]{%
	\alt<handout>{#1}{\textcolor{gray}{#1}}%
}


\begin{document}

\lecture{instructor}{instructor}
{
\usebackgroundtemplate{\includegraphics[width=\paperwidth]{hoverflies} }
\begin{frame}[b,plain]{\textcolor{orange7}{\textsc{bi} 063: Evolution and Ecology Lab}}

%\begin{center}\LARGE\textcolor{white}{Sit towards front of the room.}\end{center}

\hfill\textcolor{white}{\tiny Fir0002, Wikimedia \ccbync{3}}
\end{frame}
}

{
\usebackgroundtemplate{\includegraphics[width=\paperwidth]{mike_snake}
}
\begin{frame}[t,plain]
	\large
	\vspace{5ex}
	\hangpara \hspace{17em} Mike Taylor

	\hangpara \hspace{17em} \textsc{rh} 217

	\hangpara \hspace{17em} \textsc{m}\,9--10, \textsc{t}\,10--11, \textsc{w}\,11--12.

	\hangpara \hspace{17em} mtaylor@semo.edu
	
	\hangpara \hspace{17em} \includegraphics[width=0.4cm]{twitter_icon} @MikeTaylor\textsc{semo}\\
	\hspace{17em} \#\textsc{bi}063\textsc{taylor}

\end{frame}
}


\begin{frame}[t]{\href{http://learning.semo.edu}{learning.semo.edu}}
	\begin{center}
		\includegraphics[width=\textwidth]{moodle_logo}
		
		\medskip
		
	\end{center}
	
\end{frame}


\lecture{student}{student}
%
\begin{frame}{Scientists usually manipulate and measure variables.}
	
	\hangpara  \highlight{Explanatory variable:} a manipulated or controlled variable in an experiment
	or study whose presence or degree determines a change in the response
	variable.  
	
	\hangpara  \highlight{Response variable:} an observed variable in an experiment or
	study that changes in response to the presence or degree of one
	or more explanatory variables.  
	
\end{frame}

\end{document}
