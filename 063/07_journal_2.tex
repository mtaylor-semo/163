%!TEX TS-program = lualatex
%!TEX encoding = UTF-8 Unicode

\documentclass[12pt]{exam}
\usepackage{graphicx}
	\graphicspath{{/Users/goby/Pictures/teach/163/lab/}
	{img/}} % set of paths to search for images

\usepackage{geometry}
\geometry{letterpaper, left=1.5in, bottom=1in}                   
%\geometry{landscape}                % Activate for for rotated page geometry
\usepackage[parfill]{parskip}    % Activate to begin paragraphs with an empty line rather than an indent
\usepackage{amssymb, amsmath}
\usepackage{mathtools}
	\everymath{\displaystyle}

\usepackage{fontspec}
\setmainfont[Ligatures={TeX}, BoldFont={* Bold}, ItalicFont={* Italic}, BoldItalicFont={* BoldItalic}, Numbers={OldStyle}]{Linux Libertine O}
\setsansfont[Scale=MatchLowercase,Ligatures=TeX]{Linux Biolinum O}
\setmonofont[Scale=MatchLowercase]{Inconsolatazi4}
\usepackage{microtype}


% To define fonts for particular uses within a document. For example, 
% This sets the Libertine font to use tabular number format for tables.
 %\newfontfamily{\tablenumbers}[Numbers={Monospaced}]{Linux Libertine O}
% \newfontfamily{\libertinedisplay}{Linux Libertine Display O}

\usepackage{booktabs}
\usepackage{multicol}
\usepackage[normalem]{ulem}

\usepackage{longtable}
%\usepackage{siunitx}
\usepackage{array}
\newcolumntype{L}[1]{>{\raggedright\let\newline\\\arraybackslash\hspace{0pt}}p{#1}}
\newcolumntype{C}[1]{>{\centering\let\newline\\\arraybackslash\hspace{0pt}}p{#1}}
\newcolumntype{R}[1]{>{\raggedleft\let\newline\\\arraybackslash\hspace{0pt}}p{#1}}

\usepackage{enumitem}
\setlist{leftmargin=*}
\setlist[1]{labelindent=\parindent}

\usepackage{hyperref}
%\usepackage{placeins} %PRovides \FloatBarrier to flush all floats before a certain point.
\usepackage{hanging}

\usepackage[sc]{titlesec}

%% Commands for Exam class
\renewcommand{\solutiontitle}{\noindent}
\unframedsolutions
\SolutionEmphasis{\bfseries}

\renewcommand{\questionshook}{%
	\setlength{\leftmargin}{-\leftskip}%
}

%Change \half command from 1/2 to .5
\renewcommand*\half{.5}

\pagestyle{headandfoot}
\firstpageheader{\textsc{bi}\,063 Evolution and Ecology}{}{\ifprintanswers\textbf{KEY}\else Name: \enspace \makebox[2.5in]{\hrulefill}\fi}
\runningheader{}{}{\footnotesize{pg. \thepage}}
\footer{}{}{}
\runningheadrule

\newcommand*\AnswerBox[2]{%
    \parbox[t][#1]{0.92\textwidth}{%
    \begin{solution}#2\end{solution}}
%    \vspace*{\stretch{1}}
}

\newenvironment{AnswerPage}[1]
    {\begin{minipage}[t][#1]{0.92\textwidth}%
    \begin{solution}}
    {\end{solution}\end{minipage}
    \vspace*{\stretch{1}}}

\newlength{\basespace}
\setlength{\basespace}{5\baselineskip}


%\usepackage{mdframed}
%\mdfsetup{%
%	innerleftmargin=0pt,%
%	innerrightmargin=0pt,
%	innertopmargin=0pt,
%	innerbottommargin=0pt,
%	hidealllines=true
%}%end mdfsetup

%
%\makeatletter
%\def\SetTotalwidth{\advance\linewidth by \@totalleftmargin
%\@totalleftmargin=0pt}
%\makeatother


%\printanswers


\begin{document}

\subsection*{Journal 2: evidence from transitional forms}

\emph{Read this entire handout throughly. You are responsible for meeting all requirements of this assignment, as detailed below.}

This is your second journal assignment. This and all other journal
assignments are formal writing assignments. This journal entry will
require some writing, as you need to \emph{explain} how the evidence
from transitional forms affects your hypothesis. Use your last
phylogenetic tree and look at how your hypothesis predicts the organisms
are related (or not). Your answer to this assignment should take at
least two typed pages (double space, please!), a probably a little more,
depending on your hypothesis.

You have now seen pieces of evidence for transitional forms in the
fossil record. Does it support or falsify your hypothesis? If it
supports your hypothesis, tell how. If it falsifies your hypothesis,
tell how, and tell how you would have to revise your hypothesis to fit
the evidence. Remember to include statements that clearly identify the
predictions made by your hypothesis and whether the evidence from
transitional forms agrees or disagrees with the predictions, and then
tell whether your hypothesis was supported or falsified.

The evidence from transitional forms that you have examined to test 
your hypothesis is listed below.  Also listed is the component point value 
for this assignment.

\begin{enumerate}


	\item Fish / amphibian transition (5 points).

	\item Reptile / mammal transition (5 points).
	
	We examined various skeletal structures, including the 
	bones of the lower jaw, the type of jaw joints, the bones of 
	the palate (roof of the mouth), and relative size and shape 
	of the teeth.

	\item Reptile / bird transition (5 points).

	\item Whale / even-toed hooved mammal transition (5 points).

	\item Hominid transitions (humans and other great apes (5 points).

	\item Geological Time Scale (5 points).
	
	Did your predictions about the time of appearance for each group 
	of organisms (mammals, reptiles, birds, etc.) agree or disagree with 
	the fossil record and the geological time scale?

\end{enumerate}

\emph{Important Note:} The fossil record also shows evidence of
transitional forms from amphibians to reptiles. We did not cover this
evidence in class because to understand the differences requires more
detail than we can cover in lab (although it may get covered in your lecture).
You can view some information at this website:

\url{http://www.talkorigins.org/faqs/faq-transitional/part1b.html}. 

But, please accept that this evidence exists, which will help you to
structure your hypothesis. See the geological time scale on the course
website for the timing of when amphibians first appeared and when
reptiles first appeared.

\subsubsection*{Assignment}

Before you begin, review all the assignments on transitional forms.  
Compare the evidence from the lab exercises to the predictions 
made by your phylogenetic trees. Does the evidence support your
hypothesis? Falsify your hypothesis? Give no conclusive evidence? 
Does the evidence require you to change your hypothesis in any way? 
Refer to the Scientific Method Overview for how to use evidence to test 
hypotheses.

Some parts of your hypothesis may be supported and
some parts may be falsified by the evidence from transitional forms.
You must discuss both supported and unsupported parts of your
hypothesis. For supported parts, you must discuss which parts of your
hypothesis are supported and how the evidence supports your hypothesis.
For falsified parts of your hypothesis, you must explain which parts are
not supported, how the evidence falsifies your hypothesis, and then
explain how your hypothesis must be revised to fit the evidence examined
so far. You need only address the vertebrates (animals with a
backbone) for the written journal entry. Your new phylogenetic tree (see
below) must still include all 21 organisms.

\emph{Clarity of your written thoughts is critical for your journal entry. Writing that is not clear is evidence of
 thinking that is not clear and will be evaluated accordingly.}


\subsubsection*{Grading of Journal 2}

\begin{enumerate}

\item Evaluation of the evidence (30 points).

Each type of transition in the fossil record that we evaluated in class
has been given a point value (see the evidence above). In addition, the proper order
of the transitions, based on the geological record, must also be
discussed. The points will be assigned based on inclusion and thorough
discussion of the evidence, and how the evidence affected your
hypothesis. Did the evidence support the predictions made by your
hypothesis? Did the evidence falsify your hypothesis? If the evidence
falsified your hypothesis, then you must state how you revised your
hypothesis to agree with the evidence. Clarity counts towards the total
points in each category.

\item Spelling, grammar, and mechanics (5 points).

Spelling, grammar and mechanics (sentence structure, etc) are 
important in any writing but especially informal writing. Use spell 
check. Proof your assignment carefully before you turn it in. You 
are entitled to \emph{one free mistake.} After that, each mistake 
is a 1 point deduction.

\emph{Remember}: This course is part of the University Studies
program. The University Studies program objectives that are especially
relevant to this course are 1) the demonstration of your ability for
critical thinking, reasoning and analyzing, and 2) the demonstration of
effective written communication. Keep this in mind as you write. Points will be deducted for signs of
weak reasoning and analysis and poorly written communication.

\item New phylogenetic tree (20 points).

You must also submit a revised phylogenetic tree with this assignment. 
Your new phylogenetic tree must be consistent with \emph{every} piece of
evidence evaluated, including the anatomical evidence and the transitional
form evidence. It must also be consist with what you write
for your journal entry assignment above. You will lose 3
points for each missed homology or transitional form, or transitions that do not
agree with the geological time scale. You will also lose 3 points for each missing organism,
a missing or incomplete time scale, or any other common errors. \emph{See the first phylogenetic tree exercise for things you should not do.}

\end{enumerate}

\subsubsection*{Due Date} 

Your journal entry and new phylogenetic tree are due at the start of lab next week. Late submissions will not be accepted without a valid excuse. If necessary, review the syllabus for lab policy.

Please do not type a cover page. Just put your name and \textsc{bi} 063 at the
top, and then start typing your journal assignment.

\end{document}  