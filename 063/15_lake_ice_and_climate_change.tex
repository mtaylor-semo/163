%!TEX TS-program = lualatex
%!TEX encoding = UTF-8 Unicode

\documentclass[12pt, hidelinks, addpoints]{exam}
\usepackage{graphicx}
	\graphicspath{{/Users/goby/Pictures/teach/163/lab/}
	{img/}} % set of paths to search for images

\usepackage{geometry}
\geometry{letterpaper, left=1.5in, bottom=1in}                   
%\geometry{landscape}                % Activate for for rotated page geometry
\usepackage[parfill]{parskip}    % Activate to begin paragraphs with an empty line rather than an indent
\usepackage{amssymb, amsmath}
\usepackage{mathtools}
	\everymath{\displaystyle}

\usepackage{fontspec}
\def\mainfont{Linux Libertine O}
\setmainfont[Ligatures={TeX}, BoldFont={* Bold}, ItalicFont={* Italic}, BoldItalicFont={* BoldItalic}, Numbers={Proportional, OldStyle}]{\mainfont}
\setsansfont[Scale=MatchLowercase,Ligatures=TeX, Numbers={Proportional,OldStyle}]{Linux Biolinum O}
\setmonofont{Linux Libertine O}
\newfontface{\lining}[Numbers=Lining]{\mainfont}
\usepackage{microtype}

\usepackage{unicode-math}
\setmathfont[Scale=MatchLowercase]{Tex Gyre Pagella Math}

% To define fonts for particular uses within a document. For example, 
% This sets the Libertine font to use tabular number format for tables.
 %\newfontfamily{\tablenumbers}[Numbers={Monospaced}]{Linux Libertine O}
% \newfontfamily{\libertinedisplay}{Linux Libertine Display O}

\usepackage{booktabs}
\usepackage{multicol}

\usepackage{caption}
\captionsetup{font=small} 
\captionsetup{singlelinecheck=false}
\captionsetup[figure]{labelsep=period, format=plain}

\usepackage{longtable}
%\usepackage{siunitx}
\usepackage{array}
\newcolumntype{L}[1]{>{\raggedright\let\newline\\\arraybackslash\hspace{0pt}}p{#1}}
\newcolumntype{C}[1]{>{\centering\let\newline\\\arraybackslash\hspace{0pt}}p{#1}}
\newcolumntype{R}[1]{>{\raggedleft\let\newline\\\arraybackslash\hspace{0pt}}p{#1}}

\usepackage{enumitem}
\setlist{leftmargin=*}
\setlist[1]{labelindent=\parindent}
\setlist[enumerate]{label=\textsc{\alph*}.}
\setlist[itemize]{label=\color{gray}\textbullet}

\usepackage[hyphens]{url}
%\usepackage{hyperref}
%\usepackage{placeins} %PRovides \FloatBarrier to flush all floats before a certain point.
\usepackage{hanging}

\usepackage[sc]{titlesec}

%% Commands for Exam class
\renewcommand{\solutiontitle}{\noindent}
\unframedsolutions
\SolutionEmphasis{\bfseries}

\renewcommand{\questionshook}{%
	\setlength{\leftmargin}{-\leftskip}%
}

%Change \half command from 1/2 to .5
\renewcommand*\half{.5}

\pagestyle{headandfoot}
\firstpageheader{\textsc{bi}\,063 Evolution and Ecology}{}{\ifprintanswers\textbf{KEY}\else Name: \enspace \makebox[2.5in]{\hrulefill}\fi}
\runningheader{}{}{\footnotesize{pg. \thepage}}
\footer{}{}{}
\runningheadrule

\newcommand*\AnswerBox[2]{%
    \parbox[t][#1]{0.92\textwidth}{%
    \begin{solution}#2\end{solution}}
    \vspace*{\stretch{1}}
}

\newenvironment{AnswerPage}[1]
    {\begin{minipage}[t][#1]{0.92\textwidth}%
    \begin{solution}}
    {\end{solution}\end{minipage}
    \vspace*{\stretch{1}}}

\newlength{\basespace}
\setlength{\basespace}{5\baselineskip}

\newcommand{\hidepoints}{%
	\pointsinmargin\pointformat{}
}

\newcommand{\showpoints}{%
	\nopointsinmargin\pointformat{(\thepoints)}
}

\newcommand\chisq{$\chi^2$}

%
%\makeatletter
%\def\SetTotalwidth{\advance\linewidth by \@totalleftmargin
%\@totalleftmargin=0pt}
%\makeatother


%\printanswers


\begin{document}

\hidepoints

\subsection*{Changes in lake ice: ecosystem response to climate
change\footnote{\raggedright Data and modified lab instructions from: http://www.esa.org/tiee/vol/v3/issues/data\_sets/ lake\_ice/overview.html.} (\numpoints\ points)}

Climate observations have been recorded regularly around the world. Many of
these records date several centuries. Recent
publications have used data back to the mid 1800s from lakes and rivers in the northern
hemisphere. The data used in this exercise
are the dates of fall ``ice-on'' (the initial formation of ice cover),
spring ``ice-off'' (the break-up of winter ice cover), and duration of ice
cover on Wisconsin's lakes Mendota, Monona, and Wingra, which are part
of the North Temperate Lakes Long-Term Ecological Research site (Assel
and Robertson 1995).

Magnuson et al.~(2000) looked at river ice cover data from 39 locations
in the northern hemisphere including sites in Russia, Finland, Japan,
and the United States. The authors concluded that average rate of change 
in freeze dates over the 150-year period from 1846–1995 was 5.8 days per 100
years later and that change in breakup averaged 6.5 days per 100 years
earlier. These changes translate into increasing air temperature of
about {\lining1.2}°C per 100 years.

\subsubsection*{Long-Term Ecological Research}

In the 1970s, ecologists realized that many ecological questions could
not be answered by individual or small groups of scientists conducting
short-term research. Therefore, in 1980 the National Science Foundation
established the Long-Term Ecological Research (\textsc{lter}) network to support
research on long-term ecological phenomena in the U.S. As of 2004, the
\textsc{lter} Network includes 26 sites representing different biomes across the
U.S. and Antarctica (\url{http://www.lternet.edu/}).

The vision of the North Temperate Lakes \textsc{lter} is ``\dots to gain a
predictive understanding of the ecology of lakes at longer and broader
scales than has been traditional in limnology. Thus, we analyze and
interpret data we collect over long periods on suites of lakes''
(\url{http://lter.limnology.wisc.edu}).

\subsubsection*{Introduction}

Global warming is clearly taking place. Global temperatures
have increased by about {\lining0.8}°C ({\lining1.5}°F) during the last century,
most likely the result of ``greenhouse gases,'' such as carbon dioxide
from burning of gasoline, oil, and coal. This may not seem like a
much of a temperature change for 100 years but, in comparison, 
the largest temperature changes recorded during the last 5 million years were 
{\lining0.016–0.008}°C per 100 years. The current rate of temperature change is 
50–100$\times$ faster that any temperature change recorded in 5 million years.

There are many environmental consequences of warmer temperatures, some
unexpected. In Alaska, for instance, warmer weather allows the spruce
bark beetle to complete its normally two-year life cycle in just one
year; the result is millions of acres of spruce forest killed by the
beetle. As another example, mosquitoes carrying diseases have spread to
areas where they have never before been recorded.

One challenge to our understanding of environmental effects due to
global warming is lack of data collected over long periods of time. The
data from lakes in Wisconsin that you will work with is very unusual
because it spans \textasciitilde{}160 years.

\subsubsection*{Lake ice records for Lake Mendota}

You will work with data that include 1) the duration of ice cover, 2) dates
of spring ``ice-off'' (the break-up of winter ice cover) and 3) dates of
``ice-on'' for Wisconsin's Lake Mendota, which is part of the North
Temperate Lakes Long-Term Ecological Research site. Each of these
measures may provide different types of evidence related to global
change.

\begin{enumerate}

\item Download lake\_ice.xlsx from \url{http://mtaylor4.semo.edu/~goby/bi163/}. This spreadsheet includes the
ice data collected at Lake Mendota over a 160-year period.

\item Look at the heading at the top of the each column to make sure you
understand each one.

\item Take a look at the first row of data, for the winter of 1855–56. In
this winter, the ice froze on December 18 and melted on April 14. So the
"Ice Duration" was from Dec. 18 to April 14, a total of 118 days.

\item Notice that, in addition to being expressed as dates, ``Ice On'' and
``Ice Off'' are also expressed as numerals, the number of days since
January 1. For example, look in the sixth column. The ``Ice Off (Day of
Year)'' for 1855 is 105 (January 1, 1856 to April 14, 1856). Finally,
notice that the ``Ice On'' date for some years (e.g., 1931) is greater
than 365. That's because the ice on the lake did not form until after
the end of the year (e.g., January 31).
\end{enumerate}

In your group, discuss the three measures from the data sets and explain
what evidence related to global change each of these may provide. Briefly 
summarize your discussion on the next page.  For
example, ice-off data are especially useful for assessing long-term
trends because they integrate air temperature over many days. Therefore,
this single data point actually expresses the cumulative effects of
local weather conditions over the winter season. 

Ice on (day of the year):\vspace*{2\baselineskip}

Ice off (day of the year):\vspace*{2\baselineskip}

Ice duration (days):\vspace*{2\baselineskip}


\begin{questions}

\question[1]
Which of the three types of data do you think will give you the most useful
information or the clearest evidence of a trend? What is your hypothesis
for this data set—what do you expect to find? Make a sketch of the
pattern you predict to see.

\vspace*{14\baselineskip}

%\AnswerBox{0.1\baselineskip}{Draw a graph here.}

\textbf{Stop here.} Wait for a class discussion and to be assigned your data to graph.
\subsubsection*{Create a line graph of your data}

Each group will work with a short-term data set of \textasciitilde{}25–30
years; your teacher will tell you which years your group will graph.
Graph your years of data.

\begin{enumerate}

\item In the Excel file, click on the ``Insert'' tab. Click on ``Line'' to
choose a line graph and select ``Line with markers''.

\item Now, in the ``Chart Tools'' tab, click on ``Select data''. Under
Legend series, select ``Add''.

\item Click on the bottom blue box with the red arrow for ``Series
values''. Select/highlight the data for your \textsc{y}-axis: the years of Ice
Duration for \emph{only} the time period you were given. Click ``OK''.

\item Now select the data for your y-axis. In the box that says
``Horizontal (Category) Axis Labels'', click on Edit. Click on the blue
box with the red arrow for ``Axis label range''. Select/highlight the
Years you were given that match the \textsc{y}-axis data. Click ``OK''. Click
``OK'' again on the ``Select data source'' box to view your graph.

\item Label your axes by selecting the ``Layout'' tab at the top of the
screen, click ``Axis titles'' and label your horizontal (\textsc{x}) and vertical
(\textsc{y}) axes.

\end{enumerate}

\question[1]
How many years did you look at in your data set? What is the
\emph{average} ice duration in your data set? What is the shortest
period of ice duration in your data set (minimum)? Longest (maximum)? In
which year did each occur?

\AnswerBox{3\baselineskip}{Answer Here}

\question[1]
Is there much variability from year to year, or only a little?

\AnswerBox{3\baselineskip}{Answer Here}

\question[1]
What trend do you see? As time elapses, does the value tend to increase,
decrease, fluctuate, or stay the same?

\AnswerBox{3\baselineskip}{Answer Here}

\question[1]
Pair up with one other group (preferably the group across the table)
and compare your results. What was their time period? Compare your
average, minimum, and maximum durations and record below. Did you reach
the same or different conclusions based on your data set?

\AnswerBox{3\baselineskip}{Answer Here}


\newpage

\subsubsection*{Compare your graph to the long term data set (Questions \ref{q:compare_first_question}–\ref{q:compare_last_question})}

\question[1]\label{q:compare_first_question}
What is the average ice duration for the entire dataset? How does
this compare to your average from your data set?

\AnswerBox{1\baselineskip}{Answer Here}

Insert a line on your graph showing the average ice duration for the
\emph{entire} data set. 

\begin{enumerate}
\item Click on your graph. Choose the ``Insert'' tab from the top of the
page.

\item Under ``Shapes,'' choose the straight line.

\item Go to your graph and draw a straight line, parallel with the \textsc{x}-axis,
at the correct position on the \textsc{y}-axis (average ice duration for entire
data set).
\end{enumerate}

\vspace*{\baselineskip}

\question[1]\label{q:compare_middle_question}
How many years on your graph had shorter than average ice duration? \rule{1.5cm}{0.4pt} \\[1.5\baselineskip]
How many years on your graph had longer than average ice duration? \rule{1.5cm}{0.4pt}

%\AnswerBox{3\baselineskip}{Answer Here}

\emph{Record your values on the table drawn by your instructor on the
board.}

\question[1]\label{q:compare_last_question}
Compare the values (from \#\ref{q:compare_middle_question}) among all of the groups. Do you see a
trend in years with longer or shorter than average ice duration over
time? If so, describe the trend that you see.

\AnswerBox{1\baselineskip}{Answer Here}

\subsubsection*{Conclusions (Questions \ref{q:conclusions_first_question}–\ref{q:conclusions_last_question})}

Use your short-term data set and
the complete \textasciitilde{}160 year data set compiled by your
instructor to answer the following.

\question[1]\label{q:conclusions_first_question}
What overall trend in ice duration is shown by the long term data
set?

\AnswerBox{3\baselineskip}{The long-term data set shows a clear downward trend for ice duration. Ice remains on the lake for fewer days in recent years compared to farther back in time.}

\question[1]
Based on what you have just done, explain why using a long term data
set to discuss evidence of climate change is more useful than at a short
term data set?

\AnswerBox{3\baselineskip}{Answer Here}

\question[1]
What can short term data sets be used for with regard to
understanding climate change?

\AnswerBox{3\baselineskip}{Answer Here}

\question[1]\label{q:conclusions_last_question}
How does this data on ice cover provide evidence for global change?
What other types of data might be useful in strengthening your argument?
Be specific.

\AnswerBox{3\baselineskip}{%
Open-ended%
}
\end{questions}


\end{document}  