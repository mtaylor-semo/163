%!TEX TS-program = lualatex
%!TEX encoding = UTF-8 Unicode

\documentclass[12pt, hidelinks]{exam}

%\printanswers

\usepackage{graphicx}
	\graphicspath{{/Users/goby/Pictures/teach/163/lab/}
	{img/}} % set of paths to search for images

\usepackage{geometry}
\geometry{letterpaper, left=1.5in, bottom=1in}                   
%\geometry{landscape}                % Activate for for rotated page geometry
\usepackage[parfill]{parskip}    % Activate to begin paragraphs with an empty line rather than an indent
\usepackage{amssymb, amsmath}
\usepackage{mathtools}
	\everymath{\displaystyle}

\usepackage{fontspec}
\def\mainfont{Linux Libertine O}
\setmainfont[Ligatures={TeX}, BoldFont={* Bold}, ItalicFont={* Italic}, BoldItalicFont={* BoldItalic}, Numbers={Proportional, OldStyle}]{\mainfont}
\setsansfont[Scale=MatchLowercase,Ligatures=TeX, Numbers={Proportional,OldStyle}]{Linux Biolinum O}
\setmonofont{Linux Libertine O}
\newfontface{\lining}[Numbers=Lining]{\mainfont}
\usepackage{microtype}

\usepackage{unicode-math}
\setmathfont[Scale=MatchLowercase]{Tex Gyre Pagella Math}

% To define fonts for particular uses within a document. For example, 
% This sets the Libertine font to use tabular number format for tables.
 %\newfontfamily{\tablenumbers}[Numbers={Monospaced}]{Linux Libertine O}
% \newfontfamily{\libertinedisplay}{Linux Libertine Display O}

\usepackage{booktabs}
\usepackage{multicol}

\usepackage{caption}
\captionsetup{font=small} 
\captionsetup{singlelinecheck=false}
\captionsetup[figure]{labelsep=period, format=plain}

\usepackage{longtable}
%\usepackage{siunitx}
\usepackage{array}
\newcolumntype{L}[1]{>{\raggedright\let\newline\\\arraybackslash\hspace{0pt}}p{#1}}
\newcolumntype{C}[1]{>{\centering\let\newline\\\arraybackslash\hspace{0pt}}p{#1}}
\newcolumntype{R}[1]{>{\raggedleft\let\newline\\\arraybackslash\hspace{0pt}}p{#1}}

\usepackage{enumitem}
\setlist{leftmargin=*}
\setlist[1]{labelindent=\parindent}
\setlist[enumerate]{label=\textsc{\alph*}.}
\setlist[itemize]{label=\color{gray}\textbullet}

%\usepackage[hyphens]{url}
\usepackage{hyperref}
%\usepackage{placeins} %PRovides \FloatBarrier to flush all floats before a certain point.
\usepackage{hanging}

\usepackage[sc]{titlesec}

%% Commands for Exam class
\renewcommand{\solutiontitle}{\noindent}
\unframedsolutions
\SolutionEmphasis{\bfseries}

\renewcommand{\questionshook}{%
	\setlength{\leftmargin}{-\leftskip}%
}

%Change \half command from 1/2 to .5
\renewcommand*\half{.5}

\pagestyle{headandfoot}
\firstpageheader{\textsc{bi}\,063 Evolution and Ecology}{}{\ifprintanswers\textbf{KEY}\else Name: \enspace \makebox[2.5in]{\hrulefill}\fi}
\runningheader{}{}{\footnotesize{pg. \thepage}}
\footer{}{}{}
\runningheadrule

%\parbox[t][2\basespace][t]{0.92\textwidth}{\ifprintanswers{\bfseries The duration of ice cover should become smaller.}\fi}

\newcommand*\AnswerBox[2]{%
	\parbox[t][#1][t]{0.92\textwidth}{%
		\ifprintanswers{\bfseries #2}\fi}
	\vspace*{\stretch{1}}\par
}

\newenvironment{AnswerPage}[1]
    {\begin{minipage}[t][#1]{0.92\textwidth}%
    \begin{solution}}
    {\end{solution}\end{minipage}
    \vspace*{\stretch{1}}}

\newlength{\basespace}
\setlength{\basespace}{5\baselineskip}

\newcommand*{\subsec}[1]{\medskip \bigskip {\scshape #1} \medskip}



\begin{document}

\subsection*{Changes in lake ice: ecosystem response to climate
change\footnote{\raggedright Data and modified lab instructions from Bohanan et al. 2005. Changes in lake ice: ecosystem response to global change. Teaching Issues and Experiments in Ecology 3. \href{http://www.esa.org/tiee/vol/v3/issues/data_sets/lake_ice/abstract.html}{[online]}}}

Climate observations have been recorded regularly around the world. Many of
these records date back several centuries. The results of these observations
show that global temperatures
have increased by about {\lining0.8}°C ({\lining1.5}°F) during the last century,
most likely the result of ``greenhouse gases,'' such as carbon dioxide,
from burning of gasoline, oil, and coal. This may not seem like a
much of a temperature change for 100 years but, in comparison, 
the largest temperature changes recorded during the last 5 million years were 
{\lining0.008–0.016}°C per century. The current rate of temperature increase is 
50–100$\times$ faster that any temperature change recorded in 5 million years.

Rapid warming has many environmental consequences of warmer temperatures, some
unexpected. In Alaska, for instance, warmer weather allows the spruce
bark beetle to complete its normally two-year life cycle in just one
year; the result is millions of acres of spruce forest killed by the
beetle. As another example, mosquitoes carrying diseases have spread to
areas where they have never before been recorded.


Questions about the long-term effects of climate change are difficult
to answer by individual or small groups of scientists conducting
short-term research. Therefore, in 1980 the National Science Foundation
established the Long-Term Ecological Research (\textsc{lter}) network to support
research on long-term ecological phenomena in the U.S. 
As of 2004, the network includes 26 sites representing different biomes across the
U.S. and Antarctica (\url{http://www.lternet.edu/}).

One of the sites is the The North Temperate Lakes \textsc{lter},
which monitors 11 temperate lakes across Wisconsin. Their goal  is “\dots to gain a
predictive understanding of the ecology of lakes at longer and broader
scales than has been traditional in studies of lakes. Thus, we analyze and
interpret data we collect over long periods on suites of lakes”
(\url{http://lter.limnology.wisc.edu}). The data collected
from the Wisconsin lakes is very unusual
because it spans about 160 years.

%The data you will use in this exercise
%are the dates of fall ``ice-on'' (the initial formation of ice cover),
%spring ``ice-off'' (the break-up of winter ice cover), and duration of ice
%cover on Wisconsin's lakes Mendota, Monona, and Wingra, which are part
%of the North Temperate Lakes Long-Term Ecological Research site (Assel
%and Robertson 1995).

%\subsubsection*{Introduction}
%
%Magnuson et al.~(2000) looked at river ice cover data from 39 locations
%in the northern hemisphere including sites in Russia, Finland, Japan,
%and the United States. The authors concluded that average rate of change 
%in freeze dates over the 150-year period from 1846–1995 was 5.8 days per 100
%years later and that change in breakup averaged 6.5 days per 100 years
%earlier. These changes translate into increasing air temperature of
%about {\lining1.2}°C per 100 years.
%

%Recent publications have used data back to the mid 1800s from lakes and rivers in the northern
%hemisphere. 


\subsec{The data: ice records for Lake Mendota}

You will work with data collected from Wisconsin's Lake Mendota, which is part of the North
Temperate Lakes Long-Term Ecological Research site. The data
 include the dates of spring “ice-on” (the initial formation of ice cover), the dates of
“ice-off” (the break-up of winter ice cover), and the duration of ice cover (the number of days with ice cover on the lake). Each of these
measures may provide different types of evidence related to global
climate change.

\newpage

\subsec{Instructions}

\begin{enumerate}

\item Download lake\_ice.xlsx from \url{http://mtaylor4.semo.edu/~goby/bi163/}. This spreadsheet includes the
ice data collected at Lake Mendota over a 160-year period.

\item Look at the heading at the top of the each column to make sure you
understand each one.

\item Take a look at the first row of data, for the winter of 1855–56. In
this winter, the ice formed on December 18 and melted on April 14. So the
"Ice Duration" was from December 18 to April 14, a total of 118 days.

\item Notice that, in addition to being expressed as dates, ``Ice On'' and
``Ice Off'' are also expressed as numerals, the number of days since
January 1. For example, look in the fourth column. The ``Ice Off (Day of
Year)'' for 1855 is 105 (January 1, 1856 to April 14, 1856). Finally,
notice that the ``Ice On'' date for some years (e.g., 1931) is greater
than 365. That's because the ice on the lake did not form until after
the end of the year (e.g., January 31).
\end{enumerate}

\begin{questions}

\question
In your group, discuss the three measures from the data sets and explain
what evidence related to global change each of these may provide. 
% Briefly summarize your discussion on the next page.  
For example, ice-off data are especially useful for assessing long-term
trends because they integrate air temperature over many days. Therefore,
this single data point actually expresses the cumulative effects of
local weather conditions over the winter season. 

\bigskip

Which one measurement (ice-on, ice-off,or  duration) does your group think may be best for detecting signs of climate change? Explain why.

\AnswerBox{0.6\basespace}{Any answer acceptable.}

\textbf{{\scshape Stop here.}} Wait for a class discussion and to be assigned your data to graph.


\question \label{ques:hypothesis}
Write a hypothesis that states how the chosen measurement should change over time if mean global temperatures are increasing.
	
\AnswerBox{0.6\basespace}{Climate change/increasing temperatures will cause the duration of ice cover to decrease over time.}
	

\question
Sketch a graph of the pattern you predict to see for the chosen variable changing over time.

\AnswerBox{3\basespace}{The  duration of ice cover should decrease over time. Their graph should have years/time on \textsc{x}-axis and duration of ice cover on the \textsc{y}-axis. Line should show a negative trend.}

\subsec{Create a line graph of your data}

Each group will work with a short-term data set of about 25–30
years; your instructor will tell you which years your group will graph. 
Graph your years of data.

\begin{enumerate}

\item In the Excel file, click on the ``Insert'' tab. Click on ``Line'' to
choose a line graph and select ``Line with markers''.

\item Now, in the ``Chart Tools'' tab, click on ``Select data''. Under
Legend series, select ``Add''.

\item Click on the bottom blue box with the red arrow for ``Series
values''. Select/highlight the data for your \textsc{y}-axis: the years of Ice
Duration for \emph{only} the time period you were given. Click ``OK''.

\item Select the data for your \textsc{x}-axis. In the box that says
``Horizontal (Category) Axis Labels'', click on Edit. Click on the blue
box with the red arrow for ``Axis label range''. Select/highlight the
Years you were given that match the \textsc{y}-axis data. Click ``OK''. Click
``OK'' again on the ``Select data source'' box to view your graph.

\item Label your axes by selecting the ``Layout'' tab at the top of the
screen, click ``Axis titles'' and label your horizontal (\textsc{x}) and vertical
(\textsc{y}) axes.

\end{enumerate}

%% In future, consider teaching them MIN and MAX functions.
\question
How many years did you look at in your data set? What is the
\emph{average} ice duration in your data set? What is the shortest
period of ice duration in your data set (minimum)? Longest (maximum)? In
which year did each occur?

\AnswerBox{0.6\basespace}{Answer depends on data set plotted.}

\question
Is there much variability from year to year, or only a little?

\AnswerBox{0.6\basespace}{Answer depends on data set plotted.}

\question
What trend do you see? As time elapses, does the value tend to increase,
decrease, fluctuate, or stay the same?

\AnswerBox{0.6\basespace}{Answer depends on data set plotted.}

\question
Pair up with one other group (preferably the group across the table)
and compare your results. What was their time period? Compare your
average, minimum, and maximum durations and record below. Did you reach
the same or different conclusions based on your data set?

\AnswerBox{0.6\basespace}{Answer depends on data sets plotted by each group.}


\newpage

\subsec{Compare your graph to the long term data set}

\question \label{q:compare_first_question}
What is the average ice duration for the \textit{entire} dataset? How does
this compare to the average from your data set?

\AnswerBox{0.6\basespace}{Mean ice duration is 103 days. Their comparison will vary. Earlier years will tend to have longer durations. Recent years will tend to have shorter durations.}

Insert a line on your graph showing the average ice duration for the
\emph{entire} data set. 

\begin{enumerate}
\item Click on your graph. Choose the ``Insert'' tab from the top of the
page.

\item Under ``Shapes,'' choose the straight line.

\item Go to your graph and draw a straight line, parallel with the \textsc{x}-axis,
at the correct position on the \textsc{y}-axis (average ice duration for entire
data set).
\end{enumerate}

\bigskip

\question \label{q:compare_middle_question}
How many years in \textit{your} 27-year data set had shorter ice duration than the long-term average of the entire data set? \bigskip %\rule{1.5cm}{0.4pt} \bigskip

How many years in \textit{your} 27-year data set had longer ice duration than the long-term average of the entire data set? \bigskip %\rule{1.5cm}{0.4pt}

\emph{Record your values on the table drawn by your instructor on the
board.}

\question \label{q:compare_last_question}
Compare the values from question~\ref{q:compare_middle_question} among all of the groups. Do you see a
trend in years with longer or shorter than average ice duration over
time? If so, describe the trend that you see.

\AnswerBox{0.6\basespace}{Older dates will show more days with above average ice duration. More recent years will show more days with lower than average ice duration.}

\subsec{Conclusions}

Use your short-term data set and
the full 160-year data set to answer the following.

\question \label{q:conclusions_first_question}
What overall trend in ice duration is shown by the long term data
set?

\AnswerBox{0.6\basespace}{The long-term data set shows a clear downward trend for ice duration. Ice remains on the lake for fewer days in recent years compared to farther back in time.}

\question
Do your results combined with the class results support or falsify your hypothesis (question~\ref{ques:hypothesis}? Explain.

\AnswerBox{\basespace}{Supports if student hypothesized shorter duration.}

\question
Based on what you have just done, explain why using a long term data
set to discuss evidence of climate change is more useful than at a short
term data set?

\AnswerBox{0.6\basespace}{Long term data sets provide a better representation of overall rate of change and trends, reducing effects of short-term variability. In essence, larger data set.}

\question
What can short term data sets be used for with regard to
understanding climate change?

\AnswerBox{0.6\basespace}{Short term data can be used to compare differences among years, such as within decade variation. For example, are differences from year to year small and consistent, large and consistent, or highly variable?}

\question[Checkout] \label{q:conclusions_last_question}
How does this data on ice cover provide evidence for global change?
What other types of data might be useful in strengthening your argument?
Be specific.

\AnswerBox{0.6\basespace}{%
Open-ended%
}
\end{questions}

\vspace*{\stretch{3}}

\end{document}  