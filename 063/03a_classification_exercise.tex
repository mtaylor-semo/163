%!TEX TS-program = lualatex
%!TEX encoding = UTF-8 Unicode

\documentclass[12pt]{exam}
\usepackage{graphicx}
	\graphicspath{{/Users/goby/Pictures/teach/163/lab/}} % set of paths to search for images

\usepackage{geometry}
\geometry{letterpaper, bottom=1in}                   

\newlength{\myindent}
\setlength{\myindent}{\parindent}
\newcommand{\ind}{\hspace*{\myindent}}


%\geometry{landscape}                % Activate for for rotated page geometry
\usepackage[parfill]{parskip}    % Activate to begin paragraphs with an empty line rather than an indent
%\usepackage{amssymb, amsmath}
%\usepackage{mathtools}
%	\everymath{\displaystyle}

\usepackage{fontspec}
\setmainfont[Ligatures={TeX}, BoldFont={* Bold}, ItalicFont={* Italic}, BoldItalicFont={* BoldItalic}, Numbers={OldStyle,Proportional}]{Linux Libertine O}
\setsansfont[Scale=MatchLowercase,Ligatures=TeX, Numbers=OldStyle]{Linux Biolinum O}
\setmonofont[Scale=0.8]{Linux Libertine Mono O}
\usepackage{microtype}

%\usepackage{unicode-math}
%\setmathfont[Scale=MatchLowercase]{Asana Math}
%\setmathfont[Scale=MatchLowercase]{XITS Math}

% To define fonts for particular uses within a document. For example, 
% This sets the Libertine font to use tabular number format for tables.
\newfontfamily{\tablenumbers}[Numbers={Monospaced}]{Linux Libertine O}
\newfontfamily{\libertinedisplay}{Linux Libertine Display O}

\usepackage{booktabs}
\usepackage{multicol}

\usepackage[normalem]{ulem}

%\usepackage{tabularx}
\usepackage{longtable}
%\usepackage{siunitx}
\usepackage{array}
\newcolumntype{L}[1]{>{\raggedright\let\newline\\\arraybackslash\hspace{0pt}}p{#1}}
\newcolumntype{C}[1]{>{\centering\let\newline\\\arraybackslash\hspace{0pt}}p{#1}}
\newcolumntype{R}[1]{>{\raggedleft\let\newline\\\arraybackslash\hspace{0pt}}p{#1}}

\usepackage{enumitem}
\setlist{leftmargin=*}
\setlist[1]{labelindent=\parindent}
\setlist[enumerate]{label=\textsc{\alph*}.}

\usepackage{hyperref}
%\usepackage{hanging}

\usepackage[sc]{titlesec}


\renewcommand{\solutiontitle}{\noindent}
\unframedsolutions
\SolutionEmphasis{\bfseries}

\renewcommand{\questionshook}{%
	\setlength{\leftmargin}{-\leftskip}%
}
%Change \half command from 1/2 to .5
%\renewcommand*\half{.5}


\makeatletter
\def\SetTotalwidth{\advance\linewidth by \@totalleftmargin
\@totalleftmargin=0pt}
\makeatother



\pagestyle{headandfoot}
\firstpageheader{BI 063: Evolution and Ecology}{}{\ifprintanswers\textbf{KEY}\else Name: \enspace \makebox[2.5in]{\hrulefill}\fi}
\runningheader{}{}{\footnotesize{pg. \thepage}}
\footer{}{}{}
\runningheadrule

\newcommand*\AnswerBox[2]{%
    \parbox[t][#1]{0.92\textwidth}{%
    \begin{solution}#2\end{solution}}
    \vspace*{\stretch{1}}
}

\newenvironment{AnswerPage}[1]
    {\begin{minipage}[t][#1]{0.92\textwidth}%
    \begin{solution}}
    {\end{solution}\end{minipage}
    \vspace*{\stretch{1}}}

\newlength{\basespace}
\setlength{\basespace}{5\baselineskip}


%\printanswers

\begin{document}

\subsection*{Classification: grouping by similarity }

In 1735, Carolus Linnaeus developed a hierarchical scheme to classify 
all living organisms, grouped by increasing levels of similarity. Linnaeus 
used seven taxonomic ranks in his original hierarchy. A new rank, the Domain, 
was added in 1990.  The eight ranks, listed from most inclusive to the most 
specific, are

\ind Domain, Kingdom, Phylum, Class, Order, Family, Genus, Species.

For this exercise, you will work in groups of 2–4 students (as directed by 
your instructor) to classify “species” of nails, screws, bolts, and other 
hardware fasteners. You will classify the hardware using whatever characteristics 
you wish. Characteristics can include color, shape, the presence or absence 
of threads on the shaft, hardware type (e.g., nail, screw, bolt), type 
of head (e.g., slotted or phillips) and so on. 

You must follow only a few rules.

\begin{enumerate}

	\item Begin with one class (skip higher ranks).
	
	\item The class must have two two orders. Each order must have two
	families, each family must have two genera (plural of genus), and 
	each genus must have two species.
	
	\item The species rank can have only one species (one hardware item). 

	\item Make a descriptive name for each rank. A good name might be 
	descriptive of the character(s) you used to group items in a 
	particular rank.

	\item For each rank, write the character(s) that you used to separate 
	each rank. For example, you may have two orders that separate screws 
	and bolts, or two families that separate silver and brass items.
	
%	\item You must finish with at least two groups for your highest rank (most inclusive).
	
%	\item You must finish with 6–8 taxonomic ranks, including the species rank.
	
%	\item Every higher rank (domain through genus) must have at least one species. A phylum, for example, can have only one species. Every lower rank included in that phylum would have just that one species.
	
%	\item Each rank must include all lower ranks. That is, a family must include at least one genus and one species rank. An order must include at least one family, one genus, and one species rank.
	
%	\item You are not limited to the number of groups for each taxonomic rank. For example, you may use 3, 4, 7, or some other number of  phyla (plural of phylum).
	
%	\item Make a descriptive name (1-2) for each rank. A good name might be descriptive of the character(s) you used to group items in a particular rank.

\end{enumerate}

\bigskip

Here is an example that uses shapes with four sides. 
The example is not complete for all ranks shown but your classification 
must be complete for class to species.

Class Foursides	\\
%Domain Foursides\\
\ind Order 90degrees — sides meet at 90 degree angles.\\
\ind	\ind Family Rectangle — only opposite sides of equal length.\\
\ind	\ind	\ind Genus FilledRectangle — rectangle is filled with a solid color that is not white.\\
\ind	\ind	\ind Genus HollowRectangle  — rectangle is not filled or is white inside. \\
\ind	\ind Family Square — all four sides of equal length.\\
\ind	\ind	\ind Genus FilledSquare — square is filled with a solid color that is not white.\\ 
\ind	\ind	\ind Genus HollowSquare — square is not filled or is white inside. \\
\ind Order Not90degrees\\
\ind	\ind Family Trapezoid\\
\ind	\ind Family Rhomboid \\
\ind	\ind	\ind etc.



\begin{questions}

\question
Write or draw your classification scheme below. For each rank above species, list the
features you used to group similar hardware into the same rank. Or, list what differences
you used to separate hardware into different ranks (i.e., species, genus, family, order)?

Your group may be called on to write part of your classification scheme on the board at 
the front of the classroom. Be prepared.

\end{questions}

%\newpage

	
%	\medskip
	
%Domain Threesides\\
%	\ind Kingdom Right\\
%	\ind Kingdom Isoceles
	
%	\medskip
%	
%Domain Nosides\\
%	\ind Kingdom Oval\\
%	\ind Kingdom Circle\\


\end{document}  