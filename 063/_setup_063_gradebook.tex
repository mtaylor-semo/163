%!TEX TS-program = lualatex
%!TEX encoding = UTF-8 Unicode

\documentclass[12pt]{article}


%\printanswers


\usepackage{graphicx}
	\graphicspath{{/Users/goby/Pictures/teach/163/ta_how_to/}
	{img/}} % set of paths to search for images

\usepackage{geometry}
\geometry{letterpaper, left=1.5in, bottom=1in}                   
%\geometry{landscape}                % Activate for for rotated page geometry
\usepackage[parfill]{parskip}    % Activate to begin paragraphs with an empty line rather than an indent
\usepackage{amssymb, amsmath}
\usepackage{mathtools}
	\everymath{\displaystyle}

\usepackage[table]{xcolor}

\usepackage{fontspec}
\setmainfont[Ligatures={TeX}, BoldFont={* Bold}, ItalicFont={* Italic}, BoldItalicFont={* BoldItalic}, Numbers={OldStyle}]{Linux Libertine O}
\setsansfont[Scale=MatchLowercase,Ligatures=TeX]{Linux Biolinum O}
\setmonofont[Scale=MatchLowercase]{Inconsolatazi4}
\newfontfamily{\tablenumbers}[Numbers={Monospaced,Lining}]{Linux Libertine O}
\usepackage{microtype}

%\usepackage{bm}

% To define fonts for particular uses within a document. For example, 
% This sets the Libertine font to use tabular number format for tables.
 %\newfontfamily{\tablenumbers}[Numbers={Monospaced}]{Linux Libertine O}
% \newfontfamily{\libertinedisplay}{Linux Libertine Display O}

\usepackage{multicol}
%\usepackage[normalem]{ulem}

\usepackage{longtable}
\usepackage{caption}
	\captionsetup{format=plain, justification=raggedright, singlelinecheck=off,labelsep=period,skip=3pt} % Removes colon following figure / table number.
%\usepackage{siunitx}
\usepackage{booktabs}
\usepackage{array}
\newcolumntype{L}[1]{>{\raggedright\let\newline\\\arraybackslash\hspace{0pt}}m{#1}}
\newcolumntype{C}[1]{>{\centering\let\newline\\\arraybackslash\hspace{0pt}}m{#1}}
\newcolumntype{R}[1]{>{\raggedleft\let\newline\\\arraybackslash\hspace{0pt}}m{#1}}

\usepackage{enumitem}
\setlist{leftmargin=*}
\setlist[1]{labelindent=\parindent}
\setlist[enumerate]{label=\textsc{\alph*}.}
\setlist[itemize]{label=\color{gray}\textbullet}
%\usepackage{hyperref}
%\usepackage{placeins} %PRovides \FloatBarrier to flush all floats before a certain point.
%\usepackage{hanging}

\usepackage[sc]{titlesec}

%% Commands for Exam class

%\pagestyle{headandfoot}
%\firstpageheader{\textsc{bi}\,063 Evolution and Ecology}{}{\ifprintanswers\textbf{KEY}\else Name: \enspace \makebox[2.5in]{\hrulefill}\fi}
%\runningheader{}{}{\footnotesize{pg. \thepage}}
%\footer{}{}{}
%\runningheadrule


\begin{document}

\subsection*{How to set up the grade book for {\scshape bi} 063}

\begin{enumerate}
	\item Login to Moodle page and choose one of your lab sections. Do this for one section then import that section 
	to your other sections. Or, repeat this process for each of your lab sections.
	
	\item Select “Gradebook setup” from ``Administration'' menu on the left side of the Moodle page.
	
	\item At the top of the gradebook is a folder icon with the name of the course and the semester, similar to the one shown below.

	{\centering
		\includegraphics[width=\textwidth]{gradebook_course_settings}\par
	}

	\item Click on “Edit,” then “Edit settings” from the menu.
	
	\item Click on “Show more\dots”
	
	\item Choose “Weighted mean of grades” from the drop-down menu. Do \emph{not} choose the simple weighted mean.

	\item Make sure the check the box for “Exclude empty grades” is checked. Check the box if it is not.  Save the changes.
	
	{\centering
		\includegraphics[width=3in]{gradebook_weighted_mean}\par
	}

	\item Scroll to the bottom of the page, and then click the “Add category” button.
	
	\item Click on “Show more\dots”

	\item Type “Graded” into the Category name field, choose “Simple weighted mean of grades” from the drop-down menu, and be sure “Exclude empty grades” is \emph{not} checked.
	
	{\centering
		\includegraphics[width=3in]{gradebook_graded_settings}\par
	}
	
	\item Save the changes.
	
	\item Repeat the process by adding an “Not yet graded” category. Use simple weighted mean but be sure that “Exclude empty grades” \emph{is} checked. 
		
\end{enumerate}

\subsubsection*{Edit the Availability dates}



\end{document}  