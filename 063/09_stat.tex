In biological research, we often test for cause and effect between
\textbf{independent} and \textbf{dependent} variables.

\begin{enumerate}
\def\labelenumi{\arabic{enumi}.}
\item
  \textbf{Independent variable:} a manipulated variable in an experiment
  or study whose presence or degree determines a change in the dependent
  variable
\item
  \textbf{Dependent variable: an observed} variable in an experiment or
  study whose change(s) are determined by the presence or degree of one
  or more independent variables.
\end{enumerate}

Dr. Noymer was interested in the effect on populations of the global flu
epidemic that happened in 1918. He hypothesized that babies, with their
weak immune systems, would die in a higher percentage than adults aged
25-34 when you compare the year with the epidemic (1918) to the year
before (1917).

1. What is the IV? (2 pts)

2. What is the DV? (2 pts)

3. Write a null hypothesis based on Noymer's conjecture (2 pts).

4. Write a research hypothesis (2 pts).

Noymer's RESULTS Adapted from \emph{Age-specific death rates (per
100,000), Influenza \& Pneumonia, USA} (Noymer, 2007).

\begin{quote}
Table 9.4. U.S. Deaths per 100,000 Attributed to Influenza and
Pneumonia* During 1917-1918
\end{quote}

\begin{longtable}[c]{@{}llll@{}}
\toprule
Age & 1917 & 1918 & increase in deaths\tabularnewline
\midrule
\endhead
\textless{}1 & 2944.5 & 4540.9 & 152\%\tabularnewline
1-\/-4 & 422.7 & 1436.2 &\tabularnewline
5-\/-14 & 47.9 & 352.7 &\tabularnewline
15-24 & 78 & 1175.7 &\tabularnewline
25-34 & 117.7 & 1998 & 1707\%\tabularnewline
35-44 & 193.2 & 1097.6 &\tabularnewline
45-54 & 292.3 & 686.8 &\tabularnewline
\bottomrule
\end{longtable}

\begin{quote}
*

Note, they could not tell flu and pneumonia apart, so we count them all
as flu.
\end{quote}

6. What do these data say about the null hypothesis? (2 pts)

7. Was Noymer's research hypothesis proven true? Supported? Supported
weakly? Refuted? Explain. (3 pts)

Measures of Central Tendency a key descriptive statistic

\textbf{The Meaning of a Statistic}

A statistic is a one-number description of a set of data, or numbers
used as measurements or counts - lengths of arms, number of days, number
of fish in a catch - or, rarely, a number in that set.

Suppose that a biologist has used a 1-4 scale to describe the density of
ground-level vegetation in different habitats as part of a survey for
the local conservation district. 1 = no vegetation, 2 = a few plants
(mostly soil), 3 = mostly plants and 4 = no visible soil. Twenty samples
were taken randomly in each habitat.

The mean score of a forest habitat was 1.5, and for a meadow, the score
was 3.0. When reading the report, the administrator concludes that the
density of vegetation was two times higher in the meadow than in the
forest.

\begin{enumerate}
\def\labelenumi{\arabic{enumi}.}
\item
  Is mean the appropriate statistic?
\item
  Should the biologist accept the administrator's conclusion? Why might
  it be challenged?
\end{enumerate}

Place the left foot length measurements of males and females in the
class below

\begin{longtable}[c]{@{}llll@{}}
\toprule
Male & & Female &\tabularnewline
\midrule
\endhead
Sample number & Foot length

(cm) & & Sample number\tabularnewline
1 & & & 1\tabularnewline
2 & & & 2\tabularnewline
3 & & & 3\tabularnewline
4 & & & 4\tabularnewline
5 & & & 5\tabularnewline
6 & & & 6\tabularnewline
7 & & & 7\tabularnewline
8 & & & 8\tabularnewline
9 & & & 9\tabularnewline
10 & & & 10\tabularnewline
11 & & & 11\tabularnewline
12 & & & 12\tabularnewline
13 & & & 13\tabularnewline
14 & & & 14\tabularnewline
15 & & & 15\tabularnewline
16 & & & 16\tabularnewline
17 & & & 17\tabularnewline
18 & & & 18\tabularnewline
19 & & & 19\tabularnewline
20 & & & 20\tabularnewline
21 & & & 21\tabularnewline
22 & & & 22\tabularnewline
23 & & & 23\tabularnewline
24 & & & 24\tabularnewline
Means & & &\tabularnewline
\bottomrule
\end{longtable}

Calculate the mean foot length for each sex (2 pts each, \textbf{hint}:
total the values and divide by the number of values)

\begin{longtable}[c]{@{}l@{}}
\toprule
\textbf{In science we are often interested in the distribution of our
data.} A distribution is a graph where we put the values on the X axis,
and the frequency of the values on the Y axis. You are probably familiar
with a ``bell curve'' or normal distribution, such as the generic one
shown to the right.

\hyperdef{}{meaning}{}{}\textbf{How do you describe the center of the
distribution in a number? Three methods: }

\includegraphics{media/image1.png}\textbf{Mean:} Arithmetic mean or the
arithmetic average. Add all the scores together and divide by the number
of scores.

\textbf{Median:} In an ordered set of numbers, the number in the middle:
take the number of scores and divide by 2, count down the distribution
to find the median. The score at the 50\textsuperscript{th} percentile.

\textbf{Mode:} the most frequent score in the set of
scores.\tabularnewline
\midrule
\endhead
\bottomrule
\end{longtable}

Mean: Use when distribution is symmetric, few outliers. The mean is
sensitive to each number in the distribution, so is greatly affected by
outliers.

Median: Is less sensitive to outliers, but less stable than the mean
(slight variations in scores can change the median).

Mode: Mode is the least stable measure of central tendency.

Back to the yard gnome foot length data:

3. What are the medians for males \_\_\_\_\_\_\_\_\_\_\_\_\_\_\_\_ and
females \_\_\_\_\_\_\_\_\_\_\_\_\_\_

4. What is the mode in males \_\_\_\_\_\_\_\_\_\_\_\_\_\_\_\_ and
females \_\_\_\_\_\_\_\_\_\_\_\_\_\_\_\_\_\_\_

5. What were the means for males \_\_\_\_\_\_\_\_\_\_\_\_\_\_ and
females \_\_\_\_\_\_\_\_\_\_\_\_\_\_\_\_\_

\textbf{When the mean, median and mode are all the same value (or nearly
so) we say we have a normal distribution. }

\textbf{6. Were the foot length data normally distributed? (clearly not,
sort of, pretty good, spot on)}

\emph{CALCULATING STANDARD DEVIATION}

The standard deviation is used to tell how far on average any data point
is from the mean. The smaller the standard deviation, the closer the
scores are on average to the mean. When the standard deviation is large,
the scores are more widely spread out on average from the mean.

The \textbf{standard deviation} is calculated to find the
\textbf{average distance from the mean.}

\section{}\label{section}

\section{Practice Problem \#1: Calculate the standard deviation of the
subsample foot data by hand. Use the chart below to record the
steps.}\label{practice-problem-1-calculate-the-standard-deviation-of-the-subsample-foot-data-by-hand.-use-the-chart-below-to-record-the-steps.}

\textbf{Subsampled feet data:}

\textbf{Mean:\_\_\_\_\_\_\_\_\_\_\_\_\_
\emph{n}:\_\_\_\_\_\_\_\_\_\_\_\_\_\_\_}

Sum of (Difference from the Mean) divided by degrees of freedom (\emph{n
-- 1)}:\_\_\_\_\_\_\_ \textbf{This is called variance.}

\[\frac{{\sum_{}^{}{(x - \overset{\overline{}}{x})}}^{2}}{(n - 1)}\mathbf{=}\]

\section{Final Step:}\label{final-step}

\section{ Standard deviation = square root of what you just calculated
(variance).
}\label{standard-deviation-square-root-of-what-you-just-calculated-variance.}

\section{\texorpdfstring{Standard deviation =
\(\sqrt{\frac{{\sum_{}^{}{\mathbf{(x -}\overset{\overline{}}{\mathbf{x}}\mathbf{)}}}^{\mathbf{2}}}{\mathbf{(n - 1)}}}\mathbf{=}\)
\_\_\_\_\_\_\_\_\_\_\_\_\_\_\_.}{Standard deviation = \textbackslash{}sqrt\{\textbackslash{}frac\{\{\textbackslash{}sum\_\{\}\^{}\{\}\{\textbackslash{}mathbf\{(x -\}\textbackslash{}overset\{\textbackslash{}overline\{\}\}\{\textbackslash{}mathbf\{x\}\}\textbackslash{}mathbf\{)\}\}\}\^{}\{\textbackslash{}mathbf\{2\}\}\}\{\textbackslash{}mathbf\{(n - 1)\}\}\}\textbackslash{}mathbf\{=\} \_\_\_\_\_\_\_\_\_\_\_\_\_\_\_.}}\label{standard-deviation-sqrtfracsumux5fmathbfx--oversetoverlinemathbfxmathbfmathbf2mathbfn---1mathbf-ux5fux5fux5fux5fux5fux5fux5fux5fux5fux5fux5fux5fux5fux5fux5f.}

\textbf{PRACTICE PROBLEM:}

\textbf{For all individuals in the lab, now calculate the mean and
standard deviation of the data. Describe the mean and standard deviation
in words after calculating it (8 pts). }

\begin{enumerate}
\def\labelenumi{\alph{enumi}.}
\item
  Refer back to the full class data set on foot length.
\end{enumerate}
