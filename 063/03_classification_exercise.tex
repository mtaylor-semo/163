%!TEX TS-program = lualatex
%!TEX encoding = UTF-8 Unicode

\documentclass[12pt]{exam}
\usepackage{graphicx}
	\graphicspath{{/Users/goby/Pictures/teach/163/lab/}} % set of paths to search for images

\usepackage{geometry}
\geometry{letterpaper, bottom=1in}                   

\newlength{\myindent}
\setlength{\myindent}{\parindent}
\newcommand{\ind}{\hspace*{\myindent}}


%\geometry{landscape}                % Activate for for rotated page geometry
\usepackage[parfill]{parskip}    % Activate to begin paragraphs with an empty line rather than an indent
%\usepackage{amssymb, amsmath}
%\usepackage{mathtools}
%	\everymath{\displaystyle}

\usepackage{fontspec}
\setmainfont[Ligatures={TeX}, BoldFont={* Bold}, ItalicFont={* Italic}, BoldItalicFont={* BoldItalic}, Numbers={OldStyle,Proportional}]{Linux Libertine O}
\setsansfont[Scale=MatchLowercase,Ligatures=TeX, Numbers=OldStyle]{Linux Biolinum O}
\setmonofont[Scale=MatchLowercase]{Inconsolatazi4}
\usepackage{microtype}

%\usepackage{unicode-math}
%\setmathfont[Scale=MatchLowercase]{Asana Math}
%\setmathfont[Scale=MatchLowercase]{XITS Math}

% To define fonts for particular uses within a document. For example, 
% This sets the Libertine font to use tabular number format for tables.
\newfontfamily{\tablenumbers}[Numbers={Monospaced}]{Linux Libertine O}
\newfontfamily{\libertinedisplay}{Linux Libertine Display O}

\usepackage{booktabs}
\usepackage{multicol}

\usepackage[normalem]{ulem}

%\usepackage{tabularx}
\usepackage{longtable}
%\usepackage{siunitx}
\usepackage{array}
\newcolumntype{L}[1]{>{\raggedright\let\newline\\\arraybackslash\hspace{0pt}}p{#1}}
\newcolumntype{C}[1]{>{\centering\let\newline\\\arraybackslash\hspace{0pt}}p{#1}}
\newcolumntype{R}[1]{>{\raggedleft\let\newline\\\arraybackslash\hspace{0pt}}p{#1}}

\usepackage{enumitem}
\setlist{leftmargin=*}
\setlist[1]{labelindent=\parindent}
\setlist[enumerate]{label=\textsc{\alph*}.}

\usepackage{hyperref}
%\usepackage{hanging}

\usepackage[sc]{titlesec}


\renewcommand{\solutiontitle}{\noindent}
\unframedsolutions
\SolutionEmphasis{\bfseries}

\renewcommand{\questionshook}{%
	\setlength{\leftmargin}{-\leftskip}%
}
%Change \half command from 1/2 to .5
%\renewcommand*\half{.5}


\makeatletter
\def\SetTotalwidth{\advance\linewidth by \@totalleftmargin
\@totalleftmargin=0pt}
\makeatother



\pagestyle{headandfoot}
\firstpageheader{BI 063: Evolution and Ecology}{}{\ifprintanswers\textbf{KEY}\else Name: \enspace \makebox[2.5in]{\hrulefill}\fi}
\runningheader{}{}{\footnotesize{pg. \thepage}}
\footer{}{}{}
\runningheadrule

\newcommand*\AnswerBox[2]{%
    \parbox[t][#1]{0.92\textwidth}{%
    \begin{solution}#2\end{solution}}
    \vspace*{\stretch{1}}
}

\newenvironment{AnswerPage}[1]
    {\begin{minipage}[t][#1]{0.92\textwidth}%
    \begin{solution}}
    {\end{solution}\end{minipage}
    \vspace*{\stretch{1}}}

\newlength{\basespace}
\setlength{\basespace}{5\baselineskip}


%\printanswers

\begin{document}

\subsection*{Classification: grouping by similarity }

In 1978, Carolus Linnaeus developed a hierarchical scheme to classify all living organisms, grouped by increasing levels of similarity. Linnaeus used seven taxonomic ranks in his original hierarchy. A new rank, the Domain, was added in 1990.  The eight ranks, listed from most inclusive to the most specific, are

\ind Domain, Kingdom, Phylum, Class, Order, Family, Genus, Species.

For this exercise, you will work in groups of 2–4 students (as directed by your instructor) to classify “species” of nails, screws, bolts, and similar hardware fasteners. You will classify the hardware using whatever characteristics you wish. Some characteristics include color, shape, the presence or absence of threads on the shaft, hardware type (e.g., nail, screw, bolt, hook), type of head (e.g., slotted or phillips) and so on. 

You must follow only a few rules.

\begin{enumerate}

	\item The species rank can have only one species (one hardware item). 
	
	\item You must finish with at least two groups for your highest rank (most inclusive).
	
	\item You must finish with 6–8 taxonomic ranks, including the species rank.
	
	\item Every higher rank (domain through genus) must have at least one species. A phylum, for example, can have only one species. Every lower rank included in that phylum would have just that one species.
	
	\item Each rank must include all lower ranks. That is, a family must include at least one genus and one species rank. An order must include at least one family, one genus, and one species rank.
	
	\item You are not limited to the number of groups for each taxonomic rank. For example, you may use 3, 4, 7, or some other number of  phyla (plural of phylum).
	
	\item Make a descriptive name (1-2) for each rank. A good name might be descriptive of the character(s) you used to group items in a particular rank.

\end{enumerate}

\begin{questions}

\question
Write your hierarchical classification scheme on a separate sheet of paper, using the example on the next page as a guide. For each rank (above the species rank), describe the similarity that you used to group similar “species.” For example, if your scheme includes five families, tell what characteristic you used for each family.

\vspace*{\stretch{1}}

Your group may be called on to write part of your scheme on the board at the front of the classroom.

\end{questions}

\newpage

This example uses basic geometric shapes. The example is not complete for all ranks shown but your classification must be complete from species up to the highest (most inclusive) rank that you use. \bigskip

Domain Foursides\\
	\ind Kingdom 90degrees — sides meet at 90 degree angles.\\
	\ind	\ind Phylum Rectangle — only opposites sides of equal length.\\
	\ind	\ind	\ind Class FilledRectangle — rectangle is filled with a solid color that is not white.\\
	\ind	\ind	\ind Class HollowRectangle  — rectangle is not filled or is white inside. \\
	\ind	\ind Phylum Square — all four sides of equal length.\\
	\ind	\ind	\ind Class FilledSquare — square is filled with a solid color that is not white.\\ 
	\ind	\ind	\ind Class HollowSquare — square is not filled or is white inside. \\
	\ind Kingdom not90egrees\\
	\ind	\ind Phylum Trapezoid\\
	\ind	\ind Phylum Rhomboid
	
	\medskip
	
Domain Threesides\\
	\ind Kingdom Right\\
	\ind Kingdom Isoceles
	
	\medskip
	
Domain Nosides\\
	\ind Kingdom Oval\\
	\ind Kingdom Circle\\


\end{document}  