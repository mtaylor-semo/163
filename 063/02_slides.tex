%!TEX TS-program = lualatex
%!TEX encoding = UTF-8 Unicode

\documentclass[t]{beamer}

%%%% HANDOUTS For online Uncomment the following four lines for handout
%\documentclass[t,handout]{beamer}  %Use this for handouts.
%\usepackage{handoutWithNotes}
%\pgfpagesuselayout{3 on 1 with notes}[letterpaper,border shrink=5mm]


%%% Including only some slides for students.
%%% Uncomment the following line. For the slides,
%%% use the labels shown below the command.
%\includeonlylecture{student}

%% For students, use \lecture{student}{student}
%% For mine, use \lecture{instructor}{instructor}


%\usepackage{pgf,pgfpages}
%\pgfpagesuselayout{4 on 1}[letterpaper,border shrink=5mm]

% FONTS
\usepackage{fontspec}
\def\mainfont{Linux Biolinum O}
\setmainfont[Ligatures={Common,TeX}, Contextuals={NoAlternate}, BoldFont={* Bold}, ItalicFont={* Italic}, Numbers={OldStyle, Proportional}]{\mainfont}
\setsansfont[Scale=MatchLowercase, Numbers=OldStyle]{Linux Biolinum O} 
\usepackage{microtype}

\usepackage{graphicx}
	\graphicspath{{/Users/goby/pictures/teach/163/lecture/}
	{/Users/goby/pictures/teach/common/}} % set of paths to search for images

%\usepackage{units}
\usepackage{booktabs}
%\usepackage{textcomp}

\usepackage{tikz}
%	\tikzstyle{every picture}+=[remember picture,overlay]

\mode<presentation>
{
  \usetheme{Lecture}
  \setbeamercovered{invisible}
%  \setbeamertemplate{items}[square]
}

%\usefonttheme[onlymath]{serif}
%\usecolortheme[named=blue7]{structure}

\begin{document}

\lecture{student}{student}

\begin{frame}[t]{Our goals for this lab are to }

	\hangpara learn the \highlight{Linnaean classification system,}
	
	\hangpara learn what is \highlight{binomial nomenclature,}
	
	\hangpara learn how \highlight{taxonomy} and classification are related,
	
	\hangpara develop a classification system, and 
	
	\hangpara compare the classification of several species.	
	
\end{frame}

%\lecture{instructor}{instructor}
%
%{
%\usebackgroundtemplate{\includegraphics[width=\paperwidth]{bird_paradise.jpg}}
%\begin{frame}[b,plain]
%	\Tiny\textcolor{white}{Greater Bird of Paradise \textcopyright Tim Laman, All Rights Reserved. \href{http://www.youtube.com/watch?v=KIYkpwyKEhY}{Link to Video} }
%\end{frame}
%}
%
%{
%\usebackgroundtemplate{\includegraphics[width=\paperwidth]{goliath_beetle}}
%\begin{frame}[b,plain]
%%	\hfill\Tiny \textit{Eupatorus gracilicornis}, Didier Descouens, Wikimedia Commons.
%\end{frame}
%}

{
\usebackgroundtemplate{\includegraphics[width=\paperwidth]{beetle_fondness}}
\begin{frame}[b]
	\hfill\tiny \textit{Eupatorus gracilicornis}, Didier Descouens, \ccbysa{4}
\end{frame}
}

\lecture{student}{student}
{
\usebackgroundtemplate{\includegraphics[width=\paperwidth]{how_many_species}}
\begin{frame}[b]
	\tiny The Emirr, Wikimedia \ccbysa{3}
\end{frame}
}


\lecture{instructor}{instructor}
{
\usebackgroundtemplate{\includegraphics[width=\paperwidth]{table_of_diversity}}
\begin{frame}[b]{About 1.2 million species of \textasciitilde11 million species have been described.}

	\vspace*{13\baselineskip}
	
	{\large Should these species be organized by similarity or relatedness? }

	\vfilll

	\hfill\tiny Mora et al. 2011, PLoS ONE.
\end{frame}
}

\lecture{student}{student}
{
\usebackgroundtemplate{\includegraphics[width=\paperwidth]{systema_naturae}}
\begin{frame}[b]
\end{frame}
}

\lecture{student}{student}
{
\usebackgroundtemplate{\includegraphics[width=\paperwidth]{classification_hierarchy}}
\begin{frame}[b]
\end{frame}
}

{
\usebackgroundtemplate{\includegraphics[width=\paperwidth]{binomial_nomenclature}}
\begin{frame}[b]
\end{frame}
}

{
\usebackgroundtemplate{\includegraphics[width=\paperwidth]{taxonomic_classification}}
\begin{frame}[b]
\end{frame}
}

\end{document}
