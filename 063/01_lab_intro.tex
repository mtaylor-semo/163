%!TEX TS-program = lualatex
%!TEX encoding = UTF-8 Unicode

%\documentclass[t]{beamer}

%%%% HANDOUTS For online Uncomment the following four lines for handout
\documentclass[t,handout]{beamer}  %Use this for handouts.
%\usepackage{handoutWithNotes}
%\pgfpagesuselayout{3 on 1 with notes}[letterpaper,border shrink=5mm]


%%% Including only some slides for students.
%%% Uncomment the following line. For the slides,
%%% use the labels shown below the command.
\includeonlylecture{student}

%% For students, use \lecture{student}{student}
%% For mine, use \lecture{instructor}{instructor}


%\usepackage{pgf,pgfpages}
%\pgfpagesuselayout{4 on 1}[letterpaper,border shrink=5mm]

% FONTS
\usepackage{fontspec}
\def\mainfont{Linux Biolinum O}
\setmainfont[Ligatures={Common,TeX}, Contextuals={NoAlternate}, Numbers={Proportional, OldStyle}]{\mainfont}
\setsansfont[Ligatures={Common,TeX}, Scale=MatchLowercase, Numbers={Proportional,OldStyle}, BoldFont={* Bold}, ItalicFont={* Italic},]\mainfont

\newfontface\lining[Numbers={Lining}]\mainfont

\usepackage{graphicx}
	\graphicspath{{/Users/goby/pictures/teach/163/lecture/}
	{/Users/goby/pictures/teach/common/}} % set of paths to search for images

%\usepackage{units}
\usepackage{booktabs}
\usepackage{multicol}
\usepackage{calc}  % Used for the HiddenWord macro below.
%\usepackage{textcomp}

\usepackage{tikz}
%	\tikzstyle{every picture}+=[remember picture,overlay]

\mode<presentation>
{
  \usetheme{Lecture}
  \setbeamercovered{invisible}
  \setbeamertemplate{items}[square]
}

%\usefonttheme[onlymath]{serif}
%\usecolortheme[named=blue7]{structure}

\newcommand{\btVFill}{\vskip0pt plus 1filll}

% HiddenWord macro requires the calc package.
\newcommand\HiddenWord[1]{% 
	\alt<handout>{\rule{\widthof{#1}}{\fboxrule}}{#1}%
}

% Chance gray to black if you do not want to gray out text.
\newcommand\GrayedOut[1]{%
	\alt<handout>{#1}{\textcolor{gray}{#1}}%
}


\begin{document}

\lecture{instructor}{instructor}
{
\usebackgroundtemplate{\includegraphics[width=\paperwidth]{hoverflies} }
\begin{frame}[b,plain]{\textcolor{orange7}{\textsc{bi} 063-02: Evolution and Ecology Lab}}

%\begin{center}\LARGE\textcolor{white}{Sit towards front of the room.}\end{center}

\hfill\textcolor{white}{\tiny Fir0002, Wikimedia \ccbync{3}}
\end{frame}
}

{
\usebackgroundtemplate{\includegraphics[width=\paperwidth]{mike_snake}
}
\begin{frame}[t,plain]
	\large
	\vspace{5ex}
	\hangpara\hspace{17em} Mike Taylor

	\hangpara\hspace{17em} \textsc{rh} 217

	\hangpara\hspace{17em} \textsc{m}\,\textsc{w} 9--10~\textsc{am}, \textsc{t}\,10--11~\textsc{am}.

	\hangpara\hspace{17em} mtaylor@semo.edu

\end{frame}
}


\begin{frame}[t]{\href{http://learning.semo.edu}{learning.semo.edu}}
	\begin{center}
		\includegraphics[width=\textwidth]{moodle_logo}
		
		\medskip
		
	\end{center}
	
\end{frame}



\lecture{student}{student}

\begin{frame}[t]{Our goals for this lecture are to}

	\hangpara learn that science is a process for understanding the natural world,
	
	\hangpara learn and apply the five steps of the \highlight{scientific method.}
		
	\hangpara learn and apply \highlight{hypothesis testing,}
	
	\hangpara learn the difference between \highlight{explanatory} and \highlight{response variables,} and
	
	\hangpara perform an experiment on bacterial resistance.
	
\end{frame}

\lecture{instructor}{instructor}

{
\usebackgroundtemplate{\includegraphics[width=\paperwidth]{what_is_science}}
\begin{frame}[b]

	\hfill \tiny\textcolor{white}{J.J, \ccbysa{3}}
\end{frame}
}

\lecture{instructor}{instructor}

{
\usebackgroundtemplate{\includegraphics[width=\paperwidth]{science_tentative_new1}}
\begin{frame}[b]{Science is \highlight{tentative.}}
\end{frame}
}

\lecture{student}{student}

{
\usebackgroundtemplate{\includegraphics[width=\paperwidth]{science_tentative_new2}}
\begin{frame}[b]{Science is \highlight{tentative.}}
\end{frame}
}

{
\usebackgroundtemplate{\includegraphics[width=\paperwidth]{science_objective}}
\begin{frame}[b]{Science is \highlight{objective.}}
\end{frame}
}

{
\usebackgroundtemplate{\includegraphics[width=\paperwidth]{science_testable}}
\begin{frame}[b]{Science is \highlight{testable.}}
\end{frame}
}

{
\usebackgroundtemplate{\includegraphics[width=\paperwidth]{reasoning_inductive}}
\begin{frame}[b]

\hfill \tiny \textcolor{white}{decltype, \ccbysa{3}}
\end{frame}
}

{
\usebackgroundtemplate{\includegraphics[width=\paperwidth]{reasoning_deductive}}
\begin{frame}[b,plain]

\hfill\tiny OpenClipart Vectors, Pixabay.com, \ccby{0}
\end{frame}
}

%\begin{frame}{Let's test our reasoning skills\dots}
%
%	\hangpara I will turn over some cards.
%	
%	\hangpara Can you predict what card comes next?
%	
%	\hangpara After several observations, can you predict the final pattern?
%
%\end{frame}


\begin{frame}[t]{Scientists perform objective testing with the \highlight{scientific method.}}

	\hangpara \highlight{Observation:} some aspect of the natural world that interests you.
	\pause

	\hangpara \highlight{Hypothesis:} A plausible explanation, taking into account what is already known.
	\pause
	
	\hangpara \highlight{Prediction:} Tells what to expect if you do the experiment.\\ “\textit{If} I do X, \textit{then} I will see Y” is a prediction.
	\pause
	
	\hangpara \highlight{Experiment and results:} Set up the conditions to test the prediction.
	\pause
	
	\hangpara \highlight{Conclusion:} The results can \textit{support} the hypothesis, \textit{falsify} the hypothesis, or be inconclusive.

\end{frame}

% Sea Turtle example

{
\usebackgroundtemplate{\includegraphics[width=\paperwidth]{ascension_green_sea_turtle}}
\begin{frame}[b,plain]{\highlight{Observation:} Green sea turtles swim 1200 miles from Brazil to Ascension Island to lay their eggs.}

\hfill\tiny\textcolor{white}{Laszlo Ilyes, Flickr, \ccby{2}}
\end{frame}
}

{
\usebackgroundtemplate{\includegraphics[width=\paperwidth]{ascension_location}}
\begin{frame}[b]

\hfill\tiny\textcolor{white}{Modified from Strebe, Wikimedia, \ccbysa{3}}
\end{frame}
}


\begin{frame}[t]{Two \highlight{hypotheses} were proposed to explain the observation.}
	\setlength{\columnseprule}{0.4pt}

	\begin{multicols}{2}
	
		\hangpara Carr-Coleman (1974)
		
		\hangpara Turtles have returned to Ascension for 40–70 million years to lay eggs.
		
		\hangpara Their hypothesis is based on plate tectonics. 
		
	\columnbreak
	
		\hangpara Gould (1978)
		
		\hangpara Turtles began using Ascension within past 1 million years.
		
		\hangpara Their hypothesis is based on a rare and recent dispersal event.
	\end{multicols}
\end{frame}


{
\setbeamercolor{background canvas}{bg=black}
\begin{frame}[t]{\textcolor{white}{Plate tectonics causes continents to shift positions over time.}}
	\begin{center}
		\begin{tikzpicture}
			\foreach \Value in {1,2,3,4,5,6,7,8,9,10,11,12,13}
				\node<\Value> (img\Value) {\includegraphics[width=.5\linewidth]{tectonics\Value}};
		\end{tikzpicture}
	\end{center}

	\textcolor{white}{As Ascension Island (red dot) separated from Brazil over millions of years, generations of turtles continued to visit the island to breed.}

	\btVFill

	\hfill\tiny \textcolor{white}{\url{http://www.ucmp.berkeley.edu/geology/tectonics.html}}

\end{frame}
}

\begin{frame}[t]{Each hypothesis makes a specific \highlight{prediction.}}
	\vspace*{-\baselineskip}

	\hangpara{\small Turtle \textsc{dna} diverges at an approximate rate of 0.2\% per million years.}

	\setlength{\columnseprule}{0.4pt}

	\begin{multicols}{2}
	
		\hangpara Carr-Coleman (1974)
	
		\hangpara \textit{If} turtles have returned to Ascension for at least 40 million years,
	
		\hangpara \textit{then} \textsc{dna} should differ from other populations by at least 8\%.
	
	\columnbreak
	
		\hangpara Gould (1978)
	
		\hangpara \textit{If} turtles only started using Ascension within last 1 million years,
	
		\hangpara \textit{then} \textsc{dna} should differ from other populations by no more than 0.2\%.	
	\end{multicols}
\end{frame}

{
\usebackgroundtemplate{\includegraphics[width=\paperwidth]{ascension_dna}}
\begin{frame}[b]{The \highlight{experiment} compared \textsc{dna} of turtles sampled from Ascension, Florida, Venezuela, and Hawaii.}

\tiny\textcolor{white}{University of Michigan School of Natural Resources \textsc{dna} Lab, Flickr, \ccby{2}}
\end{frame}
}
%
{
\usebackgroundtemplate{\includegraphics[width=\paperwidth]{ascension_green_sea_turtle}}
\begin{frame}[b,plain]{Which hypothesis do you think turned out to be supported?}

\hfill\tiny\textcolor{white}{Laszlo Ilyes, Flickr, \ccby{2}.}
\end{frame}
}
%
\lecture{instructor}{instructor}
{
\usebackgroundtemplate{\includegraphics[width=\paperwidth]{ascension_results1}}
\begin{frame}[b]{Their results showed very small DNA differences between Ascension versus Florida and Venezuela.}
	
\hfill\tiny\textcolor{white}{Modified from Strebe, Wikimedia, \ccbysa{3}}
\end{frame}
}
%
{
\usebackgroundtemplate{\includegraphics[width=\paperwidth]{ascension_results2}}
\begin{frame}[b]{The results from Hawaii were also consistent.}
	
\hfill\tiny\textcolor{white}{Modified from Strebe, Wikimedia, \ccbysa{3}}
\end{frame}
}
%
\lecture{student}{student}

\begin{frame}[t]{The \highlight{conclusion} is that \HiddenWord{Gould’s} hypothesis was supported.}

	\vspace*{-\baselineskip}

	\hangpara The results agree with the prediction made by \alt<handout>{which}{Gould's} hypothesis\alt<handout>{?}{.}

	\setlength{\columnseprule}{0.4pt}

	\begin{multicols}{2}
		\hangpara \GrayedOut{Carr-Coleman (1974)}
		
		\hangpara \GrayedOut{\textit{If} turtles have returned to Ascension for at least 40 million years,}
		
		\hangpara \GrayedOut{\textit{then} \textsc{dna} should differ from other populations by at least 8\%.}

	\columnbreak
		\hangpara Gould (1978)
		
		\hangpara \textit{If} turtles only started using Ascension within last 1 million years,
		
		\hangpara \textit{then} \textsc{dna} should differ from other populations by no more than 0.2\%.	
	\end{multicols}

\end{frame}
%
\begin{frame}{\highlight{Hypothesis testing} uses data sampled from one or more populations to make inferences about those populations}
	
	\hangpara  \highlight{Null hypothesis:} There is \emph{no} difference between between populations or samples. 	Any differences are probably due to natural variation and other random effects. 
	
	\hangpara  \highlight{Research hypothesis:} There \emph{is} a difference between populations or samples. Or, changing a variable will affect the population(s). Any differences are probably not random
	and have a biological explanation. 
	
\end{frame}
%
\begin{frame}{Scientists usually manipulate and measure variables.}
	
	\hangpara  \highlight{Explanatory variable:} a manipulated or controlled variable in an experiment
	or study whose presence or degree determines a change in the response
	variable.  
	
	\hangpara  \highlight{Response variable:} an observed variable in an experiment or
	study that changes in response to the presence or degree of one
	or more explanatory variables.  
	
\end{frame}

\end{document}
