%!TEX TS-program = lualatex
%!TEX encoding = UTF-8 Unicode

\documentclass[12pt, hidelinks]{exam}

%\printanswers

\usepackage{graphicx}
	\graphicspath{{/Users/goby/Pictures/teach/163/lab/}
	{img/}} % set of paths to search for images

\usepackage{geometry}
\geometry{letterpaper, left=1.5in, bottom=1in}                   
%\geometry{landscape}                % Activate for for rotated page geometry
\usepackage[parfill]{parskip}    % Activate to begin paragraphs with an empty line rather than an indent
\usepackage{amssymb, amsmath}
\usepackage{mathtools}
	\everymath{\displaystyle}

\usepackage{fontspec}
\setmainfont[Ligatures={TeX}, BoldFont={* Bold}, ItalicFont={* Italic}, BoldItalicFont={* BoldItalic}, Numbers={Proportional, OldStyle}]{Linux Libertine O}
\setsansfont[Scale=MatchLowercase,Ligatures=TeX, Numbers={Proportional}]{Linux Biolinum O}
\setmonofont[Scale=MatchLowercase]{Linux Libertine Mono O}
\newfontfamily{\liningnum}[Numbers=Lining]{Linux Libertine O}
\usepackage{microtype}


% To define fonts for particular uses within a document. For example, 
% This sets the Libertine font to use tabular number format for tables.
 %\newfontfamily{\tablenumbers}[Numbers={Monospaced}]{Linux Libertine O}
% \newfontfamily{\libertinedisplay}{Linux Libertine Display O}

\usepackage{booktabs}
\usepackage{multicol}
%\usepackage[normalem]{ulem}

\usepackage{longtable}
%\usepackage{siunitx}
\usepackage{array}
\newcolumntype{L}[1]{>{\raggedright\let\newline\\\arraybackslash\hspace{0pt}}p{#1}}
\newcolumntype{C}[1]{>{\centering\let\newline\\\arraybackslash\hspace{0pt}}p{#1}}
\newcolumntype{R}[1]{>{\raggedleft\let\newline\\\arraybackslash\hspace{0pt}}p{#1}}

\usepackage{enumitem}
	\setlist{leftmargin=*}
	\setlist[1]{labelindent=\parindent}
	\setlist[enumerate]{label=\textsc{\alph*}.}
	\setlist[itemize]{label=\color{gray}\textbullet}

\usepackage{hyperref}
%\usepackage{placeins} %PRovides \FloatBarrier to flush all floats before a certain point.
\usepackage{hanging}

\usepackage[sc]{titlesec}

%% Commands for Exam class
\renewcommand{\solutiontitle}{\noindent}
\unframedsolutions
\SolutionEmphasis{\bfseries}

\renewcommand{\questionshook}{%
	\setlength{\leftmargin}{-\leftskip}%
}

\pagestyle{headandfoot}
\firstpageheader{\textsc{bi}\,063 Evolution and Ecology}{}{\ifprintanswers\textbf{KEY}\else Name: \enspace \makebox[2.5in]{\hrulefill}\fi}
\runningheader{}{}{\footnotesize{pg. \thepage}}
\footer{}{}{}
\runningheadrule

\newcommand*\AnswerBox[2]{%
    \parbox[t][#1]{0.92\textwidth}{%
    \begin{solution}#2\end{solution}
%    \vspace{\stretch{0.5}}
	\vskip\stretch{1}}
}

\newenvironment{AnswerPage}[1]
    {\begin{minipage}[t][#1]{0.92\textwidth}%
    \begin{solution}}
    {\end{solution}\end{minipage}
    \vspace{\stretch{1}}}

\newlength{\basespace}
\setlength{\basespace}{5\baselineskip}

\newcommand*\AnswerBlank{\rule{0.75in}{0.4pt}\kern0.67pt.}
\newcommand*\xcell[1]{cell~\liningnum{#1}}
\newcommand*\axis[1]{{\scshape #1}-axis}

%
%\makeatletter
%\def\SetTotalwidth{\advance\linewidth by \@totalleftmargin
%\@totalleftmargin=0pt}
%\makeatother


\begin{document}

%% 2010 Version
\subsection*{Creating figures with Microsoft Excel 2010\texttrademark}

Graphical representations of your data in scientific publications are called figures. For this exercise, you will learn to create scientific figures using Microsoft Excel. Excel has many options for making quality figures. Excel also has options that you should never use in scientific figures, most of which, unfortunately, are the default settings.  

This exercise will teach you how to use Excel to create one- and two-variable column charts (often called bar charts), a scatterplot, and a line graph suitable for publication. Once learned, you will use this valuable skill to present results in this course, in your future courses, and for the rest of your scientific career. 

\subsubsection*{Column chart: one explanatory variable}

Column charts are often used when you want to compare values among categorical variables. For example, you may use a column chart to compare the average \textsc{gpa} among freshmen, sophomores, juniors and seniors on campus. 

First, you will create a column chart to show the number of earthquakes of magnitude 3.0 or higher\footnote{Magnitude represents the strength of the earthquake on the Richter scale. A magnitude 4.0 earthquake releases nearly 32$\times$ more energy than a magnitude 3.0 earthquake.} recorded in Oklahoma from 1980 through last year. \bigskip

\begin{questions}

\question
Which variable is the explanatory variable? 

\AnswerBox{2\baselineskip}{Year}

\question
Which variable is the response variable?

\AnswerBox{2\baselineskip}{Number of earthquakes}


Download the file called “oklahoma\_earthquakes.xlsx” from the course data website:\\ \url{http://mtaylor4.semo.edu/~goby/bi163/}.\bigskip

The first column of data is the year. The second column is the number of earthquakes of magnitude 3.0 or higher that occurred in Oklahoma each year.

\begin{enumerate}
	\item Select cells {\liningnum B1 to B38}.

	\item Click on “Insert” on the ribbon menu. Click on the “Column” icon near the center of the ribbon, and then choose the 2-D “Clustered Column,” which is the first icon on the left. If you leave your cursor over the icon for a moment, an explanation of the chart type will appear.

	\item Resize the graph to make it larger by dragging from the corners.

	\item Right-click on the graph and choose “Select Data\dots” from the menu.

	\item Click on “Earthquakes” on the left-hand side of the dialog box, and then click the “Edit” button on the \emph{right-hand} side, for “Horizontal (Category) Axis Labels.” Select cells {\liningnum A2 to A38}. Click the “OK” button, then click the next “OK” button.

	You should now have years on the \textsc{x}-axis and the number of earthquakes for each year on the \textsc{y}-axis.
	
\end{enumerate}

The graph is adequate but it can be improved with a few simple changes. First, change and enlarge the font. The font in your graphs should be the same font you use in the body of your report. For example, if you are using Times New Roman in your Word document, you should use Times New Roman in your graph. The font size should be at least the same as you use in your report but is often a little larger. For this exercise, use 12 point font, regular style (not bold).

\begin{enumerate}[resume]
	\item Place your cursor over one of the years on the \axis{x}, then right-click. Select “Font\dots”.
	
	\item In the “Latin text font:” area, you can either delete the current font and type “Times New Roman” (exactly) or click on the small button to the right of the “Font:” area and scroll down to select it from the menu. 
	
	\item Tab over to or click in the “Size:” area and change the value to 12. Press the Enter key or click the “OK” button.
	
	\item Repeat the last step for the \axis{y}.
\end{enumerate}

Scientific figures usually do not have a symbol legend in the figure because the symbols are explained in the figure caption. So, delete the symbol legend.  

\begin{enumerate}[resume]
	\item Click on the symbol legend and press the delete key. Or, right-click on the symbol legend and select “Delete” from the menu.
\end{enumerate}

Add a labels to the \axis{x} and \axis{y}, starting with the \axis{x}.

\begin{enumerate}[resume]
	\item Click once anywhere in the chart. Click “Layout” from the “Chart Tools” section of the ribbon menu. 
	
	\item Click the “Axis Titles” icon near the center of the ribbon. Choose “Primary Horizontal Axis Title”, and then “Rotated Title” from the menus.
	
	\item Right-click on the axis label that appears to the left side of the \axis{y}. Set the font to Times New Roman and the font size to 12 points. If necessary, click on the “Font style:” menu and select “Regular” to remove the bold face.
	
	\item Right-click on the label. Select the text inside the box and type “Year.” Click anywhere outside the label to accept your change.
	
	\item Repeat the steps for the \axis{y} but choose “Primary Vertical” for the placement of the axis title. Use “Number of earthquakes” for the title. 
	
\end{enumerate}

The columns of the chart have a default color. Color can be helpful but you must consider color carefully. First, publishing color figures is very expensive (often more than \$600 \emph{per} figure) so black, white, and shades of gray are used unless color is absolutely necessary. Second, you must consider colorblindness. Colorblind readers cannot distinguish some colors. For example, the inability to distinguish between red and green is a form of colorblindness present in about 8\% of males (but rare in females). Finally, most readers
of your publications will read them as photocopies or printed from
a black and white printer, which loses the color information

For this figure, use black columns.

\begin{enumerate}[resume]
	\item Right-click on any one of columns. Choose “Format Data Series\dots” from the menu.
	
	\item Click on “Fill” from the choices on the left. Select the “Solid fill” radio button. Select one of the medium gray squares (such as the third one from the top) below “Theme Colors” that appears when you click on the “Color:” menu.
	
	\item Click on “Border Color” from the choices on the left. Select the “Solid fill” radio button. Select the small black square that appears below “Theme Color” when you click on the “Color:” menu.
	
	\item Click the “Close” button.
	
\end{enumerate}

If your figure contains grid lines, they must be removed.

\begin{enumerate}[resume]
	\item Right-click on one of the grid lines. 
	
	\item Select “Format Gridlines\dots” from the popup menu. 
	
	\item Select the “No line” radio button and close the pane.
	
	\item Or, click on one of the grid lines and press the delete key.
\end{enumerate}

You now have scientific-quality figure that would be suitable for publication.

\question
Fracking has become more common in Oklahoma. Fracking produces waste water that is injected into the ground for disposal. Fracking increased dramatically in Oklahoma in 2010 and later. Based on your figure, how does the number of earthquakes from 1980–2009 compare to the number of earthquakes since 2010?

\AnswerBox{4\baselineskip}{The number of earthquakes since 2010 is far, far greater than the 30 years before hand.}

\question
Which three years have the greatest number of earthquakes? How many earthquakes occurred in each of these years?

\AnswerBox{3\baselineskip}{See if they think to look up the numbers in the Excel sheet.}

The recent increase in the number of earthquakes suggests wastewater disposal from the fracking industry \emph{might} be the cause but this would have to be tested. Such tests are complex but you can do a visual inspection to see if there \emph{appears} to be an association.

Use your web browser to visit this official website published by the Office Of The Secretary Of Energy \& Environment for Oklahoma, and the follow the instructions below.

\url{http://earthquakes.ok.gov/what-we-know/earthquake-map/}

\begin{enumerate}
	\item Click the “Earthquakes - Past 7 days” to hide recent earthquakes.
	
	\item Click the “Arbuckle Waste Water Disposal Wells” check box. The blue circles show the locations of all active waste water disposal wells in Oklahoma.
	
	\item Click \emph{all three} checkboxes for “Earthquakes - 2000 through 2009,” “Earthquakes - 1990 through 1999,” and
	“Earthquakes - 1980 through 1989.”
	
	This shows the earthquakes recorded for 1980–2009. The size of the circles indicates the relative magnitude or strength of the earthquake. You can click on individual circles to see the date and strength of the earthquake.
\end{enumerate}

\question
Does there appear to be an association between the earthquakes and the disposal wells during 1980–2009, or do the earthquakes appear to be located randomly with respect to the wells? Explain.

\AnswerBox{3\baselineskip}{random pattern or weak association.}

\begin{enumerate}
	\item Uncheck the three boxes for 1980–2009. Leave the disposal wells box checked.
	
	\item Click the checkboxes for the three years with the greatest number of earthquakes, as shown in your Excel figure.
	
\end{enumerate}

\question
Does there appear to be an association between the earthquakes and the disposal wells during the three years with greatest earthquake activity, or do the earthquakes appear to be located randomly with respect to the wells? Explain.

\AnswerBox{2\baselineskip}{Definite association between quakes and wells.}

%\newpage

Not all disposal wells cause earthquakes but these data show a clear association between the location of earthquakes and the waste water disposal wells. Other studies using more detailed data strongly support the hypothesis that deep, rapid injection of waste water is the cause of the sudden increase in Oklahoma earthquakes.

\subsubsection*{Column chart: two explanatory variables}

Create a column chart to show the mean (average) number of insect species collected from two species of trees. Trees were sampled from a treatment plot, where the number of trees per acre was thinned, and from a control plot, where no thinning occurred. Eight trees were surveyed for each tree species.\bigskip

\question
Which two variables are the explanatory variable? 

\AnswerBox{2\baselineskip}{Tree species and whether the plot was thinned.}

\question
Which variable is the response variable?

\AnswerBox{2\baselineskip}{Insect richness.}

Download the file called “insect\_richness.xlsx” from the course data website:\\ \url{http://mtaylor4.semo.edu/~goby/bi163/}.\bigskip

The file has three columns. The first is the treatment, indicating whether the plot was thinned or not (the control). The second column represents individual trees, either Douglas Fir (\textsc{df}) or Lodgepole Pine (\textsc{lp}). The third column is insect richness, which is the number of species of insects that was found on each tree. 

The data are not in a format suitable for making a column chart. These data are the raw numbers of insects. You need to report the average richness per tree species so you must summarize the data. You also need to calculate the standard deviation.

\begin{enumerate}
	\item In \xcell{F3}, type “Douglas Fir.” In \xcell{G3}, type ‘Lodgepole Pine.” In \xcell{E4}, type “Control.” In \xcell{E5}, type “Thinned.” 

	\item \label{tree_mean} Calculate the mean insect species richness for Douglas Fir in the Control Group. Click in \xcell{F4}. Type \texttt{=average(}. Next, select the cells that you want to average. In this case, select cells {\liningnum C2} through {\liningnum C9}, as you did above. After you have selected the last cell, type “)”, and then press the Enter key. 

	What if you are not sure of the function name? Click on “Formulas” in the ribbon menu. Click on the “Insert Function” icon at the very left of the ribbon. In the dialog box that appears, type “Average” into the area where it says “Search for a function:”. Highlight \texttt{AVERAGE} in the choices and click the “OK” button. Next, a dialog box appears for you to enter the arguments for the \texttt{AVERAGE} function. For the Douglas Fir control sample, you must select cells C2:C9 by clicking and holding on \xcell{C2} and dragging down to \xcell{C9}. The range of cells is added to the dialog box. Press the Enter key or click the “OK” button. If you did this correctly, the value should be 65.875.

	\item Repeat Step~\ref{tree_mean} for the other three cells: Control Lodgepole Pine, Thinned Douglas Fir, and Thinned Lodgepole Pine. 
	
	\item \label{tree_dev} In \xcell{I3}, type ``Douglas Fir s.d.'' In \xcell{J3}, type ``Lodgepole Pine s.d.''
	
	\item Calculate the standard deviations. In \xcell{I4}, type \texttt{=stdev.s(c2:c9)} (or, select the cells as you did earlier).
	
	\item Repeat Step~\ref{tree_dev} for the other three cells: Control Lodgepole Pine (cells {\liningnum C10:C17}), Thinned Douglas Fir (cells {\liningnum C18:C25}), and Thinned Lodgepole Pine (cells {\liningnum C26:C33}). 
	
\end{enumerate}

The data are now arranged and ready for you to make the column chart.

\begin{enumerate}[resume]
	\item Select cells{\liningnum E3 to G5}.

	\item Insert a 2-\textsc{d} clustered column chart as you did above and resize it as necessary.
	
	\item Change the font and size to Times New Roman, 12 point.
	
	\item Delete the legend inside the figure.
	
	\item Add a \axis{y} label, Times New Roman, 12 point, regular style, that reads “Insect Richness.” \emph{Be sure to remove the bold face if Excel added it.}
	
	\item Remove any horizontal or vertical grid lines, if they appear on your figure.
	
	\item Change the colors of the columns to black and white (or grey). Use black for the border color of both columns.
	
\end{enumerate}

For this figure, you will add error bars to show the standard deviations, to show the variability of the data around the mean for each species and treatment.

\begin{enumerate}[resume]

	\item Click the left (control) bar for Douglas Fir. It should select both Control bars (for Douglas Fir and Lodgepole Pine). 
	
	\item Choose the “Layout” tab. Click the “Error Bars” icon on the right side of the ribbon. If necessary, move the dialog box that opens to one side so that you can see cells {\liningnum I4:J5}.
	
	\item Select “More Error Bars Options\dots”. Click on “Vertical Error Bars” on the top left side of the dialog box.
	
	\item Click on the “Custom” radio button, and then click on “Specify Value”.
	
	\item \label{fir_error} Delete the “\texttt{=\{1\}}” from the “Positive Error Value” field. With your cursor, select cells {\liningnum I4:J4}. Repeat this for the “Negative Error Value.” Select the same two cells ({\liningnum I4:J4}). Click the “OK” button.
	
	\item Repeat Step~\ref{fir_error} for Lodgepole Pine. Select cells I5:J5.
	
\end{enumerate}


\question
Which species of tree has the insect higher richness. Use numbers from the data to support your answer.

\AnswerBox{4\baselineskip}{Douglas fir has higher richness of about 67 (unthinned) or 55 (thinned species)}. 

\question
Does thinning of trees appear to affect species richness living in each tree species? Tell how you know.

\AnswerBox{4\baselineskip}{Yes. The unthinned controls for both species have greater average richness compared to thinned plots.}

\question
How do you think the length of the error bars would change if you increased sample size for each tree species and treatment?

\AnswerBox{4\baselineskip}{The amount of variability will \emph{probably} decrease so the error bars would get shorter.}
\subsubsection*{Scatterplot}

As you will learn later this semester, the distribution of ecosystems and their associated dominant plant groups, are determined primarily by mean annual temperature (\textsc{mat}) and mean annual precipitation (rain and snow; \textsc{map}).

\question
Which two variables are the explanatory variable? 

\AnswerBox{2\baselineskip}{Mean annual temperature and mean annual precipitation.}

\question
Which variable is the response variable?

\AnswerBox{2\baselineskip}{The ecosystems/dominant plant groups.}

Download the file called “ecosystems.xlsx” from the course data website:\\ \url{http://mtaylor4.semo.edu/~goby/bi163/}.\bigskip

You will make a scatterplot to see the distribution of three plant groups based on \textsc{mat} and \textsc{map}. You will plot temperature on the \axis{x} and precipitation on the \axis{y}. The plant groups are Western Redcedar (Cw), Mixed Grassland (Gr) and Subalpine Larch (La). 

\begin{enumerate}
	\item Use your cursor to select cells {\liningnum C2:D21}, which selects \textsc{mat} and \textsc{map} for Western Redcedar.

	\item Click on the “Insert” in the ribbon menu, then on the ‘Scatter” icon, then on the upper left icon (“Scatter with only Markers”).

	Enlarge the graph by dragging from the corners. Fill most of the white space on the screen to the right of the data. Move the graph if it is covering any of the data. 
\end{enumerate}

You now have one plant group plotted but it is called ‘series 1”. We need to add data for the other two plant groups and give them informative names. 

\begin{enumerate}[resume]
	\item Right-click anywhere on the chart, and then click “Select Data\dots”.

	\item Click on “Series 1,” and then click the “Edit” button. Type “Western Redcedar” (do not include the quotes) in the “Series name:” field. Press Enter or click the “OK” button.

	\item Click the “Add” button. Type “Mixed Grasslands” in the “Series name:” field.

	\item Click in the “Series X values:” field. Select cells {\liningnum C22:C51} to select \textsc{mat} for the mixed grasslands. Notice that the cells you selected appear in the “Series X values” field.

	\item Repeat for \textsc{map}. Click in the “Series Y values” field, delete the “=\{1\}”, and then select {\liningnum D22:D51} to select the \textsc{map} values for the \axis{y}. 

	\item Repeat this process for Subalpine Larch. Click the “Add” button, type “Subalpine Larch” for the name, use {\liningnum C52:C81} for “Series \textsc{x} values” and {\liningnum D52:D81} for “Series \textsc{y} values.” 

	\item Click the “OK” button. You may have to scroll back up. All three plant groups should now be displayed in your graph. 
\end{enumerate}

At this point, you have an adequate graph but you can improve it. Notice The \axis{y} crosses the \axis{x} at $0$ instead of $-4$, which places the \axis{y} in an awkward position. Change the \axis{y} to cross the \axis{x} at $-4$.

\begin{enumerate}[resume]
	\item Place your cursor over one of the numbers on the \axis{x}, then right-click. Select “Format Axis\dots”

	\item From the “Axis Options” area, click the “Axis value:” radio button, near the bottom below “Vertical axis crosses:” and then replace the $0.0$ in box with $-4.0$. Do not forget the negative sign. Click “Close.” 

	\item As you did with the previous figures, set the font to Times New Roman, 12 point, regular style for the \textsc{x}- and \textsc{y}-axes.
	
	\item Delete the symbol legend. Delete any horizontal or vertical grid lines.
	
	\item Add labels to the \textsc{x}- and \textsc{y}-axes. Format the text of each axis to Times New Roman, 12 point. Set Font style to Regular to remove the bold face.
	
	\item Edit the text of each label to read “Mean Annual Temperature (degrees C)” for the \axis{x} and “Mean Annual Precipitation (mm) for the \axis{y}.

	\item You can improve the label “(degrees C)” by using the degree symbol (°). Right-click on the label and select “Edit Text.” Double click on “degrees” to select it.

\begin{center}
	\includegraphics[width=0.5\textwidth]{11_insert_symbol}
\end{center}

	\item Click on “Insert” on the ribbon menu and then click on “Symbol” on the right side of the ribbon. Click the degree symbol near the bottom center of the various symbols. Click in the “Insert” button and then the “Close” button. 

	\item If necessary, resize your figure so that the axis label does not overlap the axis values.

\end{enumerate}

You now have an excellent scatterplot but you can improve it by enlarging the symbols and removing the color because  color is costly and is useless in photocopies.

\begin{enumerate}[resume]
	\item Place your cursor over one of the blue diamonds (Western Redcedar) and Right-click. Select “Format Data Series\dots”. 

	\item Click “Marker Options” from the menu on the left. Click the “Built-in” radio button. Increase the size to 10. Do not click the Close button.

	\item Change the Marker Fill of the diamond to solid black. Change the Marker Line Color to black. If you do not remember how to do this, review the final steps of the Column Chart section.

	\item Repeat this process for the green triangles that represent Subalpine Larch. Change the marker size to 10, change the Marker Line Color to black, and then change the Marker Fill to a medium grey by choosing the lowest small grey on the left. Click the “Close” button.

	\item Repeat this process for the red squares that represent Mixed Grasslands. Change the marker size to 9 (size 10 squares might look too big relative to the diamonds and triangles but you are welcome to try size 10), change the Marker Line Color to black, and then change the Marker Fill to white.
\end{enumerate}

\question
Which ecosystem shows the \emph{greatest} variability for mean annual temperature?

\AnswerBox{3\baselineskip}{Mixed Grasslands (squares).}

\question
Which ecosystem shows the \emph{least} variability for mean annual precipitation?

\AnswerBox{3\baselineskip}{Mixed Grasslands (squares).}

\question
Which ecosystem, on average, needs the lowest annual temperatures to be present. What do you estimate is about the average amount of annual precipitation needed by this ecosystem to be present? 

\AnswerBox{3\baselineskip}{Subalpine larch (triangles). About 1100 mm precipitation.}

\question
Which ecosystem, on average, needs the highest annual precipitation to be present. What do you estimate is about the average annual temperature needed by this ecosystem to be present? 

\AnswerBox{3\baselineskip}{Western Redcedar (diamonds). About 8–9°C temperature.}


\subsubsection*{Line Graph}

Line graphs are useful for showing trends in data over time or categories, such as increasing tree diameter with age of the tree, or the accumulation of species sampled over time. Here you will plot the mean monthly river stage (a measure of water depth) from the Mississippi River measured at the U.S. Geological Survey gauge near Thebes, IL to see how river depth changes throughout the year. The gauge records the river stage every hour of every day. The data have been reduced to mean monthly discharge.

\question
Which variable is the explanatory variable? 

\AnswerBox{2\baselineskip}{Year}

\question
Which variable is the response variable?

\AnswerBox{2\baselineskip}{Mean monthly river stage}

Download the file called “mississippi\_river\_stages.xlsx” from the course data website:\\ \url{http://mtaylor4.semo.edu/~goby/bi163/}\bigskip

By now, you should be good at graphing in Excel so the final graph should be quick.

\begin{enumerate}
	\item Use the cursor to highlight cells {\liningnum B2:C13}, which contains January to December data for the year 2011.

	\item Click “Insert” on the ribbon menu, click the “Line” icon near the left center of the ribbon, and then click the upper left icon to insert the line graph.

	\item Right-click on the graph, choose “Select Data\dots” and add the 2012 series, using the same procedure you followed with the scatterplot. Edit the Series 1 name to change it to “2011”. Name the new series “2012”. \textsc{Note}: Use the Gauge Height column for the \textsc{y} values. The Month names for the \textsc{x}-values do not change so you are not given the option to include them when you add new data series.

	\item Add a rotated primary vertical axis title. Edit the label to read “Gauge Height (feet)”. Add a primary horizontal axis title that reads “Month”. Change both to 12 point, Times New Roman font, regular style.

	\item Edit both axes to use 12 point, Times New Roman Regular font, regular style.

	\item Delete the symbol legend. Delete any horizontal and vertical grid lines.

	\item Right-click on the upper line for 2011. Format the data series so that the Line Color is solid black.

	\item Right-click on the lower line for 2012. Format the data series so that the Line Color is black and the Line Style is dashed. Select the fourth choice from the top with the uniform dash size. 
\end{enumerate}

\question
One of the years you plotted was a drought year across the U.S., including southeast Missouri. The other year was a flood year, where the flooding was so bad in southeast Missouri that they had to blow up a levee at Bird's Point to release some of the water from the river. Based on your graph, which line represents the drought year and which year represents the flood year. Explain how you decided.

\AnswerBox{4\baselineskip}{The flood year has a much higher river stage in the spring. The drought year has a much lower river stage overall.}

\question
In both years, the river stage is higher in April and May and lower in September and October. Explain why.

\AnswerBox{4\baselineskip}{Snow melt and spring rains bring the river levels up in spring. No melt and summer dryness leads to low river stages in summer.}

%\question
%About how much difference (in feet) is there between the maximum river stages for both years? About how much difference (in feet) is there between the minimum river stages for both years?
%
%\AnswerBox{4\baselineskip}{The maximum difference is about 15 feet. The minimum difference is about 6–7 feet. Again, students with a knack for the obvious can look at the raw data.}

\end{questions}

\subsubsection*{Chart types to avoid}

Pie Charts: Pie charts are not often used with scientific data although they do have some specific uses. People are not good at estimating the area of a circle or comparing the relative sizes of the pie slices. This increases the difficulty of interpreting the figure. In general, use a column chart instead of a pie chart. 

3-D charts: They look fancy but they are actually harder to interpret for most types of data. 3-D charts are rarely appropriate and must be avoided. 


\end{document}  