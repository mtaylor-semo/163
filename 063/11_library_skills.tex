%!TEX TS-program = lualatex
%!TEX encoding = UTF-8 Unicode

\documentclass[12pt, addpoints, hidelinks]{exam}
\usepackage{graphicx}
	\graphicspath{{/Users/goby/Pictures/teach/163/lab/}
	{img/}} % set of paths to search for images

\usepackage{geometry}
\geometry{letterpaper, left=1.5in, bottom=1in}                   
%\geometry{landscape}                % Activate for for rotated page geometry
\usepackage[parfill]{parskip}    % Activate to begin paragraphs with an empty line rather than an indent
\usepackage{amssymb, amsmath}
\usepackage{mathtools}
	\everymath{\displaystyle}

\usepackage{fontspec}
\setmainfont[Ligatures={TeX}, BoldFont={* Bold}, ItalicFont={* Italic}, BoldItalicFont={* BoldItalic}, Numbers={Proportional, OldStyle}]{Linux Libertine O}
\setsansfont[Scale=MatchLowercase,Ligatures=TeX, Numbers={Proportional,OldStyle}]{Linux Biolinum O}
\setmonofont[Scale=MatchLowercase]{Linux Libertine Mono O}

\usepackage{microtype}


% To define fonts for particular uses within a document. For example, 
% This sets the Libertine font to use tabular number format for tables.
 %\newfontfamily{\tablenumbers}[Numbers={Monospaced}]{Linux Libertine O}
% \newfontfamily{\libertinedisplay}{Linux Libertine Display O}

\usepackage{booktabs}
\usepackage{multicol}
%\usepackage[normalem]{ulem}

\usepackage{longtable}
%\usepackage{siunitx}
\usepackage{array}
\newcolumntype{L}[1]{>{\raggedright\let\newline\\\arraybackslash\hspace{0pt}}p{#1}}
\newcolumntype{C}[1]{>{\centering\let\newline\\\arraybackslash\hspace{0pt}}p{#1}}
\newcolumntype{R}[1]{>{\raggedleft\let\newline\\\arraybackslash\hspace{0pt}}p{#1}}

\usepackage{enumitem}
\setlist{leftmargin=*}
\setlist[1]{labelindent=\parindent}
\setlist[enumerate]{label=\textsc{\alph*}.}
\setlist[itemize]{label=\color{gray}\textbullet}

\usepackage{hyperref}
%\usepackage{placeins} %PRovides \FloatBarrier to flush all floats before a certain point.
\usepackage{hanging}

\usepackage[sc]{titlesec}

%% Commands for Exam class
\renewcommand{\solutiontitle}{\noindent}
\unframedsolutions
\SolutionEmphasis{\bfseries}

\renewcommand{\questionshook}{%
	\setlength{\leftmargin}{-\leftskip}%
}

%Change \half command from 1/2 to .5
\renewcommand*\half{.5}

\pagestyle{headandfoot}
\firstpageheader{\textsc{bi}\,063 Evolution and Ecology}{}{\ifprintanswers\textbf{KEY}\else Name: \enspace \makebox[2.5in]{\hrulefill}\fi}
\runningheader{}{}{\footnotesize{pg. \thepage}}
\footer{}{}{}
\runningheadrule

\newcommand*\AnswerBox[2]{%
    \parbox[t][#1]{0.92\textwidth}{%
    \begin{solution}#2\end{solution}}
    \vspace*{\stretch{1}}
}

\newenvironment{AnswerPage}[1]
    {\begin{minipage}[t][#1]{0.92\textwidth}%
    \begin{solution}}
    {\end{solution}\end{minipage}
    \vspace*{\stretch{1}}}

\newlength{\basespace}
\setlength{\basespace}{5\baselineskip}

\newcommand{\hidepoints}{%
	\pointsinmargin\pointformat{}
}

\newcommand{\showpoints}{%
	\nopointsinmargin\pointformat{(\thepoints)}
}

%
%\makeatletter
%\def\SetTotalwidth{\advance\linewidth by \@totalleftmargin
%\@totalleftmargin=0pt}
%\makeatother


%\printanswers


\begin{document}

\hidepoints

\subsection*{Library skills and scientific literature (\numpoints~points)}

For this exercise, you will learn more about famous
biologists and some ecological and evolutionary concepts. You will
locate and cite research that has been foundational to the development
of modern ecology or evolution. All scholarly journals have a specific format that
they follow. You will use the citation format of the journal 
Ecology for this assignment.  You will need
to examine a recent issue of Ecology from the Ecological Society of
America website for reference
(\url{http://esajournals.onlinelibrary.wiley.com/hub/issue/10.1002/ecy.2016.97.issue-10/}).
Your answers must follow this format for full credit (one point per question), as outlined below. 

%Your citations must follow this format: 

%\vspace*{0.75\baselineskip}

\hangpara{1.5em}{1}%
Last Name, FI. MI., FI. MI. Last Name, etc. Year of
Publication. Title of journal article. Name of Journal Volume number:
first page–last page.

\vspace*{0.75\baselineskip}

For example, the proper format for the scientific paper you read last week is:

\hangpara{1.5em}{1}%
Wood,~D.~L. and A.~J.~Bornstein. 2011. Notes on the
biology of \textit{Obolaria virginica} (Gentianaceae) in Southeast
Missouri, and the effects of leaf litter on emergence and flower
production. Castanea 76: 157–163.

\vspace*{0.75\baselineskip}

For the first part, you will learn about a biologist and her or his contribution(s) to their field. Pick one ecologist or evolutionary biologist from the list below,
search for their work, and find an example for each of the questions
below.

\begin{multicols}{2}
\textit{Ecology} \\
Lucy Braun \\
Rachel Carson \\
Charles Elton \\
Nelson Hairston\\
G. Evelyn Hutchinson\\
Jane Lubchenco \\
Robert A. MacArthur \\
E.~O. Wilson 

\columnbreak

\textit{Evolution}\\
Bonnie Bassler \\
Sean B. Carroll \\ 
Theodosius Dobzhansky \\
Stephen Jay Gould \\
Rosemary Grant \\
Hopi Hoekstra \\
Ernst Mayr \\
Tomoko Ohta 
\end{multicols}

\begin{questions}

\question[1]
Provide the citation for one primary literature article published in a
scholarly journal by your chosen biologist.

\vspace*{\stretch{1}}

\question[1]
Provide the citation for one secondary source authored by your
chosen biologist.

\vspace*{\stretch{0.5}}

\newpage

Next, you will search for ecological or evolutionary concepts and identify
scholarly work involving those concepts. Pick from the list
below for your keyword searches.

\begin{multicols}{2}
\textit{Ecology}\\
Ecological niche \\
Food web subsidy \\
Island biogeography \\
Keystone predation \\
Limiting similarity \\
Metapopulations \\
Succession \\
Trophic cascade

\columnbreak

\textit{Evolution}\\
Gene duplication \\
Genetic bottleneck \\
Haldane's rule \\
Kin selection \\
Mass extinction \\
Mating systems \\
Phenotypic plasticity \\
Sympatric speciation \\ 
\end{multicols}

\vspace*{-\baselineskip}

\question[1]
Provide one primary literature citation by two or more authors on your
concept.

\AnswerBox{1\baselineskip}{}

\question[1]
Provide a secondary source related to your concept.

\AnswerBox{1\baselineskip}{}

\question[1]
Find a chapter from an edited book that is related to your 
concept.

\AnswerBox{1\baselineskip}{}

Now you are you familiar with some ecological or evolutionary concepts and 
some biologists that study them. For the final part of this
assignment, you get to pick any topic or subjects related to ecology or evolution and
answer the questions below.

\question[1]
Find one primary literature article by a single author.

\AnswerBox{1.5\baselineskip}{}

\question[1]
Find a publication with an internet address only.

\AnswerBox{1\baselineskip}{}

\question[1]
Locate a government publication (any federal or state government
publication) \textit{or} a book by an organization, committee, etc.
(something other than a person or persons) that is related to your
search topic.

\AnswerBox{1\baselineskip}{}

\end{questions}

\end{document}  