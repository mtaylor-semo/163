%!TEX TS-program = lualatex
%!TEX encoding = UTF-8 Unicode

\documentclass[12pt]{exam}
\usepackage{graphicx}
	\graphicspath{{/Users/goby/Pictures/teach/163/lab/}
	{img/}} % set of paths to search for images

\usepackage{geometry}
\geometry{letterpaper, left=1.5in, bottom=1in}                   
%\geometry{landscape}                % Activate for for rotated page geometry
\usepackage[parfill]{parskip}    % Activate to begin paragraphs with an empty line rather than an indent
\usepackage{amssymb, amsmath}
\usepackage{mathtools}
	\everymath{\displaystyle}

\usepackage{fontspec}
\def\mainfont{Linux Libertine O}
\setmainfont[Ligatures={TeX}, BoldFont={* Bold}, ItalicFont={* Italic}, BoldItalicFont={* BoldItalic}, Numbers={Proportional, OldStyle}]{\mainfont}
\setsansfont[Scale=MatchLowercase,Ligatures=TeX, Numbers={Proportional,OldStyle}]{Linux Biolinum O}
\setmonofont{Linux Libertine O}
\newfontface{\lining}[Numbers=Lining]{\mainfont}
\usepackage{microtype}

\usepackage{unicode-math}
\setmathfont[Scale=MatchLowercase]{Tex Gyre Pagella Math}

% To define fonts for particular uses within a document. For example, 
% This sets the Libertine font to use tabular number format for tables.
 %\newfontfamily{\tablenumbers}[Numbers={Monospaced}]{Linux Libertine O}
% \newfontfamily{\libertinedisplay}{Linux Libertine Display O}

\usepackage{booktabs}
\usepackage{multicol}

\usepackage{caption}
\captionsetup{font=small} 
\captionsetup{singlelinecheck=false}
\captionsetup[figure]{labelsep=period, format=plain}

\usepackage{longtable}
%\usepackage{siunitx}
\usepackage{array}
\newcolumntype{L}[1]{>{\raggedright\let\newline\\\arraybackslash\hspace{0pt}}p{#1}}
\newcolumntype{C}[1]{>{\centering\let\newline\\\arraybackslash\hspace{0pt}}p{#1}}
\newcolumntype{R}[1]{>{\raggedleft\let\newline\\\arraybackslash\hspace{0pt}}p{#1}}

\usepackage{enumitem}
\setlist{leftmargin=*}
\setlist[1]{labelindent=\parindent}
\setlist[enumerate]{label=\textsc{\alph*}.}
\setlist[itemize]{label=\color{gray}\textbullet}

\usepackage[hyphens]{url}
%\usepackage{hyperref}
%\usepackage{placeins} %PRovides \FloatBarrier to flush all floats before a certain point.
\usepackage{hanging}

\usepackage[sc]{titlesec}

%% Commands for Exam class
\renewcommand{\solutiontitle}{\noindent}
\unframedsolutions
\SolutionEmphasis{\bfseries}

\renewcommand{\questionshook}{%
	\setlength{\leftmargin}{-\leftskip}%
}

%Change \half command from 1/2 to .5
\renewcommand*\half{.5}

\pagestyle{headandfoot}
\firstpageheader{\textsc{bi}\,063 Evolution and Ecology}{}{\textbf{Instructor Notes}}
\runningheader{}{}{\footnotesize{pg. \thepage}}
\footer{}{}{}
\runningheadrule

\newcommand*\AnswerBox[2]{%
    \parbox[t][#1]{0.92\textwidth}{%
    \begin{solution}#2\end{solution}}
    \vspace*{\stretch{1}}
}

\newenvironment{AnswerPage}[1]
    {\begin{minipage}[t][#1]{0.92\textwidth}%
    \begin{solution}}
    {\end{solution}\end{minipage}
    \vspace*{\stretch{1}}}

\newlength{\basespace}
\setlength{\basespace}{5\baselineskip}

\newcommand{\hidepoints}{%
	\pointsinmargin\pointformat{}
}

\newcommand{\showpoints}{%
	\nopointsinmargin\pointformat{(\thepoints)}
}

\newcommand\chisq{$\chi^2$}

%
%\makeatletter
%\def\SetTotalwidth{\advance\linewidth by \@totalleftmargin
%\@totalleftmargin=0pt}
%\makeatother


\printanswers


\begin{document}

\hidepoints

\subsection*{Changes in lake ice: ecosystem response to climate change}

\begin{itemize}
	\item \textbf{Answers in bold. Do not share with students until appropriate.}
\item
  Each group (2–3 students) will need a laptop.
\item
  Have students work through handout until Question~1. Discuss as a
  group, guiding students to the consensus that ``Ice duration'' will give you the
  most useful information, the clearest evidence of a trend. Instruct all
  students to use ``Ice duration'' for their \textsc{y}-axis in this exercise.
    
\item
  Assign students one of the \textasciitilde{}30 year time periods listed below. 
  A few groups will have the same time period. That is
  fine. Just make sure that they compare their data with a group with a
  different time period. Assign different time periods to groups at the same table. 

  1855–1881 (27 years) \quad \ifprintanswers \textbf{Mean duration: 117.5 days.}\fi
  
  1882–1908 (27 years) \quad \ifprintanswers \textbf{Mean duration: 106.7 days.}\fi
  
  1909–1934 (26 years) \quad \ifprintanswers \textbf{Mean duration: 101.7 days.}\fi
  
  1935–1961 (27 years) \quad \ifprintanswers \textbf{Mean duration: 104.0 days.}\fi
  
  1962–1988 (27 years) \quad \ifprintanswers \textbf{Mean duration: 100.2 days.}\fi
  
  1989–2015 (27 years) \quad \ifprintanswers \textbf{Mean duration: 86.6 days.}\fi
  
\item
  Have groups create their graphs and answer questions 2–8.
  
\item
  For \#9, draw a table on the board as follows. If multiple
  groups have the same years, the results only needs to be recorded on the board once. 
  
\end{itemize}

\begin{longtable}[]{@{}lL{5cm}L{5cm}@{}}
\toprule
Time period & \# years with shorter than average ice duration & \# years with longer
than average ice duration\tabularnewline
\midrule
\endhead
1855–1881 & 
\ifprintanswers\textbf{11}\fi	&
\ifprintanswers\textbf{16}\fi 
\tabularnewline[1ex]
%
1882–1908 & 
\ifprintanswers\textbf{13}\fi	&
\ifprintanswers\textbf{14}\fi 
\tabularnewline[1ex]
%
1909–1934 &  
\ifprintanswers\textbf{12}\fi & 
\ifprintanswers\textbf{14}\fi
\tabularnewline[1ex]
%
1935–1961 &
\ifprintanswers\textbf{12}\fi &
\ifprintanswers\textbf{15}\fi
\tabularnewline[1ex]
%
1962–1988 &
\ifprintanswers\textbf{13}\fi &
\ifprintanswers\textbf{14}\fi
\tabularnewline[1ex]
%
1989–2015 &
\ifprintanswers\textbf{9}\fi &
\ifprintanswers\textbf{18}\fi 
\tabularnewline[1ex]
\bottomrule
\end{longtable}

\begin{itemize}
\item
  Review the answer to \#9 as a class.
\item
  Project the long-term data set (from PowerPoint) on the board and have
  students complete questions 11–15.
  
%\item
%  Have students turn in one packet per group, 12 pts.
\end{itemize}

\end{document}  