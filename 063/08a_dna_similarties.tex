%!TEX TS-program = lualatex
%!TEX encoding = UTF-8 Unicode

\documentclass[12pt, hidelinks]{exam}
\usepackage{graphicx}
	\graphicspath{{/Users/goby/Pictures/teach/163/lab/}
	{img/}} % set of paths to search for images

\usepackage{geometry}
\geometry{letterpaper, left=1.5in, bottom=1in}                   
%\geometry{landscape}                % Activate for for rotated page geometry
\usepackage[parfill]{parskip}    % Activate to begin paragraphs with an empty line rather than an indent
\usepackage{amssymb, amsmath}
\usepackage{mathtools}
	\everymath{\displaystyle}

\usepackage{fontspec}
\setmainfont[Ligatures={TeX}, BoldFont={* Bold}, ItalicFont={* Italic}, BoldItalicFont={* BoldItalic}, Numbers={OldStyle}]{Linux Libertine O}
\setsansfont[Scale=MatchLowercase,Ligatures=TeX]{Linux Biolinum O}
\setmonofont[Scale=MatchLowercase]{Linux Libertine Mono O}
\usepackage{microtype}


% To define fonts for particular uses within a document. For example, 
% This sets the Libertine font to use tabular number format for tables.
 %\newfontfamily{\tablenumbers}[Numbers={Monospaced}]{Linux Libertine O}
% \newfontfamily{\libertinedisplay}{Linux Libertine Display O}

\usepackage{booktabs}
\usepackage{multicol}

\usepackage{tikz}
\tikzstyle{block} = [rectangle, draw, fill=white, rounded corners,
                 minimum size=2em]
\tikzstyle{branch} = [thick, draw]


\usepackage{longtable}
%\usepackage{siunitx}
\usepackage{array}
\newcolumntype{L}[1]{>{\raggedright\let\newline\\\arraybackslash\hspace{0pt}}p{#1}}
\newcolumntype{C}[1]{>{\centering\let\newline\\\arraybackslash\hspace{0pt}}p{#1}}
\newcolumntype{R}[1]{>{\raggedleft\let\newline\\\arraybackslash\hspace{0pt}}p{#1}}

\usepackage{enumitem}
\usepackage{hyperref}
%\usepackage{placeins} %PRovides \FloatBarrier to flush all floats before a certain point.
\usepackage{hanging}

\usepackage[sc]{titlesec}

%% Commands for Exam class
\renewcommand{\solutiontitle}{\noindent}
\unframedsolutions
\SolutionEmphasis{\bfseries}

\renewcommand{\questionshook}{%
	\setlength{\leftmargin}{-\leftskip}%
}

%Change \half command from 1/2 to .5
\renewcommand*\half{.5}

\pagestyle{headandfoot}
\firstpageheader{\textsc{bi}\,063 Evolution and Ecology}{}{\ifprintanswers\textbf{KEY}\else Name: \enspace \makebox[2.5in]{\hrulefill}\fi}
\runningheader{}{}{\footnotesize{pg. \thepage}}
\footer{}{}{}
\runningheadrule

\newcommand*\AnswerBox[2]{%
    \parbox[t][#1]{0.92\textwidth}{%
    \begin{solution}#2\end{solution}}
%    \vspace*{\stretch{1}}
}

\newenvironment{AnswerPage}[1]
    {\begin{minipage}[t][#1]{0.92\textwidth}%
    \begin{solution}}
    {\end{solution}\end{minipage}
    \vspace*{\stretch{1}}}

\newlength{\basespace}
\setlength{\basespace}{5\baselineskip}

%% To hide and show points
\newcommand{\hidepoints}{%
	\pointsinmargin\pointformat{}
}

\newcommand{\showpoints}{%
	\nopointsinmargin\pointformat{(\thepoints)}
}

\newcommand{\bumppoints}[1]{%
	\addtocounter{numpoints}{#1}
}

\newcommand{\dna}{\textsc{dna}}
%
%\makeatletter
%\def\SetTotalwidth{\advance\linewidth by \@totalleftmargin
%\@totalleftmargin=0pt}
%\makeatother


%\printanswers


\begin{document}

\subsection*{Organismal relationships and \textsc{dna} similarity}

The relationship of organisms can be tested by comparing the similarity of 
their \dna{}. Organisms that evolved from their common ancestor recently should 
be genetically very similar. That is, most of the nucleotides in their 
\dna{} should be the same.  Organisms that evolved farther back in time 
should be genetically more different. The differences accumulate 
in organisms over time due to random mutations that occur at a more or less 
constant rate over time. This means that 
differences in \textsc{dna} should serve as a kind of \emph{molecular clock} that 
will tell us how long ago two species had a common ancestor. Here is how this works in practice.

Pretend you are studying a rare (and imaginary) group
of birds. You want to determine the relationships among these
birds by using \textsc{dna} sequences. Here are the sequences of a small
piece of \textsc{dna} from each of the birds:

\begin{longtable}[c]{@{}ll@{}}
\toprule
Yellow-throated Garbler \textsc{(y)} & {\ttfamily CTAGC AAGTA CTACT TAGGA}\tabularnewline
California Garbler \textsc{(c)} & {\ttfamily CTGGC AAGTA CTTCT TAAGG}\tabularnewline
Blue-throated Garbler \textsc{(b)} & {\ttfamily CTAGC AACTA CTACT TAAGA}\tabularnewline
Red-footed Chump \textsc{(r)} & {\ttfamily AGGCC ATGGA CGACT ACAGA}\tabularnewline
Field Twitterer \textsc{(f)} & {\ttfamily CTGGA AAGGC CTAGT TAAGA}\tabularnewline
Alfredo's Twitterer \textsc{(a)} & {\ttfamily CTGGA AAGGA CTAGG TAAGA}\tabularnewline
\bottomrule
\end{longtable}

Fill in the table below with the number of nucleotide differences between
each pair of species. Count the differences in the \textsc{dna} sequences
between each pair of organisms.  For example, comparing \textsc{y} and \textsc{b} above, notice that the
nucleotide in the 8th position is different. It is a ``G'' in \textsc{y}, and a ``C''
in \textsc{b}. The next difference is at the 18th position. Comparing \textsc{b} and \textsc{f}, there are six positions
where the nucleotides are different. Do you see them all? (The first
one is at position 3.) Do the same for all pairs.

\begin{longtable}[c]{@{}rC{0.25in}ccccc@{}}
\toprule
& Y & C & B & R & F &
A\tabularnewline[0.15in]
Y & 0 & \rule{0.3in}{0.4pt} &
2 & \rule{0.3in}{0.4pt} &
\rule{0.3in}{0.4pt} &
\rule{0.3in}{0.4pt}\tabularnewline[0.15in]
C & ~~~~~~ & \rule{0.3in}{0.4pt} &
\rule{0.3in}{0.4pt} & \rule{0.3in}{0.4pt}
& \rule{0.3in}{0.4pt} &
\rule{0.3in}{0.4pt}\tabularnewline[0.15in]
B & ~ & ~ & \rule{0.3in}{0.4pt} &
\rule{0.3in}{0.4pt} & 6 &
\rule{0.3in}{0.4pt}\tabularnewline[0.15in]
R & ~ & ~ & ~ & \rule{0.3in}{0.4pt} &
\rule{0.3in}{0.4pt} &
\rule{0.3in}{0.4pt}\tabularnewline[0.15in]
F & ~ & ~ & ~ & ~ & \rule{0.3in}{0.4pt} &
\rule{0.3in}{0.4pt}\tabularnewline[0.15in]
A & ~ & ~ & ~ & ~ & ~ &
\rule{0.3in}{0.4pt}\tabularnewline[0.15in]
\bottomrule
\end{longtable}




\end{document}  