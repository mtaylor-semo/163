%!TEX TS-program = lualatex
%!TEX encoding = UTF-8 Unicode

\documentclass[12pt, hidelinks]{exam}
\usepackage{graphicx}
	\graphicspath{{/Users/goby/Pictures/teach/163/lab/}
	{img/}} % set of paths to search for images

\usepackage{geometry}
\geometry{letterpaper, left=1.5in, bottom=1in}                   
%\geometry{landscape}                % Activate for for rotated page geometry
\usepackage[parfill]{parskip}    % Activate to begin paragraphs with an empty line rather than an indent
\usepackage{amssymb, amsmath}
\usepackage{mathtools}
	\everymath{\displaystyle}

\usepackage{fontspec}
\setmainfont[Ligatures={TeX}, BoldFont={* Bold}, ItalicFont={* Italic}, BoldItalicFont={* BoldItalic}, Numbers={OldStyle}]{Linux Libertine O}
\setsansfont[Scale=MatchLowercase,Ligatures=TeX, Numbers=OldStyle]{Linux Biolinum O}
%\setmonofont[Scale=MatchLowercase]{Inconsolatazi4}
\usepackage{microtype}


% To define fonts for particular uses within a document. For example, 
% This sets the Libertine font to use tabular number format for tables.
 %\newfontfamily{\tablenumbers}[Numbers={Monospaced}]{Linux Libertine O}
% \newfontfamily{\libertinedisplay}{Linux Libertine Display O}

\usepackage{booktabs}
\usepackage{multicol}

\usepackage{tikz}
%\usepackage{plotmarks}

\usepackage{longtable}
%\usepackage{siunitx}
\usepackage{array}

\newcolumntype{L}[1]{>{\raggedright\let\newline\\\arraybackslash\hspace{0pt}}p{#1}}
\newcolumntype{C}[1]{>{\centering\let\newline\\\arraybackslash\hspace{0pt}}p{#1}}
\newcolumntype{R}[1]{>{\raggedleft\let\newline\\\arraybackslash\hspace{0pt}}p{#1}}

\usepackage{enumitem}
\setlist{leftmargin=*}
\setlist[1]{labelindent=\parindent}
%\setlist[enumerate]{label=\textsc{\alph*}.}
%\setlist[itemize]{label=\color{gray}\textbullet}

%\usepackage{caption}
%	\captionsetup{labelsep=period, justification=raggedright} % Removes colon following figure / table number.
%	\captionsetup{singlelinecheck=off}
%	\captionsetup[table]{skip=0pt}

\usepackage{hyperref}
%\usepackage{placeins} %PRovides \FloatBarrier to flush all floats before a certain point.
\usepackage{hanging}

\usepackage[sc]{titlesec}

%% Commands for Exam class
\renewcommand{\solutiontitle}{\noindent}
\unframedsolutions
\SolutionEmphasis{\bfseries}

\renewcommand{\questionshook}{%
	\setlength{\leftmargin}{-\leftskip}%
}

\newcommand{\hidepoints}{%
	\pointsinmargin\pointformat{}
}

\newcommand{\showpoints}{%
	\nopointsinmargin\pointformat{(\thepoints)}
}

%Change \half command from 1/2 to .5
\renewcommand*\half{.5}

\pagestyle{headandfoot}
\firstpageheader{\textsc{bi}\,063 Evolution and Ecology}{}{\ifprintanswers\textbf{KEY}\else Name: \enspace \makebox[2.5in]{\hrulefill}\fi}
\runningheader{}{}{\footnotesize{pg. \thepage}}
\footer{}{}{}
\runningheadrule

\newcommand*\AnswerBox[2]{%
    \parbox[t][#1]{0.92\textwidth}{%
    \begin{solution}#2\end{solution}}
%    \vspace*{\stretch{1}}
}

\newenvironment{AnswerPage}[1]
    {\begin{minipage}[t][#1]{0.92\textwidth}%
    \begin{solution}}
    {\end{solution}\end{minipage}
    \vspace*{\stretch{1}}}

\newlength{\basespace}
\setlength{\basespace}{5\baselineskip}

\newcommand{\dna}{\textsc{dna}}

%\printanswers

\hidepoints

\begin{document}

\subsection*{Use dna sequence to make a phylogenetic tree}

You have used homologous characters to construct phylogenetic trees that 
show the relationships among various taxonomic groups. Another character
that can be used to build phylogenetic trees is \textsc{dna} sequences.\footnote{Time
has not allowed us to demonstrate the homology of \textsc{dna.}}
D\textsc{na} is found in all living organisms, from vertebrates to bacteria. 

%% If time permits, 

Earlier, you compared the \dna{} sequence from seven species of made-up birds. 
The \dna{} sequence from each species was only 20 nucleotides long. That was useful
to show you how to use \dna{} sequences to make a phylogenetic tree. But, real studies
often use \dna{} sequences that are several hundred to several thousand nucleotides long
from dozens of species. Trying to find all of those differences by eye would be nearly impossible
to do accurately. On the other hand, hundreds of sequences can be converted into a phylogenetic tree very 
rapidly by computer. 

You will use a common analytical tool to build three phylogenetic trees using \dna{} sequence
from three different genes. Each tree will contain a subset of the 21 organisms you have been using for your hypothesis. After you have the three trees, you will draw a complete phylogeny of the 21 organisms based on the individual trees.

You will analyze three genes, one at a time, using your laptop computer or one that has been provided for you (log in with your \textsc{se k}ey).   Work in pairs (or three if necessary) to complete the following \ref{final_step} steps for \emph{each} gene.

\begin{enumerate}

	\item Open a web browser. Go to \url{http://mtaylor4.semo.edu/~goby/bi163/}. Click on one of the gene names (18\textsc{s}, \textsc{atp}6, \textsc{rag}1).\footnote{18\textsc{s} is a gene that encodes the small subunit of ribosomes used to translate m\textsc{rna} to make proteins. A\textsc{tp}6 is part of a gene that encodes an enzyme used in cellular respiration. \textsc{rag}1 is part of a gene that encodes a protein used in the vertebrate immune system.} 
	
	\item Copy the \emph{entire} sequence set (select all, or ctrl/cmd-A). Remember which sequence you copied because you will need to know in a few more steps.
	
	\item Go back to the list of genes. Click on the \textsc{ra}x\textsc{ml} link. This will take you to the \textsc{ra}x\textsc{ml} site\footnote{\textsc{ra}x\textsc{ml} uses an analytical technique called maximum likelihood. Maximum likelihood finds the best phylogenetic tree based on different models of \dna{} evolution.} to analyze the \dna{} sequences. 
	
	\item Paste your sequence set into the large area at the top of the page. If the sequences from a previous analysis is still there, delete it and then paste your new sequences.
	
	\item Scroll down to nearly the end of the ``Other Options'' section. Look for ``Outgroup name(s).'' Click that check box. Enter  one of the following outgroup names into the box \emph{exactly}, depending on which sequence you are analyzing. 
	
		\begin{tabular}{@{}ll@{}}
		\toprule
		Gene name &	Outgroup name\\
		\midrule
		18\textsc{s}	&  Ecoli\\
		\textsc{atp}6	& fungus\\
		\textsc{rag}1	& bass\\
		\bottomrule		
		\end{tabular}
	
	\item Scroll back to the top. Click the ``Compute'' button. In a few moments, the analysis will present you with a list of output files. If an \emph{Erreur} warning pops up, close it. It will not affect your analysis.
	
	\item Click on ``View tree'' to see the phylogenetic tree based on that \dna{} sequence.
	
	\item Draw the tree on a piece of paper. You can save the tree as a \textsc{png} file to your computer or flash drive. However, near the end of the assignment, you will need to see all three trees at the same time so drawing each tree on a separate piece of paper may be easiest. \label{final_step}
	
	\item Repeat this process for the other two genes.

\end{enumerate}

Now that you have drawn all three phylogenetic trees, assemble them into one large tree with all 21 organisms. Each individual tree contains some of the 21 organisms. Part of each tree ``overlaps'' with at least one of the other trees. Through careful inspection, you should be able to see how the three trees piece together. Your final tree must be consistent with all of the other evidence you have gathered. Inspect your final tree to be sure it is consistent with the homologies and transitional forms you identified in earlier labs.

When you think you have the correct tree, ask you instructor to verify it for correctness. After your instructor has verified the tree, be sure every one in your group makes a copy of the final tree.

\end{document}  