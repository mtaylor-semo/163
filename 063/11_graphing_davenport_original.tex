%!TEX TS-program = lualatex
%!TEX encoding = UTF-8 Unicode

\documentclass[12pt, hidelinks]{exam}
\usepackage{graphicx}
	\graphicspath{{/Users/goby/Pictures/teach/163/lab/}
	{img/}} % set of paths to search for images

\usepackage{geometry}
\geometry{letterpaper, left=1.5in, bottom=1in}                   
%\geometry{landscape}                % Activate for for rotated page geometry
\usepackage[parfill]{parskip}    % Activate to begin paragraphs with an empty line rather than an indent
\usepackage{amssymb, amsmath}
\usepackage{mathtools}
	\everymath{\displaystyle}

\usepackage{fontspec}
\setmainfont[Ligatures={TeX}, BoldFont={* Bold}, ItalicFont={* Italic}, BoldItalicFont={* BoldItalic}, Numbers={Proportional, OldStyle}]{Linux Libertine O}
\setsansfont[Scale=MatchLowercase,Ligatures=TeX, Numbers={Proportional,OldStyle}]{Linux Biolinum O}
\setmonofont[Scale=MatchLowercase]{Linux Libertine Mono O}
\usepackage{microtype}


% To define fonts for particular uses within a document. For example, 
% This sets the Libertine font to use tabular number format for tables.
 %\newfontfamily{\tablenumbers}[Numbers={Monospaced}]{Linux Libertine O}
% \newfontfamily{\libertinedisplay}{Linux Libertine Display O}

\usepackage{booktabs}
\usepackage{multicol}
%\usepackage[normalem]{ulem}

\usepackage{longtable}
%\usepackage{siunitx}
\usepackage{array}
\newcolumntype{L}[1]{>{\raggedright\let\newline\\\arraybackslash\hspace{0pt}}p{#1}}
\newcolumntype{C}[1]{>{\centering\let\newline\\\arraybackslash\hspace{0pt}}p{#1}}
\newcolumntype{R}[1]{>{\raggedleft\let\newline\\\arraybackslash\hspace{0pt}}p{#1}}

\usepackage{enumitem}
\setlist{leftmargin=*}
\setlist[1]{labelindent=\parindent}
\setlist[enumerate]{label=\textsc{\alph*}.}
\setlist[itemize]{label=\color{gray}\textbullet}

\usepackage{hyperref}
%\usepackage{placeins} %PRovides \FloatBarrier to flush all floats before a certain point.
\usepackage{hanging}

\usepackage[sc]{titlesec}

%% Commands for Exam class
\renewcommand{\solutiontitle}{\noindent}
\unframedsolutions
\SolutionEmphasis{\bfseries}

\renewcommand{\questionshook}{%
	\setlength{\leftmargin}{-\leftskip}%
}

%Change \half command from 1/2 to .5
\renewcommand*\half{.5}

\pagestyle{headandfoot}
\firstpageheader{\textsc{bi}\,063 Evolution and Ecology}{}{\ifprintanswers\textbf{KEY}\else Name: \enspace \makebox[2.5in]{\hrulefill}\fi}
\runningheader{}{}{\footnotesize{pg. \thepage}}
\footer{}{}{}
\runningheadrule

\newcommand*\AnswerBox[2]{%
    \parbox[t][#1]{0.92\textwidth}{%
    \begin{solution}#2\end{solution}}
    \vspace*{\stretch{1}}
}

\newenvironment{AnswerPage}[1]
    {\begin{minipage}[t][#1]{0.92\textwidth}%
    \begin{solution}}
    {\end{solution}\end{minipage}
    \vspace*{\stretch{1}}}

\newlength{\basespace}
\setlength{\basespace}{5\baselineskip}

\newcommand{\hidepoints}{%
	\pointsinmargin\pointformat{}
}

\newcommand{\showpoints}{%
	\nopointsinmargin\pointformat{(\thepoints)}
}

%
%\makeatletter
%\def\SetTotalwidth{\advance\linewidth by \@totalleftmargin
%\@totalleftmargin=0pt}
%\makeatother


%\printanswers


\begin{document}

\hidepoints

\subsection*{Working with excel spreadsheets (10~points)}

In this lab you will be working with a dataset from an actual field
experiment that was conducted on the species \emph{Obolaria virginica}
(Virginia Pennywort) at Trail of Tears State Park. \emph{Obolaria virginica} is
an herbaceous species found in forested areas throughout eastern North
America. It is a low lying, spring ephemeral and is one of the earliest
blooming species in southeastern Missouri. It emerges in late-February
to early-March in order to take advantage of the open canopy and
completes its annual cycle by mid-May.

During the life history study of \emph{Obolaria virginica} in Missouri,
it was noted that several individuals did not successfully emerge from
the leaf litter and of those individuals that did successfully emerge,
flower production was variable. It was hypothesized that the leaf litter
was having an indirect affect on flower production by prolonging
seedling emergence. The rationale was that if the seedlings emerge late
then the amount of available sunlight is decreasing as a result of the
closing canopy. Thus, if less sunlight then less energy is available for
the production of reproductive structures and seeds.

Measurements of 274 emerged plants were taken during 2006–2007. For each
plant, flowers were counted, length of main stem above the leaf litter
was taken and total length of the main stem was taken.

\subsubsection*{Working with the data set}

\textbf{Note}: There is usually more than one way to complete a task in
Excel. The following directions are basically how I use Excel. It never
hurts to learn new tricks. This goes both ways. If you have alternative
methods feel free to let me know. I am always open to learning something
new.

\textbf{Step 1}: Getting a measurement of the main stem length within
the leaf litter.

\begin{enumerate}
\item
  In row 1 cell E type in an appropriate heading. Make it brief but
  self-explanatory.

  \begin{enumerate}[label=\alph*.]
  \item
    Notice that the heading exceeds the allotted space. The text does
    not wrap like in the other cells. Right click on the cell and select
    ``format cell''
  \item
    From the box menu, click on the Alignment tab and check ``wrap
    text'' then click ``Okay''
  \end{enumerate}
\item
  Click on cell F2 and type in ``=D2-E2'' and hit enter
\item
  This gives you the within leaf litter stem length value from
  subtracting the above stem length from the total stem length for that
  plant.
\item
  Click on the F2 cell. Notice that the black outline has a small square
  in the bottom right corner. Place cursor on this and right click, hold
  and then drag down the length of the column. Once you reach the last
  cell at the bottom of the column release the mouse button. You should
  have within litter stem lengths for all plants now.
\end{enumerate}

\textbf{Step 2}: Getting ratio values between below stem length and
above litter stem lengths.

\begin{enumerate}
\item
  Create a column heading in cell G1. Remember, brief and
  self-explanatory.
\item
  Click on cell F2 and type in ``=F2/E2'' and hit enter.
\item
  Do the same thing that you did in order to fill the remaining column
  cells for ``within litter stem length''.
\item
  Highlight the entire F column and right click to format cell. In the
  format menu, click on the number tab then choose number in the
  category menu. To the right is a decimal box it should have the number
  2 in it. Do not change. Hit the Enter button. All the number values in
  the F column should be to the 2nd decimal.
\end{enumerate}

\textbf{Step 3}: Creating a scatter plot for flower numbers in relation
to \textbf{below:above} stem ratios.

\begin{enumerate}
\item
  One of the idiosyncrasies of Excel is that the first data column in a
  sequence of selected data will by default become the X axis. The ratio
  values need to form the X axis and as a result the flower count column
  needs to be moved to column H. Click on the C above the flower count
  column. This will highlight the whole column. Place the cursor on the
  black outline. The cursor will change in form. Once it does then right
  click and move the whole column from the C position to the H position.
\item
  Highlight both the G and H columns from cells 1–275. Click on the
  Insert tab at the top of the spreadsheet.
\item
  Click on the scatter option and choose the ``scatter with markers''
  only option. You should have a scatter plot graph.
\item
  Select the graph and right click on the blue border around the graph.
  Select the move chart option and check the ``New sheet'' option. This
  will move the graph to its own sheet and will be listed in the tabs at
  the bottom of the spreadsheet window.
\item
  The tab should say Chart 1. Double click the tab. This should
  highlight the tab text and as a result allow you to name the chart.
  Give your chart a name. Brief and self-explanatory.
\item
  Bring up the graph in the window. Double click on the chart title at
  the top of the graph and right click and select delete.
\item
  Select the layout tab in the chart tools at the top of the spreadsheet
  window.
\item
  In the gridline menu select the primary horizontal gridlines option
  and then select none in the pop-up menu. This should remove those
  lines.
\item
  Select the legend at the right of the graph, right click and select
  ``delete''.
\item
  Click on the black border box surrounding the chart, right click and
  select delete.
\item
  From the menu select ``Axis Titles'' and give titles for the X and Y
  axes.
\item
  Highlight one of the axes and right click. Select format axis from the
  pop-up menu. This menu will allow you to change several aspects for
  the selected axis. You can try a few different options but make sure
  you change the format back.
\item
  Right click on one of the data points and select ``Format data
  series''. Select Marker fill, then click on the solid fill button and
  select the color option for black at paint bucket icon. Next select
  Marker Line Color and select the solid line button and pick the same
  paint bucket option for black. This should now make all of your data
  points black. This is important because often ecologists are poor and
  cannot afford color in journal publications.
\item
  Select the graph and place cursor on the blue border, right click and
  select copy.
\item
  Open Microsoft Word and copy the graph into the new document by right
  clicking on the page.
\item
  Add a figure caption that best describes what the graph represents. It
  should be brief and self-explanatory.
\item
  Upon completion, put your name on the document and send it and your
  excel file to me via email for assignment points (10).
\item
  For an additional 5 points, interpret the graph results. You can type
  your interpretation/explanation below the figure caption.
\end{enumerate}

\end{document}  