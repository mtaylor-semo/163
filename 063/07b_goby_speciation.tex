%!TEX TS-program = lualatex
%!TEX encoding = UTF-8 Unicode

\documentclass[12pt, hidelinks]{exam}

%\printanswers

\usepackage{graphicx}
	\graphicspath{{/Users/goby/Pictures/teach/163/lab/}
	{img/}} % set of paths to search for images

\usepackage{geometry}
\geometry{left=1.5in, bottom=1in}                   
%\geometry{landscape}                % Activate for for rotated page geometry
\usepackage[parfill]{parskip}    % Activate to begin paragraphs with an empty line rather than an indent
%\usepackage{amssymb, amsmath}
%\usepackage{mathtools}
%	\everymath{\displaystyle}

\usepackage{pdflscape}

\usepackage{fontspec}
\setmainfont[Ligatures={TeX}, BoldFont={* Bold}, ItalicFont={* Italic}, BoldItalicFont={* BoldItalic}, Numbers={OldStyle}]{Linux Libertine O}
\setsansfont[Scale=MatchLowercase,Ligatures=TeX]{Linux Biolinum O}
\setmonofont[Scale=MatchLowercase]{Linux Libertine Mono O}
\usepackage{microtype}

%\usepackage{unicode-math}
%\setmathfont[Scale=MatchLowercase]{Asana Math}
%\setmathfont[Scale=MatchLowercase]{XITS Math}

% To define fonts for particular uses within a document. For example, 
% This sets the Libertine font to use tabular number format for tables.
\newfontfamily{\dnatable}[Numbers={Monospaced}]{Linux Libertine Mono O}
\newfontfamily{\regfont}[ItalicFont={* Italic}]{Linux Libertine O}

\usepackage{booktabs}
\usepackage{multicol}

\usepackage[justification=raggedright, labelsep=period]{caption}
\captionsetup{singlelinecheck=off}
\captionsetup{skip=0.2em}

%\usepackage{tabularx}
\usepackage{longtable}
%\usepackage{siunitx}
\usepackage{array}
\newcolumntype{L}[1]{>{\raggedright\let\newline\\\arraybackslash\hspace{0pt}}p{#1}}
\newcolumntype{C}[1]{>{\centering\let\newline\\\arraybackslash\hspace{0pt}}p{#1}}
\newcolumntype{R}[1]{>{\raggedleft\let\newline\\\arraybackslash\hspace{0pt}}p{#1}}

\usepackage{enumitem}
\usepackage{enumitem}
\setlist{leftmargin=*}
\setlist[1]{labelindent=\parindent}
\setlist[enumerate]{label=\textsc{\alph*}.}

\usepackage{hyperref}
%\usepackage{placeins} %PRovides \FloatBarrier to flush all floats before a certain point.
%\usepackage{hanging}

\usepackage[sc]{titlesec}

\renewcommand{\solutiontitle}{\noindent}
\unframedsolutions
\SolutionEmphasis{\bfseries}

\renewcommand{\questionshook}{%
	\setlength{\leftmargin}{-\leftskip}%
}

%Change \half command from 1/2 to .5
\renewcommand*\half{.5}


%% Allows fullwidth command to break across pages.
%% See 
%\makeatletter
%\def\SetTotalwidth{\advance\linewidth by \@totalleftmargin
%\@totalleftmargin=0pt}
%\makeatother


\pagestyle{headandfoot}
\firstpageheader{\textsc{bi} 063: Evolution and Ecology}{}{\ifprintanswers\textbf{KEY}\else Name: \enspace \makebox[2.5in]{\hrulefill}\fi}
\runningheader{}{}{\footnotesize{pg. \thepage}}
\footer{}{}{}
\runningheadrule

\newcommand*\AnswerBox[2]{%
    \parbox[t][#1]{0.92\textwidth}{%
    \begin{solution}#2\end{solution}}
    \vspace{\stretch{1}}
}

\newenvironment{AnswerPage}[1]
    {\begin{minipage}[t][#1]{0.92\textwidth}%
    \begin{solution}}
    {\end{solution}\end{minipage}
    \vspace{\stretch{1}}}

\newlength{\basespace}
\setlength{\basespace}{5\baselineskip}


\begin{document}

\subsection*{Behavior, color, and speciation in tropical reef fishes}

Gobies of the genus \textit{Elacatinus} form the largest group of fishes on
coral reefs of the tropical western Atlantic Ocean, including the Bahama Islands
and the Caribbean Sea. The gobies have one of three different behaviors. \emph{Cleaner} gobies
remove parasites from other fishes on the coral reef. \emph{Sponge-dweller} gobies
live inside large tube sponges, feeding on parasites that infest the sponges. \emph{Hovering} gobies
stay in the water above the reef, feeding on plankton.

The gobies have a bright stripe that, for most species, runs the length of the body
from the head to the tail. One species has a stripe that is restricted to the head.

\begin{center}
	\includegraphics[width=\textwidth]{06_elacatinus}

	{\footnotesize
	Smithsonian Institute, Flickr, public domain}
\end{center}

Work in pairs. Together,
you will use a phylogeny of 21 \textit{Elacatinus} species and
populations to explore how changes of behavior and color may have
contributed to speciation in these fishes.

The phylogeny is on page~\pageref{phylogeny}. Match the species on the phylogeny
to the behavior (cleaner, sponge-dweller, or hovering) and the stripe color 
(blue, yellow or white) shown in the table on page~\pageref{color_behavior}. 
Then, study the tree for evolutionary patterns and answer the following questions.

\emph{Apply parsimony to find the fewest number of changes for each behavior and color.}
\begin{questions}

\question
How many times did the cleaning behavior evolve? \textsc{Hint:} Trace all species with the same behavior back to their most distance common ancestor. That is when the behavior most likely evolved.

\AnswerBox{0.5\baselineskip}{Once}

\question
How many times did the sponge-dwelling behavior evolve?

\AnswerBox{0.5\baselineskip}{Once}

\question
How many times did the hovering behavior evolve?

\AnswerBox{0.5\baselineskip}{Twice}

\hfill \textbf{Questions continue on page~\pageref{questions_continued}.}

\newpage

	\ifprintanswers
		\includegraphics[height=0.75\textheight]{06_elacatinus_phylogeny_key}\label{phylogeny}
	\else
		\includegraphics[height=0.75\textheight]{06_elacatinus_phylogeny}\label{phylogeny}
	\fi
	
\newpage

% Table and Phylogeny here.
\subsubsection*{\textit{Elacatinus} behavior and stripe colors}


Match behavior and stripe colors to the species on the phylogeny. Be sure to
match the proper populations where shown. Populations are shown only for
species with different colors for different populations.

\label{color_behavior}
\begin{tabular}[c]{@{}llll@{}}
	\toprule
	\textit{Elacatinus}		&	Population		&	Behavior		&	Stripe Color \tabularnewline
	\midrule
	\textit{atronasus}		&					&	Hovering		&	Yellow	\tabularnewline
	\textit{chancei} 		&	North Caribbean	&	Sponge-dweller	&	Yellow	\tabularnewline
	\textit{chancei} 		&	South Caribbean	&	Sponge-dweller	&	Yellow	\tabularnewline
	\textit{evelynae} 		&	Bahamas			&	Cleaner			&	Yellow	\tabularnewline
	\textit{evelynae} 		&	East Caribbean	&	Cleaner			&	Blue 	\tabularnewline
	\textit{evelynae} 		&	West Caribbean	&	Cleaner			&	White	\tabularnewline
	\textit{figaro}			&					&	Cleaner			&	Yellow	\tabularnewline
	\textit{genie}			&					&	Cleaner			&	White	\tabularnewline
	\textit{horsti} 		&	Grand Cayman	&	Sponge-dweller	&	Yellow	\tabularnewline
	\textit{horsti} 		&	South Caribbean	&	Sponge-dweller	&	Yellow	\tabularnewline
	\textit{horsti} 		&	Bahamas			&	Sponge-dweller	&	Yellow	\tabularnewline
	\textit{horsti} 		&	Jamaica			&	Sponge-dweller	&	White	\tabularnewline
	\textit{illecebrosus}	&	Columbia		&	Cleaner			&	Blue	\tabularnewline
	\textit{illecebrosus} 	&	Panama			&	Cleaner			&	Yellow	\tabularnewline
	\textit{jarocho}		&					&	Hovering		&	Yellow	\tabularnewline
	\textit{lori}			&					&	Sponge-dweller	&	White	\tabularnewline
	\textit{louisae}		&					&	Sponge-dweller	&	Yellow	\tabularnewline
	\textit{oceanops}		&					&	Cleaner			&	Blue	\tabularnewline
	\textit{prochilos}		&					&	Cleaner			&	White	\tabularnewline
	\textit{randalli}		&					&	Cleaner			&	Yellow	\tabularnewline
	\textit{xanthiprora}	&					&	Sponge-dweller	&	Yellow	\tabularnewline
	\bottomrule
\end{tabular}

\textbf{You can separate the two pages to more easily match traits to the phylogeny.}

\newpage

\question \label{questions_continued}
Assuming that lateral stripe color evolved in the common ancestor
of \emph{all} of these species (the root of the tree), what is the most likely ancestral lateral
stripe color? Explain. %\textsc{Hint:} Which color is basal in each of the two main behavioral groups?

\AnswerBox{3\baselineskip}{Yellow. Most species have it so their ancestor probably had it.}

\question
How many times did each of the other colors evolve independently?

\AnswerBox{2\baselineskip}{Blue: 3 Times. \quad White: 5 times}


\question[Checkout]
Do you think northern and southern populations of 
\textit{Elacatinus chancei} should be considered a single species?
Explain.

\AnswerBox{3\baselineskip}{Probably different species but let students defend own decision. 
If guidance sought, point out that \textit{E. chancei} (both pops) has a stripe restricted to the head
but \textit{E. horsti} has a stripe that runs the full body length. Both pops of \textit{E. chancei} are also
genetically distinct. Would gene flow allow them to be genetically different?}

\question[Checkout]
Identify one other species shown in the phylogenetic tree
that could be treated as actually being more than one species. Explain why you think so.

\AnswerBox{3\baselineskip}{\textit{horsti, illecebrosus, evelynae} are most likely choices. 
The populations differ by color. You can add that the populations are genetically different, if 
they want a little guidance. Lead them by asking if gene flow between pops would allow them
to be genetically different.}


%When the fishes are preserved for study, the color of the stripe disappears. Without the color
%individuals from different populations cannot be distinguished. As a result, 
%populations with different stripe colors have always been treated as just one species. 
%Genetic studies, however, suggest the populations are actually different species.
%
%\question
%What is the term for different species that are morphologically similar but genetically different?
%
%\AnswerBox{\baselineskip}{Cryptic species.}

\end{questions}

\end{document}  