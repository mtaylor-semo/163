%!TEX TS-program = lualatex
%!TEX encoding = UTF-8 Unicode

\documentclass[12pt]{exam}
\usepackage{graphicx}
	\graphicspath{{/Users/goby/Pictures/teach/163/lab/}
	{img/}} % set of paths to search for images

\usepackage{geometry}
\geometry{letterpaper, left=1.5in, bottom=1in}                   
%\geometry{landscape}                % Activate for for rotated page geometry
\usepackage[parfill]{parskip}    % Activate to begin paragraphs with an empty line rather than an indent
\usepackage{amssymb, amsmath}
\usepackage{mathtools}
	\everymath{\displaystyle}

\usepackage{fontspec}
\setmainfont[Ligatures={TeX}, BoldFont={* Bold}, ItalicFont={* Italic}, BoldItalicFont={* BoldItalic}, Numbers={OldStyle}]{Linux Libertine O}
\setsansfont[Scale=MatchLowercase,Ligatures=TeX]{Linux Biolinum O}
\setmonofont[Scale=MatchLowercase]{Inconsolatazi4}
\usepackage{microtype}


% To define fonts for particular uses within a document. For example, 
% This sets the Libertine font to use tabular number format for tables.
 %\newfontfamily{\tablenumbers}[Numbers={Monospaced}]{Linux Libertine O}
% \newfontfamily{\libertinedisplay}{Linux Libertine Display O}

\usepackage{booktabs}
\usepackage{multicol}
\usepackage[normalem]{ulem}

\usepackage{longtable}
%\usepackage{siunitx}
\usepackage{array}
\newcolumntype{L}[1]{>{\raggedright\let\newline\\\arraybackslash\hspace{0pt}}p{#1}}
\newcolumntype{C}[1]{>{\centering\let\newline\\\arraybackslash\hspace{0pt}}p{#1}}
\newcolumntype{R}[1]{>{\raggedleft\let\newline\\\arraybackslash\hspace{0pt}}p{#1}}

\usepackage{enumitem}
\setlist{leftmargin=*}
\setlist[1]{labelindent=\parindent}

\usepackage{hyperref}
%\usepackage{placeins} %PRovides \FloatBarrier to flush all floats before a certain point.
\usepackage{hanging}

\usepackage[sc]{titlesec}

%% Commands for Exam class
\renewcommand{\solutiontitle}{\noindent}
\unframedsolutions
\SolutionEmphasis{\bfseries}

\renewcommand{\questionshook}{%
	\setlength{\leftmargin}{-\leftskip}%
}

%Change \half command from 1/2 to .5
\renewcommand*\half{.5}

\pagestyle{headandfoot}
\firstpageheader{\textsc{bi}\,063 Evolution and Ecology}{}{\ifprintanswers\textbf{KEY}\else Name: \enspace \makebox[2.5in]{\hrulefill}\fi}
\runningheader{}{}{\footnotesize{pg. \thepage}}
\footer{}{}{}
\runningheadrule

\newcommand*\AnswerBox[2]{%
    \parbox[t][#1]{0.92\textwidth}{%
    \begin{solution}#2\end{solution}}
%    \vspace*{\stretch{1}}
}

\newenvironment{AnswerPage}[1]
    {\begin{minipage}[t][#1]{0.92\textwidth}%
    \begin{solution}}
    {\end{solution}\end{minipage}
    \vspace*{\stretch{1}}}

\newlength{\basespace}
\setlength{\basespace}{5\baselineskip}


%\usepackage{mdframed}
%\mdfsetup{%
%	innerleftmargin=0pt,%
%	innerrightmargin=0pt,
%	innertopmargin=0pt,
%	innerbottommargin=0pt,
%	hidealllines=true
%}%end mdfsetup

%
%\makeatletter
%\def\SetTotalwidth{\advance\linewidth by \@totalleftmargin
%\@totalleftmargin=0pt}
%\makeatother


%\printanswers


\begin{document}

\subsection*{Journal 3: evaluating all of the evidence}

\emph{Read this entire handout throughly. You are responsible for meeting all requirements of this assignment, as detailed below.}

This assignment requires both a journal entry and your final
phylogenetic tree. This is the culmination of all of the evidence 
provided in lab  to support various homological features between 
different organisms.  Do you think \emph{all} of this evidence supports 
or falsifies your hypothesis? If it supports your hypothesis, tell how. If it
falsifies your hypothesis, tell how. Then, submit your final revised
hypothesis and tree.

If you are unclear about how to interpret any of the evidence, please 
contact your lab instructor before submitting this entry. Your instructor can 
review your previous hypothesis and guide you towards proper interpretation of
the evidence without giving you the answer. If you aren't sure whether an organism is an amphibian, reptile,
eukaryote, prokaryote, etc., use the Taxonomy Home Page
(http://www.ncbi.nlm.nih.gov/Taxonomy/) and the first exercise to review
this information.

The genetic evidence you have examined to test 
your hypothesis is listed below.  Also listed is the component point value 
for this assignment.

\begin{enumerate}

\item Universal genetic code (10 points).

	Homologous for all organisms in your hypothesis, including bacteria.
  
\item Cytochrome C data (10 points). 

	Homology for all eukaryotes (all organisms in your hypothesis 
	except bacteria). Provides homological evidence for relationships 
	among all organisms based on variable amino acids.  Additional 
	evidence includes the number of amino acid differences between 
	different organisms and the molecular clock 

\item Genetic distance between primates (5 points).

\item Geological time (10 points).

	Carefully consider the order of appearance of the different organismal 
	groups (e.g., bacteria, fishes, etc.) in   your hypothesis. This will be 
	critical to getting a correct hypothesis. Be sure to discuss whether 
	your previous hypothesis  correctly predicted the when the 
	different groups of organisms appeared.

\end{enumerate}

In each case, be sure to explain \emph{in depth} what is the specific
evidence, how it was evaluated, and how you concluded that it provided
evidence of the appropriate relationships.

The following evidence \emph{does not need to be discussed} at all in
your journal assignment, but your final phylogenetic tree must be
consistent with \emph{all} of the evidence listed above and below.

\begin{enumerate}[resume]

\item Anatomical homologies: skeletons and leaves.
  
\item Embryological and larval homologies: tail and pharyngeal arches
  are homologous for all vertebrates. Homologous trochophore larvae for
  marine worm and land snail.
  
\item Transitional forms in agreement with the geological time
  scale.
  
\end{enumerate}





\subsubsection*{Assignment}

Before you begin, review all the assignments you completed throughout
the semester.   
Compare the evidence from the lab exercises to the predictions 
made by your phylogenetic trees. Does the evidence support your
hypothesis? Falsify your hypothesis? Give no conclusive evidence? 
Does the evidence require you to change your hypothesis in any way? 
Refer to the Scientific Method Overview for how to use evidence to test 
hypotheses.

Some parts of your hypothesis may be supported and
some parts may be falsified by the evidence from transitional forms.
You must discuss both supported and unsupported parts of your
hypothesis. For supported parts, you must discuss which parts of your
hypothesis are supported and how the evidence supports your hypothesis.
For falsified parts of your hypothesis, you must explain which parts are
not supported, how the evidence falsifies your hypothesis, and then
explain how your hypothesis must be revised to fit the evidence examined
so far. You need only address the vertebrates (animals with a
backbone) for the written journal entry. Your new phylogenetic tree (see
below) must still include all 21 organisms.

\emph{Clarity of your written thoughts is critical for your journal entry. Writing that is not clear is evidence of
 thinking that is not clear and will be evaluated accordingly.}


\subsubsection*{Grading of Journal 3}

\begin{enumerate}

\item Evaluation of the evidence (35 points).

Each type of genetic evidence that we evaluated in class
has been given a point value (see the evidence above). In addition, the proper order
of the organisms appearing in the fossil record must also be
discussed. The points will be assigned based on inclusion and thorough
discussion of the evidence, and how the evidence affected your
hypothesis. Did the evidence support the predictions made by your
hypothesis? Did the evidence falsify your hypothesis? If the evidence
falsified your hypothesis, then you must state how you revised your
hypothesis to agree with the evidence. Clarity counts towards the total
points in each category.

\item Spelling, grammar, and mechanics (5 points).

Spelling, grammar and mechanics (sentence structure, etc) are 
important in any writing but especially informal writing. Use spell 
check. Proof your assignment carefully before you turn it in. You 
are entitled to \emph{one free mistake.} After that, each mistake 
is a 1 point deduction.

\emph{Remember}: This course is part of the University Studies
program. The University Studies program objectives that are especially
relevant to this course are 1) the demonstration of your ability for
critical thinking, reasoning and analyzing, and 2) the demonstration of
effective written communication. Keep this in mind as you write. Points will be deducted for signs of
weak reasoning and analysis and poorly written communication.

\item New phylogenetic tree (20 points).

You must also submit a final, revised phylogenetic tree with this assignment. 
Your final phylogenetic tree must be consistent with \emph{every} piece of
evidence evaluated, including the anatomical evidence, the transitional
form evidence, and the genetic. It must also be consist with what you write
for your journal entry assignment above. You will lose 3
points for each missed piece of genetic evidence, homology or transitional form, or disagreements
 with the geological time scale. You will also lose 3 points for each missing organism,
a missing or incomplete time scale, or any other common errors. \emph{See the first phylogenetic tree exercise for things you should not do.}

\end{enumerate}

\subsubsection*{Due Date} 

Your journal entry and new phylogenetic tree are due at the start of lab next week. Late submissions will not be accepted without a valid excuse. If necessary, review the syllabus for lab policy.

Please do not type a cover page. Just put your name and \textsc{bi} 063 at the
top, and then start typing your journal assignment.

\end{document}  