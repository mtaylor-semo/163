%!TEX TS-program = lualatex
%!TEX encoding = UTF-8 Unicode

%\documentclass[t]{beamer}

%%%% HANDOUTS For online Uncomment the following four lines for handout
\documentclass[t,handout]{beamer}  %Use this for handouts.
%\usepackage{handoutWithNotes}
%\pgfpagesuselayout{3 on 1 with notes}[letterpaper,border shrink=5mm]
%	\setbeamercolor{background canvas}{bg=black!5}

\includeonlylecture{student}

%%% Including only some slides for students.
%%% Uncomment the following line. For the slides,
%%% use the labels shown below the command.

%% For students, use \lecture{student}{student}
%% For mine, use \lecture{instructor}{instructor}



% FONTS
\usepackage{fontspec}
\def\mainfont{Linux Biolinum O}
\setmainfont[Ligatures={Common,TeX}, BoldFont={* Bold}, ItalicFont={* Italic}, Numbers={Proportional, OldStyle}]{\mainfont}
\setsansfont[Ligatures={Common,TeX}, Scale=MatchLowercase, Numbers={Proportional,OldStyle}, BoldFont={* Bold}, ItalicFont={* Italic},]{Linux Biolinum O} 
\usepackage{microtype}

\usepackage{amsmath,amssymb}
%\usepackage{unicode-math}
%\setmathfont[Scale=MatchLowercase]{TeX Gyre Termes Math}

\usepackage{graphicx}
	\graphicspath{{/Users/goby/pictures/teach/163/case_studies/}
	{/Users/goby/pictures/teach/163/lecture/}
	{/Users/goby/pictures/teach/common/}} % set of paths to search for images

\usepackage{color}
\definecolor{biolum}{RGB}{111,253,254}

\usepackage{multicol}
\usepackage{longtable}
\usepackage{booktabs}
\usepackage{array}
\newcolumntype{L}[1]{>{\raggedright\let\newline\\\arraybackslash\hspace{0pt}}p{#1}}
\newcolumntype{C}[1]{>{\centering\let\newline\\\arraybackslash\hspace{0pt}}p{#1}}
\newcolumntype{R}[1]{>{\raggedleft\let\newline\\\arraybackslash\hspace{0pt}}p{#1}}
%\usepackage{textcomp}
%\usepackage{mhchem}
\usepackage{enumitem}
%\usepackage[export]{adjustbox}


%\usepackage{tikz}
%	\tikzstyle{every picture}+=[remember picture,overlay]
%	\usetikzlibrary{arrows}
%\usetikzlibrary{positioning}

\mode<presentation>
{
  \usetheme{Lecture}
  \setbeamercovered{invisible}
  \setbeamertemplate{items}[default]
}


\begin{document}

%
\begin{frame}[t]{How many killer whales would it take to eat 40,000 otters in six years? (extra credit)}

	\includegraphics[width=\textwidth]{missing_otter2_orca_seal}

\end{frame}
%
\begin{frame}[t]{How many killer whales would it take to eat 40,000 otters in six years? (extra credit)}

	\hangpara For otter energy provided, use \newline	
	\hspace*{1em} an average otter weight of 29 kg (29,000 g), and \newline
	\hspace*{1em} 1.7 kilocalories of energy per gram of otter per day.

	\hangpara For killer whale energy requirements, use \newline
	\hspace*{1em} an average killer whale weight of 4,000 kg, and \newline
	\hspace*{1em} 54 kilocalories of energy per kilogram of killer whale per day.

\end{frame}
%
\lecture{instructor}{instructor}

\begin{frame}[t]{An average otter has 49,300 kilocalories of energy.}

	\begin{multicols}{2}
	
	\hangpara Average otter weight = 29,000 g.
	
	\hangpara Energy = 1.7 kcal / gram.

	\hangpara 29,000 g $\times$ 1.7 kcal/g = 49,300 kcal.

	\columnbreak
	
		\includegraphics[width=0.45\textwidth]{missing_otter3_swarm}
		
	\end{multicols}
	
\end{frame}
%
\begin{frame}[t]{An average killer whale needs 216,000 kilocalories of energy.}

	\begin{multicols}{2}
	
	\hangpara Average killer whale weight = 4,000 kg.
	
	\hangpara Energy needs = 54 kcal / kg

	\hangpara 4,000 kg $\times$ 54 kcal/kg = 216,000 kcal.

	\columnbreak
	
		\includegraphics[width=0.45\textwidth]{missing_otter3_kw_eating}
		
	\end{multicols}
	
\end{frame}
%
\begin{frame}[t]{Only 4--5 killer whales can cause the sea otter population decline by 40,000 individuals in six years.}

	\vspace*{-\baselineskip}
	
	\hangpara How many otters per day would one killer whale need?
		
		\[216,000\ \mathrm{kcal} / 49,300\ \mathrm{kcal} = 4.38\ \mathrm{otters} \approx 4\mathrm{–}5\ \mathrm{otters}\]
		
	\pause
	\hangpara How many otters could one killer whale eat in six years?
	\begin{align*}4 \times 365 \times 6 &= 8,760\ \mathrm{otters}\\
	5 \times 365 \times 6 &= 10,950\ \mathrm{otters}\end{align*}

	\pause
	\hangpara How many killer whales could 40,000 otters in six years?
	\begin{align*}40,000 / 8,760 &= 4.57 \approx 5\ \mathrm{killer\ whales.}\\
	40,000 / 10,950 &= 3.65 \approx 4\ \mathrm{killer\ whales.}\end{align*}

\end{frame}
%
\lecture{student}{student}

\begin{frame}[t]{Why did killer whales suddenly start eating sea otters?}

	\hangpara Otters and killer whales coexisted for thousands of years without any predation.

	\hangpara Killer whales only recently switched to otters as prey.

	\hangpara Could the decline of otters have further effects on the marine community in southern Alaska?

\end{frame}
%
{
\usebackgroundtemplate{\includegraphics[width=\paperwidth]{missing_otter3_otter_and_urchins} }
\begin{frame}[b]{\textcolor{white}{I am essential to the community. Now, let me eat.}}
%	\hfill \tiny Edward Rooks, Flickr Creative Commons.
\end{frame}
}
%
{
\usebackgroundtemplate{\includegraphics[width=\paperwidth]{missing_otter3_red_sea_urchin} }
\begin{frame}[b]
	\hfill \tiny \textcolor{white}{Kirk L. Onthank, Wikimedia, \ccby{3}}
\end{frame}
}
%
{
\usebackgroundtemplate{\includegraphics[width=\paperwidth]{missing_otter3_kelp_forest} }
\begin{frame}[b]
	\hfill \tiny \textcolor{white}{Peter Southwood, Wikimedia, \ccbysa{3}}
\end{frame}
}
%
{
\usebackgroundtemplate{\includegraphics[width=\paperwidth]{missing_otter3_urchin_front} }
\setbeamercolor{background canvas}{bg=black}
\begin{frame}[b]
	 \tiny \textcolor{white}{\copyright\ National Geographic}
\end{frame}
}
%
{
\usebackgroundtemplate{\includegraphics[width=\paperwidth]{missing_otter3_kelp_graphs} }
\begin{frame}[b]
	\tiny Based on Estes and Duggins 1995. Ecological Monographs 65: 75--100.
\end{frame}
}
%
{
\usebackgroundtemplate{\includegraphics[width=\paperwidth]{missing_otter3_algae_graphs} }
\begin{frame}[b]
	\tiny Based on Estes and Duggins 1995. Ecological Monographs 65: 75--100.
\end{frame}
}
%
\begin{frame}[t]{Sea otters are needed for healthy kelp forest communities.}

\vspace*{-0.5\baselineskip}

%\begin{longtable}{@{}L{25mm}C{18mm}C{18mm}C{18mm}C{18mm}@{}}
\begin{longtable}{@{}L{0.21\textwidth}C{0.15\textwidth}C{0.15\textwidth}C{0.15\textwidth}C{0.15\textwidth}@{}}
\toprule
	& \multicolumn{2}{C{0.3\textwidth}}{Amchitka Is.\newline (Otters Present)} & \multicolumn{2}{C{0.3\textwidth}}{Shemya Is.\newline (Otters Absent)}\tabularnewline
\cmidrule(r){2-3} \cmidrule(l){4-5}
	&	1972	&	1987	&	1972	& 1987 \tabularnewline
\midrule	 
Kelp Density (inds./0.25 m2) & 5.1\,±\,0.7 & 4.7\,±\,1.2 & 0 & 0.5\tabularnewline[2em]
Urchin Density (inds./0.25 m\textsuperscript{2}) & 27.9\,±\,14.4 & 23.4\,±\,7.5 & 50\,±\,14.6 & 38.6\,±\,1.4\tabularnewline[2em]
Urchin Biomass (g/0.25 m\textsuperscript{2}) & 45.1\,±\,16.9 & 36.7\,±\,15.0 & 368.2\,±\,151.7 & 369\,±\,14.3\tabularnewline
\bottomrule
\end{longtable}

\vspace*{-1.5\baselineskip}

\hangpara Fewer otters in the kelp forest allowed the population size of sea urchins to increase.  More sea urchins ate more kelp so kelp density decreased.  This is called a \highlight{trophic cascade.}

\end{frame}
%
{
\usebackgroundtemplate{\includegraphics[width=\paperwidth]{missing_otter3_trophic_cascade} }
\begin{frame}[t]

	\vspace*{4\baselineskip}
	
	\hangpara \parbox{0.35\textwidth}{\raggedright%
	A \highlight{trophic cascade} occurs when the abundance of a predator at one trophic level affects the abundance of organisms at lower trophic levels.	
	}

	\vfilll
	
	\tiny \copyright\ McGraw-Hill.
\end{frame}
}
%
{
\usebackgroundtemplate{\includegraphics[width=\paperwidth]{missing_otter3_diversity} }
\begin{frame}[b]

	\textcolor{white}{\tiny (t): Mike Liu. (b): Ed Bierman. \hfill All: Flickr \ccbyncsa{2} \hfill (t): Ed Bierman. (b): Brenna Green. }
\end{frame}
}
%
\end{document}


% kelp diver Ed Bierman Flickr \ccby{2}
% sea slug Ed Bierman Flickr \ccby{2}

% Garibaldi Mike Liu, Flickr \ccbysa{2}

% Kelp bass Briant Gratwicke Flickr \ccby{2}

% Snail on batstar Brenna Green, Flickr \ccbncsa{2}