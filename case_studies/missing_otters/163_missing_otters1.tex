%!TEX TS-program = lualatex
%!TEX encoding = UTF-8 Unicode

\documentclass[t]{beamer}

%%%% HANDOUTS For online Uncomment the following four lines for handout
%\documentclass[t,handout]{beamer}  %Use this for handouts.
%\usepackage{handoutWithNotes}
%\pgfpagesuselayout{3 on 1 with notes}[letterpaper,border shrink=5mm]
%	\setbeamercolor{background canvas}{bg=black!5}

%\includeonlylecture{student}

%%% Including only some slides for students.
%%% Uncomment the following line. For the slides,
%%% use the labels shown below the command.

%% For students, use \lecture{student}{student}
%% For mine, use \lecture{instructor}{instructor}



% FONTS
\usepackage{fontspec}
\def\mainfont{Linux Biolinum O}
\setmainfont[Ligatures={Common,TeX}, Contextuals={NoAlternate}, BoldFont={* Bold}, ItalicFont={* Italic}, Numbers={Proportional, OldStyle}]{\mainfont}
\setsansfont[Ligatures={Common,TeX}, Scale=MatchLowercase, Numbers={Proportional,OldStyle}, BoldFont={* Bold}, ItalicFont={* Italic},]{Linux Biolinum O} 
\usepackage{microtype}

\usepackage{amsmath,amssymb}
%\usepackage{unicode-math}
%\setmathfont[Scale=MatchLowercase]{TeX Gyre Termes Math}

\usepackage{graphicx}
	\graphicspath{{/Users/goby/pictures/teach/163/case_studies/}
	{/Users/goby/pictures/teach/163/lecture/}
	{/Users/goby/pictures/teach/common/}} % set of paths to search for images

\usepackage{color}
\definecolor{biolum}{RGB}{111,253,254}

\usepackage{multicol}
\usepackage{booktabs}
\usepackage{array}
\newcolumntype{L}[1]{>{\raggedright\let\newline\\\arraybackslash\hspace{0pt}}p{#1}}
\newcolumntype{C}[1]{>{\centering\let\newline\\\arraybackslash\hspace{0pt}}p{#1}}
\newcolumntype{R}[1]{>{\raggedleft\let\newline\\\arraybackslash\hspace{0pt}}p{#1}}
%\usepackage{textcomp}
%\usepackage{mhchem}
\usepackage{enumitem}
%\usepackage[export]{adjustbox}


\usepackage{tikz}
	\tikzstyle{every picture}+=[remember picture,overlay]
	\usetikzlibrary{arrows}
\usetikzlibrary{positioning}

\mode<presentation>
{
  \usetheme{Lecture}
  \setbeamercovered{invisible}
  \setbeamertemplate{items}[default]
}


\begin{document}


\lecture{student}{student}
\begin{frame}{The case of the missing sea otters.}

	\centering\includegraphics[height=0.8\textheight]{missing_otter1_intro}
	
\end{frame}
%
{\setbeamercolor{background canvas}{bg=black}
\begin{frame}[t]{The case of the missing sea otters.}

	\includegraphics[width=\textwidth]{missing_otter1_location}
	
	\begin{tikzpicture}
	
	\node [yellow] (here) at (10.5,4.2) {\small You are here!};
	
	\draw [thick, yellow, ->] (here) -- (10.8,3.45);
	
	\node [yellow] (study) at (4.4,3.6) {Study area};
	
	\draw [ultra thick, yellow] (1.8,4.6) rectangle (2.7,5.5) ;;
	
	\draw [yellow, thick, ->] (study.north west) -- (2.75,4.55);
	\end{tikzpicture}
	
\end{frame}
}
%
\begin{frame}

	\centering\includegraphics[height=0.95\textheight]{missing_otter1_sites}
	
\end{frame}
%
{
\usebackgroundtemplate{\includegraphics[width=\paperwidth]{missing_otter1_decline} }
\begin{frame}[b]{How has the size of the sea otter population changed?}
	\hfill \tiny \textcolor{white}{J.J. Harrison, Wikimedia, \ccbysa{3}}
\end{frame}
}
%
\begin{frame}[t]{What could have caused the otter population to decline?}

	\begin{multicols}{2}
	
	\includegraphics[width=0.5\textwidth]{missing_otter1_decline_small}

	\columnbreak
	
	What are the four factors that determine whether 
	a population is getting larger or smaller?

	\vspace*{\baselineskip}

	In groups (3–4 students), write four hypotheses that might 
	explain otter population decline. Write one hypothesis
	for each factor. 

	\vspace*{\baselineskip}

	Be specific. For example, do not write “Fewer 
	individuals are giving birth.”  What do you think 
	could be specific causes of decreased birth rates?

	\end{multicols}
		
\end{frame}
%
{
\setbeamercolor{background canvas}{bg=black}
\begin{frame}{\phantom{phantom}}
	\begin{multicols}{2}
	
		\hangpara\textcolor{white}{Four factors that determine population size.}
	
	\columnbreak
	
		\hangpara\textcolor{white}{Possible causes of decline.}
	
	\end{multicols}
\end{frame}
}
%
\begin{frame}[t]
	\begin{multicols}{2}
	
		Based on this article, which potential causes 
		of population decline can we eliminate?
		
		\vspace*{0.65\baselineskip}
		
		\includegraphics[width=0.5\textwidth]{missing_otter1_intro}

	\columnbreak
	
		\includegraphics[width=0.5\textwidth]{missing_otter1_article}

	\end{multicols}
	
	\vfilll
	
	\tiny Excerpted from Stevens, W.K. Search for missing sea otters turns up a few surprises. New York Times, January 5, 1999.

\end{frame}
%
\begin{frame}[t]{Orca don't eat otters\dots right? \hfill Right\char"203D}

	\includegraphics[width=\textwidth]{missing_otter1_otter_orca}

	\vfilll
	
	\hfill \tiny Is this thing on?
\end{frame}
%
\end{document}
