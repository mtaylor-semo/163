%!TEX TS-program = lualatex
%!TEX encoding = UTF-8 Unicode

\documentclass[t]{beamer}

%%%% HANDOUTS For online Uncomment the following four lines for handout
%\documentclass[t,handout]{beamer}  %Use this for handouts.
%\usepackage{handoutWithNotes}
%\pgfpagesuselayout{3 on 1 with notes}[letterpaper,border shrink=5mm]

%\includeonlylecture{student}

%%% Including only some slides for students.
%%% Uncomment the following line. For the slides,
%%% use the labels shown below the command.

%% For students, use \lecture{student}{student}
%% For mine, use \lecture{instructor}{instructor}


%\usepackage{pgf,pgfpages}
%\pgfpagesuselayout{4 on 1}[letterpaper,border shrink=5mm]

% FONTS
\usepackage{fontspec}
\def\mainfont{Linux Biolinum O}
\setmainfont[Ligatures={Common,TeX}, Contextuals={NoAlternate}, BoldFont={* Bold}, ItalicFont={* Italic}, Numbers={OldStyle}]{\mainfont}
\setsansfont[Ligatures={Common,TeX}, Scale=MatchLowercase, Numbers=OldStyle]{Linux Biolinum O} 
\usepackage{microtype}

\usepackage{graphicx}
	\graphicspath{{/Users/goby/pictures/teach/163/case_studies/}}

%\usepackage{units}
\usepackage{booktabs}
%\usepackage{textcomp}

\usepackage{tikz}
	\tikzstyle{every picture}+=[remember picture,overlay]
	\usetikzlibrary{arrows}

\mode<presentation>
{
  \usetheme{Lecture}
  \setbeamercovered{invisible}
  \setbeamertemplate{items}[square]
}


\begin{document}

{
\usebackgroundtemplate{\includegraphics[width=\paperwidth]{columbine_intro}}
\begin{frame}[t]{\textcolor{white}{Reproductive isolation in columbines.\newline{\small Written by J. Phil Gibson, University of Oklahoma. Modified by Michael S. Taylor.}}}

\vfilll

\tiny \textcolor{white}{KimonBerlin, Wikimedia \ccbysa{2}}
\end{frame}
}
%
%{
%\usebackgroundtemplate{\includegraphics[width=\paperwidth]{sunflowers}
%}
%\begin{frame}[b,plain]
%	\hfill \tiny \textcolor{white}{Trey Ratcliff, Flickr, \ccbyncsa{2}}
%\end{frame}
%}
%
%{
%\usebackgroundtemplate{\includegraphics[width=\paperwidth]{spider_bee}
%}
%\begin{frame}[b,plain]
%\tiny Alvesgaspar, Wikimedia, \ccbysa{4}
%\end{frame}
%}
%
%{
%\usebackgroundtemplate{\includegraphics[width=\paperwidth]{marine_iguana}}
%\begin{frame}[b,plain]
%	\textcolor{white}{\tiny Les Williams, Flickr, \ccbysa{2}}
%\end{frame}
%}
%
%{
%\usebackgroundtemplate{\includegraphics[width=\paperwidth]{bird_paradise.jpg}}
%\begin{frame}[b,plain]
%	\tiny\textcolor{white}{\href{http://www.youtube.com/watch?v=KIYkpwyKEhY}{Link to Video} \hfill \copyright\,Tim Laman, All Rights Reserved}
%\end{frame}
%}

%% Genotype and phenotype
%%\lecture{instructor}{instructor}
%{
%\usebackgroundtemplate{\includegraphics[width=\paperwidth]{chromosomes}}
%\begin{frame}[c,plain]
%	\begin{tikzpicture}[remember picture, overlay]
%
%	\alt<handout>{}{\visible <1,6->{
%		\node at (0.45,2) [right] {The \highlight{genotype} is the genetic makeup of the organism.};}
%	
%	\visible <7>{
%		\node at (0.45,1.4)[right] {The \highlight{phenotype} is the physical expression of the genotype.};
%	}}
%		
%	
%	\alt<handout>{}{\visible <2>{
%		\draw (3,-2.5) -- (3,0.5) -- (3.9,0.5) -- (3.9,-2.5) -- cycle ;
%		\draw (6,-2.5) -- (6,0.5) -- (6.9,0.5) -- (6.9,-2.5) -- cycle ;
%		\draw (9,-2.5) -- (9,0.5) -- (9.9,0.5) -- (9.9,-2.5) -- cycle ;
%	
%		\node at (6.4, 2) (gene) {Gene or Locus};
%	
%		\draw (gene.south west) -- (3.9,0.5);
%		\draw (gene.south) -- (6.45,0.5);
%		\draw (gene.south east) -- (9,0.5);
%	}}
%	
%	\visible <3-4>{
%		\node at (3.45,0.25) (Callele) {$C$};
%		\node at (3.45,-2.25) {$C$};
%
%		\node at (6.45,0.25) (aallele) {$a$};
%		\node at (6.45,-2.25) {$a$};
%	}
%
%	\visible <3,5>{
%		\node at (9.45,0.25) (Tallele) {$T$};
%		\node at (9.45,-2.25) {$t$};
%	}
%	
%	\alt<handout>{}{\visible <3>{
%		\node at (6.4, 2) (allele) {Allele};
%		\draw (allele.south west) -- (Callele.north east);
%		\draw (allele.south) -- (aallele.north);
%		\draw (allele.south east) -- (Tallele.north west);
%	}}
%
%	\visible <4> {
%		\draw (3,-2.5) -- (3,0.5) -- (3.9,0.5) -- (3.9,-2.5) -- cycle ;
%		\draw (6,-2.5) -- (6,0.5) -- (6.9,0.5) -- (6.9,-2.5) -- cycle ;
%
%		\node at (3.45, 1.25) (hc) {\highlight{Homozygous,} $C$ allele};
%		\node at (6.45, -3.5) (ac) {Homozygous, $a$ allele};
%
%		\draw (hc.south) -- (3.45,0.5);
%		\draw (ac.north) -- (6.45,-2.5);
%	}
%	
%	\visible <5> {
%		\draw (9,-2.5) -- (9,0.5) -- (9.9,0.5) -- (9.9,-2.5) -- cycle ;
%
%		\node at (9.45, 1.25) (tc) {\highlight{Heterozygous,} $T$ allele};
%
%		\draw (tc.south) -- (9.45,0.5);
%	}
%	\end{tikzpicture}
%\end{frame}
%}


\end{document}
