%!TEX TS-program = lualatex
%!TEX encoding = UTF-8 Unicode

%\documentclass[t,hidelinks]{beamer}

%%%% HANDOUTS For online Uncomment the following four lines for handout
\documentclass[t,handout]{beamer}  %Use this for handouts.
%\usepackage{handoutWithNotes}
%\pgfpagesuselayout{3 on 1 with notes}[letterpaper,border shrink=5mm]

%\includeonlylecture{student}

%%% Including only some slides for students.
%%% Uncomment the following line. For the slides,
%%% use the labels shown below the command.

%% For students, use \lecture{student}{student}
%% For mine, use \lecture{instructor}{instructor}


%\usepackage{pgf,pgfpages}
%\pgfpagesuselayout{4 on 1}[letterpaper,border shrink=5mm]

% FONTS
\usepackage{fontspec}
\def\mainfont{Linux Biolinum O}
\setmainfont[Ligatures={Common,TeX}, Contextuals={NoAlternate}, BoldFont={* Bold}, ItalicFont={* Italic}, Numbers={OldStyle}]{\mainfont}
\setsansfont[Ligatures={Common,TeX}, Scale=MatchLowercase, Numbers=OldStyle]{Linux Biolinum O} 
\usepackage{microtype}

\usepackage{graphicx}
	\graphicspath{{/Users/goby/pictures/teach/163/case_studies/}}

%\usepackage{units}
\usepackage{booktabs}
\usepackage{enumitem}
	\setlist[enumerate]{label = \textsc{\alph*.}}
\usepackage{multicol}

\usepackage{array}
\newcolumntype{L}[1]{>{\raggedright\let\newline\\\arraybackslash\hspace{0pt}}p{#1}}
\newcolumntype{C}[1]{>{\centering\let\newline\\\arraybackslash\hspace{0pt}}p{#1}}
\newcolumntype{R}[1]{>{\raggedleft\let\newline\\\arraybackslash\hspace{0pt}}p{#1}}

\newcommand{\ques}[1]{\highlight{\textsc{q#1:}}}

\usepackage{tikz}
	\tikzstyle{every picture}+=[remember picture,overlay]
	\usetikzlibrary{arrows.meta}
%	\usetikzlibrary{trees}


\mode<presentation>
{
  \usetheme{Lecture}
  \setbeamercovered{invisible}
  \setbeamertemplate{items}[default]
}


\begin{document}

{
\usebackgroundtemplate{\includegraphics[width=\paperwidth]{columbine_intro}}
\begin{frame}[t]{\textcolor{white}{Reproductive isolation in columbines.\newline{\small Written by J. Phil Gibson, University of Oklahoma. Modified by Michael S. Taylor.}}}

\vfilll

\tiny \textcolor{white}{KimonBerlin, Wikimedia \ccbysa{2}}
\end{frame}
}
%
\begin{frame}[t]{What will this case study address?}

	\begin{multicols}{2}
	\hangpara How can a phylogenetic perspective provide insights on evolution and ecology of plant reproduction?
	
	\hangpara How can we identify and test for reproductive isolation in plant species?

	\hangpara How do flowers, pollinators, and pollination systems work?

	\columnbreak
	
	\includegraphics[width=0.45\textwidth]{coneflower_butterfly}
	\end{multicols}
	 
	\vfilll
	
	\hfill \tiny \href{https://www.flickr.com/photos/usdagov/12837726385/in/photolist-kyqGN6-fh8AaV-kyTDd4-i44Hma-dMyexJ-U9oNPh-8MVKpC-9iEtgR-pfUkfN-i6BB1D-WeaPsw-aqUmTV-6R2QfY-e1GG39-JdQ1fs-YNbHzz-9gp9eM-att94W-wqBjH2-ePxjLW-kyWRcE-Ydheej-8MSKyg-bfTrkF-9gsens-8FgBqc-WBMVv8-XR5juK-Kcompi-FoKX42-8Pk8CA-RLnLix-XK1ySw-pxAjzB-okRa8S-WTUtFQ-f8KyKZ-pqdijo-pbz8dh-8FjQDY-8MVKru-cEHMp7-eY8rmA-JcEdDd-f9epwz-8FgBtV-br2Mtc-WmM4uM-e5Cc3s-WeaQP9}{\textsc{usda}, Flickr, \ccby{2}}
\end{frame}
%
\begin{frame}[t]{\ques1 The diagram below is a phylogeny for five species in the same genus. This phylogeny shows\dots}
		

	\begin{multicols}{2}
	\begin{enumerate}
		\item the degree to which organisms look like one another.
		
		\item \alert<2>{a hypothesis of how different organisms are related through common ancestors.}

		\item a hypothesized sequence in which species evolved from left to right.

		\item the amount of change each species has experienced.

		\item which species are most closely related to the species at point 1. 

	\end{enumerate}
	
	\columnbreak

		\hfill \includegraphics[width=0.4\textwidth]{columbine_case_study_phylogeny1}
\end{multicols}
\end{frame}
%
\begin{frame}[t]{\ques2 What do the vertical lines in this phylogeny represent?}

	\begin{multicols}{2}
	\begin{enumerate}

		\item Each line represents a single individual.

		\item The history of mating between individuals.

		\item \alert<2>{Lineages of organisms.}

		\item Which species evolved first, second, third, etc.

		\item All of the above. 

	\end{enumerate}

	\columnbreak
	
	\hfill \includegraphics[width=0.4\textwidth]{columbine_case_study_phylogeny_plain}
	\end{multicols}
\end{frame}
%
\begin{frame}[t]{\ques3 What do the nodes (indicated by 1, 2, 3, and 4) represent in the phylogeny?}

	\begin{multicols}{2}
	\begin{enumerate}

		\item The sequence in which species \textsc{a}–\textsc{e} evolved.
		
		\item The point in time where two species hybridized.
		
		\item The ancestral lineages of organisms.
		
		\item \alert<2>{Common ancestors of different species at the time of speciation.}
		
		\item None of the above.

	\end{enumerate}

	\columnbreak
	
	\hfill \includegraphics[width=0.4\textwidth]{columbine_case_study_phylogeny_labeled}
		
	\end{multicols}
\end{frame}
%
{
\begin{frame}[t]{\ques4  What does the node labeled 2 indicate?}
	
	\begin{multicols}{2}
	\begin{enumerate}
		\item Species~\textsc{e} in the past.
		
		\item Where species~\textsc{e} evolved into species~\textsc{b}.
		
		\item Species~\textsc{b} in the past.

		\item Where species~\textsc{b} evolved into species~\textsc{e}.

%		\item Where species \textsc{b} and species \textsc{e} hybridized.

		\item \alert<2>{The common ancestor of species~\textsc{b}, \textsc{c}, \textsc{d}, \& \textsc{e}.}

	\end{enumerate}
	
	\columnbreak
	
		\hfill \includegraphics[width=0.4\textwidth]{columbine_case_study_phylogeny_labeled}
	\end{multicols}
		
\end{frame}
}
%
\begin{frame}[t]{\ques5 Which statement below accurately describes what happened at the nodes labeled 1–4?}

	\begin{multicols}{2}
	\begin{enumerate}
		\item \alert<2>{Speciation occurred in the common ancestor.}
		
		\item Mutation and natural selection made the best new species.  
		
		\item One species turned into another species.
		
		\item Two species came together to make a new species.
	\end{enumerate}
	
	\columnbreak
	
	\hfill \includegraphics[width=0.4\textwidth]{columbine_case_study_phylogeny_labeled}
	\end{multicols}
		
\end{frame}
%
\begin{frame}[t]{Phylogenies are \emph{hypotheses} about relationships.}
	
	\begin{multicols}{2}
		\hangpara Vertical branches represent different lineages.\vspace*{-1ex}

		\hangpara Nodes, where branches diverge, indicate common ancestors at the time of speciation.\vspace*{-1ex}

		\hangpara A clade is a group of lineages that all share a common ancestor. The clade includes all descendants of the ancestor.\vspace*{-1ex}
			
		\hangpara Members of a clade share unique features (derived traits) not present in other clades.\vspace*{-1ex} %Different clades may share features due to common ancestry (ancestral traits), but 

	\columnbreak
		
		\hfill \includegraphics[width=0.4\textwidth]{columbine_case_study_phylogeny_plain}
		
	\end{multicols}
\end{frame}
%
\begin{frame}[t]{Flowering plants form a clade that share a unique feature, \highlight{flowers.}}
	
	{\centering 
	\includegraphics[width=\textwidth]{flower_parts}
	}
	
	\vfilll
	
	\hfill \tiny Modified from \href{https://pixabay.com/en/diagram-flower-mature-anatomy-41571/}{Clker-Free-Vector-Images, Pixabay, \cc{}}
	
\end{frame}
%
\begin{frame}[t]{Pollination is the transfer of pollen from an anther to a receptive stigma.}

	\begin{multicols}{2}
	\hangpara Biotic pollination is often a \highlight{mutualism} between a plant and animals attracted to the flowers for food or other benefits. 
	
	\hangpara In the process, the pollinators pick up pollen and then carry it to another flower, resulting in reproduction for the plant. 

	\columnbreak
	
%	\includegraphics[width=0.46\textwidth]{pollen_covered_bee}
	\includegraphics[width=0.45\textwidth]{hummingbird_anthers}
	\end{multicols}
	
	\vfilll
	
%	\hfill \tiny Smudge 9000, Flickr, \ccbysa{2} % Bee
	\hfill \tiny \href{https://commons.wikimedia.org/wiki/File:RubyThroatedHummingbird.jpg}{Joe Schneid, Wikimedia, \ccby{3}} %  Hbird
	
\end{frame}
%
\begin{frame}[t]{\ques6 What do you think is the most important feature of a flower for attracting pollinators?}
	
	{\raggedcolumns
	\begin{multicols}{2}

	\begin{enumerate}
		\item Flower color.
		
		\item Flower orientation.
		
		\item Flower shape.
		
		\item Flower smell.
		
		\item A reward, such as nectar.
		
		\item A reward, such as pollen.

	\end{enumerate}

	\columnbreak
	
		\includegraphics[width=0.465\textheight]{carrion_flower}
	\end{multicols}}

	\vfilll
	
	\hfill \tiny \href{https://en.wikipedia.org/wiki/Carrion_flower}{Lothar Grünz, Wikimedia, Public Domain}

\end{frame}
%
\begin{frame}[t]{Flowers have \highlight{attractants} to attract pollinators.}


	{\raggedcolumns
	\begin{multicols}{2}
	\hangpara \highlight{Primary attractants}
	\begin{itemize}
		\item Food (nectar, pollen)
		\item Shelter (heat in solar tracking flowers)
		\item Other needed materials (e.g,. waxes, pheromones, repellants)
	\end{itemize}

	\hangpara \highlight{Secondary Attractants}
	\begin{itemize}
		\item Odor (scents and fragrances)
		\item Visual cues (color, shape, nectar guides)
	\end{itemize}
	
	\vfill
	
	\columnbreak
	
		\includegraphics[width=0.30\textwidth]{nectar_guide}
	
	\end{multicols}
	}
	
	\vfilll
	
	\hfill \tiny \copyright \href{http://photographyoftheinvisibleworld.blogspot.com/}{Dr. Klaus Schmitt, Photography of the Invisible World.}
\end{frame}
%
\begin{frame}[t]{Biotic pollination syndromes}

	\begin{multicols}{2}
	
	\hangpara \highlight{Flies:} no characteristic preferences for particular color, scent, or structural features of flowers, generalists (myophily).
	
	\hangpara \highlight{Carrion and dung flies:} purple-brown flowers with scent of decaying protein, flowers have deep traps in petals to temporarily capture pollinators (sapromyophily).

	\columnbreak
	
		\includegraphics[width=0.41\textwidth]{fly}\\
		\includegraphics[width=0.41\textwidth]{dung_fly}
		
	\end{multicols}
	
	\vfilll
	
	\hfill \tiny Top: \href{https://en.wikipedia.org/wiki/File:Chrysomya_megacephala_male.jpg}{Muhammad Mahdi Karim, Wikimedia \textsc{gnu} 1.2}.\\ \hfill Bottom: \href{https://commons.wikimedia.org/wiki/File:Dung_Fly_-_Scatophaga_stercoraria.jpg}{Wikimedia, Public Domain}
\end{frame}
%
\begin{frame}[t]{Biotic pollination syndromes}

	\begin{multicols}{2}
	\hangpara \highlight{Hawkmoths:} white or pale green color, strong sweet scent, deep narrow tubes contain nectar they access with unrolled proboscis (sphignopily).
	
	\hangpara \highlight{Butterflies:} preferences for red, yellow, and blue flowers with moderately strong sweet scent, deep narrow tubes contain nectar they access with unrolled proboscis (psychopily).
	
	
	\columnbreak
	
		\includegraphics[width=0.45\textwidth]{hawkmoth_pollinator}
		
	\end{multicols}
	
	\vfilll
	
	\hfill \tiny \href{https://pixabay.com/en/hummingbird-hawk-moth-butterfly-542500/}{Magdebeurger, Pixabay, \cc{}}
\end{frame}
%
\begin{frame}[t]{Biotic pollination syndromes}

	\begin{multicols}{2}
	\hangpara \highlight{Bees:} wide color preference but lower preference for pure red flowers, sweet scented flowers with open or tubular petals, often with prominent anthers (melittophily).
	
	\hangpara \highlight{Birds:} preference for bright colored flowers particularly red, no scent, wide deep tubes with nectar (ornithophily).

	\columnbreak
	
		\includegraphics[width=0.45\textwidth]{bee_pollinator}
	
	\end{multicols}
	
	\vfilll
	
	\hfill \tiny \href{https://commons.wikimedia.org/wiki/File:Diadasia_Bee_Straddles_Cactus_Flower_Carpels_close-up.jpg}{Jessie Eastland, Wikimedia, \ccbysa{4}}
\end{frame}
%
\begin{frame}[t]{Biotic pollination syndromes}

	\begin{multicols}{2}
	\hangpara \highlight{Bats:} preference for large, white or light colored flowers that open at night and have strong musty odors (chiropterophily).
	
	\hangpara \highlight{Lizards:} A recent discovery. The gecko functions as a pollinator \emph{and} a seed disperse for the same species of plant, which is rare.
	
	\columnbreak
	
		\includegraphics[width=0.38\textwidth]{bat_pollinator}\\
		\includegraphics[width=0.38\textwidth]{gecko_pollinator}

	\end{multicols}

	\vfilll
	
	\hfill \tiny Top: \textsc{usda}, Merlin Tuttle, Flickr, \ccby{2}.\\ \hfill Bottom: Hansen \& Müller 2009. Intl. J. Plant Sci. 170: 42
\end{frame}
%
\begin{frame}[t]{Floral features can promote reproductive isolation.}
	
	\begin{multicols}{2}
	\hangpara Floral structure can promote reproductive isolation and speciation by affecting either pollinator behavior (\highlight{behavioral isolation}) or pollen transfer (\highlight{mechanical isolation}). This may be particularly important in columbines which have a number of unique floral features.  

	\columnbreak
	
		\includegraphics[width=0.45\textwidth]{flyorchidwiththreewasps}	
	\end{multicols}

	\vfilll
	
	\hfill \tiny \copyright \href{http://www.orpingtonfieldclub.org.uk/ofc-article002.html}{Grant Hazelhurst, Orpington Field Club}
\end{frame}
%
\begin{frame}[t]{Columbines (\textit{Aquilegia} spp.) are represented by about 70 species.}
	
	\begin{multicols}{2}
	\hangpara All columbine flowers have \highlight{spurs} and other structural similarities.
	
	\hangpara The spurs contain nectar, a sugar-rich reward for pollinators who visit flowers.  

	\columnbreak
	
		\includegraphics[width=0.45\textwidth]{columbine_parts}
	\end{multicols}
	
	\vfilll
	
	\hfill \tiny \href{https://www.fs.fed.us/wildflowers/beauty/columbines/flower.shtml}{\textsc{usda} Forest Service, Public Domain}
\end{frame}
%
\begin{frame}[t]{\ques7 In columbines, color is example of a \rule{0.5in}{0.4pt} attractant and nectar is an example of a \rule{0.5in}{0.4pt} attractant.}

	\begin{multicols}{2}
	\begin{enumerate}
		\item primary/secondary
		\item floral/pollinator
		\item \alert<2>{secondary/primary}
		\item bird/insect
		\item reward/visual
	\end{enumerate}

	\columnbreak
	
		\includegraphics[width=0.45\textwidth]{hawkmoth_pollinating_columbine}
	\end{multicols}
	
	\vfilll
	
	\hfill \tiny \href{https://commons.wikimedia.org/wiki/File:Whitelined_Sphinx_Hummingbird_Moth_Colorado.JPG}{Meredith Tally, Wikimedia, \ccbysa{3}}
\end{frame}
%
\begin{frame}[t]{Nectar spurs are innovations that promoted reproductive isolation and speciation in columbines.}
	
	\begin{multicols}{2}
	\hangpara Changes in spur length may result in pollinator shifts.

	\hangpara How could this work? 

	\hangpara How might hummingbirds, bees, and moths differ in why \& how they visit flowers? 


	\vfilll
	
	\raisebox{2pt}{\textcolor{blue}{\rule{0.25in}{2pt}}} {\footnotesize Bee}\\
	\raisebox{2pt}{\textcolor{red}{\rule{0.25in}{2pt}}} {\footnotesize Hummingbird} \\
	\raisebox{2pt}{\textcolor{yellow!85!orange}{\rule{0.25in}{2pt}}} {\footnotesize Hawkmoth}
	
	\columnbreak
	
%		\vspace*{-1\baselineskip}
		
		\includegraphics[width=0.44\textwidth]{columbine_phylogeny}
	\end{multicols}
	
	\pause
	
	\tikz \draw [ultra thick] (10.1,2.3) ellipse (1.9cm and 0.7cm);
	
\end{frame}
%
{
\usebackgroundtemplate{\includegraphics[width=\paperwidth]{aquilegia_formosa}}
\begin{frame}[t]{\hfill \textcolor{white}{\textit{Aquilegia formosa}}}

	\vspace*{1\baselineskip}
	
	%
	\hfill \parbox{0.34\textwidth}{\raggedright%
	\textcolor{white}{Spur length of 10–20 mm. \\
	Moist habitats at low-mid \\ \hspace*{5pt} elevation. \\
	Pendant (points down).
	} 
	}
	
	\vfilll
	
	\hfill \tiny \textcolor{white}{Daniel Schwen, Wikimedia, \ccbysa{4}}
\end{frame}
}
%
{
\usebackgroundtemplate{\includegraphics[width=\paperwidth]{aquilegia_pubescens}}
\begin{frame}[t]{\textcolor{white}{\textit{Aquilegia pubescens}}}

%	\vspace*{1\baselineskip}
	
	\vspace*{1em}\parbox{0.4\textwidth}{\raggedright %
	\textcolor{white}{Spur length of 25–40 mm. \\
	Dry habitats at high elevation. \\
	Upright (points up).
	} 
	}
	
	\vfilll
	
	\hfill \tiny \textcolor{white}{Daniel Schwen, Wikimedia, \ccbysa{4}}
\end{frame}
}
%
\begin{frame}[t]{\textit{Aquilegia formosa} and \textit{A.~pubescens} are \highlight{sister species.}}
	
	\begin{multicols}{2}
	
	\hangpara What differences between \textit{A.~formosa} and \textit{A.~pubescens} flowers may be important for interacting with pollinators?
	
	\hangpara Why might their sharing a recent common ancestor be important?


	\vfilll
	
	\raisebox{2pt}{\textcolor{blue}{\rule{0.25in}{2pt}}} {\footnotesize Bee}\\
	\raisebox{2pt}{\textcolor{red}{\rule{0.25in}{2pt}}} {\footnotesize Hummingbird} \\
	\raisebox{2pt}{\textcolor{yellow!85!orange}{\rule{0.25in}{2pt}}} {\footnotesize Hawkmoth}

	\columnbreak
	
	\includegraphics[width=0.45\textwidth]{columbine_phylogeny}
%	{\centering
%	\includegraphics[width=0.25\textwidth]{a_formosa_small}\\
%	\includegraphics[width=0.25\textwidth]{a_pubescens_small}\\
%	}
	\end{multicols}

	\tikz \draw [ultra thick] (10.1,2.35) ellipse (1.9cm and 0.7cm);

\end{frame}
%
\begin{frame}[t]{\ques8 Because they share a common ancestor\dots}
	
	\begin{multicols}{2}
		\begin{enumerate}
			\item both species are completely different from one another and share no traits. 
			
			\item \alert<2>{both columbine species share some traits, but also have other unique features that differentiate them.}
			
			\item both columbine species will attract the same pollinators and grow in the same places.
			
			\item both columbine species are 100\% reproductively isolated from one another.
		\end{enumerate}

	\columnbreak

		{\centering
		\includegraphics[width=0.24\textwidth]{a_formosa_small}\\
		\includegraphics[width=0.24\textwidth]{a_pubescens_small}\\
		}
	\end{multicols}

	\vfilll
	
	\hfill \tiny Top: \href{http://species.wikimedia.org/wiki/File:Aquilegia_formosa_14962.JPG}{Walter Siegmund, Wikimedia, \ccbysa{3}}. \\ \hfill Bottom: \href{http://calphotos.berkeley.edu/cgi/img_query?enlarge=0000+0000+1209+2492
}{CalPhotos, \ccbyncsa{3}}
\end{frame}
%
\begin{frame}[t]{How could you test the importance of nectar spurs and other floral features on columbine speciation?}
	
	\begin{multicols}{2}
	%\hangpara Given the proposed importance of nectar spurs and other floral features, how would you test their influence on columbine speciation?

	\hangpara \highlight{Develop a hypothesis and design an experiment} to explore potential causes and function of behavioral isolation and mechanical isolation between \textit{A.~pubescens} and \textit{A.~formosa} flowers.

	\columnbreak
	
		\includegraphics[width=0.45\textwidth]{butterflyweed_spicebush}
	\end{multicols}

	\vfilll
	
	\hfill \tiny \href{https://www.flickr.com/photos/drphotomoto/3638805249}{John Flannery, Flickr, \ccbysa{2}}
\end{frame}
%
\begin{frame}[t]{Researchers designed an experiment to test pollinator preference for each columbine species.}
	
	\begin{multicols}{2}
	\hangpara In an initial study, researchers presented both columbine species in a hexagonal array with nine flowering individuals of each spp. 
	
	\hangpara Arrays were placed near \textit{A.~pubescens} and \textit{A.~formosa} populations and pollinator visits recorded.
	
	\hangpara \highlight{Why did they use this design?}

	\columnbreak
	
		\includegraphics[width=0.45\textwidth]{experimental_design}
	\end{multicols}

\end{frame}
%
\begin{frame}[t]{What do you predict?}
	
	\vspace*{-\baselineskip}
	
	\hangpara Draw a bar graph to show your predicted mean visits per flower per hour by different pollinators to \textit{A.~formosa} and \textit{A.~pubescens}.
	
	\bigskip
	
	{\centering
		\includegraphics[width=0.9\textwidth]{blank_graph}\par
	}
	
\end{frame}
%
\begin{frame}[t]{Mean visits per flower per hour by different pollinators to \textit{A.~formosa} and \textit{A.~pubescens}.}

	{\centering
	\includegraphics[width=0.85\textwidth]{columbine_pollinator_results}\par
	}

	\vfilll
	
	\tiny Fulton and Hodges 1999. Proc.~R.~Soc.~Lond.~B 266: 2247.
\end{frame}
%
\begin{frame}[t]{Pollinators visited one species more often than the other.}

	\vspace*{-1\baselineskip}
	
	\begin{multicols}{3}
		
		\vspace*{3\baselineskip}
		
		\includegraphics[width=0.27\textwidth]{a_formosa_small}\\
		\textit{A.~formosa}
				
	\columnbreak
	
		\begin{center}
			
		\includegraphics[width=0.12\textwidth]{hummingbird}\\
		\vspace*{-0.5ex}{\footnotesize Hummingbirds}
		
		\bigskip

		\includegraphics[width=0.12\textwidth]{bee}\\
		{\footnotesize Bees}
		
		\bigskip
		
		\includegraphics[width=0.12\textwidth]{fly}\\
		{\footnotesize Flies}
		
		\bigskip
	
		\includegraphics[width=0.12\textwidth]{hawkmoth}\\
		{\footnotesize Hawkmoths}
		
		\end{center}
	\columnbreak
	
		\vspace*{3\baselineskip}

		\hfill \includegraphics[width=0.27\textwidth]{a_pubescens_small}\\
		\hfill \textit{A.~pubescens}
		
		\begin{tikzpicture}
		
		% Left side
		\tikz \draw [line width = 3pt, -{Latex[length=4mm]}] (-3.2,5.3) -- (-4.9,4.5); % Hbirds to formosa
		
		\tikz \draw [line width = 3pt, -{Latex[length=4mm]}] (-3.2,3.7) -- (-4.9,3.7); % Bees to formosa

		\tikz \draw [line width = 3pt, -{Latex[length=4mm]}] (-3.2,1.9) -- (-4.9,1.9); % Flies to formosa

		\tikz \draw [line width = 0.25pt, -{Latex[length=1.5mm]}] (-3.2,0.2) -- (-4.9,1.3); % Hawkmoth to formosa

		% Right side
		\draw [line width = 0.25pt, -{Latex[length=1.5mm]}] (-1.3,5.3) -- (0.4,4.5); % Hbirds to pubescens

		\draw [line width = 1.5pt, -{Latex[length=2.25mm]}] (-1.3,3.7) -- (0.4,3.7); % Bees to pubescens

		\draw [line width = 1.5pt, -{Latex[length=2.25mm]}] (-1.3,1.9) -- (0.4,1.9); % Flies to pubescens

		\draw [line width = 3pt, -{Latex[length=4mm]}] (-1.3,0.2) -- (0.4,1.3); % Hawkmoth to pubescens

		\end{tikzpicture}

	\end{multicols}
\end{frame}
%
\begin{frame}[t]{Pollinators showed significant preferences for a specific species}
		
	\begin{center}
	\begin{tabular}{lrrr}
	\toprule
		 	& Hummingbirds & Hawkmoths & Bees \tabularnewline
	\midrule
	\textit{A.~formosa}		&	81	&	0	&	85 \tabularnewline
	\textit{A.~pubescens}	&	9	&	115	&	19 \tabularnewline
	$\chi^2$				&	57.6	&	115	&	41.8 \tabularnewline
	$p$						&  \textless0.0001 & \textless0.0001 & \textless0.0001 \tabularnewline
	\bottomrule
	\end{tabular}
	\end{center}

	\hangpara What can the researchers conclude so far?

	\vfilll
	
	\hfill \tiny Fulton and Hodges 1999. Proc.~R.~Soc.~Lond.~B 266: 2247.
	
\end{frame}
%
\begin{frame}[t]{The researchers used \textit{A.~pubescens} for another study.}
	
	\vspace*{-\baselineskip}
	
	\hangpara They planted two arrays of unmodified (control) and modified flowers.

	\hangpara \highlight{Array 1:} pedicels for ½ of the \textit{A.~pubescens} flowers staked to make flowers pendent (point downwards)
	
	\hangpara \highlight{Array 2:} spurs for ½ of the \textit{A.~pubescens} flowers shortened (squeezed nectar from bottom of spurs, tied \& clipped spur)
	
	\centering
	\includegraphics[width=0.9\textwidth]{experimental_design2}

\end{frame}
%
\begin{frame}[t]{\ques9 If floral orientation in \textit{A.~pubescens} is important for reproductive isolation, then you would predict that\dots}
	
	
	\begin{enumerate}
		\item the floral modifications will have no effect on pollinator visitation and pollen removal.
		\item hawkmoths will be better able to remove pollen from pendent flowers.
		\item hummingbirds will be attracted to the pendent flowers less than unmodified flowers.
		\item \alert<2>{the unmodified flowers will have higher visitation by hawkmoths than pendent flowers.}
	\end{enumerate}
\end{frame}
%
\begin{frame}[t]{\ques{10} In the spur shortening experiment, you should expect\dots}
	\begin{enumerate}
		\item \alert<2>{unmanipulated flowers will have more pollen removed by hawkmoths than the shortened spur flowers.}
		\item unmanipulated  flowers will have less pollen removed by hawkmoths than the shortened spur flowers.
		\item hawkmoths will visit shortened spur flowers more than the unmanipulated flowers.
		\item hawkmoths will remove more pollen from shortened spur flowers than the unmanipulated flowers.
	\end{enumerate}

\end{frame}
%
\begin{frame}[t]{\ques{11} Are these results consistent with the researchers’ prediction about flower orientation and visitation?}
	
	\begin{columns}[t]
		\begin{column}{0.28\textwidth}
			\begin{enumerate}
				\item No
				\item \alert<2>{Yes}
			\end{enumerate}
		\end{column}
		\begin{column}{0.5\textwidth}
	Visits by hawkmoths to \textit{A.~pubescens} with differing floral orientation.
	
			\begin{tabular}{L{0.6in}R{1in}}
				\toprule
				& Number of Observed Visits \tabularnewline
				\midrule
				Upright		&	51	 \tabularnewline
				Pendent		&	5	 \tabularnewline
				$\chi^2$	&	45.5 \tabularnewline
				$p$			&  \textless0.0001 \tabularnewline
				\bottomrule
			\end{tabular}
		\end{column}
	\end{columns}

	\vfilll
	
	\hfill \tiny Fulton and Hodges 1999. Proc.~R.~Soc.~Lond.~B 266: 2247.

\end{frame}
%
\begin{frame}[t]{\ques{12} Did spur length affect the number of visits by hawkmoths?}
	
	\begin{columns}[t]
		\begin{column}{0.28\textwidth}
			\begin{enumerate}
				\item \alert<2>{No}
				\item Yes
			\end{enumerate}
		\end{column}
		\begin{column}{0.5\textwidth}
			Visits by hawkmoths to \textit{A.~pubescens} with long or short nectar spurs.
			
			\begin{tabular}{L{0.75in}R{1in}}
				\toprule
				& Number of Observed Visits \tabularnewline
				\midrule
				Long spurs		&	17	 \tabularnewline
				Short spurs		&	19	 \tabularnewline
				$\chi^2$		&	0.11 \tabularnewline
				$p$				&  \textgreater0.05 \tabularnewline
				\bottomrule
			\end{tabular}
		\end{column}
	\end{columns}

	\vfilll
	
	\hfill \tiny Fulton and Hodges 1999. Proc.~R.~Soc.~Lond.~B 266: 2247.

\end{frame}
%
\begin{frame}[t]{\ques{13} Are these results consistent with the researchers' prediction?}
		\begin{columns}[t]
		\begin{column}{0.28\textwidth}
			\begin{enumerate}
				\item No
				\item \alert<2>{Yes}
			\end{enumerate}
		\end{column}
		\begin{column}{0.5\textwidth}
			Mean number of pollen grains \textbf{remaining} in \textit{A.~pubescens} anthers after hawkmoth visitation.\\[1ex]
			\includegraphics[width=0.9\textwidth]{pollen_grains_remaining}
		\end{column}
	\end{columns}
	
	\vfilll
	
	\tiny Fulton and Hodges 1999. Proc.~R.~Soc.~Lond.~B 266: 2247.
\end{frame}
%
\begin{frame}[t]{\ques{14} From these studies, the researchers could conclude that\dots}
	
	\begin{enumerate}
		\item floral structure causes reproductive isolation after pollination occurs.
		
		\item \alert<2>{orientation promotes behavioral isolation and spur length promotes mechanical isolation.}

		\item orientation promotes mechanical isolation and spur length promotes behavioral isolation.
		
		\item spur length is a primary attractant and color is a secondary attractant.

		\item the species have few floral features that would promote reproductive isolation.
		
	\end{enumerate}

\end{frame}
%
\begin{frame}[t]{\ques{15} Based upon the floral manipulation studies, spurs act to maintain separate \textit{Aquilegia} species by\dots}
	
	{\raggedcolumns
	\begin{multicols}{2}
	\begin{enumerate}
		\item causing flowers to grow in different habitats.
		\item offering different rewards to different pollinators.
		\item \alert<2>{influencing pollen removal from and depositing on flowers.}
		\item attracting different pollinators.
	\end{enumerate}

	\columnbreak
	
		\includegraphics[width=0.45\textwidth]{bee_pollinating_columbine}
	\end{multicols}
	}
	
	\vfilll
	
	\hfill \tiny \href{https://commons.wikimedia.org/wiki/File:Bee_pollinating_Aquilegia_vulgaris.JPG}{Roo72, Wikimedia, \ccbysa{3}}
\end{frame}
%
\begin{frame}[t]{\textit{A.~formosa} and \textit{A.~pubescens} can form viable hybrids.}
	
	\vspace*{-\baselineskip}
	
	\begin{multicols}{2}
	\hangpara \textit{Aquilegia pubescens} typically grows at higher elevations and in drier habitats than \textit{A.~formosa} which tends to grow in more moist habitats at lower elevations.

	\hangpara Yet, hybrid populations of viable, reproductively functioning plants with floral traits and molecular markers characteristic of both species have been identified at intermediate elevations and habitats!

	\hangpara \highlight{How could this happen?}

	\columnbreak
	
		{\centering
		\includegraphics[height=0.8\textheight]{columbine_hybrids}\par}
	\end{multicols}
	
	\vfilll
	
	\hfill \tiny \href{https://commons.wikimedia.org/wiki/File:Aquilegia_pubescens-formosa_hybrid-swarm_flowers_close.jpg}{Dcrjsr, Wikimedia, \ccby{4}}
\end{frame}
%
\begin{frame}[t]{\ques{16} The occurrence of hybrids indicates\dots}
	
	\begin{multicols}{2}
	\begin{enumerate}
		\item that \textit{A.~formosa} and \textit{A.~pubescens} are really just one species.
		\item reproductive isolation does not matter for plant species.
		\item pollinator behavior is not important for maintaining species.
		\item reproductive isolating barriers are not always absolute between species.
		\item that habitat is not important for maintaining species.
	\end{enumerate}
	
	\columnbreak
	
		\includegraphics[width=0.45\textwidth]{pubescens_hybrid}
	\end{multicols}
	
	\vfilll
	
	\hfill \tiny \href{https://www.flickr.com/photos/63093099@N02/32184751505/in/photolist-R34gjH}{text}plantpollinator, Flickr, \ccbyncsa{2}
\end{frame}
%
\begin{frame}[t]{Summary}
	
	\begin{multicols}{2}
	\hangpara Floral structural differences can influence pollinator behavior, pollinator effectiveness, and, consequently, reproductive isolation between species. 

	\hangpara Although floral features can promote reproductive isolation between species for some pollinators, generalist pollinators may visit both species, resulting in hybrids that can survive in intermediate habitats. 

	\columnbreak
	
		\includegraphics[width=0.45\textwidth]{hummingbird_pollinator_finale}
	\end{multicols}

	\vfilll
	
	\hfill \tiny \href{https://www.flickr.com/photos/jeffreyww/17421811000/in/photolist-egCnAb-W5cqTb-JmQo5J-NYPC2-7Tve9a-aCd9bM-7YoLY2-eeKix9-dhnGmv-nh85vv-sAP76s-6PkrUP-6r6qhK-driKBF-drdFtw-WS8wEY-drdwbv-7Ys1Pu-7YoLMx-egwBPz-9L5NHb-sqRuEw-sUo69r-9L5NHU-f9MG8d-RUtzU5-sxvkrJ-exZVce-sUcJns-rSeypr-oim7rn-drdzeb}{jeffreyw, Flickr, \ccby{2}}
\end{frame}
%
\end{document}
