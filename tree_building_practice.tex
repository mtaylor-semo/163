%!TEX TS-program = lualatex
%!TEX encoding = UTF-8 Unicode

\documentclass[12pt]{article}


\usepackage{graphicx}
	\graphicspath{{/Users/goby/Pictures/teach/163/lab/}
	{img/}} % set of paths to search for images

\usepackage{geometry}
\geometry{letterpaper, left=1.5in, bottom=1in}                   
%\geometry{landscape}                % Activate for for rotated page geometry
\usepackage[parfill]{parskip}    % Activate to begin paragraphs with an empty line rather than an indent
\usepackage{amssymb, amsmath}
\usepackage{mathtools}
	\everymath{\displaystyle}

\usepackage{fontspec}
\setmainfont[Ligatures={TeX}, BoldFont={* Bold}, ItalicFont={* Italic}, BoldItalicFont={* BoldItalic}, Numbers={Lining}]{Linux Libertine O}
\setsansfont[Scale=MatchLowercase,Ligatures=TeX]{Linux Biolinum O}
\setmonofont[Scale=MatchLowercase]{Linux Libertine Mono O}
\usepackage{microtype}


\usepackage{tikz}
\usetikzlibrary{trees}

\usepackage{forest}

% To define fonts for particular uses within a document. For example, 
% This sets the Libertine font to use tabular number format for tables.
 %\newfontfamily{\tablenumbers}[Numbers={Monospaced}]{Linux Libertine O}
% \newfontfamily{\libertinedisplay}{Linux Libertine Display O}

\usepackage{booktabs}
\usepackage{multicol}

\usepackage{caption}
	\captionsetup{labelsep=period, justification=raggedright} % Removes colon following figure / table number.
	\captionsetup{singlelinecheck=off}
	\captionsetup[table]{skip=0pt}

\usepackage{tikz}
\tikzstyle{block} = [rectangle, draw, fill=white, rounded corners,
                 minimum size=2em]
\tikzstyle{branch} = [thick, draw]

\usetikzlibrary{positioning, backgrounds}


\forestset{
	every leaf node/.style={
		if n children=0{#1}{}
	},
	every tree node/.style={
		if n children=0{}{#1}
	},
	mytree/.style={
		for tree={
			edge path={
				\noexpand\path [draw, thick, \forestoption{edge}] (!u.parent anchor) |- (.child anchor)\forestoption{edge label};
			},
			every tree node={draw=none,inner sep=0, outer sep=0, minimum size=0},
			every leaf node/.style={align=left},
			grow'=0,
			parent anchor=east, 
			child anchor=west,
			anchor=west,
			l sep=0.5cm,
			s sep=3mm,
			draw=none,
			if n children=0{tier=word}{}
		}
	}
}


\usepackage{longtable}
%\usepackage{siunitx}
\usepackage{array}
\newcolumntype{L}[1]{>{\raggedright\let\newline\\\arraybackslash\hspace{0pt}}p{#1}}
\newcolumntype{C}[1]{>{\centering\let\newline\\\arraybackslash\hspace{0pt}}p{#1}}
\newcolumntype{R}[1]{>{\raggedleft\let\newline\\\arraybackslash\hspace{0pt}}p{#1}}


\usepackage{fancyhdr}
\fancyhf{}
\pagestyle{fancy}
\lhead{}
\chead{}
\rhead{\footnotesize pg. \thepage }
\renewcommand{\headrulewidth}{0.4pt}

\fancypagestyle{plain}{%
	\fancyhf{}
	\lhead{\textsc{bi} 163: Evolution and Ecology}
	\rhead{Phylogenetic Tree Practice}
	\renewcommand{\headrulewidth}{0pt}
}

\fancypagestyle{answers}{%
	\fancyhf{}
	\lhead{\textsc{bi} 163: Evolution and Ecology}
	\rhead{\textbf{Solutions}}
	\renewcommand{\headrulewidth}{0pt}
}

\begin{document}

\thispagestyle{plain}

Make phylogenetic trees from the following presence/absence data 
sets. Each data set contains eight species and at least seven 
characters. Answers are provided after the character matrices.
Complete each tree, then compare it to the answer. 

Use only 1s to determine relationships. Zeros represent absence 
of data so you cannot use 0s to determine relationships. 

The last five problems get slightly more involved and provide more
information for you to think about.

\emph{Remember!} Branches can pivot without changing relationships 
so your trees might look slighly different from the trees in the 
answer key. Check the \emph{relationships} to be sure your answer is 
correct.

\bigskip

\textsc{Tree 1}

\begin{longtable}[l]{@{}rccccccc@{}}
\toprule
Species	& C1	& C2	& C3	& C4	& C5	& C6	& C7 \tabularnewline
\midrule
A 		& 1 	& 0 	& 0 	& 0		& 0 	& 0 	& 1  \tabularnewline
B 		& 1 	& 1 	& 1 	& 1		& 1 	& 1 	& 1  \tabularnewline
C 		& 0 	& 0 	& 0 	& 0 	& 0 	& 0 	& 1  \tabularnewline
D 		& 1 	& 1 	& 1 	& 1 	& 1 	& 1 	& 1  \tabularnewline
E 		& 1 	& 0 	& 0 	& 0 	& 0 	& 1 	& 1  \tabularnewline
F 		& 1 	& 0 	& 0 	& 1 	& 0 	& 1 	& 1  \tabularnewline
G 		& 1 	& 0 	& 1 	& 1 	& 1 	& 1 	& 1  \tabularnewline
H 		& 1 	& 0 	& 0 	& 1 	& 1 	& 1 	& 1  \tabularnewline
\bottomrule
\end{longtable}

\quad

\textsc{Tree 2}

\begin{longtable}[l]{@{}rccccccc@{}}
\toprule
Species	& C1	& C2	& C3	& C4	& C5	& C6	& C7 \tabularnewline
\midrule
A 		& 1 	& 0 	& 1 	& 1		& 0 	& 1 	& 1  \tabularnewline
B 		& 1 	& 0 	& 0 	& 1 	& 1 	& 1 	& 1  \tabularnewline
C 		& 1 	& 0 	& 0 	& 0 	& 0 	& 0 	& 0  \tabularnewline
D 		& 1 	& 1 	& 0 	& 0 	& 0 	& 1 	& 0  \tabularnewline
E 		& 1 	& 0 	& 1 	& 1 	& 0 	& 1 	& 1  \tabularnewline
F 		& 1 	& 0 	& 0 	& 0 	& 0 	& 1 	& 1  \tabularnewline
G 		& 1 	& 0 	& 0 	& 1 	& 1 	& 1 	& 1  \tabularnewline
H 		& 1 	& 1 	& 0 	& 0 	& 0 	& 1 	& 0  \tabularnewline
\bottomrule
\end{longtable}

\newpage

\textsc{Tree 3}

\begin{longtable}[l]{@{}rccccccc@{}}
\toprule
Species	& C1	& C2	& C3	& C4	& C5	& C6	& C7 \tabularnewline
\midrule
A 		& 0 	& 0 	& 1 	& 1		& 1 	& 1 	& 0  \tabularnewline
B 		& 1 	& 1 	& 0 	& 1 	& 0 	& 0 	& 0  \tabularnewline
C 		& 1 	& 1 	& 0 	& 1 	& 0 	& 0 	& 0  \tabularnewline
D 		& 1 	& 0 	& 0 	& 1 	& 0 	& 0 	& 1  \tabularnewline
E 		& 0 	& 0 	& 1 	& 1 	& 1 	& 1 	& 0  \tabularnewline
F 		& 0 	& 0 	& 0 	& 1 	& 0 	& 1 	& 0  \tabularnewline
G 		& 1 	& 0 	& 0 	& 1 	& 0 	& 0 	& 1  \tabularnewline
H 		& 0 	& 0 	& 1 	& 1 	& 0 	& 1 	& 0  \tabularnewline
\bottomrule
\end{longtable}

\quad

\textsc{Tree 4}

\begin{longtable}[l]{@{}rccccccc@{}}
\toprule
Species	& C1	& C2	& C3	& C4	& C5	& C6	& C7 \tabularnewline
\midrule
A 		& 0 	& 0 	& 0 	& 0		& 0 	& 1 	& 0  \tabularnewline
B 		& 1 	& 0 	& 0 	& 1 	& 0 	& 1 	& 0  \tabularnewline
C 		& 1 	& 1 	& 0 	& 1 	& 1 	& 1 	& 1  \tabularnewline
D 		& 1 	& 1 	& 1 	& 1 	& 0 	& 1 	& 1  \tabularnewline
E 		& 1 	& 1 	& 1 	& 1 	& 0 	& 1 	& 1  \tabularnewline
F 		& 1 	& 1 	& 0 	& 1 	& 1 	& 1 	& 1  \tabularnewline
G 		& 0 	& 0 	& 0 	& 1 	& 0 	& 1 	& 0  \tabularnewline
H 		& 1 	& 0 	& 0 	& 1 	& 0 	& 1 	& 1  \tabularnewline
\bottomrule
\end{longtable}

\quad

\textsc{Tree 5}

\begin{longtable}[l]{@{}rccccccc@{}}
\toprule
Species	& C1	& C2	& C3	& C4	& C5	& C6	& C7 \tabularnewline
\midrule
A 		& 0 	& 1 	& 1 	& 1		& 1 	& 0 	& 1  \tabularnewline
B 		& 1 	& 1 	& 1 	& 1 	& 1 	& 1 	& 1  \tabularnewline
C 		& 0 	& 0 	& 1 	& 0 	& 1 	& 0 	& 0  \tabularnewline
D 		& 1 	& 1 	& 1 	& 1 	& 1 	& 1 	& 1  \tabularnewline
E 		& 0 	& 1 	& 1 	& 0 	& 1 	& 0 	& 1  \tabularnewline
F 		& 0 	& 1 	& 1 	& 0 	& 1 	& 0 	& 0  \tabularnewline
G 		& 1 	& 1 	& 1 	& 1 	& 1 	& 0 	& 1  \tabularnewline
H 		& 0 	& 0 	& 0 	& 0 	& 1 	& 0 	& 0  \tabularnewline
\bottomrule
\end{longtable}

\quad

\textsc{Tree 6}

Species can share more than one homology.

\begin{longtable}[l]{@{}rcccccccc@{}}
\toprule
Species	& C1	& C2	& C3	& C4	& C5	& C6	& C7 & C8 \tabularnewline
\midrule
A 		& 1 	& 0 	& 0 	& 1		& 0 	& 0 	& 0  & 0  \tabularnewline
B 		& 1 	& 1 	& 1 	& 0 	& 0 	& 0 	& 0  & 0  \tabularnewline
C 		& 1 	& 0 	& 1 	& 0 	& 1 	& 1 	& 1  & 1  \tabularnewline
D 		& 1 	& 0 	& 1 	& 0 	& 1 	& 0 	& 1  & 1  \tabularnewline
E 		& 1 	& 0 	& 1 	& 0 	& 1 	& 0 	& 0  & 1  \tabularnewline
F 		& 1 	& 1 	& 1 	& 0 	& 0 	& 0 	& 0  & 0  \tabularnewline
G 		& 1 	& 0 	& 0 	& 1 	& 0 	& 0 	& 0  & 0  \tabularnewline
H 		& 1 	& 0 	& 1 	& 0 	& 1 	& 1 	& 1  & 1  \tabularnewline
\bottomrule
\end{longtable}

\quad

\textsc{Tree 7}

\begin{longtable}[l]{@{}rcccccccc@{}}
\toprule
Species	& C1	& C2	& C3	& C4	& C5	& C6	& C7 & C8 \tabularnewline
\midrule
A 		& 0 	& 0 	& 0 	& 0		& 1 	& 1 	& 1  & 0  \tabularnewline
B 		& 1 	& 0 	& 0 	& 1 	& 1 	& 1 	& 1  & 0  \tabularnewline
C 		& 1 	& 1 	& 0 	& 1 	& 1 	& 1 	& 1  & 0  \tabularnewline
D 		& 0 	& 0 	& 1 	& 0 	& 0 	& 0 	& 1  & 1  \tabularnewline
E 		& 1 	& 1 	& 0 	& 1 	& 1 	& 1 	& 1  & 0  \tabularnewline
F 		& 0 	& 0 	& 0 	& 0 	& 0 	& 1 	& 1  & 0  \tabularnewline
G 		& 1 	& 0 	& 0 	& 0 	& 1 	& 1 	& 1  & 0  \tabularnewline
H 		& 0 	& 0 	& 1 	& 0 	& 0 	& 0 	& 1  & 1  \tabularnewline
\bottomrule
\end{longtable}

\quad


\textsc{Tree 8}

\begin{longtable}[l]{@{}rccccccccc@{}}
\toprule
Species	& C1 & C2 & C3 & C4	& C5 & C6 & C7 & C8 & C9 \tabularnewline
\midrule
A 		& 0  & 1  & 0  & 0	& 1  & 1  & 1  & 1  & 0  \tabularnewline
B 		& 1  & 1  & 1  & 1 	& 1  & 1  & 0  & 1  & 1  \tabularnewline
C 		& 0  & 1  & 0  & 0 	& 0  & 0  & 0  & 0  & 0  \tabularnewline
D 		& 0  & 1  & 0  & 0 	& 1  & 0  & 0  & 0  & 0  \tabularnewline
E 		& 1  & 1  & 0  & 1 	& 1  & 1  & 0  & 1  & 1  \tabularnewline
F 		& 0  & 1  & 0  & 1 	& 1  & 1  & 0  & 1  & 0  \tabularnewline
G 		& 0  & 1  & 0  & 0 	& 1  & 1  & 1  & 1  & 0  \tabularnewline
H 		& 1  & 1  & 1  & 1	& 1  & 1  & 0  & 1  & 1  \tabularnewline
\bottomrule
\end{longtable}

\quad

\textsc{Tree 9}

Characters that are not shared (unique to one species) cannot be used to determine relationships.

\begin{longtable}[l]{@{}rcccccccc@{}}
\toprule
Species	& C1 & C2 & C3 & C4	& C5 & C6 & C7 & C8 \tabularnewline
\midrule
A 		& 1  & 0  & 0  & 0	& 0  & 0  & 0  & 0  \tabularnewline
B 		& 1  & 1  & 0  & 1 	& 1  & 1  & 1  & 0  \tabularnewline
C 		& 1  & 1  & 0  & 1 	& 1  & 1  & 0  & 0  \tabularnewline
D 		& 1  & 1  & 0  & 1 	& 1  & 1  & 1  & 0  \tabularnewline
E 		& 1  & 0  & 1  & 1 	& 1  & 1  & 0  & 0  \tabularnewline
F 		& 1  & 0  & 1  & 1 	& 1  & 1  & 0  & 0  \tabularnewline
G 		& 1  & 0  & 0  & 0 	& 0  & 1  & 0  & 1  \tabularnewline
H 		& 1  & 0  & 0  & 0 	& 1  & 1  & 0  & 0  \tabularnewline
\bottomrule
\end{longtable}

\quad


\textsc{Tree 10}


\begin{longtable}[l]{@{}rcccccccccc@{}}
\toprule
Species	& C1 & C2 & C3 & C4	& C5 & C6 & C7 & C8 & C9 & C10 \tabularnewline
\midrule
A 		& 1  & 1  & 0  & 0	& 1  & 0  & 1  & 0  & 0  & 1   \tabularnewline
B 		& 1  & 0  & 0  & 0 	& 0  & 0  & 0  & 0  & 0  & 1   \tabularnewline
C 		& 1  & 1  & 0  & 1 	& 0  & 1  & 1  & 1  & 0  & 1   \tabularnewline
D 		& 1  & 1  & 1  & 0 	& 0  & 1  & 1  & 0  & 0  & 1   \tabularnewline
E 		& 1  & 1  & 0  & 1 	& 0  & 1  & 1  & 1  & 0  & 1   \tabularnewline
F 		& 1  & 1  & 0  & 0 	& 1  & 0  & 1  & 0  & 1  & 1   \tabularnewline
G 		& 0  & 0  & 0  & 0 	& 0  & 0  & 0  & 0  & 0  & 1   \tabularnewline
H 		& 1  & 1  & 1  & 0 	& 0  & 1  & 1  & 0  & 0  & 1   \tabularnewline
\bottomrule
\end{longtable}

\newpage

\thispagestyle{answers}

\textsc{Tree 1}

\begin{forest} mytree
	[[,name=base
	[C]
	[,name=split1
	[A]
	[,name=split2
	[E]
	[,name=split3
	[F]
	[,name=split4
	[H]
	[,name=split5
	[G]
	[,name=split6
	[D]
	[B]
	]
	]
	]
	]
	]
	]]]
	\filldraw (base) circle [radius=3pt, fill=black, xshift=-5mm] node [below, xshift=-5mm, yshift=-2pt] {C7};
	\filldraw (split1) circle [radius=3pt, fill=black, xshift=-5mm] node [below, xshift=-5mm, yshift=-2pt] {C1};
	\filldraw (split2) circle [radius=3pt, fill=black, xshift=-5mm] node [below, xshift=-5mm, yshift=-2pt] {C6};
	\filldraw (split3) circle [radius=3pt, fill=black, xshift=-5mm] node [below, xshift=-5mm, yshift=-2pt] {C4};
	\filldraw (split4) circle [radius=3pt, fill=black, xshift=-5mm] node [below, xshift=-5mm, yshift=-2pt] {C5};
	\filldraw (split5) circle [radius=3pt, fill=black, xshift=-5mm] node [below, xshift=-5mm, yshift=-2pt] {C3};
	\filldraw (split6) circle [radius=3pt, fill=black, xshift=-5mm] node [below, xshift=-5mm, yshift=-2pt] {C2};
\end{forest}

\quad

\textsc{Tree 2}


\begin{forest} mytree
[
 [,name=base
  [,name=AEGBFHD
   [,name=AEGBF
    [,name=AEGB
     [,name=AE
	  [A]
	  [E]
     ]
     [,name=GB
      [G]
      [B]
     ]
    ]
  [F]
  ]
  [
   [[,name=HD
    [H]
    [D]
   ]]
   ]]
   [
    [C]
  ]
 ]
]
	\filldraw (base) circle [radius=3pt, fill=black, xshift=-5mm] node [below, xshift=-5mm, yshift=-2pt] {C1};
	\filldraw (AEGB) circle [radius=3pt, fill=black, xshift=-5mm] node [above, xshift=-5mm, yshift=2pt] {C4};
	\filldraw (AE) circle [radius=3pt, fill=black, xshift=-5mm] node [above, xshift=-5mm, yshift=2pt] {C3};
	\filldraw (GB) circle [radius=3pt, fill=black, xshift=-5mm] node [below, xshift=-5mm, yshift=-2pt] {C5};
	\filldraw (HD) circle [radius=3pt, fill=black, xshift=-5mm] node [below, xshift=-5mm, yshift=-2pt] {C2};
	\filldraw (AEGBF) circle [radius=3pt, fill=black, xshift=-5mm] node [below, xshift=-5mm, yshift=-2pt] {C7};
	\filldraw (AEGBFHD) circle [radius=3pt, fill=black, xshift=-5mm] node [below, xshift=-5mm, yshift=-2pt] {C6};
\end{forest}

\newpage

\textsc{Tree 3}


\begin{forest} mytree
[
 [,name=base
  [,name=AEHF
   [,name=AEH
    [,name=AE
     [A]
     [E]
    ]
    [
     [H]
    ]
   ]
   [
    [F]
   ]
  ]
  [,name=DGCB
   [,name=DG
	[D]
	[G]
   ]
   [,name=CB
    [C]
    [B]
   ]
  ]
 ]
]
	\filldraw (base) circle [radius=3pt, fill=black, xshift=-5mm] node [below, xshift=-5mm, yshift=-2pt] {C4};
	\filldraw (AE) circle [radius=3pt, fill=black, xshift=-5mm] node [below, xshift=-5mm, yshift=-2pt] {C5};
	\filldraw (AEH) circle [radius=3pt, fill=black, xshift=-5mm] node [below, xshift=-5mm, yshift=-2pt] {C3};
	\filldraw (AEHF) circle [radius=3pt, fill=black, xshift=-5mm] node [below, xshift=-5mm, yshift=-2pt] {C6};
	\filldraw (DGCB) circle [radius=3pt, fill=black, xshift=-5mm] node [below, xshift=-5mm, yshift=-2pt] {C1};
	\filldraw (DG) circle [radius=3pt, fill=black, xshift=-5mm] node [below, xshift=-5mm, yshift=-2pt] {C7};
	\filldraw (CB) circle [radius=3pt, fill=black, xshift=-5mm] node [below, xshift=-5mm, yshift=-2pt] {C2};
\end{forest}

\quad

\textsc{Tree 4}

\begin{forest} mytree
[
 [,name=base
  [,name=HDEFCBG
   [,name=HDEFCB
    [,name=HDEFC
     [H]
      [,name=DEFC
       [,name=DE
        [D]
        [E]
       ]
       [,name=FC
        [F]
        [C]
       ]
      ]
     ]
     [B] 
    ]
   [G]
  ]
  [A]
 ]
]
	\filldraw (base) circle [radius=3pt, fill=black, xshift=-5mm] node [below, xshift=-5mm, yshift=-2pt] {C6};
	\filldraw (DE) circle [radius=3pt, fill=black, xshift=-5mm] node [below, xshift=-5mm, yshift=-2pt] {C3};
	\filldraw (FC) circle [radius=3pt, fill=black, xshift=-5mm] node [below, xshift=-5mm, yshift=-2pt] {C5};
	\filldraw (DEFC) circle [radius=3pt, fill=black, xshift=-5mm] node [below, xshift=-5mm, yshift=-2pt] {C2};
	\filldraw (HDEFC) circle [radius=3pt, fill=black, xshift=-5mm] node [below, xshift=-5mm, yshift=-2pt] {C7};
	\filldraw (HDEFCB) circle [radius=3pt, fill=black, xshift=-5mm] node [below, xshift=-5mm, yshift=-2pt] {C1};
	\filldraw (HDEFCBG) circle [radius=3pt, fill=black, xshift=-5mm] node [below, xshift=-5mm, yshift=-2pt] {C4};
\end{forest}

\newpage

\textsc{Tree 5}

\begin{forest} mytree
[
[,name=base
 [,name=GBDAEFC
  [,name=GBDAEF
   [,name=GBDAE
    [,name=GBDA
     [,name=GBD
      [G]
      [,name=BD
       [B]
       [D]
      ]
     ]
     [A]
    ]
    [E]
   ]
   [F]
  ]
  [C]
  ]
 [H]
 ]
]
	\filldraw (base) circle [radius=3pt, fill=black, xshift=-5mm] node [below, xshift=-5mm, yshift=-2pt] {C5};
	\filldraw (BD) circle [radius=3pt, fill=black, xshift=-5mm] node [below, xshift=-5mm, yshift=-2pt] {C6};
	\filldraw (GBD) circle [radius=3pt, fill=black, xshift=-5mm] node [below, xshift=-5mm, yshift=-2pt] {C1};
	\filldraw (GBDA) circle [radius=3pt, fill=black, xshift=-5mm] node [below, xshift=-5mm, yshift=-2pt] {C4};
	\filldraw (GBDAE) circle [radius=3pt, fill=black, xshift=-5mm] node [below, xshift=-5mm, yshift=-2pt] {C7};
	\filldraw (GBDAEF) circle [radius=3pt, fill=black, xshift=-5mm] node [below, xshift=-5mm, yshift=-2pt] {C2};
	\filldraw (GBDAEFC) circle [radius=3pt, fill=black, xshift=-5mm] node [below, xshift=-5mm, yshift=-2pt] {C3};
\end{forest}

\quad

\textsc{Tree 6}

\begin{forest} mytree
[
 [,name=base
  [,name=CHDEFB
   [,name=CHDE
    [,name=CHD
     [,name=CH
      [C]
      [H]
     ]
     [D]
    ]
    [E]
   ]
   [,name=FB
    [F]
    [B]
   ]
  ]
  [[,name=AG
   [A]
   [G]
  ]]
 ]
]
	\filldraw (base) circle [radius=3pt, fill=black, xshift=-5mm] node [below, xshift=-5mm, yshift=-2pt] {C1};
	\filldraw (AG) circle [radius=3pt, fill=black, xshift=-5mm] node [below, xshift=-5mm, yshift=-2pt] {C4};
	\filldraw (FB) circle [radius=3pt, fill=black, xshift=-5mm] node [below, xshift=-5mm, yshift=-2pt] {C2};
	\filldraw (CH) circle [radius=3pt, fill=black, xshift=-5mm] node [below, xshift=-5mm, yshift=-2pt] {C6};
	\filldraw (CHD) circle [radius=3pt, fill=black, xshift=-5mm] node [below, xshift=-5mm, yshift=-2pt] {C7};
	\filldraw (CHDE) circle [radius=3pt, fill=black, xshift=-6mm] node [below, xshift=-6mm, yshift=-2pt] {C5,C8};
	\filldraw (CHDEFB) circle [radius=3pt, fill=black, xshift=-5mm] node [below, xshift=-5mm, yshift=-2pt] {C3};
\end{forest}

\newpage

\textsc{Tree 7}

\begin{forest} mytree
[
 [,name=base
  [,name=AGCEBF
   [,name=AGCEB
   [A]
    [,name=GCEB
    [G]
     [,name=CEB
      [,name=CE
       [C]
       [E]
      ]
      [B]
     ]
    ]
   ]
   [F]
  ]
  [,name=HD[
   [H]
   [D]
  ]]
 ]
]
	\filldraw (base) circle [radius=3pt, fill=black, xshift=-5mm] node [below, xshift=-5mm, yshift=-2pt] {C7};
	\filldraw (HD) circle [radius=3pt, fill=black] node [below, yshift=-2pt] {C3,C8};
	\filldraw (CE) circle [radius=3pt, fill=black, xshift=-5mm] node [below, xshift=-5mm, yshift=-2pt] {C2};
	\filldraw (CEB) circle [radius=3pt, fill=black, xshift=-5mm] node [below, xshift=-5mm, yshift=-2pt] {C4};
	\filldraw (GCEB) circle [radius=3pt, fill=black, xshift=-5mm] node [below, xshift=-5mm, yshift=-2pt] {C1};
	\filldraw (AGCEB) circle [radius=3pt, fill=black, xshift=-5mm] node [below, xshift=-5mm, yshift=-2pt] {C5};
	\filldraw (AGCEBF) circle [radius=3pt, fill=black, xshift=-5mm] node [below, xshift=-5mm, yshift=-2pt] {C6};
\end{forest}

\quad

\textsc{Tree 8}

\begin{forest} mytree
[
 [,name=base
  [,name=HBEFAGD
   [,name=HBEFAG
    [,name=HBEF
     [,name=HBE
      [,name=HB
       [H]
       [B]
      ]
      [E]
     ]
     [F]
    ]
    [[,name=AG
     [A]
     [G]
    ]]
   ]
   [D]
  ]
  [C]
 ]
]
	\filldraw (base) circle [radius=3pt, fill=black, xshift=-5mm] node [below, xshift=-5mm, yshift=-2pt] {C2};
	\filldraw (HB) circle [radius=3pt, fill=black, xshift=-5mm] node [below, xshift=-5mm, yshift=-2pt] {C3};
	\filldraw (HBE) circle [radius=3pt, fill=black, xshift=-6mm] node [below, xshift=-6mm, yshift=-2pt] {C1,C9};
	\filldraw (HBEF) circle [radius=3pt, fill=black, xshift=-5mm] node [below, xshift=-5mm, yshift=-2pt] {C4};
	\filldraw (AG) circle [radius=3pt, fill=black, xshift=-5mm] node [below, xshift=-5mm, yshift=-2pt] {C7};
	\filldraw (HBEFAG) circle [radius=3pt, fill=black, xshift=-6mm] node [below, xshift=-6mm, yshift=-2pt] {C6,C8};
	\filldraw (HBEFAGD) circle [radius=3pt, fill=black, xshift=-5mm] node [below, xshift=-5mm, yshift=-2pt] {C5};
\end{forest}

\newpage

\textsc{Tree 9}

Character 8 (in gray) is unique to species G. Character 8 cannot be
used to determine relationships because it is not shared
by two or more species.

\begin{forest}mytree
[
 [,name=base
  [,name=BDCFEHG
   [,name=BDCFEH
    [,name=BDCFE
     [,name=BDC
      [,name=BD
       [B]
       [D]
      ]
      [C]
     ]
     [,name=FE
      [F]
      [E]
     ]
    ]
    [H]
   ]
   [G,name=G]
  ]
  [A]
 ]
]
	\filldraw (base) circle [radius=3pt, fill=black, xshift=-5mm] node [below, xshift=-5mm, yshift=-2pt] {C1};
	\filldraw (BD) circle [radius=3pt, fill=black, xshift=-5mm] node [below, xshift=-5mm, yshift=-2pt] {C7};
	\filldraw (BDC) circle [radius=3pt, fill=black, xshift=-5mm] node [below, xshift=-5mm, yshift=-2pt] {C2};
	\filldraw (FE) circle [radius=3pt, fill=black, xshift=-5mm] node [below, xshift=-5mm, yshift=-2pt] {C3};
	\filldraw (BDCFE) circle [radius=3pt, fill=black, xshift=-5mm] node [below, xshift=-5mm, yshift=-2pt] {C4};
	\filldraw (BDCFEH) circle [radius=3pt, fill=black, xshift=-5mm] node [below, xshift=-5mm, yshift=-2pt] {C5};
	\filldraw (BDCFEHG) circle [radius=3pt, fill=black, xshift=-5mm] node [below, xshift=-5mm, yshift=-2pt] {C6};
	\filldraw [gray] (G) circle [radius=3pt, xshift=-15mm] node [below, xshift=-15mm, yshift=-2pt, color=gray] {C8};
\end{forest}

\quad

\textsc{Tree 10}

\begin{forest}{mytree}
[
 [,name=base
  [,name=AFDHECB
   [,name=AFDHEC
    [,name=AF
     [A]
     [F,name=F]
    ]
    [,name=DHEC
     [,name=DH
      [D]
      [H]
     ]
     [,name=EC
      [E]
      [C]
     ]
    ]
   ]
   [B]
  ]
  [G]
 ]
]
	\filldraw (base) circle [radius=3pt, fill=black, xshift=-5mm] node [below, xshift=-5mm, yshift=-2pt] {C10};
	\filldraw (AF) circle [radius=3pt, fill=black, xshift=-5mm] node [below, xshift=-5mm, yshift=-2pt] {C5};
	\filldraw (DH) circle [radius=3pt, fill=black, xshift=-5mm] node [below, xshift=-5mm, yshift=-2pt] {C3};
	\filldraw (EC) circle [radius=3pt, fill=black, xshift=-6mm] node [below, xshift=-6mm, yshift=-2pt] {C4,C8};
	\filldraw (AFDHEC) circle [radius=3pt, fill=black, xshift=-6mm] node [below, xshift=-6mm, yshift=-2pt] {C2,C7};
	\filldraw (AFDHECB) circle [radius=3pt, fill=black, xshift=-5mm] node [below, xshift=-5mm, yshift=-2pt] {C1};
	\filldraw (DHEC) circle [radius=3pt, fill=black, xshift=-5mm] node [below, xshift=-5mm, yshift=-2pt] {C6};
	\filldraw [gray] (F) circle [radius=3pt, xshift=-10mm] node [below, xshift=-10mm, yshift=-2pt, color=gray] {C9};
\end{forest}


\end{document}  