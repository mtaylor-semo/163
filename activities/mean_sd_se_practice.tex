%!TEX TS-program = lualatex
%!TEX encoding = UTF-8 Unicode

\documentclass[12pt]{exam}


%\printanswers

\usepackage{fontspec}
\setmainfont[Ligatures={TeX}, BoldFont={* Bold}, ItalicFont={* Italic}, BoldItalicFont={* BoldItalic}, Numbers={Monospaced, Lining}]{Linux Libertine O}
\setsansfont[Scale=MatchLowercase,Ligatures=TeX]{Linux Biolinum O}
\setmonofont[Scale=MatchLowercase]{Inconsolatazi4}
\newfontfamily{\tablenumbers}[Numbers={Monospaced,Lining}]{Linux Libertine O}
\usepackage{microtype}

\usepackage{amsmath}

\usepackage{unicode-math}
\setmathfont[Scale=MatchLowercase]{TeX Gyre Termes Math}

\usepackage{geometry}
\geometry{letterpaper, left=1.5in, bottom=1in}                   
%\geometry{landscape}                % Activate for for rotated page geometry
\usepackage[parfill]{parskip}    % Activate to begin paragraphs with an empty line rather than an indent

\usepackage{longtable}

%\usepackage{siunitx}
\usepackage{booktabs}
\usepackage{array}
\newcolumntype{L}[1]{>{\raggedright\let\newline\\\arraybackslash\hspace{0pt}}m{#1}}
\newcolumntype{C}[1]{>{\centering\let\newline\\\arraybackslash\hspace{0pt}}m{#1}}
\newcolumntype{R}[1]{>{\raggedleft\let\newline\\\arraybackslash\hspace{0pt}}m{#1}}

\usepackage{enumitem}
\setlist{leftmargin=*}
\setlist[1]{labelindent=\parindent}
\setlist[enumerate]{label=\textsc{\alph*}.}
\setlist[itemize]{label=\color{gray}\textbullet}
%\usepackage{hyperref}
%\usepackage{placeins} %PRovides \FloatBarrier to flush all floats before a certain point.
%\usepackage{hanging}

\usepackage[sc]{titlesec}

%% Commands for Exam class
\renewcommand{\solutiontitle}{\noindent}
\unframedsolutions
\SolutionEmphasis{\bfseries}


\pagestyle{headandfoot}
\firstpageheader{\textsc{bi}\,163 Evolution and Ecology}{}{\ifprintanswers\textbf{KEY}\else Name: \enspace \makebox[2.5in]{\hrulefill}\fi}
\runningheader{}{}{\footnotesize{pg. \thepage}}
\footer{}{}{}
\runningheadrule



\begin{document}

\subsection*{Basic statistics practice problems}

%$s_{\overline{Y}}$

Here are five practice problems to calculate mean $\overline{Y}$, standard deviation $s$, and standard error of the mean $s_{\overline{Y}}$.  Answers are on the second page.

$\overline{Y}$ is the mean. $s$ is the standard deviation. The final answer is rounded to two decimal places. I did not round any values during the calculations.


\begin{tabular}[l]{@{}R{0.5in}R{0.75in}R{0.75in}R{0.75in}R{0.75in}R{0.75in}@{}}	
\toprule
	& Problem 1	& Problem 2	& Problem 3	& Problem 4	& Problem 5\\
\midrule
$Y_1$	& 32	& 43	& 48	& 16	& 14\\
$Y_2$	& 23	& 35	& 34	& 15	& 46\\
$Y_3$	& 50	& 12	& 26	& 30	& 27\\
$Y_4$	& 32	& 11	& 29	& 12	& 30\\
$Y_5$	& 29	& 35	& 13	& 50	& 38\\
& 	& 	& 	& \\
$\overline{Y}$	& \rule{0.5in}{0.4pt}	& \rule{0.5in}{0.4pt}	& \rule{0.5in}{0.4pt}	& \rule{0.5in}{0.4pt}	& \rule{0.5in}{0.4pt}\\
&	&	&	& \\
$s$	& \rule{0.5in}{0.4pt}	& \rule{0.5in}{0.4pt}	& \rule{0.5in}{0.4pt}	& \rule{0.5in}{0.4pt}	& \rule{0.5in}{0.4pt}\\
&	&	&	& \\
$s_{\overline{Y}}$	& \rule{0.5in}{0.4pt}	& \rule{0.5in}{0.4pt}	& \rule{0.5in}{0.4pt}	& \rule{0.5in}{0.4pt}	& \rule{0.5in}{0.4pt}\\
\bottomrule
\end{tabular}

\newpage

\subsubsection*{Answer Key}

\begin{tabular}[l]{@{}R{0.5in}R{0.75in}R{0.75in}R{0.75in}R{0.75in}R{0.75in}@{}}	
	\toprule
	& Problem 1	& Problem 2	& Problem 3	& Problem 4	& Problem 5\\
	\midrule
	$Y_1$	& 32	& 43	& 48	& 16	& 14\\
	$Y_2$	& 23	& 35	& 34	& 15	& 46\\
	$Y_3$	& 50	& 12	& 26	& 30	& 27\\
	$Y_4$	& 32	& 11	& 29	& 12	& 30\\
	$Y_5$	& 29	& 35	& 13	& 50	& 38\\
	& 	& 	& 	& \\
	$\overline{Y}$	& 33.2	& 27.2	& 30.0	& 24.6	& 31.0\\
	&	&	&	& \\
	$s$	& 10.08	& 14.70	& 12.71	& 15.81	& 12.04\\
	&	&	&	& \\
	$S_{\overline{Y}}$	& 4.5	& 6.6	& 5.7	& 7.1	& 5.4\\
	\bottomrule
\end{tabular}

	\bigskip
	
	\bigskip
	
	Mean:
	\begin{equation*}
	\overline{Y} = \dfrac{\sum\limits^n_{i=1} Y_i}{n}
	\end{equation*}
	
	\bigskip
	
	Standard Deviation:
	\begin{equation*}
	 s = \sqrt{\dfrac{\sum\left(Y_i - \overline{Y}\right)^2}{n-1}}
	\end{equation*}

	\bigskip

	Standard Error:
	\begin{equation*}
	S_{\overline{Y}} = \dfrac{s}{\sqrt{n}}
	\end{equation*}

\end{document}  