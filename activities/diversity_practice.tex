%!TEX TS-program = lualatex
%!TEX encoding = UTF-8 Unicode

\documentclass[12pt]{exam}


%\printanswers

\usepackage{fontspec}
\setmainfont[Ligatures={TeX}, BoldFont={* Bold}, ItalicFont={* Italic}, BoldItalicFont={* BoldItalic}, Numbers={Monospaced, Lining}]{Linux Libertine O}
\setsansfont[Scale=MatchLowercase,Ligatures=TeX]{Linux Biolinum O}
\setmonofont[Scale=MatchLowercase]{Inconsolatazi4}
\newfontfamily{\tablenumbers}[Numbers={Monospaced,Lining}]{Linux Libertine O}
\usepackage{microtype}

\usepackage{amsmath}

\usepackage{unicode-math}
\setmathfont[Scale=MatchLowercase]{TeX Gyre Termes Math}

\usepackage{geometry}
\geometry{letterpaper, left=1.5in, bottom=1in}                   
%\geometry{landscape}                % Activate for for rotated page geometry
\usepackage[parfill]{parskip}    % Activate to begin paragraphs with an empty line rather than an indent

\usepackage{longtable}

%\usepackage{siunitx}
\usepackage{booktabs}
\usepackage{array}
\newcolumntype{L}[1]{>{\raggedright\let\newline\\\arraybackslash\hspace{0pt}}m{#1}}
\newcolumntype{C}[1]{>{\centering\let\newline\\\arraybackslash\hspace{0pt}}m{#1}}
\newcolumntype{R}[1]{>{\raggedleft\let\newline\\\arraybackslash\hspace{0pt}}m{#1}}

\usepackage{enumitem}
\setlist{leftmargin=*}
\setlist[1]{labelindent=\parindent}
\setlist[enumerate]{label=\textsc{\alph*}.}
\setlist[itemize]{label=\color{gray}\textbullet}
%\usepackage{hyperref}
%\usepackage{placeins} %PRovides \FloatBarrier to flush all floats before a certain point.
%\usepackage{hanging}

\usepackage[sc]{titlesec}

%% Commands for Exam class
\renewcommand{\solutiontitle}{\noindent}
\unframedsolutions
\SolutionEmphasis{\bfseries}


\pagestyle{headandfoot}
\firstpageheader{\textsc{bi}\,163 Evolution and Ecology}{}{\ifprintanswers\textbf{KEY}\else Name: \enspace \makebox[2.5in]{\hrulefill}\fi}
\runningheader{}{}{\footnotesize{pg. \thepage}}
\footer{}{}{}
\runningheadrule



\begin{document}

\subsection*{Shannon-Wiener diversity practice problems}


%The species diversity of a community can be estimated by random sampling of individuals from the community. You can calculate \textit{species richness}, which is the total number of unique species in your sample. You can also calculate \textit{relative abundance}, which the proportion of individuals for each species out of the total number of individuals in your sample.
%
%Species richness and relative abundance (also known as evenness) can be combined mathematically to represent the \textit{species diversity} of a community with a single value (the diversity index). The number by itself does not tell you too much but communities with higher indices have higher diversity.

Species diversity for a sample can be calculated with any of several different equations. The equation we will use is called the Shannon-Wiener diversity index, or Shannon index for short. The Shannon equation is

\[H^{\prime} = -\sum p_i \ln p_i,\] 

where $H^{\prime}$ is the index, $p_i$ is the relative abundance of each species, $ln$ is the natural logarithm, and $\sum$ indicating summing across all species. Notice the negative sign before the summation symbol. This will make the final index positive, as you will see below.

Here is an example calculation with the number of individuals from each of 5 species.

\begin{tabular}{llllll}
Species & A & B & C & D & E \tabularnewline
Number & 11 & 23 & 17 & 24 & 28 \tabularnewline
\end{tabular}

The calculations are summarized in the table at the end of the list.

\begin{enumerate}
\item Calculate the total number of individuals in the sample, $11 + 23 + 17 + 24 + 28 = 103.$

\item Calculate $p_i$ by dividing the number of individuals for each species by the total number of individuals, e.g., 11/103 = 0.107, etc.
%
%\begin{tabular}{lrr}
%Species & $n$ & $p_i$ \tabularnewline
%\toprule
%A & $11$ & $0.107$ \tabularnewline
%B & $23$ & $0.223$ \tabularnewline
%C & $17$ & $0.165$ \tabularnewline
%D & $24$ & $0.233$ \tabularnewline
%E & $28$ & $0.272$ \tabularnewline
%\bottomrule
%\end{tabular}

\item Calculate the natural log of each $p_i$. This is the “ln” key on your calculator. If your calculator does not have a natural log key, use the “log” key for $\log_{10}$. You will get a different result but that is OK. You are comparing numbers (larger = more diverse) so the actual values do not matter much. If your calculator does not have either key, get a better calculator! 

%\begin{tabular}{lrrr}
%Species & $n$ & $p_i$ & $\ln p_i$\tabularnewline
%\toprule
%A & $11$ & $0.107$ & $-2.235$ \tabularnewline
%B & $23$ & $0.223$ & $-1.501$ \tabularnewline
%C & $17$ & $0.165$ & $-1.802$ \tabularnewline
%D & $24$ & $0.233$ & $-1.457$ \tabularnewline
%E & $28$ & $0.272$ & $-1.302$ \tabularnewline
%\bottomrule
%\end{tabular}
%
\item Multiply the two values together, e.g., $0.107 \times -2.235 = -0.239$ and so on.

%\begin{tabular}{lrrrr}
%Species & $n$ & $p_i$ & $\ln p_i$ & $p_i \ln p_i$ \tabularnewline
%\toprule
%A & $11$ & $0.107$ & $-2.235$ & $-0.239$ \tabularnewline
%B & $23$ & $0.223$ & $-1.501$ & $-0.335$ \tabularnewline
%C & $17$ & $0.165$ & $-1.802$ & $-0.297$ \tabularnewline
%D & $24$ & $0.233$ & $-1.457$ & $-0.339$ \tabularnewline
%E & $28$ & $0.272$ & $-1.302$ & $-0.354$ \tabularnewline
%\bottomrule
%\end{tabular}

\item Sum together the values in the final column.

\item Notice the final sum is a negative value. That is why the negative sign appears before the summation symbol in the equation above. It turns the final value positive. 
\end{enumerate}

\begin{tabular}{lrrrr}
Species & $n$ & $p_i$ & $\ln p_i$ & $p_i \ln p_i$ \tabularnewline
\toprule
A & $11$ & $0.107$ & $-2.235$ & $-0.239$ \tabularnewline
B & $23$ & $0.223$ & $-1.501$ & $-0.335$ \tabularnewline
C & $17$ & $0.165$ & $-1.802$ & $-0.297$ \tabularnewline
D & $24$ & $0.233$ & $-1.457$ & $-0.339$ \tabularnewline
E & $28$ & $0.272$ & $-1.302$ & $-0.354$ \tabularnewline
\midrule
  &  $103$    &         & \hfill $\sum$ & $-1.564$\tabularnewline
\bottomrule
\end{tabular}

The Shannon diversity for this sample is $H^{\prime} = 1.564.$

\subsection*{Practice Problems}

Calculate $H^{\prime}$ for each community, then identify the most and least diversity communities.  The correct answers for $\ln$ and $\log_{10}$ are given for you to check your work. Be sure to compare values calculated by the same logarithm function (i.e., only $\ln$ or only $\log_{10}$). Your numbers may be off slightly due to rounding but you should not be off by more than 0.010 or so.

\begin{enumerate}[label=\textsc{Community \alph*:}]

\item 74 71 75 84 55 56 71 54

 Answer $\ln: 2.067$
 
 Answer $\log_{10}: 0.898$

\item 90 93 65 85 74 91 73 85

 Answer $\ln: 2.072$
 
 Answer $\log_{10}: 0.900$

\item 65 65 90 51 83 99 50 52

 Answer $\ln: 2.047$
 
 Answer $\log_{10}: 0.889$

\item 80 43 37 77 94 10 67 63

 Answer $\ln: 1.965$
 
 Answer $\log_{10}: 0.854$

\item 80 82 66 50 38 87 31 38

 Answer $\ln: 2.014$
 
 Answer $\log_{10}: 0.875$

\end{enumerate}

Community B is the most diverse. Community D is the least diverse.

\end{document}  