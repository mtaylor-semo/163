%!TEX TS-program = lualatex
%!TEX encoding = UTF-8 Unicode

\documentclass[12pt, addpoints]{exam}

%\printanswers

\usepackage{fontspec}
\setmainfont[Ligatures={TeX}, BoldFont={* Bold}, ItalicFont={* Italic}, BoldItalicFont={* BoldItalic}, Numbers={OldStyle, Lining}]{Linux Libertine O}
\setsansfont[Scale=MatchLowercase,Ligatures=TeX]{Linux Biolinum O}
\setmonofont[Scale=MatchLowercase]{Inconsolatazi4}
\newfontfamily{\tablenumbers}[Numbers={Monospaced,Lining}]{Linux Libertine O}
\usepackage{microtype}

\usepackage{graphicx}
	\graphicspath{{/Users/goby/Pictures/teach/163/activities/}}
	
\usepackage{amsmath}


%\usepackage{unicode-math}
%\setmathfont[Scale=MatchLowercase]{TeX Gyre Termes Math}

\usepackage{geometry}
\geometry{letterpaper, left=1in, bottom=1in}                   
\usepackage[parfill]{parskip}    % Activate to begin paragraphs with an empty line rather than an indent

%\usepackage{siunitx}
\usepackage{booktabs}

\usepackage{array}
\newcolumntype{L}[1]{>{\raggedright\let\newline\\\arraybackslash\hspace{0pt}}m{#1}}
\newcolumntype{C}[1]{>{\centering\let\newline\\\arraybackslash\hspace{0pt}}m{#1}}
\newcolumntype{R}[1]{>{\raggedleft\let\newline\\\arraybackslash\hspace{0pt}}m{#1}}

\usepackage{multicol}

\usepackage{threeparttable}

\usepackage[justification=raggedright, labelsep=period]{caption}
\captionsetup{singlelinecheck=off}
\captionsetup{skip=0.2em}

\usepackage{enumitem}
\setlist{leftmargin=*}
\setlist[1]{labelindent=\parindent}
\setlist[enumerate]{label=\textsc{\alph*}.}
%\setlist[itemize]{label=\color{gray}\textbullet}

%\usepackage{hanging}

\usepackage[sc]{titlesec}

\newcommand{\sws}{\textsc{sws}}
\newcommand{\mws}{\textsc{mws}}
\newcommand{\lws}{\textsc{lws}}

%% Commands for Exam class
\newlength{\myindent}
\setlength{\myindent}{\parindent}
\newcommand{\ind}{\hspace*{\myindent}}

\pointpoints{ pt}{ pts}

\renewcommand{\solutiontitle}{\noindent}
\unframedsolutions
\SolutionEmphasis{\bfseries}

\renewcommand{\questionshook}{%
	\setlength{\leftmargin}{-\leftskip}%
}

\pagestyle{headandfoot}
\firstpageheader{\textsc{bi}\,163 Evolution and Ecology}{}{\ifprintanswers\textbf{KEY}\else Name: \enspace \makebox[2.5in]{\hrulefill}\fi}
\runningheader{Evolution of primate trichromacy}{}{\footnotesize{pg. \thepage}}
\footer{}{}{}
\runningheadrule

\newcommand*\AnswerBox[2]{%
    \parbox[t][#1]{0.92\linewidth}{%
    \begin{solution}#2\end{solution}}
    \vspace{\stretch{1}}
}


\begin{document}

\subsection*{Evolution of trichromacy in primates (10~points)}

\textbf{You may work in pairs but all explanations must be your own words.}

\begin{questions}

\begin{multicols}{2}

\question
The mutation rate for the \textit{cytochrome~b} gene in primates is 0.2\% per million years. %Irwin~et~al.~(1991)
Divide the mutation rate into the genetic differences listed in Table~\ref{tab:divergence} (page~\pageref{tab:divergence}) to estimate the time (\textsc{mya}) since each clade last shared an ancestor (Fig.~\ref{fig:primate_phylogeny}, lettered branches). 

\bigskip


\question
Plot the relationship for genetic difference and time since common ancestor in Fig.~\ref{fig:divergence_plot} (page~\pageref{tab:divergence}). Add a trend line to show the “average” relationship between genetic difference and time since common ancestry for the nine primate clades listed in Table~\ref{tab:divergence}.

\bigskip

\question
Table~\ref{tab:amino_acid_properties} lists some non-polar and polar amino acids that differ between the \mws{} and \lws{} opsins. Table~\ref{tab:amino_acid_diffs} lists the differences between \mws{} and \lws{}~opsins. Place an {\large $\times$} below each \textit{non-polar to polar} amino acid change in the table.

\bigskip

\question\label{ques:total_shift}
Each type of non-polar to polar amino acid changes the sensitivity by an average of about 10~nm. From your results in Table~\ref{tab:amino_acid_diffs}, what is the total nanometer shift?

\medskip

Total shift: \rule{1.5cm}{0.4pt}~nm.

\question\label{ques:long_peak}
 The peak sensitivity of \mws{}~opsin is around 541 nanomters. Add the value you calculated in question~\ref{ques:total_shift} to the \mws{} peak to calculate the peak sensitivity for \lws{}~opsin.

\medskip

\lws{} peak sensitivity: \rule{1.5cm}{0.4pt}~nm.

\question
Add a relative sensitivity curve for \lws{}~opsin to Fig.~\ref{fig:relative_sensitivity} (page~\ref{fig:relative_sensitivity}). Draw your curve just like the curve for \mws{}~opsin but shifted to the right by the value you calculated in question~\ref{ques:long_peak}.

\columnbreak

{\centering\includegraphics[width=0.99\linewidth]{primate_phylogeny_vertical}
\par
\captionof{figure}{Phylogeny of the primates. Letters indicate clade ancestors listed in Table~\ref{tab:divergence}.}\label{fig:primate_phylogeny}
}


\vfill
\bigskip

\hfill{\large See page~\pageref{last_page} for final questions.}

\end{multicols}


{\renewcommand{\arraystretch}{1.0}
\begin{threeparttable}
\caption{Nine primate clades and the genetic difference within each clade. Not all ancestors are labeled on the tree.}\label{tab:divergence}
\begin{tabular}{@{}llrR{1in}@{}}
\toprule
Node & Ancestor & Genetic difference & Time (\textsc{mya}) \tabularnewline
\midrule
A & Primate & 16.4\% & \ifprintanswers 82.0 \else \newline\rule{0.9in}{0.4pt}\fi \tabularnewline
B & Haplorhini & 15.5\% & \ifprintanswers 77.5 \else \newline\rule{0.9in}{0.4pt}\fi \tabularnewline
C & Stepsirrhini & 12.5\% & \ifprintanswers 62.5 \else \newline\rule{0.9in}{0.4pt}\fi \tabularnewline
D & Anthropoidea & 8.9\% & \ifprintanswers 44.5 \else \newline\rule{0.9in}{0.4pt}\fi \tabularnewline
E & Catarrhini & 6.0\% & \ifprintanswers 30.0 \else \newline\rule{0.9in}{0.4pt}\fi \tabularnewline
F & Platyrrhini & 4.4\% & \ifprintanswers 22.0 \else \newline\rule{0.9in}{0.4pt}\fi \tabularnewline
— & \textit{Cebus\,–\,Callothrix} & 4.8\% & \ifprintanswers 24.0 \else \newline\rule{0.9in}{0.4pt}\fi \tabularnewline
G & Cercopithecoidea & 2.9\% & \ifprintanswers 14.5 \else \newline\rule{0.9in}{0.4pt}\fi \tabularnewline
— & \textit{Homo\,–\,Pongo} & 3.6\% & \ifprintanswers 18.0 \else \newline\rule{0.9in}{0.4pt}\fi \tabularnewline
\bottomrule
\end{tabular}
\end{threeparttable}}

\bigskip

\parbox{\linewidth}{%
\ifprintanswers
\includegraphics[width=0.82\linewidth]{primate_divergence_plot_key}
\else
\includegraphics[width=0.82\linewidth]{primate_divergence_plot_blank}
\fi
\captionof{figure}{Plot of genetic differences and time since common ancestor for nine primate clades.}\label{fig:divergence_plot}\par
}
\bigskip

\begin{threeparttable}[t]
\captionsetup{type=table, position=top,skip=0pt}
\captionof{table}{Some non-polar and polar amino acids, and their abbreviations.}\label{tab:amino_acid_properties}
\begin{tabular}{@{}ll@{}}
\toprule
Non-polar 			& Polar \tabularnewline
\midrule
Alanine (Ala) 		& Asparagine (Asn)	\tabularnewline
Isoleucine (Ile)	& Glutamine	(Gln)	\tabularnewline
Leucine	(Leu)		& Serine (Ser)		\tabularnewline
Methionine (Met)	& Threonine (Thr)	\tabularnewline
Phenylalanine (Phe)	& Tyrosine (Tyr)	\tabularnewline
Proline	(Pro)		&					\tabularnewline
Valine (Val)		&					\tabularnewline
\bottomrule
\end{tabular}\par
\end{threeparttable}\par

\bigskip

\begin{threeparttable}
\caption{Some non-polar and polar amino acid differences between \mws{} and \lws{}~opsins. Numbers indicate amino acid positions in the opsin protein.}\label{tab:amino_acid_diffs}
\begin{tabular}{@{}*{14}{l}@{}}
\toprule
 & 111 & 116 & 153 & 171 & 180 & 236 & 274 & 275 & 277 & 279 & 285 & 296 \tabularnewline
\midrule
\textsc{mws} & Val & Tyr & Met & Ile & Ala & Val & Val & Leu & Phe & Phe & Ala & Pro \tabularnewline
\textsc{lws} & Ile & Ser & Leu & Val & Ser & Met & Ile & Phe & Tyr & Val & Thr & Ala \tabularnewline
\midrule\noalign{\medskip}
 {\Large $\times$} 	   & \rule{0.5cm}{0.4pt}
 & \rule{0.5cm}{0.4pt} & \rule{0.5cm}{0.4pt}
 & \rule{0.5cm}{0.4pt} & \ifprintanswers{\large $\times$}\else\rule{0.5cm}{0.4pt}\fi
 & \rule{0.5cm}{0.4pt} & \rule{0.5cm}{0.4pt} 
 & \rule{0.5cm}{0.4pt} & \ifprintanswers{\large $\times$}\else\rule{0.5cm}{0.4pt}\fi 
 & \rule{0.5cm}{0.4pt} & \ifprintanswers{\large $\times$}\else\rule{0.5cm}{0.4pt}\fi  
 & \rule{0.5cm}{0.4pt} \tabularnewline
\bottomrule
\end{tabular}
\end{threeparttable}

\bigskip

 
\parbox{\linewidth}{
\ifprintanswers
	\includegraphics[width=0.93\linewidth]{trichromat_plain_plot}
\else
	\includegraphics[width=0.93\linewidth]{dichromat_plain_plot}
\fi
\captionof{figure}{Relative sensitivity for short, medium, and long wavelength opsins.}\label{fig:relative_sensitivity}\par
}


\question
Studies across many genera from all of the primate clades shown in Fig.~\ref{fig:primate_phylogeny} have found that only great apes and Old World monkeys have the long wavelength gene. Based on the phylogeny and your calculations from the molecular clock, how long ago do you estimate that the gene duplication must have occurred?  Explain.

\AnswerBox{0.3\textheight}{No more than about 40 million years ago because all great apes and Old World monkeys have the gene. It probably occurred in the ancestor labeled E.}

\question
\textsc{Discussion question:} New World monkeys do not have the \lws{}~opsin gene yet females (and a few males) in some New World genera can be at least partially sensitive to long wavelengths, and sometimes as sensitive as Old World monkeys and great apes.  How can this be so? \textsc{Hints:} The \mws{}~opsin is on the X~chromosome. \emph{Each} non-polar to polar mutation at the three sites causes a slight increase in the wavelength sensitivity.

\AnswerBox{0.3\textheight}{Mutations create alleles of the \mws{} gene. Some mutations have occurred at the amino acid sites shown above. Females have two X~chromosomes so can end up with two different alleles, creating variation in the color vision. Males have only one X~chromosome so are less likely to have a shifted allele.}

\label{last_page}\end{questions}

\end{document}  