%!TEX TS-program = lualatex
%!TEX encoding = UTF-8 Unicode

\documentclass[t]{beamer}

%%%% HANDOUTS For online Uncomment the following four lines for handout
%\documentclass[t,handout]{beamer}  %Use this for handouts.
%\usepackage{handoutWithNotes}
%\pgfpagesuselayout{3 on 1 with notes}[letterpaper,border shrink=5mm]

%\includeonlylecture{student}

%%% Including only some slides for students.
%%% Uncomment the following line. For the slides,
%%% use the labels shown below the command.

%% For students, use \lecture{student}{student}
%% For mine, use \lecture{instructor}{instructor}

% FONTS
\usepackage{fontspec}
\def\mainfont{Linux Biolinum O}
\setmainfont[Ligatures={Common,TeX}, Contextuals={NoAlternate}, BoldFont={* Bold}, ItalicFont={* Italic}, Numbers={OldStyle}]{\mainfont}
\setsansfont[Ligatures={Common,TeX}, Scale=MatchLowercase, Numbers=OldStyle, BoldFont={* Bold}, ItalicFont={* Italic},]{Linux Biolinum O} 
\usepackage{microtype}

\usepackage{graphicx}
	\graphicspath{{/Users/goby/pictures/teach/163/activities/}
	{/Users/goby/pictures/teach/common/}} % set of paths to search for images

\usepackage{caption}

\usepackage{multicol}
%\usepackage{booktabs}
%\usepackage{textcomp}

\usepackage{tikz}
	\tikzstyle{every picture}+=[remember picture,overlay]
%	\usetikzlibrary{arrows}


\mode<presentation>
{
  \usetheme{Lecture}
  \setbeamercovered{invisible}
}


\begin{document}
%
\lecture{student}{student}
%
\begin{frame}{Our goal for this activity is to explore the evolution of trichromacy in primates.}
	
	\begin{tabular}{ccc}
	\includegraphics[width=0.3\linewidth]{colobus_monkey} &
	\includegraphics[width=0.3\linewidth]{capuchin_monkey} &
	\includegraphics[width=0.3\linewidth]{yellow_cheeked_gibbon} \tabularnewline
	Colobus Monkey & Capuchin Monkey & Buff-cheeked Gibbon \tabularnewline
	\end{tabular}
		
	\vfilll
	
	\tiny Eric Kilby, Wikimedia Commons, \ccbysa{2} \hfill
	David M. Jensen, Wikimedia Commons, \ccbysa{2}\hfill
	Derek Keats, Flickr, \ccby{2}
\end{frame}
% 
\begin{frame}{Phylogeny of major primate clades.}


\includegraphics[width=\linewidth]{primate_phylogeny}

\vfilll

\hfill \tiny Perelman et~al.~2011. PLoS Genetics 7(3):~e1001342.

\end{frame}
%
\begin{frame}[t]{Humans have three color opsins (trichromacy)}
	
	\includegraphics[width=\linewidth]{spectrum_trichromat}
	
\end{frame}
%
\begin{frame}[t]{Most mammals have two color opsins (dichromacy)}
	
	\includegraphics[width=\linewidth]{spectrum_dichromat}
	
\end{frame}
%
{
\setbeamercolor{background canvas}{bg=black}
\begin{frame}{\textcolor{white}{Opsins have an amino acid “pocket” surrounding a retinal molecule.}}
\vspace{-1.5\baselineskip}
{\centering
\includegraphics[height=0.83\textheight]{rhodopsin_structure}
\par}
\vfilll
\hfill \tiny \textcolor{white}{Roland Deschain, Wikimedia Commons, public domain}
\end{frame}
}
%
\begin{frame}{Amino acid polarity in the pocket can alter wavelength sensitivity.}

\vspace{-0.5\baselineskip}
{\centering
\includegraphics[height=0.79\textheight]{amino_acid_structures}\par}

\vfilll

\hfill \tiny \href{https://commons.wikimedia.org/wiki/File:Figure_03_04_02.jpg}{C\textsc{nx} OpenStax, Wikimedia Commons} \ccby{4}
\end{frame}
%
%
\begin{frame}{You will }
	
	\hangpara use a molecular clock to calculate the time of the last common ancestor for different primate clades,
	
	\hangpara calculate the shift of wavelength sensitivity in a duplicated opsin gene, and
	
	\hangpara estimate when the gene duplication event happened.
	
\end{frame}
% 
\begin{frame}{Genetic difference and time since common ancestor.}

\includegraphics[width=\linewidth]{primate_divergence_plot_key}
\end{frame}
%
\begin{frame}[t]{Relative sensitivity for dichromats.}

	\vspace{-0.5\baselineskip}

	\includegraphics[width=\linewidth]{dichromat_plot}
	
\end{frame}
%

\begin{frame}[t]{A 30~nm shift increases long-wavelength sensitivity.}
	
	\vspace{-0.5\baselineskip}

	\includegraphics[width=\linewidth]{trichromat_plot}
	
\end{frame}
%

%
\end{document}
