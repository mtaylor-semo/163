%!TEX TS-program = lualatex
%!TEX encoding = UTF-8 Unicode

\documentclass[t]{beamer}

%%%% HANDOUTS For online Uncomment the following four lines for handout
%\documentclass[t,handout]{beamer}  %Use this for handouts.
%\usepackage{handoutWithNotes}
%\pgfpagesuselayout{3 on 1 with notes}[letterpaper,border shrink=5mm]


%%% Including only some slides for students.
%%% Uncomment the following line. For the slides,
%%% use the labels shown below the command.
%\includeonlylecture{student}

%% For students, use \lecture{student}{student}
%% For mine, use \lecture{instructor}{instructor}


%\usepackage{pgf,pgfpages}
%\pgfpagesuselayout{4 on 1}[letterpaper,border shrink=5mm]

% FONTS
\usepackage{fontspec}
\def\mainfont{Linux Biolinum O}
\setmainfont[Ligatures={Common,TeX}, Contextuals={NoAlternate}, Numbers={Proportional, OldStyle}]{\mainfont}
\setsansfont[Ligatures={Common,TeX}, Scale=MatchLowercase, Numbers={Proportional,OldStyle}, BoldFont={* Bold}, ItalicFont={* Italic},]\mainfont

\newfontface\lining[Numbers={Lining}]\mainfont

\usepackage{graphicx}
	\graphicspath{{/Users/goby/pictures/teach/163/activities/}
	{/Users/goby/pictures/teach/common/}} % set of paths to search for images

%\usepackage{units}
\usepackage{booktabs}
\usepackage{multicol}
%\usepackage{textcomp}

\usepackage{tikz}
%	\tikzstyle{every picture}+=[remember picture,overlay]

\mode<presentation>
{
  \usetheme{Lecture}
  \setbeamercovered{invisible}
  \setbeamertemplate{items}[square]
}

%\usefonttheme[onlymath]{serif}
%\usecolortheme[named=blue7]{structure}

\begin{document}
%
%
\begin{frame}{I will show you six organisms, each with a distinctive trait.}

\hangpara \highlight{Thought question:} Why did each trait evolve?

\end{frame}
%
{
\usebackgroundtemplate{\includegraphics[width=\paperwidth]{lizard_green_blood} }
\begin{frame}[t]{Some lizards in Papua New Guinea have \textcolor{green6}{green} blood.}

\vfilll 

\hfill \textcolor{white}{\tiny \copyright~Christoper Austin}
\end{frame}
}
%
{
\usebackgroundtemplate{\includegraphics[width=\paperwidth]{banana_slug} }
\begin{frame}[t]{Banana slugs are hermaphrodites (both female and male).}

\vfilll 

\hfill \textcolor{white}{\tiny Ben Stanfield, Flickr, \ccby2}
\end{frame}
}
%
{
\usebackgroundtemplate{\includegraphics[width=\paperwidth]{human_hand} }
\begin{frame}[t]{The human hand has five digits.}

\vfilll 

\hfill \textcolor{black}{\tiny Zephyris, Wikimedia Commons, \ccby3}
\end{frame}
}
%
{
\usebackgroundtemplate{\includegraphics[width=\paperwidth]{skunk_cabbage} }
\begin{frame}[t]{Skunk cabbage can elevate its internal temperature.}

\vfilll 

\tiny \textcolor{white}{dogtooth77, Flick, \ccbyncsa2}
\end{frame}
}
%
{
\usebackgroundtemplate{\includegraphics[width=\paperwidth]{anglerfish} }
\begin{frame}[t]{\textcolor{white}{Male anglerfish bite and fuse to the body of the female.}}

\vfilll 

\hfill\tiny\textcolor{white}{\textcopyright\,Theodore W. Pietsch, University of Washington.}
\end{frame}
}
%
{
\usebackgroundtemplate{\includegraphics[width=\paperwidth]{red_backed_spider} }
\begin{frame}[t]{Red-backed spiders commit sexual suicide.}

\vfilll 

\hfill \tiny David McClenaghan, Wikimedia Commons, \ccby3
\end{frame}
}
%
\begin{frame}

\hangpara \begin{tabular}{@{}ll@{}}
\toprule
Organism			& Trait \\
\midrule
Lizard				& Green Blood \\
Banana slug			& Hermaphrodites \\
Humans				& Five digits \\
Skunk cabbage		& Elevated temperature \\
Anglerfish			& Fused males \\
Red-backed spider	& Sexual suicide \\
\bottomrule
\end{tabular}

\bigskip

\hangpara Work in pairs. Pick any three species.

\hangpara For each species, write a \emph{detailed} hypothesis that explains why each trait evolved. Try to think of the function of each trait and explain how the trait might have evolved to serve that function.

\hangpara  Do not tell me the trait increases their ability to survive. Do you think five fingers \emph{really} increases your chance of surviving compared to four or six fingers?

\end{frame}
%

\end{document}
